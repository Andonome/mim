\section{Archwarp Fallen}
\label{lostcity}

\begin{multicols}{2}

\mapPic{t}{Dyson_Logos/forgotten_city}{
  \ref{fallen_tower}/36/78,
  \ref{basementTrapdoor}/22/70,
  \ref{campfire}/28/70,
  \ref{lostGriffins}/73/80,
  \ref{shiningLake}/66/49,
  \ref{lostDoor}/56/35,
}

\histEvent{299}{4}{The Shadow Gate drives traders between Sixshadow and Archwarp}
\begin{exampletext}
  Some centuries ago, a town known as Archwarp rose to prominence and wealth due to the magical gateway created by alchemists.

  Since then, the gateway has closed, the population left, and most of the stone requisitioned for other structures and sent downstream.

  A few patches of stone building, or artistic structures remain to this day, causing all sorts of rumours among the rare rangers who pass through the area.
\end{exampletext}

\begin{boxtext}
  The trees and vines here cling to blocks of stone, as if trying to crush the last remnants of civilization.
  You sometimes wander for so long without seeing a single brick that you forget you're anywhere but a normal forest, but then another errant stone, or the marble hand of a statue juts up not far from the riverbank, to remind you of the ghosts resting in this place.
\end{boxtext}

\begin{exampletext}
  \noindent
  \Pgls{southSeeker} of the \gls{wolfhead} came here in search of the old alchemy library, and found it.
  However, he didn't dare enter the old door to the alchemy basement's library he found, so he hired a few of the \gls{guard} to enter, and check the barrow for safety.
  He planned to have them `split the loot', and then help them sell whatever valuables remained inside, and hoped they didn't see (or didn't understand) the alchemical gateway.
\end{exampletext}

\subsection{Journeying to Archwarp}
If the troupe manage to procure a boat, they might arrive by journeying upriver; otherwise they must walk.

\paragraph{Anyone ferrying them by boat}
will demand at least 5~\glspl{sp} per day, and will not hang around to be attacked (so if a wandering monster happens by, they will leave or die).

\paragraph{During the journey towards Archwarp,}
the troupe will spot \gls{southSeeker} leaving.
He will give them an awkward stare, as he sits in a sail-boat big enough for half a dozen men at least.
He knows that they must be going to investigate something, and he knows that they know he was there.

If the \glspl{pc} don't know him, he will say nothing, wave frostily, and continue to let the currents guide him.
However, if he considers them reasonable acquaintances, he will stop and arrange to get what he wants while giving away as little information as possible.

\southSeeker

\subsection{Campfire Embers}

\begin{enumerate}
  \item
  This old, fallen tower supports the last roof (made of stone) in all of Archwarp.
  \Gls{southSeeker}'s bedding remains here.
  \label{fallen_tower}
  \item
  This hidden doorway has so much topsoil grown over it that nobody could imagine it was there.
  However, anyone in the basement below, in room \vref{alchemySecret}, may notice the underside, and dig their way up.
  \label{basementTrapdoor}
  \item
  The campfire shows evidence of a dozen people.
  The \glspl{pc} can roll \roll{Intelligence}{Vigilance} to understand what the various debris around the campfire means.
  \begin{nametable}{Campfire Investigation}
  \textbf{Roll} & \textbf{Result} \\
  \hline
    7 & A dozen men were here. \\
    8 & They ate a griffin. \\
    9 & \glspl{guard}. \\
    10 & Only one person left the area. \\
    11 & Someone prepared griffin feathers to make \pgls{boon} over the camp fire. \\
  \end{nametable}
  \label{campfire}
  \item
  These blossoming fruit trees hold a trap: mouthdiggers.%
  \exRef{judgement}{Judgement}{mouthdigger}
  The amount of fruit depends upon the season.
  \begin{description}
    \item[Cold]
    seasons mean nothing but snow on the trees.
    \item[Hot]
    seasons see the tree nearly bursting with food.
    The trees hold 6 days' worth of food (which goes rotten within 2 days)
    and the mouthdiggers here have two litters of 6 sprogs each.%
    \footnote{Mouthdigger children can be used as \glspl{ingredient}.}
    \item[Mild]
    seasons mean the trees contain 4 days' worth of rations.

  \end{description}
  \mouthdigger[\npc{\T[3]\A}{\arabic{noAppearing} Mouthdiggers}]
  \label{lostGriffins}
  \item
  This glistening lake holds an acidic ooze.%
  \exRef{judgement}{Judgement}{ooze}
  However, during cold seasons, the rivers and lakes freeze over, leaving the ooze in a silent and pure contemplation.

  \jelly
  \label{shiningLake}
  \item
  This old basement door lay under the mud for some centuries, but it became a good deal more obvious once \gls{southSeeker} opened it.
  The \glspl{pc} can roll \roll{Wits}{Vigilance} (\tn[8]) to locate it.
  \label{lostDoor}
\end{enumerate}

\subsection{Before the Journey Back}

\begin{exampletext}
  The \glspl{guard} who came with \gls{southSeeker} to investigate the old wooden door already had a mission from their \gls{jotter}, and \gls{investigator} noticed their disappearance instantly.
  He ferreted out a dozen drunken \glspl{guard} from \gls{town} and told them he'd send them all to \gls{paik}%
  \footnote{\ldots meaning `they would hang'.}
  personally if they failed to come with him and hunt down the deserters.

\end{exampletext}

\investigator

\humanthief[\npc{\T[3]\Hu}{\arabic{noAppearing} \Glspl{guard}}]

\humansoldier[\npc{\T[3]\Hu}{\arabic{noAppearing} \Glspl{guard}}]

\Gls{investigator} will arrive \pgls{interval} after the \glspl{pc} arrive, so he may wait at the camp site while they investigate the alchemical basement, or he may wander up behind them, and decide that \emph{the \glspl{pc} are the \gls{guard} deserters}.

\paragraph{If \gls{southSeeker} passed the \glspl{pc} silently on the river,}
he became wary that they might double-back and come after him, so he pulled his boat to the bank, and stayed hidden at the side.
This means that \gls{investigator} did not see \gls{southSeeker}, and if the \glspl{pc} mention him going downriver, he will decide they are all liars, and then inform them of his decision.

