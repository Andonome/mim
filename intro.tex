\chapter{Protagonists \& Plots}
  \epigraph{He attacked everything in life with a mix of extraordinary genius and na\"ive incompetence, and it was often difficult to tell which was which.}{Douglas Adams}
\label{sideQuestIntro}

\label{Irina/greylands}

\section{Plots \& Schemes}

\begin{multicols}{2}

\mapPic[\large]{t}{Dyson_Logos/map_circles}{
  \rule[.3em]{.16\textwidth}{2pt}/86/04,
  \Large 10 miles/86/08,
  \N\normalsize\glsentrytext{redfall}/40/52,
  \N\nameref{lostcity}/32/96,
  \Glsentrytext{town}/54/67,
  \D\nameref{necromancers_lair}/13/22,
  \rotatebox{55}{\normalsize\nameref{lakeside}}/85/49,
  \rotatebox{10}{\normalsize crossroads}/70/67,
}

\noindent
Trouble has arrived at \gls{town}, as two great organizations begin their grab for power.

\subsection{The Woodspy Bandits}

\paragraph{\gls{townmaster}} feels dissatisfied with only ruling \gls{town}, and wants to expand the lands he owns, but \gls{king} hasn't granted him an army, so he has begun patronizing a local bandit group known as `the woodspy bandits'.

The forest possesses a wealth of stone from a forgotten civilization, so his men will use this to create outposts, from where they can safely live, and later begin the work of creating stone-walled towns.

\paragraph{\gls{traitor}}
secretly helps \gls{townmaster} organize the woodspy bandits, porting messages, and organizing weekly food drops.
He hopes to become \pgls{village} master or town master himself one day, once the area expands enough.
He zealously inspects \gls{lostcity} for any precious or informative artefacts.

\paragraph{\Gls{nurabaron}}
lives in \gls{redfall}, and believes strongly in \gls{townmaster}'s vision of expanding the world, so his soldiers ferry supplies to the woodspy bandits.

\subsection{The Immortal Bandits}

\paragraph{\Gls{necromancer}}
once worked as a priest of Eldren in \gls{lostcity}, but simply refused to die.
His slightly rotten, but still moving body, inhabits a crumbling, forgotten church.
The centuries of isolation have given him a phobia of living people, so he wants to gather an army of ghouls to keep himself safe from the living.

Now that the immortal bandits have formed, he has fashioned half a dozen magical to help them walk and rob tirelessly.
In return, they bring him any corpses they find (or create).

\paragraph{\gls{banditking}}
grew up as a noble of Whiteplains, but had to flee, and now lives as a bandit.
After discovering \gls{necromancer}'s lair, he made an alliance, and now stays there, along with various bandits, known as `the immortal bandits'.

\paragraph{\gls{sewerking}}
is \gls{banditking}'s brother, but stays in the forgotten underbelly of \gls{town}.
He knows that \gls{lostcity} contained magical portals to the Nura Realm, and hopes to find one near Lakeside.
He knows that if he can cut off the only road to \gls{town}, the entire land will become isolated, and the immortal bandits can then take power without \gls{king}'s army interfering.

\subsection{The Outsider}

\paragraph{\gls{forestpriest}}
has heard \gls{townmaster}'s anti-elven rants, seen traders drop food in the forest (for the woodspy bandits), and seen the immortal bandits organizing themselves with more of a plan than any other bandit group.
He doesn't know what any of this means yet, and has decided to discover the truth.

\Gls{forestpriest} has exceptional powers, and will come down hard on one side, if he can only feel certain about whether or not the local elves, or the nura present the most real danger to the area.
However, until he knows the truth, he has decided to use the miracles granted to him by Alass\"e to turn anyone he finds aiding bandits into a creature of some kind.

\iftoggle{aif}{
  The stories begin during the season of \season\ (take a look at \autoref{astronomy} for more on how seasons work in Fenestra).
}{}

\subsection{In their Own Words}
\label{expanding_wilderness}

In the words of \gls{townmaster}:

\begin{exampletext}

  Our city has burnt to the ground so long ago that people have forgotten where it is.
  They just call it `\gls{lostcity}'.
  We have all heard of the \gls{guard} finding marble statues, or vine-covered walls beyond the \gls{edge}, but do you know why?
  Back before Rex Dalius, \glsentrylong{town} stretched across every part of the forest, through all the \glspl{village}.
  Every road and hamlet around \gls{town} once rested safely, surrounded by towering stone walls.
  It had its own little branch of \gls{alchemists}, and a great logging industry.
  It had temples to all the gods, and great shipyards.
  The great city held the surrounding farmlands inside its great rings of walls.
  It went so far out that a man could not walk across it in a single day.

  Then the evil elves came.
  Elves cannot lift great rocks to build walls, like men.
  They feel jealous of everything we do, so they destroyed \gls{lostcity}, and with evil magics toppled our walls, and made bushes and vines grow to cover everything we had built!

  The common folk returned here a few generations ago, and picked up all the stone along the roads for their houses and little walls, but if you journey just a little out into the darkness, you'll find ancient artefacts still resting, or half-buried.

  I will rebuild our heritage, brick and stone, walls and farms, temples and hearths.

  Of course, I require an army, and of course \gls{king} no longer permits armies.
  His clear negligence has allows beasts and bandits to run loose throughout the land\ldots which gave me a plan.
  What about simply employing the bandits? -- genius!
  After all, I do not have to wander out and speak with them personally.
  Hell, I don't even need to reveal myself.
  I don't need to trust them either.
  My two loyal servants, \gls{traitor} and \gls{alchemist}, take deliveries of food, coin, and sometimes even weapons, into the forest.
  A separate note explains where the reward sits, and what they must do.
  In this way, I have kept all the bandits busy, and the people truly have less to fear.
  People see them seldom, but everyone knows they still exist, so the locals have taken to calling them the `Woodspy Bandits'.
  I rather like the name.

  So far, we have built an outpost and I have ordered it be painted green, for camouflage.
  We cannot let anyone in on the action until some fortifications have been placed.
  
  I will bring up defence towers, rebuild the old walls in the forest, and finally reclaim our land!

\end{exampletext}

In the words of \gls{traitor}:

\begin{exampletext}

  My Lord,
  I have found a map in \gls{town}'s library, with markings around \gls{lostcity} (I have enclosed a copy).
  I cannot see where these locations lie, but I will attempt to find each of them.

  \Gls{captain} does not suspect me of anything wrong.
  He is an idiot, and I have simply told him I am scouting, and whenever I return, I bring him more news of the bandits we ran into, now known as `the immortal bandits'.
  Even if I see nothing, I still return with some news of them.
  Right now, he feels they are everywhere, and even has a map showing them surrounding the city on all sides.
  He has sent men to hunt for them everywhere.

  \Gls{greentower}'s construction is going as planned, and will finish soon.
  We have brought limestone paint from Redfall, and the outside will be painted soon.
  It already looks invisible from far away.

  I remain your faithful servant,

  \gls{traitor}

\end{exampletext}

In the words of \gls{banditking}:

\begin{exampletext}

  We came from something, and we'll return.

  After \gls{king} sent his the \gls{guard} to kill our parents, take our land, and hand it to bureaucrats, my brother \gls{sewerking} fled from \gls{college}, and I guided him safely to Mount Arthur.
  \iftoggle{aif}{
    \footnote{See \textit{Fenestra}, \autopageref{whiteplainsWar} for the massacre of Whiteplains.}
  }{}%
  We've seen ten cycles pass, and started planning our return to nobility.
  And if \gls{king} can play rough, then so can we.

  We've robbed a lot of people, and I don't feel good about that, but we really have no choice.
  We've done nothing worse than \gls{king}.

  I chose \gls{town} as it was out of the way, and we started to explore the area.
  Just him and me, wandering with our rations and picking up food.
  We were essentially living like bandits even before we became bandits.
  Soon enough, I found our home-to-be, where the dead priest lives.
  He's a shy creature, but we've gained an understanding.

  We came to his temple at night, so I saw an arrow stabbing into the earth before I saw the building.
  I could clearly see it was a warning, but didn't understand how anyone could possibly see us walking in the darkness.
  Any idiot could tell that nobody sat there with an army, or we would have heard them, and they could have killed us.
  So we kept back until daylight, and went out again, cloaks waving about, hailing whoever might be there.
  And another arrow went into the ground -- another warning shot.
  When we saw the crumbling temple, alone in the forest, we knew we'd found a new home, despite the arrows.

  It took a long time to make friends with the dead priest.

  We stayed put, camping nearby, and the dead priest came to me in my dreams.
  We spoke for a while -- from a distance.
  In dreams, the dead can fear the living.
  After spending so much time alone, I honestly thing he's gained a fear of the living.
  Like people who hate spiders, but he just can't stand to see anything animate and moving about on its own.

  We don't want to kill people, just get enough money to get by, and we picked up a few friends to realize \gls{sewerking}'s vision.
  But of course, accidents happen, and at those times, we return with the bodies and give them to \gls{necromancer}.
  \Gls{necromancer} eventually gifted me a ring which would make me invisible to the dead, and another for my brother.
  He made us a few more for our comrades.
  We can't wear them for too long, but we can stay a night in the temple without worry, and the dead do not attack our horses.

  And today, things have changed.
  We discovered another team of bandits, who call themselves the Woodspies.
  We found a letter they sent as well, along with a map showing old locations from \gls{lostcity}, and the idiots don't know what they've found.
  So I've passed it onto \gls{sewerking}, and he had some interesting insights.

\end{exampletext}

In the words of \gls{sewerking}:
\label{blightConspiracy}

\begin{exampletext}

  Think about \gls{town}'s roads.
  It has only one, so if nobody can walk that road, nothing moves in or out.

  But let me back up a little.
  I had to return from my studies in \gls{college} to rescue \gls{banditking}, and found him hiding in some seedy bar, as if the \gls{guard} had such a high opinion of him that they wouldn't think to look for him there.
  I organized our escape, and found us lodgings with the little coin we still had.

  It started with \gls{pig}.
  \Gls{pigowner} heard our coins a mile away -- people with money don't normally stay in \gls{pig} for more than they need to get the gossip, and don't take their money in with them.
  We stayed in a little area she had tucked away, underneath, and she told us the tale of how there was once an underground library, and temple, and all sorts down there.

  We started cleaning the place out.
  The entire place had flooded simply due to one blockage which built up over time.
  Nobody could get down that passage far enough to find the problem, but those deathly rings \gls{banditking} received from \gls{necromancer} allowed me and a few other likely lads who stayed with \gls{pigowner} to hold our breath long enough to journey down and dig away at the blockage.
  We went down a few minutes at a time, always with ropes tied around our waists, and soon dug the blockage out.

  We have our own underground palace now.
  It's dirty, and wet, it's dark and dank, but it's ours, and we can store anything we need down there.
  And it has treasures.

  So. Many. Treasures.

  It has an old brazier which infuses any corpse with a malicious spirit.
  It has an alter of skulls which whisper your fortunes and mishaps.
  And it has old scrolls, secreted away, which describe the forbidden magics of the nura.
  
  So I've begun to study that magic, and create magical items.
  And I've asked the skulls about my fortune, and sometimes they tell me I will kill \gls{townmaster}, so now I know I am destined to rule this land.
  And I've begun to steal the bodies to create an army  of ghouls down here, all locked in little wooden cages.
  We have over a dozen.

  And we have used the dead to create tunnels.
  We've dug into the catacombs of \gls{town}'s temple of Eldren and started removing the bodies of the dead, to resurrect as ghouls.
  My men replace them each time with sacks of dirt and rocks.
  They're starting to look rather realistic.
  Once I have enough ghouls, I can release them into the citadel (from below), and take out \gls{townmaster}.

  Tunnels and skulls, forgotten magics in scrolls, and secret passages carved by the dead.
  Everything will come together soon, but we need one more piece before we rule \gls{town}.

  Just one portal to the Nura realm on the right place along the main road from \gls{town} will make a \gls{blight}.
  Perhaps that idiot \gls{traitor} will open one of the portals on his travels across \gls{lostcity}.
  If not, I'll find one on my own.

  Once the road to the outside has closed, \gls{town} will sit alone, with no way to call for outside help, and no way for \gls{king} to interfere.
  So long as \gls{town} remains divided from the rest of the world, we can simply kill \gls{townmaster}, destroy the woodspy bandits, and rule \gls{town} and the surrounding \glspl{village}.
  If the nura prove too much trouble, \gls{necromancer} can help protect us, and strengthen his army.

  We will regain our birthrights in this new land before \season\ has ended.

\end{exampletext}

In the words of \gls{forestpriest}:

\begin{exampletext}

  I walk a lot.
  The last of my friends died of old age ten years ago -- that's the curse of long life inherited from my father.
  But I've started to think of it differently recently -- perhaps I just live many lives, one after another.
  And in this one I walk a lot.

  I quite like being a bird when I'm not walking.
  You can see a lot when you fly.
  It felt dangerous, because sometimes a hawk would try to eat me, but then I learned how to become a hawk.
  I like shouting a lot.
  And recently I've seen a lot of people dropping off military supplies -- food, armour and weapons mostly.
  And when I waited, I saw men in green, picking up the shipments.
  
  I don't usually interfere with this sort of thing -- in fact in the priesthood we swear ourselves to neutrality wherever possible, but that was my last life.
  So I turned the delivery man into a goat.

  He ran away fast, so I think he'll make a good goat.

  Before the deliveryman starting oinking, he started shouting about how I'm an evil elf.
  That's only half true at best, but it got me wondering if he was correct, and if the elves did destroy the city that spanned this area.
  The elves tell me another story -- that the humans opened portals to the nura realms in order to trade with them.
  So I walk around the ruins, and try to see if I can find something that looks like an alchemical portal.
  And I found those green bandits searching the ruins too, and building things.
  Sometimes questions just get stuck in my head like that, so I'm going to keep searching until I can find the answers.

\end{exampletext}

\end{multicols}

\section{Side Quest Summaries}
\label{sqSummaries}

\begin{multicols}{2}

\noindent
This complete list of Side~Quests allows you to see which Side~Quests are available at a glance, in each area.
Once the \glspl{pc} encounter some Side~Quest which combines with another, pick another Side Quest in the area to continue or start.

Many Side Quests start in the town, then change to the \glspl{village}, and back again.
This summary gives you an overview of what happens per-area.

Remember to write a `\sqr' in front of the Side Quest encounters which have become available, and `\ding{55}'~out encounters which you've completed.

\end{multicols}

\foreach \x in {Town, Villages, Forest}{
  \center\subsection*{\x}
  \printcontents[\x]{l}{2}{\setcounter{tocdepth}{3}}
}

