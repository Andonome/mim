\chapter{Protagonists \& Plots}
\label{sideQuestIntro}

\section{A Tapestry of Karma}
\label{sqSummaries}

\noindent
On one level, this \gls{campaign} presents a farce where some people argue about the tax on a river, and monsters start eating them, because they're too busy arguing to notice.
On another level, the \gls{campaign} weaves together nine stories with one theme: the subtle line between karma and vengeance.

\begin{multicols}{2}

%\playCommentarySQ

\subsection{Structure}

`\nameref{spiderSong}' begins with \gls{spiderqueen} killing people with her monstrous pets, then she feeds an entire hamlet to them.
She seems too powerful for anyone to kill her, but once she makes too many enemies, someone volunteers to help the \glspl{pc} kill her.
So the \glspl{pc} may become instruments of karmic vengeance.

While that thread is still in development, `\nameref{wolfHeads}' begins, and the \glspl{pc} meet a series of suspicious characters.
They may argue with one, become friends with another, or just ignore every troublesome character they meet.
Soon after, the \glspl{wolfhead} go out together and find the \glspl{pc} are in trouble; and so the \glspl{wolfhead} decide what counts as karma.
And long after the \glspl{wolfhead} help, hinder, or mock the \glspl{pc}, the situation reverses when the \glspl{pc} find the \glspl{wolfhead} suffering an attack on the road.
Vengeance on either side can lead to escalation.

But before either of those two threads resolve, the \gls{campaign} has another, and another.
Because whenever the \glspl{pc} don't head towards the next part of the story, BIND simply starts a new \gls{thread}, while the old one lies in wait.

\subsubsection{Player Interference}
should modify \glspl{thread}, especially the later \glspl{segment}.
\Glspl{segment} won't ask anyone's permission to \textit{start}, but once started, the \glspl{pc} actions' should have repercussions.
For example, if \pgls{pc} somehow kills \gls{spiderqueen} during the second \gls{segment} of her \gls{thread}, the rest of the \glspl{segment} would disappear instantly.
However, most \glspl{thread} will not fade away so easily, because most of the protagonists come from one of five political factions.
If \gls{banditking} dies, \gls{traitor} (or even his brother \gls{sewerking}) can take his place.

Since the \glspl{pc} begin in the \gls{guard}, that makes them part of the establishment at the \gls{campaign}'s start.
Of course, this may change soon, as \gls{sewerthief} will gently inquire how the \glspl{pc} feel about joining the \glspl{whiteBandits}, or they may find themselves making concrete plans with the \glspl{wolfhead}.

\subsubsection{Wandering Monsters}
should keep wandering throughout the \gls{campaign}, because their presence provides the \glspl{pc} with a reason to go out and hunt, and because their presence is the crux of the matter -- they are the doom which comes if \gls{valley} tears itself apart.

Most \glspl{segment} require insight and negotiation, rather than combat.
This helps the \glspl{thread} run alongside random encounters, without making dangerous combat-encounters repetitive.

The players may take a while to fully understand the grander political schemes threatening \gls{valley}, and even longer to come up with counter-schemes.
Bog-standard \gls{guard} missions can help kick-start journeys, as they often involve entering the forest.
\iftoggle{judgement}{%
  You can find some random tables to generate missions in the \textit{Book of Judgement}, \autopageref{NGmissions}.
}{}

\subsection{Three Circles}

Look at the map \vpageref{Irina/greylands}.
\Gls{valley} has three \glspl{region} -- \gls{town} sits in the centre, the Roads surround it, and the eternal Forest surrounds everything.

Each \gls{region} comes with a list of \glspl{segment} so you can tick cross them off once they're done, and tick the next \gls{segment} in the \gls{thread} to show it's ready (although the next available \gls{segment} may not be in the same \gls{region}).

\subsubsection{The Outer Circle}
is the endless forest which surrounds every civilization.
Whenever people enter this outer darkness, they invoke a little more curiosity from the forest, and encourage it to explore a little more of \gls{valley}.

The map icons with a single tower are \glspl{broch}.
Each morning and evening, the \glspl{guard} of these tall towers play loud pipes to attract the forest's \glspl{monster} towards them, and away from the inner farmlands.

Have a glance `\nameref{threadList:Forest}' \vpageref{threadList:Forest}, with the \gls{thread} list and locations, then fold the page-corner over, so you can find the next \gls{segment} and location descriptions easily.

\subsubsection{The Second Circle}
includes everything from \gls{town} until the towering forest at the \gls{edge}.
The little houses on the inside are scattered hamlets, each controlling large amounts of farmland, full of grazing livestock.
\Gls{town} lives on the farmland here, which produces surplus meat for the outer \glspl{village}, and enough food for the \glspl{temple}, wanderers, and urchins of \gls{town}.

The outer structures (shown with multiple rooves on the map) are \glspl{village}, with tall walls to defend themselves from the constant probes from \gls{sylf}'s children.

Note the thin crust of protective structures surrounding delicate fields.
Over the course of the \glspl{thread}, some of the outer structures will fall to the \glspl{fiend} beyond the \gls{edge}.
When \pgls{fiend} destroys any \gls{broch} or \gls{village}, you should take a pen, and score out the settlement.

Fold the corner of \vpageref{threadList:Roads} so you can check available \glspl{segment} quickly.


%\playCommentaryMarket

\subsubsection{The Centre}
is \gls{town}, where the \gls{temple} leaders organize, barter, and bicker.
Low-ranking \glspl{guard} should not enter towns, but many do anyway, to buy weapons, drink ale, or sell stolen goods.

Fold the corner of \autopageref{townChapter}.


%%%%%%%%%%%%%%%%%%%% Map %%%%%%%%%%%%%%%%%%%%

\renewcommand\csComments{
  \mapCircle{16}{76}{1.7}{Dyson_Logos/bandit_camp}
  \mapCircle{35}{88}{2}{Dyson_Logos/forgotten_city}
  \mapCircle{27}{09}{2}{Dyson_Logos/qualme_temple}
  \mapCircle[4]{56}{52}{2.5}{Dyson_Logos/town}
  \mapCircle{44}{41}{2}{Dyson_Logos/redfall}
  \mapCircle{83}{09}{1.7}{Dyson_Logos/shadow_gate}
  \mapCircle{86}{45}{1.7}{Dyson_Logos/lochside}
  \draw[very thick,white] (11,0.6) -- (12,0.6) node[anchor=north]{\outline{10 Miles}} -- (13,0.6) ;
}

\mapNotes{
  \nameref{banditLair}/16/73,
  \rotatebox{-20}{$\huge\Longleftarrow$}/05/59,
  \rotatebox{-20}{\small\Glsentrytext{whiteplains}}/07/55,
  \nameref{lostcity}/35/86,
  \glssymbol{yonder}/35/94,
  \glssymbol{paik}/56/61,
  \Glsentrytext{town}/56/50,
  \Large\glssymbol{paik}/44/49,
  \glsentrytext{redfall}/44/41,
  \glssymbol{abderian}/83/13,
  \nameref{shadowVault}/83/05,
  \small\nameref{silentGorge}/83/30,
  \rotatebox{10}{\small\gls{southDale}}/95/33,
  \rotatebox{10}{$\huge\Longrightarrow$}/94/29,
  \normalsize\glsentrysymbol{sylf}/70/87,
  \normalsize\nameref{cinderfilch}/70/83,
  \glsentrysymbol{sylf}/86/49,
  \gls{lochside}/86/45,
  \D/27/15,
  \glsfmttext{oldTemple}/25/08,
}

\widePic[b]{Irina/greylands}

\end{multicols}

\setglossarypreamble[people]{
  These are the movers and shakers, the doers and thinkers.
  You can find each of them in the \glspl{thread}, sometimes as primary actors, other times wandering into the background of multiple threads and \glspl{segment}.
}

\printglossary[
  type=people,
  style=topicmcols,
  title={Places \& Politics},
]

\Needspace{20em}

\section{Getting Started}
\label{theOpeningSession}

\begin{multicols}{2}

\subsection{A Quick Tour}

You don't need to read the whole book to start using it -- just get a feel for the first parts the players will see.
I recommend jumping through the first few \glspl{segment} to get a feel for how the plot wanders about the map.

\begin{enumerate}
  \item
  Grab \pgls{pc} sheet, or roll up a random character, then check their background \vpageref{greyBackgrounds}'s `\nameref{greyBackgrounds}'.
  \item
  Start in \nameref{cinderfilch} \vpageref{cinderfilch}, and check the first \gls{segment} available in the \gls{region} (\vpageref{threadList:Roads}).
  \item
  Read and resolve each \gls{segment} quickly (reading box-text aloud helps the memory).
  Use no more than one roll, then decide on a resolution and move onto the next \gls{interval}, and mark the next \gls{segment} as ready (\gls{sgr}).
  \item
  Ignore combat -- just remove 2~\glspl{hp} from the \gls{pc} you're touring with, and move onto the next \gls{segment}.
  \item
  Give the character all the \glspl{gp} they might need for \glspl{ration} or new \glspl{weapon}, then check the market handouts before the Roads \gls{region}.
\end{enumerate}

By the time you've taken the character into the forest on a mission, then into \gls{town} for new weapons, and back to \pgls{broch}, you'll have a solid idea of how \gls{valley} works.

\subsection{Character Introductions}
\label{greyBackgrounds}

\subsubsection{In Established \Glsfmtplural{campaign}}
you might open with the false promise of an idyllic land, so free of crime that the \gls{templeOfBeasts} has no criminals to use as \gls{guard} \glspl{fodder}, and so rich that nobody signs up as \pgls{guard} \gls{soldier}.

\begin{speechtext}
  It's your lucky day!
  The \gls{warden}'s received a letter from \gls{town}, in \gls{valley}, saying they don't have enough \glspl{guard}.
  \Gls{townmaster} has paid a handsome sum to have a few more sent over to guard \gls{valley}.

  It's a lovely place; crime-free, and open fields as you can see from a mountain-top.
  Yous can take the first boat upriver, and ask for orders when you get to \gls{lochside}.

  Have fun, and try not to get bored.
\end{speechtext}

\subsubsection{New \Glsfmtplural{campaign}}
should begin as usual; give an overview of \gls{fenestra},%
\exRef{judgement}{Judgement}{openingLine}
then have each player craft \pgls{pc}.%
\exRef{stories}{Stories}{randomCharacterCreation}
Once they make a basic character, we add one more roll to check where they come from.

\paragraph{Homeland}

\begin{dlist}
  \item
  \begin{description}
    \item[Humans and Gnomes]
  come from \gls{town}, born and raised.
  If they entered the \gls{templeOfBeasts} as punishment for a crime, then \gls{townmaster} must have given the sentence in \gls{town}'s \gls{court}.
    \item[Others]
    come from a land far to the South of \gls{sixshadow}, but close enough to see those mountains on a clear day.
    They lost some family members last year to \gls{spiderqueen} and her early experiments with \glspl{crawler}.
  \end{description}
  \item
  \begin{description}
    \item[Dwarves, Gnolls, and Humans]
    come from \gls{southDale}, but the \gls{warden} at their trial ordered them to serve in \gls{valley}.
    They have heard of the \glspl{wolfhead}, and heard that some have engaged in violent crimes, but were never convicted.
    \item[Others]
    come from a distant island, across the sea.
    Their family escaped to avoid a war between \pgls{lich} and \pgls{dryad} which consumed the civilized areas.
    They took various jobs to support their travelling family before ending in the \gls{guard}.
  \end{description}
  \item
  \begin{description}
    \item[Humans]
    come from \gls{redfall}, but have no idea about the current crisis.
    They entered the \gls{guard} when \gls{hungrywarden} sentenced them.
    \item[Others]
    entered \gls{valley} with the `Champions of Levity', a travelling circus.
    The circus then abandoned them in \gls{southDale}, and conscription followed soon after.
  \end{description}
  \item
  \begin{description}
    \item[Humans]
    come from one of the outer \glspl{village} in \gls{valley}, and have grown up with old-fashioned songs about \gls{lostcity}.
    \item[Others]
    once traded irons goods along the river which goes from \gls{whiteplains}, through \gls{valley}, and down to \gls{southDale}.
    \Gls{townmaster}'s heavy taxes lead them to destitution, and criminal activities.
  \end{description}
  \item
  \begin{description}
    \item[Elves, Gnolls and Humans]
    come from \gls{whiteplains}, with bitter memories of the \glspl{keeper} there pushing far too many humans into the \gls{guard} without good reason.
    \item[Dwarves and Gnomes]
    once lived in the \gls{deep}, and still don't speak the \gls{tradeTongue} very well, though they lived in \gls{southDale} for a year before conscription in the \gls{guard}.
  \end{description}
  \item
  \begin{description}
    \item[Elves, Gnolls and Gnomes]
    come from a distant land, North of \gls{southDale}, where a dragon has declared himself king, eats the local basilisks.
    The player should describe that settlement in any way they please, although nobody is required to believe it.
    \item[Dwarves and Humans]
    grey tired of the cold weather in the far North, where elasmotheria storm across snowy plains.
    They heard wondrous tales of \gls{lostcity}, and decided to investigate, taking odd jobs along the way, until landing in the \gls{templeOfBeasts}.
  \end{description}
\end{dlist}

\Glspl{pc} with a shared result should have a shared past.
The origins should also inform how players spend \glspl{storypoint},%
\exRef{stories}{Stories}{stories}
since newcomers to \gls{valley} are less likely to have local allies, but may still have allies who come from the same land.

\begin{multicols}{2}

\paragraph{\Gls{southDale} Names}

\begin{dlist}
  \item
  Fathom
  \item
  Edgefore
  \item
  Takbrine
  \item
  Groundswell
  \item
  Idlelub
  \item
  Logbane
\end{dlist}

\paragraph{\Gls{whiteplains} Names}

\begin{dlist}
  \item
  Herof
  \item
  Mortsafe
  \item
  Knox
  \item
  Rotfilch
  \item
  Magwhite
  \item
  Caskfind
\end{dlist}

\end{multicols}

\vspace{-4em}
\subsection{The Opening Scene}

Find \nameref{cinderfilch} \vpageref{cinderfilch} and give the players the \gls{broch}'s handout image, then point to their lodgings in room~\ref{cinderBeds}.

\begin{boxtext}
  Dawn has arrived, giving you the first sight of the surrounding area in daylight.
  The outside air feels \showTemperature.
  The forest dominates your vision, despite the \gls{broch}'s height.

  Two roads lead out -- the South road, which you came from last night, and the West road, which lies empty.
  Below, a droning noise indicates that the piper is calling \gls{sylf}'s children to the \gls{broch}.
\end{boxtext}

The sound of the pipes has a 2-in-6 chance of attracting a wandering \gls{monster} from past the \gls{edge}.
The \glspl{soldier} in room~\ref{cinderBows} will kill whatever emerges, but you can still make the encounter roll and note what the \glspl{pc} see (if anything).

With the game started, just follow the standard cycle from the \textit{Book of Judgement}:

\begin{enumerate}
  \item
  The characters receive a mission, perhaps from \Gls{ranger}~\composeHumanName\ (room~\ref{cinderUpper}) or \Gls{jotter}~\composeHumanName\ (room~\ref{cinderOffice}).
  Find the missions in the \textit{Book of Judgement}\iftoggle{judgement}{ (\autopageref{NGmissions})}{}.
  \item
  Roll to find the weather and next encounter \iftoggle{judgement}{(\textit{Book of Judgement}, pages \pageref{weather} and \pageref{randomEncounters})}{on the \gls{gm}-shield}.
  \item
  Check the next available \gls{thread} \gls{segment} (\vpageref{threadList:Roads}).
\end{enumerate}

The first available \gls{segment} on the roads is
\ifnum\value{cycle}>2%
  `\nameref{spiderSong}', where a trader arrives from the South after a harrowing encounter, delighted to be alive, and \emph{at the same time} one of the \glspl{soldier} notes a wanted poster on the \gls{broch}'s wall which kicks off `\nameref{wolfHeads}', and \emph{at the same time} \gls{traitor} arrives on horseback from the West to tell the \glspl{pc} to immediately abandon their mission and help cure his disease.
\else
  `\nameref{wolfHeads}', as the \glspl{pc} spot a wanted poster; then only a mile up the road, bandits attack in the first \gls{segment} of the `\nameref{desperatemeasures}' \gls{thread}.
\fi

That should be enough to convey the feeling of `too much to do, never enough time'.
And it marks a good moment to remind the players that the \glspl{pc} must return to civilization before the session's end.

\end{multicols}
