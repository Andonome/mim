\sidequest[Roads,Forest]{Entitlement}

\noindent
The \gls{whiteplains} \glspl{warden} cannot survive the tax hikes imposed on them when taking their boats through \gls{town}, so they have sent two of their most devious agitators to stir up trouble.
They begin by making close alliances, and giving lofty promises to the local \gls{guard}, then bandits.

Throughout this story, they try to slowly replace or destroy existing power structures, and to make an alliance with the \glspl{pc}.

\sqpart{Roads}% AREA
{Rotten Breath}% NAME
{\Glsentrytext{traitor} sends the party to find a cure for his breath}% SUMMARY
\label{rottenBreath}

\begin{exampletext}
  When \pgls{guard} Rider reported finding a ruined temple%
  \footnote{See \vpageref{necromancers_lair}.}
  to \gls{traitor}, \gls{traitor} went to investigate with \gls{sewerking}.
  At first, it appeared abandoned, so \gls{sewerking} picked up some interesting items he found in the hall.

  A moment later, a swarm of stirges swarmed around \gls{traitor}, and their bite carried an infection.
  \Gls{traitor} now has \textit{Breath Rot}.
  The two fled, carrying statues and a magical \gls{artefact}%
  \footnote{See \vpageref{The Bladder Skull}.}
  but \gls{traitor} still has foul breath from a disease which can slowly suffocate people to death.
\end{exampletext}

\noindent
The \gls{jotter} awakens the troupe rudely, and gives them a foul lecture, with stinking breath.
He needs them to march, as fast as they can, and go to a spot past the \gls{edge} where Screeching Moss%
\exRef{judgement}{Judgement}{screeching_moss}
grows.
Screech Moss is the only thing \gls{traitor} knows about which can cure his disease.

How far the \glspl{pc} must march depends on where they start, and how well they can navigate,%
\exRef{core}{Core Rules}{navigate_land}
but the moss should be somewhere close.

\paragraph{If the troupe fetch the moss within a couple of days,}
\gls{traitor} rewards them with a longsword each, then asks them to go immediately to \gls{town} to request \gls{forestpriest} create a cure for him, and gives them a single note stating they have official business.%
\footnote{The \gls{guard} should not usually enter towns without official business.}

\paragraph{If the troupe enter \gls{town},}
they will find \gls{forestpriest} in the \gls{doulaShop}, not far from \gls{pig}.

\paragraph{If the troupe fail to make it back within a couple of days,}
\gls{traitor} will be furious with them, but will find another group of \glspl{guard} to bring him a cure.

\begin{boxtext}
  Creeping closer to the light, you spot glimpses of the tip of a fire, then a second not far ahead, but no voice, or any movement.
  You find it impossible to move without making constant crunching and grating sounds.
\end{boxtext}

\sqpart{Forest}% AREA
{Guild Consciousness}% NAME
{\gls{sewerthief} talks politics}% SUMMARY

\Gls{sewerthief} wants a promotion, so he has taken on the dangerous mission of tracking down a griffin eyrie.
To survive the night, he lights two fires, fairly far apart, then crawls under a little bush to hide.
His hope is that any predators in the area will wander up to the fire, and not him.

\sewerthief

\Gls{sewerthief} speaks to the troupe about \gls{lostcity}, then soon after -- politics.

\begin{speechtext}
  ``They say all of \gls{valley} used to be one big city, and you could just walk anywhere.
  Just think about how wonderful it must be if all of this were a big city, with gardens, just like \glspl{warden} have, and we weren't sitting in the dark, away from the fires.

  You know they say we can become \glspl{warden} by building a new \gls{village}, but I've never really seen it.
  We don't become \glspl{warden} really, and none of the \glspl{warden} I've seen ever started in the \gls{guard}.''
\end{speechtext}

\boxPair[t]{
  \humanarcher[\npc{\T[6]\Hu}{\arabic{noAppearing} \glspl{whiteBandits} Archers}]
}{
  \humanthief[\npc{\T[3]\Hu}{\arabic{noAppearing} \glspl{whiteBandits}}]
}

\sqpart{Forest}% AREA
{New Kids}% NAME
{The \glsentrytext{whiteBandits} have an argument with the \glsentrytext{guard}}% SUMMARY

\ifnum\value{temperature}=0
  \newcommand\localMonster{griffins}
\else
  \newcommand\localMonster{a basilisk}
\fi

\begin{exampletext}
  \Gls{banditking} tours around \gls{valley} with a dozen men, trying to ingratiate himself to as many \glspl{warden} as he can.
  He presents the entourage as his \glspl{sunGuard}, but in reality, each of them once worked as bandits, or as the gutter-scum of \gls{town}.

  When a traveller informed him that \localMonster\ had attacked \pgls{village} for the last few nights, he marched hard with his entourage to help.
  Within the day, they arrived, and stayed the night.
  Once it attacked, they loosed arrows at it, then followed the trail of blood towards \gls{town}.

  Shortly after, the local \gls{guard} arrived, and began their own volley of arrows.
\end{exampletext}

The \glspl{pc} arrive to find \localMonster, full arrows, at least one of which came from the \gls{guard}.
Both groups understand that the flesh of \localMonster\ has value to alchemists, and plenty of meat, which could fetch a good sum of money.

Sorting the disagreement won't be easy.
The exact roll depends on what the \glspl{pc} attempt.
They might try to threaten one group with \roll{Strength}{Tactics}, or calm the situation with \roll{Charisma}{Empathy}.

\paragraph{If the encounter occurs far from \gls{town},}
the meat will soon spoil anyway, so resolving the disagreement requires a roll at \tn[9].

If this part occurs closer to \gls{town} (or even a large \gls{village}), the rolls will be at \tn[12].

\paragraph{If the \glspl{pc} cannot resolve the situation within two rolls,}
the two groups begin to fight.

Remember to roll only for skirmishes the \glspl{pc} have engaged in, and use the simplified combat rules for \glspl{npc} at the end of each round.%
\exRef{judgement}{Judgement}{npcfights}

\banditking

\paragraph{However this resolves,}
\gls{banditking} will focus on his own safety above all else, even if it means running away, with any surviving men.

If the \glspl{pc} capture him, \gls{townmaster} will personally pass judgement on him in \gls{town}'s \gls{court}.
In this case, give \gls{banditking} a \roll{Charisma}{Empathy} roll (\tn[8]).
Failure means death, but the \glspl{pc} should hear the result no matter which way it goes.

\paragraph{Highly observant \glspl{pc}}
may notice that \gls{banditking} carries the stench of marching mushrooms.%
\exRef{judgement}{Judgement}{marching_mushroom}

\begin{boxtext}
  Bestial roars ahead force your neck around, and a mere 20 \glspl{step} away, you spot \gls{banditking}, with half a dozen archers, taking aim at something just beyond them.
\end{boxtext}

\sqpart{Forest}% AREA
{The Beast's Treasure}% NAME
{\Gls{banditking} displays alchemical knowledge}% SUMMARY

In the distance, the \glspl{pc} hear the cries of \localMonster\ once again.
This time, \gls{banditking} has fewer soldiers in his entourage, but apparently enough to hunt \localMonster.

\ifnum\value{temperature}=0
  \griffin
\else
  \basilisk
\fi

\Gls{banditking}'s entourage loose arrows at their prey, while \gls{banditking} himself uses powerful Earth spells to trap every foot which hits the ground.

The fight lasts only a couple of rounds, even if the \glspl{pc} don't help.
\Gls{banditking} remains jovial and understanding with them, even if they parted on bad terms.

\paragraph{If the \glspl{pc} parted on bad terms,}
\gls{banditking} will ask them not to approach to close.
He knows his archers can't do much at close-range.

\paragraph{Once \gls{banditking} feels safe,}
he kneels down with a knife and starts cutting into flesh.
He knows how to use the flesh of \localMonster\ to create alchemical \glspl{boon}.%
\exRef{judgement}{Judgement}{manaIndex}

\sqpart{Roads}% AREA
{\gls{vlg} Dead Nobles}% NAME
{\Gls{banditking} begins his assault on the nobility}% SUMMARY

We start with \glspl{warden}' bodies floating down a river, so if the troupe will not pass any river which lies downstream from a \gls{village}, skip this scene, and return to it later.

\begin{exampletext}
  \Gls{banditking} collected a trove of of diseased marching mushrooms, then sneaked the vile powder into a trader's wine stash, knowing the family and three guards would drink it.
  With the \gls{warden} house's defences down, \gls{banditking} and his most trusted soldiers entered and slew the lot in the middle of the night.
\end{exampletext}

Water has bloated the corpses, leaving them barely identifiable (\roll{Intelligence}{Medicine}, \tn[11], on a tie the character recognizes just one).
Whether they come from \pgls{village} or (more likely) an inner hamlet, the people who served them never liked them, and will not feel ready to cooperate with the \glspl{pc}.

\paragraph{Searching for bandit tracks}
may lead the \glspl{pc} to the bandits' lair%
\footnote{See \autopageref{banditLair}.}
with a \roll{Wits}{Wyldcrafting} roll (\tn[12]).%
\exRef{core}{Core Rules}{tracking}

\sqpart{Roads}% AREA
{Political Positions}% NAME
{\Gls{sewerthief} gauges how the troupe feel about dead \glspl{warden}}% SUMMARY

\Gls{sewerthief} approaches the crew shortly (or immediately) after they saw the \gls{warden} family dead.
He offers no firm opinion, but only gauges how they feel about the situation, then reports their attitudes to \gls{banditking}.

\paragraph{If the \glspl{pc} somehow notice his connection with \gls{banditking},}
he does his best to `play the numpty' -- a skill he excels at.
Instead of denying it, he accepts it, then talks about all the rich and famous people who speak with him, with a series of obvious lies.

\paragraph{If the \glspl{pc} seem sympathetic to the notion of murdering unpopular \glspl{warden},}
\gls{sewerthief} suggests coming for a drink in \gls{town} to discuss the matter further.
He ends up at \gls{whitehorse}, where he introduces them to \gls{sewerking}.
