\sidequest[town,Roads,Forest]{Wolf Heads}
\label{wolfHeads}

\noindent
The \glspl{wolfhead} want to find the \gls{shadowGate}.
Unfortunately, they lack manpower and finances.
\Gls{southCook} has plans for gaining wealth (covered in \nameref{troubleAle}, \autopageref{troubleAle}).

Over the course of this \gls{sq}, the troupe will meet the various members of the \gls{wolfhead}, and form allies or enemies.

\sqpart{Roads}% AREA
{\squash Wanted Poster}% NAME
{The troupe see a wanted poster for \glsfmttext{southRogue}}% SUMMARY

A poster proclaims a 10~\glspl{gp} reward for returning \gls{southRogue} to the \gls{guard} of \gls{southDale}.

\begin{boxtext}
  A new poster in the \gls{broch} proclaims a 10~\glspl{gp} reward for a young woman.
  The picture seems useless, but then it describes her as `exhuberant', `wicked', `short-haired', and `expert archer'.
  The instructions say to hand her over at \gls{lochside}, and not to let her near oils of any kind.

  \Pgls{guard} is staring into the poster, and starts explaining that she must be a wayward soul, and that he `\textit{could fix her\ldots}', then looks sad as he continues `\textit{\ldots but the \gls{guard} never end in a happy marriage\ldots}'.
\end{boxtext}

Combine this \gls{sq} with the next, or run the next immediately.

\paragraph{If the \glspl{pc} hunt for \gls{southRogue},}
they will find her eventually (just keep running through the \glspl{sq}).
Otherwise, just leave this as a feint half-memory for the players.

\paragraph{If the troupe ask about finding her,}
everyone recommends \gls{town} or \gls{lochside}.

\sqpart{town}% AREA
{\squash Rumours of Glory}% NAME
{The troupe hear about the \glsfmtplural{wolfhead}}% SUMMARY

The locals gossip about local heroes, known as the \glspl{wolfhead}, and how they protected \pgls{village} from \pgls{basilisk}.
Most people say they're heroic, but some mention how the \gls{templeOfBeasts} won't be happy they killed \pgls{monster}, and gutted it, without paying any dues.

You might add this as a comment by some trader, recently arrived in town, then interrupt the story with the next \gls{sq} \gls{segment}; or perhaps the next \gls{segment} involves a conversation with someone who could add this rumour to whatever they already wanted to say.

\sqpart{Forest}% AREA
{A Lonely Tower}% NAME
{The troupe discover \glsentrytext{southSeeker}'s secret tower in the forest, where he searches for locations in \glsentrytext{lostcity}}% SUMMARY

\label{green_tower_sq}

\Gls{greentower} does not have an exact location on the map.
Wherever the \glspl{pc} go \gls{rambling} in the forest, \textit{that's} where it is, and where it has always been.

``Where do you go next?'', you ask innocently.
And no matter where the \glspl{pc} decide to travel through the forest, they stumble into somewhere they've never been before, and spot \gls{greentower}.

\begin{exampletext}
  \Gls{southSeeker} found the tower's remains while searching for one of the magical gateways in \gls{lostcity}.
  Currently, he thinks a magical gateway lies in tunnels underneath \gls{greentower}, or that a gateway lies close to this tower.
  He has commissioned men from a nearby \gls{village} to work as masons, and start rebuilding \gls{greentower}, so he can use it as a base of operations.%
  \footnote{You'll have to play this one by ear.
  If \gls{southSeeker} has already found the magical gate in `\nameref{old_alchemy_basement}' (\autopageref{old_alchemy_basement}), then he must be looking for the other gate, behind the mountains.
  Or if he's found the latter already, he must be searching for the former.
  And if he's found both, then he's checking in on his half-way house\ldots unless \gls{greentower} isn't half way between anything, in which case\ldots you'll figure something out.}
\end{exampletext}

\paragraph{If a gateway really does lie nearby,}
then \gls{southSeeker} has deduced the proper location, and will use this as a half-way house and base of operations when he finally discovers the gateways.

\paragraph{If the gateways lie nowhere remotely close to the tower,}
\gls{southSeeker} will soon realize this, and leave the tower abandoned; but he will never forget its location and potential uses.

Check \autopageref{green_tower} for details on \gls{greentower}.

\sqpart{Roads}% AREA
{New Friends}% NAME
{\Glsentrytext{southRogue} follows the troupe to find out what the local \glsfmttext{guard} know about the \glsfmttext{lostcity}}% SUMMARY

\Gls{southCook} suspects the local \gls{guard} could know something about \gls{lostcity}, so she sends \gls{southRogue} out to find out what the score is.
She has bought \glsentrytext{southRogue} some black, hardened leather armour, so she looks the part, then explained the excuse she should use.
\Gls{southRogue} joined the first trader caravan leaving \gls{town}.

\paragraph{Once she meets the troupe,}
she insists on joining them.

\paragraph{If the troupe recognize her from the poster,}
she interrupts before they can complete a sentence on the matter.

\begin{exampletext}
  Listen-but-no!
  
  They \textbf {told me}, that I had to join a trader's caravan.
  I was in Bogpeak (that's in \gls{southDale}), but they went West (\gls{southDale} lies East of here), and I'm not meant to go with them into \gls{town} because I'm \pgls{gDigger}.
  
  And listen, LISTEN\ldots

  I can't register with \pgls{jotter} because he won't have my name, because I'm not in the \gls{guard}.

  I mean, not here I'm not.
  But I'm here now, and I need to get some silver to get back to Bogpeak (which is in \gls{southDale}).
\end{exampletext}

\paragraph{If the troupe take her to \pgls{jotter},}
the \gls{jotter} will begin writing a letter, and ask the troupe to keep a \textit{very} close eye on her in the meantime.

\southRogue % The resisted \tn pulls from this block
\label{southRogue}

\paragraph{If the troupe take her to the \gls{court} in \gls{town},}
then \gls{southCook} intervenes, explaining that \glsentrytext{southRogue} must return to her father, the \gls{warden} of Bogpeak, and that she suffers from `Hysterophantasia', and joined the \gls{guard} due to going unmedicated.

However it goes, the troupe must travel with her for a few weeks, at least.
She attempts to steal anything that looks like `information', but does not read well, so any piece of paper that `looks like something' may go missing.
Have the \glspl{pc} make \gls{resistedaction} of \roll{Wits}{Vigilance} against her \roll{Dexterity}{Larceny}, \tn.

\paragraph{If danger emerges,}
she keeps her distance, and uses her bow.

\paragraph{After a couple of weeks,}
she plans her escape, however she can.
She will act on the spur of the moment, at just the right time, and it will probably involve fire.

\sqpart{Roads}% AREA
{\glsentrysymbol{night}~Just Out for a Walk}% NAME
{\Glsfmttext{southSmith} wanders the forest, secretly looking for key locations in \glsfmttext{lostcity}}% SUMMARY

\begin{exampletext}
  \Gls{southSmith} has hired eight people from the nearest \gls{village} to join him (at 150~\glspl{cp} per day), to hunt for \gls{lostcity}.
  Night has fallen, and despite the dangers, the farmers have decided to light a fire, talk and cook.
  None of them know how close they've come to the road.
\end{exampletext}

\begin{boxtext}
  In the distant, darkness of the forest, just past the \gls{edge}, a fire burns.
\end{boxtext}

\humanarcher[\npc{\T[3]\Hu}{8 Archers}]

Of course, the farmers are aware of the dangers, and are keeping \pgls{vigil}.
Sneaking up requires a \roll{Dexterity}{Stealth} roll against their \roll{Wits}{Vigilance} (\tn, with +2 if snowing).

\paragraph{If the successfully sneak up on the group,}
they overhear the farmers demanding to know what they're doing out here in the forest.

\begin{speechtext}
  Like I said, I'm hunting for a mushroom called `\pgls{dryadsKiss}'.
  But when we find it, you need to be careful not to even look at it, because even a single glance at the mushroom will make you incredibly gullible for the rest of the day.
\end{speechtext}

\Gls{southSmith} isn't lying so that people will believe him.
He's lying so he doesn't have to tell anyone what he's doing, and so that the farmers have something to say when people ask them where they've been.

\southSmith[\npcQuote{You don't need to trust us to speak politely.  Can we start again?}\label{southSmith}]

\paragraph{If the \glspl{pc} engage,}

\Gls{southSmith} tries the following:

\begin{enumerate}
  \item
  Yelling a polite request to stay away (`\textit{Please stay back, no offence is meant, but we do not know you, and do not wish to approach people wandering deep in the forest}').
  \item
  Asking the archers to fire on the troupe.
  \item
  Negotiating, politely.
\end{enumerate}

\paragraph{If the \glspl{pc} ask \gls{southSmith} about his business in the forest,}
he continues to lie about his real intentions, then returns the question to them.

\begin{speechtext}
  I'm searching for some \textit{\gls{dryadsKiss}}\ldots

  It's a mushroom which makes people believe whatever you say.
  Someone doesn't believe the things I say (would you believe that?), so I need the mushrooms to make sure they understand I'm telling the truth.
\end{speechtext}

Once spotted, he returns with the farmers to their \gls{village}, pays them, and abandoned the mission, at least for now.

\sqpart{Forest}% AREA
{\squash~The Guilded Party}% NAME
{The troupe encounter the \glsfmtplural{wolfhead} while dealing with some other situation}% SUMMARY

\begin{exampletext}
  The \glspl{wolfhead} went out together, to secretly scout out a potential location of one of the magical gateways in \glspl{lostcity}.%
  \footnote{They now believe there were multiple towns, not just a single, large city.}
\end{exampletext}

The \glspl{wolfhead} come across the troupe at the same time as another encounter.
How they react depends entirely on their previous dealings.

\sqpart{Roads}% AREA
{A Cry for Help}% NAME
{The troupe find the \glsfmtplural{wolfhead} under attack}% SUMMARY

The \gls{wolfhead} have located \gls{shadowVault}, so they journeyed out to find it, each of them with six days' worth of \glspl{ration}.
But before they reached the \gls{edge} six griffins surrounded them.

\Gls{southSmith}'s arm bleeds from a nasty wound, and \gls{southCook} isn't doing much better.

\setcounter{wounds}{3}
\southCook

After this \gls{segment}, the \glspl{wolfhead} return to \gls{town} to rest and heal, at \gls{town}'s \gls{templeOfSickness}.

\griffin[\npc{\T[6]\A}{\arabic{noAppearing} griffins}]

\sqpart{town}% AREA
{The Deal}% NAME
{The \glsfmtplural{wolfhead} make their move, befriending or attacking the troupe}% SUMMARY

The \gls{wolfhead} have found both magical gateways, and need to secure their assets.
What happens here depends on so much\ldots

Perhaps the \gls{wolfhead} feel indifferent to the \glspl{pc}.
Perhaps they want to kill the \glspl{pc}.
However, if they feel they can trust the \glspl{pc}, then they will propose an alliance.

As luck would have it, \gls{southRogue} meets up with them first.

\paragraph{If all goes well,}
the \glspl{pc} and \glspl{wolfhead} can make a plan.
\Gls{southCook} will agree to finance the plan (assuming the troupe did not conclude \nameref{troubleAle} and leave her penniless), \glsentrytext{southSeeker} will agree to find any information required (or at least find out how to find it), and \gls{southSmith} will agree to complete any task he can.
\Gls{southRogue} will agree to complete any task, then not do it.

The rest of the tale depends on your table.

\paragraph{If all does not go well,}
and the \glspl{wolfhead} dislike the \glspl{pc}, they will try to keep an eye on them.
If the \gls{wolfhead} find out that the \glspl{pc} know where \gls{shadowVault} lies, they will walk on ahead, and make a long stay at the last \gls{village} before the mountains in front.

As anyone could guess, \gls{shadowVault} has plenty of dangers, so the \gls{wolfhead} simply wait for the \glspl{pc} to go ahead, then follow on a day later.
The \glspl{pc} won't even be able to roll, since the \gls{wolfhead} will follow so far behind that neither group sees the other.

The \gls{wolfhead} can then ambush the \glspl{pc} once they emerge from \gls{shadowVault}.

