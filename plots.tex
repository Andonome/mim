\chapter{Protagonists \& Plots}
  \epigraph{He attacked everything in life with a mix of extraordinary genius and na\"ive incompetence, and it was often difficult to tell which was which.}{Douglas Adams}
\label{sideQuestIntro}

\section{Twisting Threads}
\label{sqSummaries}

\begin{multicols}{2}

\noindent
BIND has its own way of structuring events and plots.

\Glspl{segment} are small scenes from a story.
Each one sits in the Forests, Roads, or \gls{town}, waiting for the \glspl{pc} to arrive, and they must be comfortable waiting.
\Pgls{segment} might have the \glspl{pc} meet a \gls{seeker} of the \gls{templeOfLight} in the forest, who's up to no good, but it will never start with the \glspl{pc} in \gls{whitehorse} -- that would be presumptuous.

The next \gls{segment} in \pgls{sq} will never demand the \glspl{pc} come to them, their domains have enough room that the \glspl{pc} will always come eventually.
Perhaps the next time the troupe visit town, they will spot that \gls{seeker} again, speaking with a known crook; but for now, the troupe feel and sore after a fight with a giant arachnid, so they stay in the forest a little longer.
This means it's time to look down the list of Forest \glspl{segment}, and start another story.

The little \glspl{segment} slowly form \pgls{sq}, but at the start, the players probably won't see any form of `narrative' -- just random events unfolding randomly.
However, the more \glspl{segment} occur, the more \glspl{sq} activate, which increases the chances of the \glspl{pc} contacting another \gls{segment} from \pgls{sq} they've already begun.
These tales build slowly, to something larger.

Some of these \glspl{sq} have locations attached -- a castle, cave, or a simple street might hold importance to \pgls{npc} in \pgls{sq}, but the locations must sit even more politely than the \glspl{segment} -- some of them may go unexplored by the \glspl{pc}, and unknown by the players; and that's fine.

\subsubsection{Two Perspectives}

From the point-of-view of the \pgls{sq}, the following events each wait their turn.
The first has a `\sqr' symbol, meaning it is ready.
Once part 1 ends, you can score it out, and mark part 2 with a `\sqr'.

From the point of view of \pgls{sq}, it goes like this:

\begin{list}{\scshape Part \Roman{enc}}{\usecounter{enc}}
  \item
  occurs in the Forest. (\sqr)
  \item
  occurs on the Roads. (\sqn)
  \item
  squashes down with the next \gls{sq} in the Forest, so both \glspl{segment} happen in the same scene. (\sqn~\gls{squash})
  \item
  occurs once the \glspl{pc} reach town. (\sqn)
\end{list}


Now imagine how things might seem from the players' point of view.
Scan down the list \vpageref{sq:Roads} and look at the \glspl{segment} on the Roads, then imagine what the players would encounter if they went from the Roads, to the Forest ($\times 2$), then back on the Roads ($\times 2$), then finally, to Town.
Each time \pgls{segment} should occur, scan down the list of \glspl{segment} in that region, stopping at the first readied \gls{sq}, then marking the next one as ready.

Now imagine a different sequence -- they go from the Roads to Town ($\times 3$) until someone chucks them out, then they pass through the Roads, back into the Forest, then hide on the Roads again ($\times 2$).

Sometimes players will know where the next \gls{segment} in \pgls{sq} will take place, and decide to chase after it.
At other times, these \glspl{segment} rise randomly, like bad weather.

Some \glspl{sq} \glspl{segment} have a `\gls{vlg}' symbol in their summary, indicating that the scene may involve destroying \pgls{village}.
When this happens, mark the \gls{village} off the map.
If the \glspl{pc} find \pgls{village} destroyed, this often presents a dangerous situation.
They will find themselves outside, at night, without the protection of any civilization.

\subsubsection{Getting Started}

Start somewhere on the `Roads'.
This area covers everywhere outside of town, up to the forest, so it constitutes most of the map \vpageref{Irina/greylands}.

Standard \gls{guard} duties%
\exRef{judgement}{Judgement}{NGmissions}
include transporting trader caravans on their ways between \gls{town} and the outer \glspl{village}.
The pleasant areas, closer to \gls{town}, see more traders and travellers than monsters, while the distance \glspl{village} have roads piercing the forest, and must push back massive predators almost daily.

Once the \glspl{jotter} send them this way and that, they will instantly find \glspl{sq} assaulting them daily, or weekly, or at any pace you like, and the players will eventually investigate at least one lead.

Throughout this time, \glspl{pc} should receive all the standard encounter rolls.
Mixing the lot -- encounters and plots, \glspl{sq} and \gls{guard} missions -- creates a maelstrom of twists and turns.
This can be a lot of fun, but it also threatens the players' focus, so it's best to remember to pull that focus back to whatever the players think of a their primary mission.
This may change half way through the mission, or they may disagree about what's important; and both of those results are fine.
However, if the players ever feel \emph{confused} about what their characters should be doing, they will quickly disconnect from \gls{fenestra}, and start staring into their rectangles.

Chaos, and focus should flow like breathing in and out.

\subsubsection{Let Players `Ruin' the Mission}

The \glspl{sq} below can withstand a lot of punishment from the players.
If \gls{segment} I of a \gls{sq} goes haywire, \gls{segment} II can usually take place unscathed, or with minor adjustments.
But you should let the players change the world, otherwise what is the point of this world?

When \glspl{pc}' actions threaten to derail \pgls{sq}, you have a few options.

\begin{description}
  \item[If the \glspl{pc} kill a major character]
  use another.
  It's no accident that two brothers lead the \gls{whiteBandits}.
  If \gls{sewerking} dies, \gls{banditking} can take over most of his duties.
  Similarly, \gls{alemaster} can also take over any of \gls{pigowner}'s \glspl{segment}, as they're both members of the \gls{wheatGuild}.
  \item[If the group decide to join the \gls{whiteBandits}]
  adjust!
  So they don't want to be in the \gls{guard} any more -- that's quite understandable.
  So they will have to listen to \gls{sewerking} instead of \gls{traitor}.
  \Gls{sewerking} still has plenty of missions for them.
  \item[When all else fails]
  just cut out the \gls{sq}.
  If the \glspl{pc} somehow kill \gls{spiderqueen} during Part 1, the troupe only lose 6 scenes.
  The plot has plenty of other scenes!
\end{description}

\renewcommand\csComments{
  \mapCircle{16}{76}{1.7}{Dyson_Logos/bandit_camp}
  \mapCircle{35}{88}{2}{Dyson_Logos/forgotten_city}
  \mapCircle{27}{09}{2}{Dyson_Logos/qualme_temple}
  \mapCircle[4]{56}{52}{2.5}{Dyson_Logos/town}
  \mapCircle{44}{41}{2}{Dyson_Logos/redfall}
  \mapCircle{83}{09}{1.7}{Dyson_Logos/shadow_gate}
  \mapCircle{86}{45}{1.7}{Dyson_Logos/lakeside}
}

\mapNotes{
  \nameref{banditLair}/16/73,
  \nameref{lostcity}/35/86,
  \Glsentrytext{curiosity}/35/94,
  \Glsentrytext{justice}/56/61,
  \Glsentrytext{town}/56/50,
  \Large\gls{justice}/44/49,
  \nameref{redfallVillage}/44/41,
  \Glsentrytext{poison}/83/13,
  \nameref{shadowVault}/83/05,
  \gls{beasts}/86/49,
  \gls{lakeside}/86/45,
  \D/27/15,
  \nameref{necromancers_lair}/25/08,
}


\widePic[t]{Irina/greylands}

\end{multicols}

\label{sqList}
\printAllSideQuests{Roads, Forest, Town}

%%%%%%%%%%%%%%%%%%%%
\noindent
Each day the \glspl{pc} will awaken to new plots unfolding.
Most days, they will hear rumours, on other days they will see the consequences of the plots.

The \glspl{pc} most likely work as members of the \glspl{guard}, but whatever they want to do, they will have their own tasks to attend to, and the little scenes here may seem like little distractions at first -- just a bar-fight here, and a tall story there.
But the plots of our various factions will soon begin to interfere with the lives of everyone in \gls{valley} -- humans, elves, and gnomes alike.

Resolution will demand hunting down the right locations, but the locations will not make any demands on the players or \glspl{pc}.
They will remain free to ignore any `call to adventure', which will leave the various actors below free to resolve their plots.

\setglossarypreamble[people]{
  These are the movers and shakers, the doers and thinkers.
  You can find each of them in the \glspl{sq}, sometimes as primary actors, other times wandering into the background of multiple threads and \glspl{segment}.
}

\printglossary[
  type=people,
  style=topicmcols,
]

\bigLine

\begin{multicols}{2}

\subsection{Cartography}

Take a look at the map \vpageref{Irina/greylands}.
Note how long a journey between two \glspl{village} might take.
Travellers going West to East can go downriver, which decreases their travel times significantly, while other journeys will go around walking speed.

If the smallest of the \glspl{pc} have 8~\glspl{hp}, they can travel 16 miles before gaining \gls{fatigue} Penalties.
Of course, that would push them to their limits, which invites dangerous situations.
Alternatively, they may travel for 8~miles (gaining 4~\glspl{fatigue} when walking on the roads), but then they would have to travel in the darkness of the evenings, which also invites danger\ldots

\subsubsection{Primordial Forests}
Outside, a dense, primordial forest surrounds everything.
Travelling beyond here leads nowhere for a week or more, and navigation could not be more difficult.

\subsubsection{Outer Civilization}
Brave, little \glspl{village} protrude a little into the wilderness, and every few nights some new creature emerges from the forest to crawl over their rooves, and claw at the doors.

Some of the outer buildings only show a single tower.
These are the \glspl{bothy}, where traders rest overnight, and where the new \glspl{guard} often stay.
In fact, new \glspl{guard} may not enter towns or \glspl{village}, so they must remain here until they increase their rank.%
\exRef{judgement}{Judgement}{fodder}

\subsubsection{Inner Hamlets}
A little further inside, quiet hamlets rest peacefully.
Any beasts which emerge from the forests would have wandered towards the noise of the outer \glspl{village} or \glspl{bothy}.

These little areas have a `civilization rating' of 15 to 17, depending on their proximity to town, so you can replace any encounters which roll that number or lower with traders.%
\exRef{judgement}{Judgement}{encounters}

\end{multicols}


