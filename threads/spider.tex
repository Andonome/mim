\thread[Forest,town,Roads]{The Spider's Song}
\label{spiderSong}

\noindent
When elves become old, they get weird.
\Gls{spiderqueen} has left her people, and devoted her life to enchanting animals with song and fostering a kinship with them.
Currently, she has collected three pet \glspl{crawler}, but she's having trouble keeping them, because they require too much food.
She has also begun magically deforming her own body to look progressively more like \pgls{crawler}.
Currently, her lower body has six legs and a thorax, which produces thin webbing.
Her skin has begun to harden, and sprout thick, brown hair.
In short, she has become \pgls{dryad}.

Over the course of these encounters, \gls{spiderqueen} becomes progressively more irritating to the troupe and the local area, until the players finally happen upon her lair.

\ifnum\value{cycle}>2
  \Glspl{crawler} hibernate over \gls{cTwo} (\gls{fenestra}'s `Winter'), so if your game takes place around \gls{cOne} or \gls{cTwo}, just run \gls{segment} \ref{spiderDreams}, and leave the rest until \gls{fenestra} returns to the \gls{ainumar}'s warmth in \gls{cThree}.%
  \footnote{Check the main glossary \vpageref{fenestraGlosPreamble} for ana overview of the \glspl{cycle}.}
\else
  \sqpart{Forest}% AREA
{A Field of Geometric Dreams}% NAME
{The troupe find a dozen hibernating \glsfmtplural{crawler} in the snow}% SUMMARY
\label{spiderDreams}

This \gls{segment} only works during \gls{cTwo}, when \gls{snow} covers \gls{fenestra}, and all the \glspl{crawler} hibernate.
During any other \gls{cycle}, you can ignore this \gls{segment} -- it will only play out if nobody has dealt with \gls{spiderqueen} before \gls{sable} curses the land with frost.

\begin{boxtext}
	Eight shiny berries in a bush above you glimmer in the snow.
	They look black, but full of moisture.
\end{boxtext}

If the \gls{pc} picks the `berries', they find out those berries are in fact the eyes of \pgls{crawler}.
It will not try to wake the others, but every noise anyone makes will wake one more \gls{crawler}.

\paragraph{If the \glspl{pc} stay silent,}
this counts as holding their their breath.%
\exRef{core}{Core Rules}{holdingBreath}

\paragraph{If they try to kill all the \glspl{crawler}}
then ask them about their plans very carefully.
The \glspl{crawler} hibernate while packed in tight next to each other -- they can move out gently, without waking the others, but if any feel hurt, it will insinctively spread its legs suddenly, and wake up two more.

\chitincrawler[\npc{\T[3]\A\ifnum\value{Strength}>4\F\else\M\fi}{\Glsfmtplural{crawler}}\addtocounter{weight}{2}]

\chitincrawler[\npc{\T[3]\A\ifnum\value{Strength}>4\F\else\M\fi}{\Glsfmtplural{crawler}}\addtocounter{weight}{2}]

If they manage to harvest all the \glspl{crawler}'s parts, this could prove very lucrative indeed.
But then again, if they ask anyone for help, they will want a cut of the profits.

\paragraph{If the \glspl{pc} wait for \gls{spiderqueen},}
then they will have to wait until the \gls{storm} heralds \gls{cThree}, but she will come for her pets, and will not be expecting an ambush, so long as they do not look damaged from afar.

\paragraph{The next \gls{segment}}
cannot begin until \gls{cTwo} ends.

\fi

\segment[\squash]{Roads}% AREA
{Luckiest Trader Alive}% NAME
{A seller tells the troupe about her luck escaping \glsfmtplural{crawler}}% SUMMARY
\label{spiderPt2}

The \glspl{pc} meet a trader who seems infectiously happy after escaping a near-death experience.

\begin{exampletext}
  \Gls{spiderqueen} sent her \glspl{crawler} after the trader -- Barkridge -- along with illusionary crawlers.
  The real ones became a little confused, so the plan didn't amount to much.
\end{exampletext}

\begin{boxtext}
  \begin{speechtext}
  I just stopped by a stream, near the \glsentrytext{edge}, to let the horses drink, and I heard a funny song.
  Next thing I know, two-dozen crawlers came out.
  \end{speechtext}
\end{boxtext}

\paragraph{If the troupe ask about the song,}
she knows nothing except how odd it sounded, unless they ask if it was elvish.
At that point, she can confirm.

\paragraph{If the troupe buy anything from her,}
she sells at a discount, because today is a good day.
She has ropes (2~\glspl{sp}), shortswords (4~\glspl{sp}), and three travelling \glspl{ration} (\rations, \rations, and \rations, 5~\glspl{cp} each).

\boxPair{
  \spiderqueen
  \label{spiderqueen}
  \index{Elves!\Glsentrytext{spiderqueen}}
}{
  \showStdSpells[
    \input{config/spells/Life3.tex}
    \input{config/spells/Light2.tex}
    \input{config/spells/Mind2.tex}
  ]
}

\segment{Roads}% AREA
{Sheep Stampede}% NAME
{\Glsentrytext{spiderqueen} summons sheep to be eaten by her \glsfmtplural{crawler}}% SUMMARY

\begin{exampletext}
  \Gls{spiderqueen} has gathered more \glspl{crawler}, and she needs to feed them again, so they have laid out their webs, and await their dinner.
\end{exampletext}

If the troupe are near town, they spot a shepherd on his own.
If they are near the \gls{edge}, then the shepherd is travelling with a two trading wagons and four \gls{guard} \glspl{fodder}. 
The \glspl{guard} take a morale check to see if they enter combat.%
\exRef{judgement}{Judgement}{morale}

\humansoldier[\npc{\T[4]\Hu}{\Glspl{guard}}]

\begin{boxtext}
  A nearby shepherd suddenly shouts out ``Hey!'', as he loses control of his flock.
  A distant song entices the sheep to run toward its discordant melody. It is as if the singer caused the music itself to decay.
\end{boxtext}

\paragraph{If the \glspl{pc} do nothing,}
the \glspl{crawler} feed for thirty minutes, then leave.

\begin{boxtext}
  The sheep go beyond sight, and into the distant trees, then the song stops, and they begin to cry out in a way you've never heard sheep cry before.  Half of them flee straight back out of the forest.
\end{boxtext}

\paragraph{If the characters pursue,}
they encounters webs (\roll{Wits}{Vigilance} (\tn[9]) to spot), then see the \glspl{crawler} feasting on the sheep.
They will disengage and attack the characters if they take any Damage.

\paragraph{If the players mention specifically to look out for webs,}
their characters should spot them immediately.

As before, \gls{spiderqueen} waits in the distance, and flees at the first sign of trouble.

\chitincrawler[\npc{\T[3]\A\ifnum\value{Strength}>4\F\else\M\fi}{\Glsfmtplural{crawler}}]

\chitincrawler[\npc{\T[3]\A\ifnum\value{Strength}>4\F\else\M\fi}{\Glsfmtplural{crawler}}]

\segment{Forest}% AREA
{The Arachnid Double Cross}% NAME
{\Glsentrytext{spiderqueen} double-bluffs the troupe, attacking with illusory \glsfmtplural{crawler}, mixed in with real ones}% SUMMARY
\label{spiderqueenssong}

\Gls{spiderqueen} sat with her babies, serenely in the forest, wondering what to feed them.
She cast a \textit{Witness Life} spell, and discovered the \glspl{pc}, then hatched a cunning plan\ldots

\begin{enumerate}
  \item
    She will cast illusions of \glspl{crawler}, who attack beside her real babies.
  \item
  The characters will spot the illusions (they aren't very good) and think they are in no danger.
  \item
  A moment later, she creates an illusion of a gnome, so the characters will \emph{think} that they understand where the illusions came from.
  \item
  As the real \glspl{crawler} grab the characters, they will be taken by surprise.
\end{enumerate}

She casts her \glspl{spell} by singing songs.

\begin{boxtext}
  You can always tell elven music by a sort of off-beat, where the beat goes wrong in a regular way.
  This one is soft and high-pitched, and interrupted by the sound of snapping twigs.
  More crackles come from in front.
  The setting Sun casts a red shimmer over the armoured bodies of a dozen man-sized black-clad, crawling creatures.
  The trees drop a small platoon of arachnids, and in a moment a hundred eyes are calculating how you taste.
\end{boxtext}

The \glspl{pc} roll \roll{Wits}{Vigilance} to understand their environment.

Thirty \glspl{crawler} attack (3 real, 27 fake).

\paragraph{If any \gls{pc} cannot tell the real \glspl{crawler} from the fakes,}
they will have real problems in combat as they will expend all their \glspl{ap} on the fake ones (which attack first).

\paragraph{If they kill \pgls{crawler},}
\gls{spiderqueen} `changes her tune', and sings another song to bring them back to her.

\paragraph{If the \glspl{pc} approach \gls{spiderqueen},}
she casts another illusion to make herself appear as a bush (\tn[10] to spot).

\paragraph{If two \glspl{pc} die,}
the \glspl{crawler} all stop to feed.

\begin{nametable}{What they see\ldots}

  \textbf{\glsentrytext{tn}} & \textbf{Result} \\\hline
  8 & The Sunset red on the chitin is too much, like the creatures don't look right.  You instantly spot that these are illusions. \\
  9 & On a nearby branch a little gnome sits, quietly giggling to himself, then looks shocked as you spot him. \\
  11 & The distant song seems to be coming from a single \gls{crawler} in the distance, \\
  12 & though she looks different from the rest. \\
  13 & Looking past the poor \gls{crawler} illusions in front of you, you notice that the rest are completely and definitively real. \\
  14 & The little gnome, however, is entirely fake. \\

\end{nametable}

\chitincrawler[\npc{\T[3]\A\ifnum\value{Strength}>4\F\else\M\fi}{\Glsfmtplural{crawler}}]

\widePic{Dyson_Logos/ruined_village}

\spell{Spider Hive}% Name
  {Duplicated,Detailed}% Enhancements
  {Warp}% Action
  {Air, Fire}% Spheres
  {\roll{Wits}{Vigilance}}% Resist with
  {The caster rapidly lists every known property of spiders, and \arabic{spellTargets} \gls{crawler} illusions emerge from every shadow}% Description
  {Each \gls{crawler} has Body Attributes at +\arabic{spellPlusOne}, and vanishes once anything touches them.}

\segment[\gls{vlg}]{Roads}% AREA
{Quiet Little Hamlet}% NAME
{\Glsentrytext{spiderqueen}'s \glsfmtplural{crawler} have eaten an entire hamlet}% SUMMARY

\Glspl{crawler} don't usually reach the inner hamlets -- they get distracted along the way by traders or \glspl{village}.
But \gls{spiderqueen} can lead her babies anywhere, including past the dangerous, outer \glspl{village}, and into the quiet little areas inside.
So no matter which road the \glspl{pc} walk, they can encounter \pgls{village} destroyed by \gls{spiderqueen}.

If the \glspl{pc} have arrived at a known \gls{village}, \gls{broch}, or \gls{bothy}, use that one.
Otherwise, use this example.

\begin{boxtext}
  The little hamlet rests quietly.
  The air is cool, but then a single cockerel lets off half a crow in the distance, and goes suddenly silent before he's finished.
  It's only then you really notice: the fields have no animals, and the farmhouse chimneys don't give out any smoke.
\end{boxtext}

However, inside each of the four farmhouses, rooms are filled wall-to-wall with webbing.  Each house contains the same thing:

\begin{enumerate}
  \item
  Dead farmers in webs.
  \item
  Half-dead farmers in webs, waiting to be eaten.
  \item
  Great sacks of \gls{crawler} eggs, ready to burst out and feed.
  \item
    One male and one female \gls{crawler}.
\end{enumerate}

\chitincrawler[\npc{\T[3]\A\ifnum\value{Strength}>4\F\else\M\fi}{\Glsfmtplural{crawler}}]

\chitincrawler[\npc{\T[3]\A\ifnum\value{Strength}>4\F\else\M\fi}{\Glsfmtplural{crawler}}]

\Gls{spiderqueen} herself has since moved away and left her creatures to multiply.

\paragraph{If the troupe want to leave}
then they can, without issue.

\paragraph{If the \glspl{pc} make a lot of noise,}
all of the male \glspl{crawler} come out, and pursue.
The females remain with their eggs.

\segment[\gls{vlg}]{town}% AREA
{Another Hamlet Down}% NAME
{Everyone in \glsentrytext{town} talks about \glsfmtplural{crawler} eating everyone in a small, local hamlet, not far from \glsentrytext{town}}% SUMMARY

Select an inner settlement on the map and cross it out.
\Gls{spiderqueen} has raided it, and the place has become so full of \glspl{crawler} that people now avoid it, and take another road.

If the \glspl{pc} journey to kill the \glspl{crawler}, they find 10 \glspl{crawler} in total, and a lot of remains of dead farmers.

\segment{Forest}% AREA
{The Lone Ranger}% NAME
{A \glsentrytext{guard} ranger describes \glsentrytext{spiderqueen}'s fortress}% SUMMARY

Bilefen, a ranger in the \gls{guard}, went out to track down \gls{spiderqueen}.
He has succeeded, but won't approach her alone.

\begin{boxtext}
  A man ahead, dressed in black, stares at you, then slowly wanders forward.
  He puts his finger to his mouth, indicating you need to be silent.
\end{boxtext}

\humansoldier[\NPC{\M\Hu}{Bilefen Liskill}{tall, lanky, and agitated}{purses lips}{to get promoted to builder}]
\index{Liskill}

Bilefen approaches slowly, and explains his solo mission to track down \gls{spiderqueen}.

\begin{speechtext}
  \small
  There's that creature -- looks like \gls{sylf} went and fucked an elf -- she just went past with half a dozen \glspl{crawler} following her.
  We should get a peek at her situation and report back to \pgls{jotter}.
  But\ldots

  And I {\large\scshape cannot} stress this enough\ldots

  \noindent
  quietly\ldots
\end{speechtext}

What the troupe see depends on how far they're willing to go.
Ask the players to make a \roll{Dexterity}{Stealth} roll, without \pgls{tn}, give them the results, then ask if they want to press on while raising the~\gls{tn}.
If they fail, or press their luck too many times, trouble strikes.

%!

\begin{nametable}{What they see\ldots}
  \textbf{\gls{tn}} & \textbf{Result} \\
  \hline
  6 & In the distance, you see trees covered in so much webbing it seems like a fortress of goo. \\
  7 & There's a feint silhouette of \gls{spiderqueen} in the very centre, with two \glspl{crawler} next to her. \\
  8 & You can't see a clear path to the centre of the mess of webs yet. \\
  9 & Bilefen decides to \gls{retreat}. \\
  10 & Getting closer, you can count the motionless black spots in the fortress -- about a dozen. \\
  --- & {\large\scshape They've heard you, they climb down, three are out and running\ldots now six.} \\
\end{nametable}

\paragraph{If the \glspl{pc} push for Bilefen to join them,}
someone needs to roll \roll{Charisma}{Melee}, \tn[11].

\segment[\squash]{town}% AREA
{The Disappearing Fortress}% NAME
{\Glsentrytext{spiderqueen} has moved her fortress}% SUMMARY

While the next \gls{thread} plays out, drop the bad news on the \glspl{pc} -- thirty \glspl{guard} moved out to defeat \gls{spiderqueen}, but by that time she had moved her home and brood elsewhere.
Nobody has any idea where she might build her new home, but they know the walls of civilization cannot take much more.

\null
\segment{Forest}% AREA
{The Cunning Plan}% NAME
{Three gnomes have an elaborate plan for the troupe to kill \glsentrytext{spiderqueen}}% SUMMARY
\index{Gnomes!\Glsfmttext{spiderqueen}}

Three gnomes have been debating about how to approach the troupe about their plan.
\Gls{keras} thinks that it's best to honest, and just approach the troupe and ask if they would like to fight giant spiders.
However, Leta is the decision-maker, and her nose is longer than \gls{keras}'s,\footnote{Gnomes consider this to be an important point.} so she says there's no use talking to the troupe without testing if they really can fight \glspl{crawler}.
Mayelo, meanwhile, just wants both of them to stop fighting and make a decision.
He's been depressed ever since his \gls{village} was eaten by \glspl{crawler}, forcing him to move to \gls{oolery}.

Their final plan is to cast an illusion of \pgls{crawler} and see how the troupe react.
If they appear as skilled warriors, the gnomes approach and tell them the plan to defeat \gls{spiderqueen}.

\begin{boxtext}
  As you nip to the side to take a quick piss, a rustle above you shows that a giant arachnid has suddenly appeared, and looks down at you with dripping fangs.
  In the distance, high-pitched snickering can be heard.
\end{boxtext}

\paragraph{If the troupe flee,}
the encounter ends.

\paragraph{If the illusion of \pgls{crawler} has been vanquished,}
the three gnomes step forward.
Leta begins talking like she's some kind of trader.

\null
\begin{speechtext}
  So you don't like the \glspl{crawler}?

  You really hate them?

  How much would it be worth to you to be rid of \gls{spiderqueen}, who guides them through the human \glspl{village}?

  And you seem to be adventurers, in the employment of destroying monsters, is that so?

  And what if I told you that we could aid you pushing back against \gls{spiderqueen}?
\end{speechtext}

It's only after the characters emphatically agree that they do want to kill \gls{spiderqueen} that Leta informs them that she's feeling so generous that she's going to help them for free, and indeed has already laid plans.

If the players ask how the gnomes know exactly where she is, they explain they have triangulated her position through her periodic singing.
If they ask how the gnomes can be so certain that a half-kilometre tunnel, going somewhere the gnomes have never seen, can be so precisely dug, Leta shows them her calculations.
An \roll{Intelligence}{Academics} roll at \tn[11] shows that they are correct.

The players should be aware that if they jump out \emph{near} \gls{spiderqueen}, but not near enough, the dozen \glspl{crawler} will sprint towards them instantly.
Their only hope is to break out of the earth, kill her in an instant, and hope the \glspl{crawler} flee once her spell has been broken.

%\null
\begin{speechtext}
  It's simple.
  Anyone wandering close to that pit of spiders will be eaten by spiders.
  Any large army approaches, and she will flee, leaving option to track her whereabouts.
  The only way to be rid of her is a fast, decisive attack.
  But she has herself covered there too -- not yesterday we spoke to an elf who had spoken to local birds, who informed us that even the tops of the trees there are covered in webs.
  Her mobile fortress is impregnable, and hungry, and they will feed again soon.

  However, with our compass and our calculations, we have found a different way.
  We know that she rests not a kilometre \emph{that} way, and so half a kilometre that way there is a tunnel which we have almost completed.
  Once done, it will open \emph{directly} beneath the very place \gls{spiderqueen} sits.

  You know what you need to do.
\end{speechtext}

If the characters agree to squeeze through the tunnel, dig the last few feet, then burst out, then each one has to roll \roll{Speed}{Athletics} at \tn[7].
Success indicates that the character can spend 3~\glspl{ap} to climb out of the hole.
Failure indicates that the character pushes up a little, and becomes stuck,%
\exRef{core}{Core Rules}{prone}
while a tie indicates they exit badly, and become Prone, but those behind them can still exit, after reducing their \glspl{ap} to match the prone character's \glspl{ap}.

\begin{boxtext}
  You look up at the wide eyes of \gls{spiderqueen}.
  She immediately starts climbing higher up the tree, as dozens of \glspl{crawler} all around race towards you.
\end{boxtext}

Once out, they can shoot at \gls{spiderqueen}, climb the tree, or otherwise attack her.
Two of the \glspl{crawler} will arrive to attack each round, but once \gls{spiderqueen} dies, any who have not yet come forward do not attack.

\keras

\showStdSpells

\gnome[\NPC{\Gn\F}{Leta}{Inquisitive}{Picks nose}{less chatter, more action}]

\gnome[\NPC{\Gn\M}{Mayelo}{Creepy}{Scratches Adams apple}{to double-check the Maths}]

\chitincrawler[\npc{\T[3]\A\ifnum\value{Strength}>4\F\else\M\fi}{\Glsfmtplural{crawler}}]

\chitincrawler[\npc{\T[3]\A\ifnum\value{Strength}>4\F\else\M\fi}{\Glsfmtplural{crawler}}]

\paragraph{Success} means \gls{spiderqueen} has been killed or quelled.
If she's damaged and her chitinous children pushed back, she flees to seek new adventures elsewhere, and without killing any more people.

\paragraph{Failure} occurs when the characters fail to damage \gls{spiderqueen} before she flees.
Things get difficult here.
She destroys two more \glspl{village}, then wanders off to find something else to entertain herself, as elves often do.

\spiderqueen

\showStdSpells[
  \input{config/spells/Fate2.tex}
  \input{config/spells/Fire2.tex}
  \setcounter{enc}{3}
]

\ifnum\value{cycle}>3
  \sqpart{Forest}% AREA
{A Field of Geometric Dreams}% NAME
{The troupe find a dozen hibernating \glsfmtplural{crawler} in the snow}% SUMMARY
\label{spiderDreams}

This \gls{segment} only works during \gls{cTwo}, when \gls{snow} covers \gls{fenestra}, and all the \glspl{crawler} hibernate.
During any other \gls{cycle}, you can ignore this \gls{segment} -- it will only play out if nobody has dealt with \gls{spiderqueen} before \gls{sable} curses the land with frost.

\begin{boxtext}
	Eight shiny berries in a bush above you glimmer in the snow.
	They look black, but full of moisture.
\end{boxtext}

If the \gls{pc} picks the `berries', they find out those berries are in fact the eyes of \pgls{crawler}.
It will not try to wake the others, but every noise anyone makes will wake one more \gls{crawler}.

\paragraph{If the \glspl{pc} stay silent,}
this counts as holding their their breath.%
\exRef{core}{Core Rules}{holdingBreath}

\paragraph{If they try to kill all the \glspl{crawler}}
then ask them about their plans very carefully.
The \glspl{crawler} hibernate while packed in tight next to each other -- they can move out gently, without waking the others, but if any feel hurt, it will insinctively spread its legs suddenly, and wake up two more.

\chitincrawler[\npc{\T[3]\A\ifnum\value{Strength}>4\F\else\M\fi}{\Glsfmtplural{crawler}}\addtocounter{weight}{2}]

\chitincrawler[\npc{\T[3]\A\ifnum\value{Strength}>4\F\else\M\fi}{\Glsfmtplural{crawler}}\addtocounter{weight}{2}]

If they manage to harvest all the \glspl{crawler}'s parts, this could prove very lucrative indeed.
But then again, if they ask anyone for help, they will want a cut of the profits.

\paragraph{If the \glspl{pc} wait for \gls{spiderqueen},}
then they will have to wait until the \gls{storm} heralds \gls{cThree}, but she will come for her pets, and will not be expecting an ambush, so long as they do not look damaged from afar.

\paragraph{The next \gls{segment}}
cannot begin until \gls{cTwo} ends.

\fi


\segment[\squash]{town}% AREA
{Natural Balance}% NAME
{The lack of \glsfmtplural{crawler} have lead to a growth in griffins}% SUMMARY

Play this part only if \gls{spiderqueen} and a significant number of \glspl{crawler} die.

\Gls{greyLibrarian}%
\footnote{Find her \vpageref{greyLibrarian}.}
accosts the troupe on the streets, asking how many griffins they've seen, how many others have seen, how many feathers they've seen, and so on.

\begin{speechtext}
  As so many \glspl{crawler} died, the local griffin populations have grown to\ldots well, `monstrous' proportions.

  Frightful!

  Skies are no longer safe!

  Maybe we could plant some eggs around the forest.

  Could you track down some \glspl{crawler} eggs, and just spread them about the forest, until the bioarachnid balance restores?
\end{speechtext}

For the next \gls{cycle}, replace all \gls{crawler} encounters with $1D6$ griffins.

