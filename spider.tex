\resumecontents[Town]
\resumecontents[Villages]
\sidequest{\Glsentrytext{spiderqueen}'s Song}
\stopcontents[Villages]
\stopcontents[Town]

\startcontents[sq]

\sqminitoc

\noindent
When elves become old, they get weird.
\Gls{spiderqueen} has left her people, and devoted her life to enchanting animals with song and fostering a kinship with them.
Currently, she has collected four pet chitincrawlers, but she's having trouble keeping them, because they require too much food.%
\footnote{\iftoggle{aif}{See \autopageref{chitincrawler} for chitincrawlers.}{Chitincrawlers are gargantuan spiders, about the size of a horse.}}
She has also begun Polymorphing her own body to look progressively more like a chitincrawler.

Over the course of these encounters, \gls{spiderqueen} becomes progressively more irritating to the party and the local area, until the players finally happen upon her lair.

Chitincrawlers don't operate during the Winter, so if you run into a Winter season, just miss these encounters until the world is warmer.

\sqpart{Forest}% AREA
{The Arachnid Double Cross}% NAME
{\Gls{spiderqueen} double-bluffs the party, attacking with illusory chitincrawlers, mixed in with real ones}% SUMMARY
\label{spiderqueenssong}

\textbf{Background:}
\Gls{spiderqueen} wants her chitincrawlers to feed well (on the PCs, since they're here), but doesn't want them to get hurt.
She begins by casting an illusion of seven chitincrawlers, so the PCs will spot the fakery, and stop trying to resist so the real ones can eat them more easily.
To complete this illusion, she casts yet another illusion of a gnome sitting in a tree, so the PCs will feel convinced that nothing they see is worrying.

\begin{boxtext}

  You can always tell elven music by a sort of off-beat, where the beat goes wrong in a regular way.
  This one is soft and high-pitched, and interrupted by the sound of snapping twigs.
  More crackles come from in front.
  The setting Sun casts a red shimmer over the armoured bodies of a dozen man-sized chitinous, crawling creatures.
  The trees drop a small platoon of arachnids, and in a moment a hundred eyes are calculating how you taste.

\end{boxtext}

The PCs roll Wits + Vigilance to understand their environment.

\begin{rollchart}

  \textbf{TN} & \textbf{Result} \\\hline
  8 & The Sunset red on the chitin is too much, like the creatures don't look right.  You instantly spot that these are illusions. \\
  9 & On a nearby branch a little gnome sits, quietly giggling to himself, then looks shocked as you spot him. \\
  11 & The distant song seems to be coming from a single chitincrawler in the distance, \\
  12 & though she looks different from the rest. \\
  13 & Looking past the poor chitincrawler illusions in front of you, you notice that the rest are completely and definitively real. \\
  14 & The little gnome, however, is entirely fake. \\

\end{rollchart}

Ten chitincrawlers attack (3 real, 7 fake).

\paragraph{If any PC cannot tell the real chitincrawlers from the fakes,}
they will have real problems in combat as they will expend all their AP on the fake ones (which attack first).

\paragraph{If they kill a chitincrawler,}
\gls{spiderqueen} `changes her tune', and sings another song to bring them back to her.

\paragraph{If the PCs get close to \gls{spiderqueen},}
she polymorphs into a bird and flies away.

\chitincrawler[\npc{\A\T[3]}{3 Chitincrawlers}]

\sqpart{Villages}% AREA
{Sheep Stampede}% NAME
{\Glsentrytext{spiderqueen} summons sheep to be eaten by her chitincrawlers}% SUMMARY

\begin{boxtext}

  A nearby shepherd suddenly shouts out ``Hey!'', as he loses control of his flock.
  A distant song entices the sheep to run toward its discordant melody. It is as if the singer caused the music itself to decay.

\end{boxtext}

\textbf{Background:}
\Gls{spiderqueen} has gathered more chitincrawlers, and she needs to feed them again, so they have laid out their webs, and await their dinner.

\begin{boxtext}

  The sheep go beyond sight, and into the distant trees, then the song stops, and they begin to cry out in a way you've never heard sheep cry before.  Half of them flee straight back out of the forest.

\end{boxtext}

\paragraph{If the PCs do nothing,}
the chitincrawlers feed for thirty minutes, then leave.

\paragraph{If the characters pursue,}
they encounters webs (Wits + Vigilance, TN 9 to spot), then see the chitincrawlers feasting on the sheep.
They will disengage and attack the characters if they take any Damage.

\paragraph{If the players mention specifically to look out for webs,}
their characters should spot them immediately.

As before, \gls{spiderqueen} waits in the distance, and flees at the first sign of trouble.

\chitincrawler[\npc{\A\T[8]}{8 Chitincrawlers}]

\sqpart{Villages}% AREA
{Quiet Little Hamlet}% NAME
{An entire hamlet has been eaten by chitincrawlers}% SUMMARY

\widePic{Dyson_Logos/ruined_village}

\textbf{Background:}
Most villages in the inner circle, closer to a town, feel safe.
If woodspies or chitincrawlers attack, they generally engage with earlier, walled towns, rather than those close to the centre of the circle of civilization.

Knowing this, \gls{spiderqueen} coaxed her babies past the dangerous walled towns, full of armoured archers, and brought them to a quiet little hamlet where they could feed safely, and lay lots of eggs.

The plan went perfectly, so she left them to sleep and mate, and plans to return when she has more little babies to bring home with her.

\begin{boxtext}

  The little hamlet rests quietly.
  The air is cool, but then a single cockerel lets off half a crow in the distance, and goes suddenly silent before he's finished.
  It's only then you really notice: the fields have no animals, and the farmhouse chimneys don't give out any smoke.

\end{boxtext}

However, inside each of the four farmhouses, rooms are filled wall-to-wall with webbing.  Each house contains the same thing:

\begin{enumerate}
  \item
  Dead villagers in webs.
  \item
  Half-dead villagers in webs, waiting to be eaten.
  \item
  Great sacks of chitincrawler eggs, ready to burst out and feed.
  \item
  One male and one female chitincrawler.
\end{enumerate}

\Gls{spiderqueen} herself has since moved away and left her creatures to multiply.

\paragraph{If the party want to leave}
then they can, without issue.

\paragraph{If the PCs make a lot of noise,}
all of the male chitincrawlers come out, and pursue.
The females remain with their eggs.

\chitincrawler[\npc{\A\T[3]}{Male Chitincrawlers}]

\chitincrawler[\npc{\A\T[3]}{Female Chitincrawlers}]

\sqpart{Forest}% AREA
{The Lone Ranger}% NAME
{A member of the \glsentrytext{guard} stalks \glsentrytext{spiderqueen}}% SUMMARY

\textbf{Background:}
Gregory, a scout in the \gls{guard}, went out to track down \gls{spiderqueen}.
He has succeeded, but won't approach her alone.

\begin{boxtext}

  A man ahead, dressed in greens, stares at you, then slowly wanders forward.  He puts his finger to his mouth, indicating you need to be silent.

\end{boxtext}

Gregory approaches slowly, and explains his solo mission to track down \gls{spiderqueen}.
He confesses he feels scared, and asks the party to come with him (``but quietly.!.'').

\paragraph{If the party continue following the tracks,}
they encounter \gls{spiderqueen}'s lair.

\begin{boxtext}
  In the distance, you see trees covered in so much webbing it seems like a fortress of goo.
  No clear path presents itself, and shortly after the webs you can see five, then ten, arachnid silhouettes.
  A few move slowly down their trees.
\end{boxtext}

Having found the location, Gregory only wants to return to \gls{town} and give his report.

\paragraph{If the PCs fight,}
the chitincrawlers don't move out quickly -- they feel safer in their lair, but they may come out if fire begins to burn their home.

Twelve chitincrawlers in total remain in the fortress of webs, and two emerge each round.
\Gls{spiderqueen} then moves out to cast aggressive Polymorph spells, turning them into goats, birds, or other creatures.

\paragraph{If the PCs push for Gregory to join them,}
someone needs to roll Charisma + Combat, TN 10.
Whether or not they succeed, Gregory impresses on them that they have no duty to fight.

\humansoldier[\NPC{\M}{Gregory}{Suspicious}{Purses lips}{Tribe}]

\paragraph{If the party attempt to light the forest on fire,}
they will have a hard time.
The forest is muggy and damp at the best of times.
Even in the warmer seasons, lighting a fire requires an Intelligence + Wyldcrafting roll, TN 13.
No fire will spread fast, so the chitincrawlers will still have time to attack.

\sqpart{Town}% AREA
{\squash The Disappearing Fortress}% NAME
{The \gls{spiderqueen} has moved her fortress}% SUMMARY

While the next Side Quest plays out, drop the bad news on the PCs -- the \gls{guard} moved out en masse to defeat \gls{spiderqueen}, but by that time she had moved her home and brood elsewhere.

Nobody has any idea where she might build her new home, but they know the walls of civilization cannot take much more.

\sqpart{Forest}% AREA
{The Cunning Plan}% NAME
{Three gnomes have an elaborate plan for the party to kill \gls{spiderqueen}}% SUMMARY

\textbf{Background:}
Three gnomes have been debating about how to approach the party about their plan.
\Gls{keras} thinks that it's best to honest, and just approach the party and ask if they would like to fight giant spiders.
However, Holly is the chief illusionist, and her nose is longer than \gls{keras}'s,\footnote{Gnomes consider this to be a very important point.} so she says there's no use talking to the party without testing if they really can fight chitincrawlers.
Greg, meanwhile, just wants both of them to stop fighting and make a decision.
He's been depressed ever since their village was eaten by chitincrawlers.

The final plan is to cast an illusion of a chitincrawler and see how the party react.
If they appear as skilled warriors, the gnomes approach and tell them the plan to defeat \gls{spiderqueen}.

\begin{boxtext}

  As you nip to the side to take a quick piss, a rustle above you shows that a giant arachnid has suddenly appeared, and looks down at you with dripping fangs.
  In the distance, high-pitched snickering can be heard.

\end{boxtext}

\paragraph{If the party flee,}
the encounter ends.

\paragraph{If the illusion of a chitincrawler has been vanquished,}
the three gnomes step forward.
Holly begins talking like she's some kind of trader.

\begin{speechtext}

  So you don't like the chitincrawlers?

  You really hate them?

  How much would it be worth to you to be rid of \gls{spiderqueen}, who guides them through the human villages?

  And you seem to be adventurers, in the employment of destroying monsters, is that so?

  And what if I told you that we could aid you pushing back against \gls{spiderqueen}?

\end{speechtext}

It's only after the characters emphatically agree that they do want to kill \gls{spiderqueen} that Holly informs them that she's feeling so generous that she's going to help them for free, and indeed has already laid plans.

If the players ask how the gnomes know exactly where she is, they explain they have triangulated her position through her periodic singing.
If they ask how the gnomes can be so certain that a half-kilometre tunnel, going somewhere the gnomes have never seen, can be so precisely dug, Gregory shows them his calculations.
An Intelligence + Academics roll at TN 11 shows that they are correct.

The players should be aware that if they jump out \emph{near} \gls{spiderqueen}, but not quite at her, they will be attacked instantly, with little hope of survival.
Their only hope is to break out of the earth, kill her in an instant, and hope the chitincrawlers flee once her spell has been broken.

\begin{speechtext}

  It's simple really.
  Anyone wandering close to that pit of spiders will be eaten by spiders.
  Any large army approaches, and she will flee, with no option to track here whereabouts.
  The only way to be rid of her is a fast, decisive attack.
  But she has herself covered there too -- not yesterday we spoke to an elf who had spoken to local birds, who informed us that even the tops of the trees there are covered in webs.
  Her mobile fortress is impregnable, and hungry, and they will feed again soon.

  However, with our compass and our calculations, we have found a different way.
  We know that she rests not a kilometre \emph{that} way, and so half a kilometre that way there is a tunnel which we have almost completed.
  Once done, it will open \emph{directly} beneath the very place \gls{spiderqueen} sits.

  You know what you need to do.

\end{speechtext}

If the characters agree to squeeze through the tunnel, dig the very last few feet, then burst out, then each one has to make a Speed + Athletics check at TN 7.
Success indicates that the character can spend 4 Initiative to climb out of the hole.
Failure indicates that the character will not be able to get out of the hole that round, and neither will anyone behind them.

\begin{boxtext}

  You look up at the wide eyes of \gls{spiderqueen}.
  She immediately starts climbing higher up the tree, as dozens of chitincrawlers all around race towards you.

\end{boxtext}

Once out, they can shoot at \gls{spiderqueen}, climb the tree, or otherwise attack her.
Two of the chitincrawlers will arrive to attack each round, but once \gls{spiderqueen} dies, any who have not yet come forward do not attack.

\keras

\gnome[\NPC{\F}{Holly}{Inquisitive}{Picks nose}{Tribe}]

\gnome[\NPC{\M}{Greg}{Creepy}{Scratches Adams apple}{Acquisition}]

\spiderqueen

\Gls{spiderqueen} has spent 4 MP to gain a spider-like body, with +3 Strength and DR 3.

\chitincrawler[\npc{\A\T[6]}{Chitincrawlers}]

\paragraph{Success} means \gls{spiderqueen} has been killed or quelled.
If she's damaged and her chitinous children pushed back, she flees to seek new adventures elsewhere, and without killing random villagers.

\paragraph{Failure} occurs when the characters fail to damage \gls{spiderqueen} or her children before they flee.
Things get difficult here.
She attacks neighbouring villages twice, then gains her fifth level of Polymorph, and decides to become an air spirit for a while.
These two attacks play out as above, so she can only be stopped by a full-on assault at her lair, without the aid of a gnomish tunnel.

\stopcontents[sq]


