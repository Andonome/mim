\sidequest[Roads,Town]{Sewer Bandits}
\label{sewerking}

\histEvent{1}{1}{The \glsfmtplural{diggers} begin to fill \glsfmttext{town}'s catacombs with ghasts}
\begin{exampletext}
  \noindent
  The \glspl{diggers} under \gls{town} have been working on a plan.
  Ever since \gls{sewerking} returned with the `\gls{bskulls}' \gls{artefact}, they noticed that people who get too close to it see strange visions and lose much of the sensation in their bodies.
  Once a member of the \gls{sunGuard} fell victim to this effect, while on a raid, they killed him, and found that he turned into a ghast -- a sentient undead creature, with a broken mind and a killer instinct.%
  \exRef{judgement}{Judgement}{ghast}

  Besides enjoying the humiliation of the guard, this gave \gls{sewerking} an idea, and eventually, all the \glspl{diggers} agreed to it:

\end{exampletext}

%!
\null
\label{ghastPlan}
\begin{description}
  \item[Phase I]
  \Gls{healerLeader}, a beaurocrat in the \gls{templeOfSickness} slowly poisons rich patrons who are old, or have fallen ill.
  The poison makes their minds empty and their bodies stiff, and \gls{healerLeader} declares them dead.
  Other \glspl{diggers} then come to swap out the barely-living person with a fake body-bag full of earth, rocks, and fabrics.

  The \glspl{diggers} then lock these sick, rich people in little alcoves in their \nameref{sewers}, and use \pgls{artefact} to turn them into ghasts -- sentient undead, ready to kill.
  \item[Phase II]
  involves digging a tunnel underneath \gls{town}'s citadel (\vpageref{citadel}).
  \item[Phase III]
  is to release the ghasts into the citadel while lighting a rancid fire underneath, and let undead-nature take its course.
\end{description}
\null

In the meantime, \gls{sewerking} has been working as a fence, using the tunnel to \gls{traitor}'s house, just outside \gls{town}'s wall.

\sqpart{Roads}% AREA
{Art Collectors}% NAME
{\Glsentrytext{sewerking} introduces himself as a fence}% SUMMARY

\begin{exampletext}
  After raiding \gls{necromancer}'s crumbling temple with \gls{traitor} (\vpageref{rottenBreath}), \gls{sewerking} stashed the \gls{artefact}, and decided to sell his more mundane (and valuable) prizes in \gls{lochside}.
  He travels by road to avoid the heavy taxes levied in \gls{town}.
  \Gls{lochside} always has someone willing to buy valuable items, without too many questions.
\end{exampletext}

\begin{boxtext}
  In the far distance, a trading caravan approaches.
  You can see the head rider's long hair trailing behind him in the \showTemperature\ wind.
  Both coaches have the usual archer on the top, though they don't look like \glspl{guard}.
\end{boxtext}

\Gls{sewerking} greets the troupe in a friendly way, and explains he's going to \gls{lochside} to sell his wares for a `client'.
He refuses to divulge any information about this client, then begins hinting that if the troupe ever run into anything valuable, and need to sell it, he can be discrete.

\setcounter{wounds}{2}

\sewerking

\humanarcher[\npc{\T[6]\Hu}{\arabic{noAppearing} \glspl{whiteBandits}}]

\null
\begin{speechtext}
  If you brave warriors ever need to sell any `paintings' you find, you can always find me in \gls{pig}, in \gls{town}.

  Anyway, it's a dangerous road.
  Fancy turning the other way and joining me?
  You know, all the work you do in the \gls{guard} has real value.
  People don't tell you that enough, but you guys save everyone, every day, and nobody says `thank you'.
  Why don't you come and protect people on this road, instead of that one, and I'll say `thank you', at the end.
\end{speechtext}

\Gls{sewerking} then offers the troupe 1~\gls{sp} per mile, in total, if they join him till \gls{lochside}.
Boating may be faster than the road, but boating means river-taxes.

The length of the journey depends on how far this scene takes place from \gls{lochside}.
\Gls{sewerking} takes a route not too far, or close to town, so encounters should consist of traders as much as beasts.

\paragraph{If anyone asks about the archers,}
\gls{sewerking} explains jovially.

\null
\begin{speechtext}
  We come from the Temple of `Odd-Job', to save people from the god of messing-about and time-wasting!
\end{speechtext}

The characters will understand that this is a joke, as they will all be familiar with standard temples.

\paragraph{If the party press the issue,}
\gls{sewerking} explains that he comes from \gls{whiteplains}, and while his family were \glspl{warden}, they have become poor, due to the high taxes imposed by \gls{town}, so now he travels and takes work where he can, and hires people to help him.

\paragraph{If the \glspl{pc} abandon their current mission,}
the next time they see \pgls{jotter}, the leader will have some explaining to do.
They should roll \roll{Intelligence}{Deceit} (\tn[12]), or suffer a demotion.
\Glspl{fodder} who behave in this way (the lowest rank) will find themselves escorted to the \gls{court} in \gls{town}, where \gls{townmaster} will sentence them.

\paragraph{If the \glspl{pc} look for him in \gls{pig},}
they may or may not find him -- he keeps a low profile.

\paragraph{The wagons hold}
the following items:

\null
\begin{itemize}
  \item
  20 days' rations.
  \item
  2 ancient human skulls, with red stones inserted around the crown.
  \label{skullCrown}
  \item
  A stone statue of a half-dead man in repose, in a coffin.
\end{itemize}

\begin{boxtext}
  As the Sun casts red over the forest, darkness closes around you.
  Then in the distance, fire, then another, not far off.
\end{boxtext}

\sqpart{Town}% AREA
  {Nobody Bags}% NAME
  {During a funeral, a body-bag spills open, revealing nothing but dirt and rocks}% SUMMARY

During a funeral procession, a coffin spills open (perhaps a horse gets spooked, perhaps one of the \glspl{pc} do something\ldots), the `corpse' falls out, but then spills open, revealing nothing but dirt and rocks.%
\footnote{Find the plan \vpageref{ghastPlan}.}

Everyone around stops, shocked into silence, and the dirge ends.
Eventually, a few members of the family march to the local \gls{templeOfSickness} (area \vref{townHealersGuild}) and demand to know what happened.
\Glsfmtname{healerLeader} mutters something about a `clerical error', then shuts the door with a look of deep embarrassment.

\boxPair[t]{
  \ghast[\npc{\D\Hu}{\Glsfmttext{armourHall} Ghast}]
}{
  \newGhast[\npc{\D\Hu}{Elderly \Glsfmttext{warden} Ghast}]
}

\sqpart{Roads}% AREA
  {Beyond the Pale}% NAME
  {\glsentrytext{traitor} passed too close to the \glsentrytext{bskulls}, and now looks semi-dead}% SUMMARY

\begin{exampletext}
  \Glsentryname{traitor} transported some barrels through his house, down to the \nameref{sewers}, so \gls{sewerking} could pick them up.
  Unfortunately for him, the \gls{bskulls} \gls{artefact} had a spell waiting for him.
\end{exampletext}

\Gls{traitor} will recover within a day, but the next time the \glspl{pc} visit \pgls{broch} (or anywhere on the road) they may find him giving a mission while looking\ldots not altogether \textit{there}.
An \roll{Intelligence}{Medicine} roll (\tn[8]) will show he's beyond unhealthy, while an \roll{Intelligence}{Academics} roll (also \tn[8]) lets \pgls{pc} know he's suffering from the Death \gls{sphere}.

Of course, if \gls{bskulls} has already departed \nameref{sewers}, skip this \gls{segment}.

\sqpart{Town}% AREA
{\gls{night}~Unexpected Ghasts}% NAME
{The dead run through the streets}% SUMMARY

\Gls{sewerthief} forgot to lock a door properly, and a room of ghasts escaped via \nameref{slum_exit} (area \vref{slum_exit}).
If the \glspl{pc} are anywhere near the area, the ghasts arrive, and kill quickly, quietly, and indiscriminately.

\begin{boxtext}
  The dock worker takes lets his pipe rest while his hands come up to his ears.
  He asks if you can hear it\ldots the irritable footsteps, like someone's digging their anger into the ground.
\end{boxtext}

The ghasts move together, but slightly spread out across the alleys, no more than 20~\glspl{step} apart.
They have enough sentience remaining to try to sneak up on the troupe (or anyone the troupe are speaking with).

\paragraph{If the \glspl{pc} shout for the \glspl{sunGuard},}
a roll of \roll{Strength}{Empathy} (\tn[7]) alerts two \glspl{sunGuard}, who arrive after 5~\glspl{round}.

\paragraph{At the end of the day,}
the ghasts will be put to the sword one way or another, but the question remains; what brought them here?

\paragraph{A little investigation}
reveals that these ghasts were high-ranking members of various temples, previously thought dead.

\newGhast[\npc{\T[3]\D\Hu}{Other Ghasts}]

