\subsection{The Old Alchemy Basement}
\label{old_alchemy_basement}

The entire basement of the old magical laboratory is sodden with water, resting knee-height to a human.

\begin{boxtext}
  Descending the stairs, you find a low ceiling, and a moment later correct yourself.  It's not a low ceiling -- black, stagnant, water has flooded the entire hall.

  The torch picks up a great stone pillar in the distance, and another a little farther along.
  Great double doors along the hall, to the right.

\end{boxtext}

\paragraph{Wading through the water}
is difficult.
All movement is limited by 2 squares minus the character's Strength Bonus (minimum of 2), so some will receive no penalty, while those with Strength -2 receive a 4 square penalty to movement each round.
For many, this will mean they cannot move at all without spending a full round pushing forward, or simply swimming.

Remember to note who has torches when underground, and that carrying a torch in one hand means the character is effectively duel-wielding.

\paragraph{The cold water and foetid air}
inflicts 2 \glspl{fatigue} per \gls{interval}.

\paragraph{All doors}
have swollen due to the many years of water-logging.
Opening them requires a Strength + Crafts Team Roll (TN 7).

As usual, each roll remains as-is, so if someone fails to open a door they will need to find a way to gather more strength, or give up opening that door.

\paragraph{Narrow hallways}
make wielding long weapons challenging.
The \textit{Enclosure Rating} for this place is 5, so any weapon which requires 6 Initiative to wield takes a -1 penalty to Strike.

\paragraph{The Dead Chant} when not in combat.
If they stand at the other end of a hallway, they chant.
If the characters lock them in a room, the dead stand outside and chant while clawing at the door.

This strange behaviour could vex any necromancer.
The simple spirits which inhabit and animate ghouls do not usually speak.
Their strange behaviour is the result of a powerful necromancer living in the catacombs.
The undead necromancer%
\footnote{See room \ref{undead_ogre}.}
felt the words engraved on top of the magical portal which can open a portal to the Realm of Darkness and Fire.
There was only one problem: the creature's tongue was too rotten to speak the words properly.
Its body has only been preserved due to the high content of peat in the water, but that was not enough to allow it to speak properly, so it tried teaching the dead to chant.

The words themselves simply mean `open to trade', but the characters will hear only ``Opena trei, opena trei, opena trei!''.

\paragraph{Cave-ins} present a real danger here.  If the ceiling ever collapses while the characters are inside, the falling rocks from above at first deals $1D6-2$ Damage to everyone in the room, then $1D6$, and so on, increasing by 2 each round, until it's unliveable.

\mapentry[alchemyHall]{Drowned Hallway}

\textbf{Background:}
When \gls{lostcity} was still burning from the nura attack some centuries ago, one necromancer raised a powerful undead spirit into the body of an ogre.
That ogre demi-lich then used the corpses around him to raise a regiment of ghouls.
The door was sealed during that time, peat-filled water flooded in, and the dead rested there, perfectly embalmed and perfectly still.

\paragraph{When the party enter,}
they move over an uneven floor, and some of the debris below them are ghouls.

The ghouls' stiff bodies move slowly, and by the time the characters have all moved into the hall, the dead rise between them, separating members of the troupe.

Five ghouls rest at the start of the hallway, and another five later on.
A further five at the other end of the hallway begin walking towards them immediately.
The entire situation makes for the perfect ambush, though the dead have not planned for it.

\ghoul[\npc{\T[15]\D}{15 Ghouls}]

\paragraph{If the party attempt to fell any pillars,}
have them roll Strength + Crafts, TN 13 (or less, if they use the right equipment, such as rope).
Once the pillars fall, the entire area collapses within two rounds.

\paragraph{If anyone searches the bodies,}
they find one of the ghouls which were once Woodspy Bandits has \gls{forestpriest}'s payment of \lootMedium.

\mapPic{b}{Dyson_Logos/under_lost_city}{
  \Large A/18/37,
  \Large B/42/37,
  \Large C/30/18,
  \huge S/355/18,
  \huge\rotatebox{90}{S}/73/71,
  \huge S/64/86,
  \ref{alchemyHall}/18/84,
  \nameref{alchemyHall}/18/93,
  \rotatebox{45}{\nameref{alchemyEquipment}}/14/61,
  \ref{alchemyEquipment}/20/59,
  \rotatebox{25}{\nameref{alchemyLibrary}}/46/76,
  \ref{alchemyLibrary}/52/72,
  \rotatebox{45}{\nameref{alchemyRooms}}/29/41,
  \ref{alchemyRooms}/31/37,
  \rotatebox{-45}{\nameref{alchemySecret}}/76/88,
  \ref{alchemySecret}/71/83,
  \rotatebox{45}{\nameref{alchemyGift}}/66/31,
  \ref{alchemyGift}/60/39,
  \rotatebox{-90}{\nameref{summoningRoom}}/93/58,
  \ref{summoningRoom}/87/58,
}

\mapentry[alchemyEquipment]{Equipment}

\textbf{Background:}

The standard alchemist's equipment -- gold dust, rubies, beechwood, chitin, and black soil -- have mostly been removed from the area during the panic when people fled.

Some of those panicked people returned and were dragged back into the portal, only to return as ogres.
Those ogres were resurrected as undead, along with everyone else.

\begin{boxtext}

  The shelves in this wide room are full of smashed and broken equipment, but it looks generally alchemical.

\end{boxtext}

\begin{boxtext}

  A powerful force grabs your ankle, and squeezes.
  A creature, taller than any man, stands up and turns you upside down, then pulls you in towards his teeth.

\end{boxtext}

\npc{\T[3]\N\D}{3 Undead Ogres}
\animal{6}{-3}{0}{-3}{3}{2}{}{Death Sight}{}

\paragraph{Characters who scour the room,}
can find rare gems on the ground, although wading through all the sludge will not be pleasant.
This requires a Dexterity + Academics roll (TN 6) to correctly identify coinage, and what items might be useful, by feel.
Each marginal point means 15sp worth of items has been found.

\mapentry[alchemyLibrary]{Library}
\index{Library!Ancient}

\begin{boxtext}

  These two doors stand locked and refusing to budge.  A simple brass lock stands rusted on the front.

\end{boxtext}

Opening \emph{this} door requires a Strength + Crafts roll, TN 10.

\begin{boxtext}
  The doors throw inwards, revealing row upon row of rotten books.
\end{boxtext}

\paragraph{Careful perusal of the books,}
allows a few items to be discovered.
Anyone searching joins the Group Roll of Dexterity + Academics, TN 6.
Each margin allows a particular book to be carefully extracted, but destroyed the book above, so rolling `9' means the valuable book on Invocation is found, but the letter is destroyed.
Destroyed books simply fall apart due to rotten spines and damp pages, but if taken back and cared for, some could be preserved.
The party can make any number of rolls, but if they want to use a resting action, each new roll requires a new \gls{interval}, and the cold water around them will make them fatigued.

\begin{enumerate}

\setcounter{enumi}{6}
  \item
  A book detailing a nearby treasure, guarded by a dragon and her two children.
  The mother leaves for a holiday every Alassea.
  A greedy dwarf penned this complete fabrication during the War of Lies.%
  \exRef{aif}{\textit{Fenestra}}{warOfLies}
  \item
  An ancient city map, detailing sites of interest such as a Temple of Qualm\"{e}, which holds beautifully decorated, and unaging corpses (see page \pageref{green_tower}).
  \item
  A valuable book on Invocation, worth 20sp.
  \item
  A letter stating that a portal to an unknown labyrinth realm has been opened, and that trade has opened with various dwarves in exchange for food.  It also states which word will activate this portal.
  \item
  Letters of complaint from the Dean of Conjuration, stating that the Dean of Illusion must tidy his room, and that the rats he's brought in have become so bad that he's ordering no food to be permitted in the area, under any circumstances.
  \item
  A valuable book on late-stage Conjuration, worth 100sp, covering the fifth level of the Conjuration sphere.
  \item
  Threatening letters from elves saying to be wary of opening portals.
  \item
  Hidden behind some other books: a book on Nuramancy.  This is highly illegal, but allows anyone to gain up to a single level in Nuramancy with a little study, and the right (or wrong) attitude.
  \item
  A book on opening portals to distant places using a clever combination of the Force and Conjuration spheres.
  \item
  A letter granting permission to open a portal to the Realm of Darkness and Fire, with the hopes of trading magical items for food.
  This proves the humans were opening dangerous portals!

\end{enumerate}

\mapentry[alchemyRooms]{Dark Pit}

\begin{boxtext}
  These doors swing open effortlessly, showing a new room with three more doors; right, left and centre.

\end{boxtext}

\textbf{Background:}
Three rooms here used to house the various masters of alchemy.
Stairs reached down to a lower floor, then back up.
Since then it flooded, though the water will not show that.

One of the ogres in this area was raised as a ghast, and the spirit inhabiting that body knows powerful necromantic magic.
It tried to teach the dead how to say the password to open the door, and resurrect the Woodspy Bandits who came in earlier.

Mostly, it simply sits at the bottom of the pool.

\paragraph{If anyone steps into the water,}
they need to pass a Dexterity + Caving check (TN 10).

Success means the ghastly ogre reaches out, and attempts to grab them.
Failure means they fall into the water, \emph{then} get grabbed (with a -2 penalty to resist the attack).

\npc{\M\N\D}{Ghastly Ogre Mage}
\label{undead_ogre}

\person{6}% STRENGTH
  {-3}% DEXTERITY
  {0}% SPEED
  {{2}% INTELLIGENCE
  {0}% WITS
  {-5}}% CHARISMA
  {3}% DR
  {1}% COMBAT
  {
    Aggression 2, Academics 2, Xenomology 3, Medicine 1
  }% SKILLS
  {\longsword}% EQUIPMENT
  {
    \addtocounter{xpbonus}{3}
    \setcounter{Fate}{2}
    \setcounter{Air}{2}
    \setcounter{Water}{1}
  }

\paragraph{Room A} used to house the master of Conjuration, who built the portal in room \ref{summoningRoom}.
The ogre has kept him around for his own amusement, as a ghoul.
He still has \lootBig\ in his pockets.

\paragraph{Room B} houses nothing but broken furniture and sludge.
The last room, however, is different.

\paragraph{Room C} used to house an illusionist, and his spells are still going ever since he died.
Instead of cleaning his room, he would simply cast an illusion of cleanliness.
The room looks immaculate, and full of light.

\begin{boxtext}
  The heavy door creaks open to an attractive room, like an expensive upstairs room in a tavern, complete with a bed, a study, and a freshly cooked breakfast on the table.
\end{boxtext}

Within the room, under the comfortable-looking (but filthy) bed, is a hidden little tunnel, which leads up to a secret room.
The ogrish mage cannot follow the characters here, even if he wanted to put up with the irritating light, because he is far too large to fit through the narrow opening.

\mapentry[alchemySecret]{Secret Study}

Up the stairs the area remains dry, safe and eventually leads to a regular door (no roll required to open it).
Inside, the room contains tables with extremely old scrolls, dust, and a series of very out-of-date books on alchemical theory.

\begin{boxtext}

  The stairs reach up, and finally you step your muddy boots out of the water and along a cold, but dry corridor.

\end{boxtext}

\paragraph{Resting here}
causes no \glspl{fatigue}, as the place is not full of cold water.

\paragraph{Reading the old language}
requires an Intelligence + Academics roll (TN 10).
The scrolls say this:

\begin{exampletext}

  I shall see you by Laiquea.  Have the portal completed.  We have no funds.  Five lands mapped.

\end{exampletext}

\begin{exampletext}

  Some funding came through.  They want mutton, beef, bread and soup.  Everything must be prepared before sending, except the meat.

  Prepare the food.  Destroy this letter.

\end{exampletext}

\begin{exampletext}

  The portal has been established.  Negotiations are going well, but please have more guards available than last time.  Excuses aside, we can't have a repeat of the last incident.  Three women.  It doesn't sound good in song.

  Of course if you want my advice we would put every bard in the kingdom to the sword and be done with the matter.

\end{exampletext}

A Wits + Crafts check, TN 10, reveals a loose wooden board in the ceiling.
It used to be an exit to the ground floor of the Citadel above,%
\footnote{See area \ref{fallen_tower} in `\nameref{lostcity}', page \pageref{fallen_tower}.}
but now the upper floor is just the ground outside\ldots after a lot of digging upwards.

\mapentry[alchemyGift]{Giftschrank}

\textbf{Background:}
This bare room used to store various books, including the words which open the portal.
It's flooded, like every other room on its level.

The two skeletons on the table have aged worse than the other corpses, as they were never preserved in the peat-water.
They died of hunger rather than facing the dead they knew to be outside.
One holds a book of poetry, and the other holds a book of Enchantment spells, which she never managed to understand before dying.

\begin{boxtext}

  The bricks fall away easily, revealing a full new room.  Two skeletons rest on a table, each clutching a book.

\end{boxtext}

\paragraph{The book of conjuration}
is outdated, but still worth at least 20gp to \gls{alchemists}.

\paragraph{The book of poetry is pleasant,}
and hides one spell-song -- a poem which still functions to stop the user fearing any type of problem and regenerates 1D6+1 \glspl{fp} (it holds 3 mana, and costs 2 to cast).

\paragraph{Finding the words which unlock the portal}
requires an Intelligence + Vigilance roll, TN 9.
It's hidden among a dozen rather dull books on proper etiquette with alchemy, and accountancy books concerning what the guild brings in and what is can produce.

The door to the Summoning Room is only blocked by a crude wooden panel, so exiting only requires a kick

\paragraph{Spotting the hidden door from the outside}
requires a Wits + Vigilance (TN 8).

\mapentry[summoningRoom]{Summoning Room}

\textbf{Background:}
This is where the magic happened.
When anyone said `open to trade', the portal came to life and allowed a trade of foodstuffs for magical items and knowledge.

The language is old but an Intelligence + Academics roll, TN 9, will allow anyone to understand it.

The dungeon's necromancer (in room \ref{undead_ogre}) has laid a trap for anyone entering this room.
He chained ten ghouls to each of the front pillars.


\begin{boxtext}

  The massive double doors slowly swing inwards, and the torchlight reveals a flooded hallway of six stone pillars, two enclaves, and a stairway leading up to a stage.
  The stage shows a grand stone arch, like a doorway, leading to darkness.
  You can see an writing across the top.

\end{boxtext}

\paragraph{If any of the players say the words out loud,}
so do their characters; the portal opens and the ghouls in the room begin chanting along with them in unison.%
\footnote{As usual, speech costs 2 Initiative points, so if the ghouls are in combat once the words are spoken, the party should enjoy the unexpected advantage they get.}

\paragraph{If the PCs remove the gemstones in the portal,}
it breaks forever.
The gems are worth 30gp in total.

If undiscovered, the dead stand and begin their chant, then slowly walk towards the characters.

\paragraph{After 2 rounds,}
the dead stand up.

\begin{boxtext}

  You look behind, and note two-dozen dead men standing from the water and staring at you.
  Their skin has gone brown with age, and they look barely able to move.
  Each drags a chain behind it, tied around one of the entrance pillars.
  They pull together towards you, each uttering the same strange, chanting moan, and then stop as the chains go tight around the stone pillars.

\end{boxtext}

\paragraph{If a pillar has six or more ghouls pulling at it,}
then it collapses after 3 rounds.

\paragraph{If a group of ghouls grab a character,}
they stop pulling at their chains and focus on attacking that character.

\paragraph{If the characters open the portal,}
they see a dark room, with a distant light.
What might be less obvious is that the portal opens on the \emph{ceiling} of a room in the Realm of Darkness and Fire.
Anyone throwing an item in notices it flies, then `sticks' to the far `wall' (meaning, the ground).
Characters may notice the discrepancy from the odd appearance of the doors on the other side, with a Wits + Crafts roll, TN 10.

Ten hobgoblins immediately arrive with a ladder and start making their way up, into the dungeon.
They know the portal can open, and they know they need a password.
They fight, but try to keep the characters alive so that they can learn the magic word which opens the portal (they cannot read).

\paragraph{If the characters drop through the portal,}
you're on your own.

Perhaps they will survive a while in the nura realm.
Perhaps they will make it back through a different portal.

But probably not.


