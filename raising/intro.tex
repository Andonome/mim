The next \gls{thread} --- \nameref{troubleAle} --- dominates the action in \gls{town} for some time.
It forms a `\gls{segment}-stopper', just to slow things down a little, so later \glspl{thread} won't advance too far until most of the dangling \glspl{segment} from \autoref{threadsI} reach a conclusion.

When the \glspl{pc} find \glspl{whiteBandits} assaulting an elf, they may rescue the elf and gain an ally, or may sensibly avoid the issue.
After a couple of stops in \gls{town}, early clues about the \gls{diggers} plans to use undead as \glspl{weapon} emerge, and the \glspl{pc} will find an avenue to investigate, or simply ignore the issue.
Of course, each problem they ignore festers, and returns as a much bigger problem.

By this point, the players should have begun to investigate the many locations around \gls{valley}, and begun thinking about its history (or at least they should have noticed that it \emph{has} history to be found).

These \glspl{thread} have an assumed start-date of \showCycle,
\ifnum\thecycle<4%
  in the year \thefenestraYear,
\fi%
but \autoref{threadsI} and \ref{threadsIII} will overlap with these \glspl{thread}, and small adjustments may need to be made for \gls{snow}, \glspl{heatwave}, or \pgls{storm}.
