\sidequest{Interruptions}
\label{interruptions}

\noindent
The deep forest is no place to build relationships or get into prolonged battles -- it is a chaotic environment, where one never knows what the next day brings.
These disjointed Side Quests don't fit with anything in particular, but exist to provide little clues to other quests, or simple distractions.

\sqpart{Forest}% AREA
{\squash Broken Sword}% NAME
{One of the characters' weapons breaks}% SUMMARY

\begin{boxtext}

  Your sword plunges into the chitincrawler's face, but as you pull it out, the creature twists its body, and your sword shatters.
  You pull out the handle with a metal stump, and the next creature attacks.

\end{boxtext}

One of the characters' weapons shatters at just the wrong time.
The next time any character uses a m\^el\'ee weapon, that weapon shatters unless there is some reason it cannot shatter, such as being a magical item, or a weapon renowned for being of excellent quality.
If one weapon cannot shatter, move to the next which is used.

This encounter combines with the next Side Quest, so the players will most likely find a weapon shattering during combat.
This might happen when the character smashes their weapon into an enemy's, or perhaps when stabbing at an enemy so deeply the weapon embeds in a creature's hide, and then snaps off when the weapon is withdrawn.


\sqpart{Forest}% AREA
{The Curious Crawler}% NAME
{A chitincrawler tries to dig up a little gnome}% SUMMARY

A chitincrawler pulls up the earth beside a tree, as if trying to dig under it.
Shrewd characters might spot that under the tree a little gnome lives.
The side of the tree opens, revealing a very small staircase.
The chitincrawler has smelled the gnome cooking food, and has decided to stay up top and dig until he catches the little creature.

\begin{boxtext}

  A distant shuffling past some trees starts, then stops, then starts then stops.
  In the far distance, you see the dim silhouette of a chitincrawler scratching around the base of a tree, as if trying to dig something up.

\end{boxtext}

Wits + Crafts (TN 9) to understand that the tree leads to an underground home.

\chitincrawler

\keras

\paragraph{If rescued,}
\gls{keras} is delighted, and gifts the characters a scroll which -- once read aloud -- will cast an illusion of a chitincrawler.  It was studying chitincrawlers for his spells that got him into this mess.

\sqpart{Forest}% AREA
{Random Traders}% NAME
{Three tradesmen are lost in the forest}% SUMMARY

\textbf{Background:}
Aaron, carrying various flowers, and over a hundred eggs, started the day late, and knew that his cargo would be bad before reaching \gls{town}, so he convinced Jason (carrying uncured meats) and Steve (carting blood sausage) to go with him via an old road his grandfather told him about.
However, the road is completely overgrown, so the traders are now stuck in the woods, and lost.

\begin{boxtext}
  In the distance, you see a group of a dozen men trying to get their first wagon out of a muddy ditch.
  Two more wagons sit behind.
\end{boxtext}

\humantrader[\npc{\T[3]\M\Hu}{Aaron, Jason, and Steve}]

\sqpart{Forest}% AREA
{The Elven Party}% NAME
{The party are told to dance, and dance they must}% SUMMARY

\begin{figure*}[b]
\begin{speechtext}

  They never attacked, but cleaned up the mess.  Alchemists back in those days had no legal restrictions, and many were originally tradesmen.  They opened portals to various other lands, and soon began trading goods with the strange creatures there.

  Nura came through at one point, invaded the city, and laid waste to it.  My grandfather came to save people, but upon seeing the complete ruin of the city, devoted himself to ridding the area of nura.  He died in that war, as so many other elves did.

  Men did not repopulate the area because of their grief, and their fear of undiscovered portals.
  Perhaps four in total were created before the assault began.
  If people populate the area, it will only be a matter of time before they find those now-buried portals, and try to make use of them.

  But this is a dark conversation, and we wandered to dance, and celebrate the changing season.

\end{speechtext}
\end{figure*}

Elves have better eyesight than most, so many of their feasts take place in the darkness, and involve games of hide and seek, or enchantment.

\begin{boxtext}

  Off-kilter music and half rhyming words, wander out from the forest, then gentle footsteps to the far right, and more in the distant left.

  The scent of fresh fruit, salads and salmon hit you.  There's a low-burning fire in the distance, looking enticing.

\end{boxtext}

The elves hear the characters, and quickly hide as a game.  Those at the farthest reaches of the gathering shout out that the game is on, and everyone between hides quickly.

\begin{boxtext}

  The noise of little feet darts around the silent forest, but nobody responds.
  A feast lies on the blanket.

\end{boxtext}

\paragraph{If the characters eat the food, nothing bad happens.}
It tastes great.
The game doesn't end until the characters settle down to eat or they find an elf.

The characters can roll Wits + Vigilance, TN 8 to see how quickly they find an elf, but there are two dozen, so it's only a matter of time before they see one.

Once the game is up, all the elves come out of hiding and laugh.  They dance, and sing, and feast.  However, the elves get a little too carried away, and eventually enchant the party to continue dancing all night.  The elven illusions make sure that the songs echo long past when the singers have gone for the night, and the characters just continue dancing.
 
\elf[\NPC{\F}{Aiw\"{e}}{Jester}{Looks upwards}{Tribe}]

Aiw\"{e} loves a laugh but never learnt when she's gone too far, and will fashion leaf-crowns for dancing characters, adorning them while they dance.

\elf[\NPC{\M}{Taurestel}{Pedagogue}{``For example\ldots''}{Experience}]

\elvenenchanter[\NPC{\F}{Erende}{Curt}{Raises Eyebrow}{Acquisition}]

\paragraph{If the characters ever ask about \gls{lostcity},}
and whether or not elves destroyed it, the elves present say that they were never there, but have heard the story from their elders (see below).

\paragraph{Once the party is in full swing,}
ask for a Wits + Academics roll, TN 12.
Each time the party fail, they dance for another \gls{interval} and lose 3 \glspl{fatigue}.
They dance until they pass out or until someone succeeds in the roll.

\paragraph{But if they become violent,}
Erende responds with a \textit{Fast, Wide, Sleep} spell.
There are twenty elves, so the characters have little chance of victory.

\sqpart{Forest}% AREA
{Furry Traders}% NAME
{Three gnolls are here to trade}% SUMMARY

\textbf{Background:}
The gnolls have caught four deer in a trap, eaten one, and cured the meat from the other three.  They are willing to trade.

\begin{boxtext}

  In the distance, hunched humanoids carrying spears and heavy loads on their backs stop suddenly.
  They eye you up, then come forward, with ears pricked up high.
  It's a group of gnolls, carrying large sacks.

\end{boxtext}

\gnollhunter[\npc{\T[6]\Nl}{6 Gnolls}]

\sqpart{Forest}% AREA
{\N The Mouth of Hell}% NAME
{A thousand woodspies have gathered around a portal to Hell}% SUMMARY

Some encounters cannot be bested. The only thing for the party to do is take running away as the best possible victory.

\begin{boxtext}

  Pushing more foliage aside, you notice this area looks strange somehow, like the trees are made of wax.

\end{boxtext}

\textbf{Background:}
An underground kingdom of nura have found a portal to Fenestra, and one might expect hobgoblins and nastier things to pour out and devour people.
However, the first few were caught by a woodspy, who bred, and started a family, and the same happened to the next few.
At this point, around 300 woodspies live in an entirely unnatural (and slightly inbred) alliance around a single hole in the ground, where their food comes from.

On the other side of the portal, the nura have begun using the mysterious hole to dispose of their unwanted.
Criminals, wounded and useless nura, or just those who annoy everyone, find themselves thrown into the hole.

\begin{boxtext}

  Just ahead of you, you see a pit lined with stones, each with expensive gems and covered in alchemical writing, carved into the rock.
  An intense heat emanates from the pit.

  Suddenly, the waxy parts of the trees start to move, revealing itself to be a three-limbed creature, shifting across the bark.
  Behind you is another, and another, and then an entire tree pulls itself apart, revealing another two dozen of the shape-shifting creatures.

\end{boxtext}

\paragraph{If the woodspies ever leave the area,}
the nura will have a safe portal to Fenestra, and they will raid the local area.
If the portal ever closes, the woodspies will become ravenous, and invade the local population.
The two stand in a tasty equilibrium, and the best the characters can do is flee.

The woodspies are fat and unconcerned with chasing the characters far.

\woodspy[\npc{\T[4]\A}{300 Woodspies}]

\paragraph{Take a pen,}
and add this nura portal to your map of the area, wherever the PCs have wandered.

\paragraph{If the local nura rating ever reaches 7,}
the nura will raise an army big enough to break out, and drag any woodspies in the area below to become nura themselves.
Raise the local Nura Ratking by 1.

\morphwoodspy[\npc{\T[4]\N}{Nura Woodspies}]

\stopcontents[sq]


