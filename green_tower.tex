\subsection{\Glsfmttext{greentower}}
\label{green_tower}

\mapPic[\Large]{t}{Dyson_Logos/green_tower}{
  \Huge\ref{greenGrounds}/62/55,
  \ref{greenConservatory}/26/35,
  \ref{green1}/10/35,
  \ref{green2}/23/85,
  \ref{green3}/84/84,
  \ref{greenTop}/90/24,
  \ref{underGreenTower}/01/35,
  \ref{underWatcher}/44/02,
  \ref{underPriests}/50/02,
  \ref{underDoors}/58/02,
  \ref{underSpiral}/68/07,
  \ref{Abyss}/65/00,
}

A Temple of Ohta once stood here, but was destroyed with the rest of \gls{lostcity}.
Now \gls{townmaster} has sent masons (members of the Woodspy Bandits), to build a base of operations for him to begin rebuilding the old human city.

However, everyone building here is unaware that the lower parts of the temple are still active.
While \gls{lostcity} still stood, the priests of Ohta wanted to make their own trade-routes with the nura realm, but eventually fell to war, just like everyone else.
The priests of Qualm\"e (who were close with the temple's priests) volunteered to remain, and suppress the horde which came up from below.

The priests still fight -- every few years the nura make some attempt to journey up through the eternally-open portal, and the golden priests of Qualm\"e push them back, to protect a civilization they have no idea has fallen.

\mapentry[greenGrounds]{Outer Grounds}

The area around the tower contains piles of rock which labourers have collected from the surrounding ruins.

\paragraph{If anyone in the party casts spells,}
they noticed immediately that there is a magical hum of energy in the area.
Everyone here regenerates 3 \glspl{mp} per \gls{interval}.

The source is a mana lake, slightly underground.
Any spell-caster with depleted mana can easily locate where the energy is strongest by simply wandering about, and eventually identifying a large rock which blocks the path underground (and has done, for decades).
The rock has a Weight Rating of 8, so a total Strength of +4 is required to lift it.%
\footnote{Of course another option is a lot of time, digging, or the proper use of a lever with an Intelligence + Crafts roll (TN 8).}

\humanfarmer[\npc{\T[6]\M\Hu}{6 Masons}]

\mapentry[greenConservatory]{Conservatory}

Overnight, the labouring equipment rests here.
The lock is a simple knot tied on the inside, and anyone slipping a knife inside can get in (Intelligence + Larceny, TN 5).

\mapentry[green1]{First Floor}

Basic straw-stuffed beds, clothes, and some bows lie here.

\mapentry[green2]{Second Floor}

The men sleep here, though it's eventually planned as a station for lower-level archers.

\mapentry[green3]{Third Floor}

The top floor provides a place for \gls{traitor}, overseer of the operation, to get a good look at the surrounding area.
This is also where he keeps a stockpile of weapons:

\begin{itemize}

  \item{20 longbows}
  \item{50 longswords}
  \item{50 suits of partial leather armour}

\end{itemize}

\mapentry[greenTop]{Top Floor}

This green roof meshes with the local trees for some distance around, but like any other camouflage, it seems obvious once you look at it.

From the top of the roof, anyone can see for miles around.
They can't see people or creatures sneaking about, but they can see basilisks and groups of men without fail, even by moonlight.

\mapPic{b}{Dyson_Logos/green_secret}{
  \rotatebox{15}{\nameref{underGreenTower}}/09/55,
  \ref{underGreenTower}/17/43,
  \rotatebox{-15}{\nameref{underWatcher}}/09/02,
  \ref{underWatcher}/18/09,
  \rotatebox{15}{\nameref{underPriests}}/20/78,
  \ref{underPriests}/28/57,
  \rotatebox{0}{\nameref{underDoors}}/40/02,
  \ref{underDoors}/375/31,
  \rotatebox{-15}{\nameref{underSideRoom}}/68/85,
  \ref{underSideRoom}/56/73,
  \rotatebox{-90}{\nameref{underSpiral}}/90/43,
  \ref{underSpiral}/81/40,
  \rotatebox{25}{\nameref{Abyss}}/67/49,
  \ref{Abyss}/68/36,
}

\mapentry[underGreenTower]{The Stairs}

\textbf{Background:}
Years of growth and soil-spillage have left a thin layer of mucus on the stairs.

\paragraph{Anyone descending}
must make a Dexterity + Athletics roll, TN 8, or fall down one staircase, taking $1D6-2$ Damage.
Each of the three staircases require a different roll.
Characters can get a bonus for proper equipment, such as rope.

The air down in the tomb has become so dry and foetid, that anyone spending time there gains 2 \glspl{fatigue} per \gls{interval}.
Of course, this can be offset with sufficient rest (or just leaving the place to air out for a few days).

The stairs and all hallways here are rather narrow, making weapons difficult to swing.%
\footnote{%
  \iftoggle{core}%
  {\nameref{enclosedcombat}.}%
  {See the Core rules on Enclosed Spaces.}%
}

\mapentry[underWatcher]{The Watcher}

\textbf{Background:}
The tomb is guarded by a Pious Ghast who stands in a side-cupboard, sworn to guard the golden priests.
He has spent a long time down here in the dry air, and his body so seized up that he cannot move.
A little time, however, will allow him to regain the use of his limbs, and open the door.

\paragraph{Opening the door}
requires a Strength + Crafts roll at TN 14, as the door is bolted shut from the inside, and will have to be busted in.

\paragraph{If a player insists on `picking the lock'}
(despite the lack of lock) they roll Intelligence + Larceny, TN 16.

\begin{boxtext}

  You swing the door open to find a highly decorated corpse with a pendant to Qualm\"e, made of gold, with bone strung along the copper thread.
  The clothes look like they were once silk, and the helmet's leather covering seems to be a human face, stretched out and tanned.

\end{boxtext}

\paragraph{If the PCs stab the corpse,}
they can kill it.

\paragraph{If they pass the door,}
the Pious Ghast animates and comes to kill them.
He is intelligent enough to know to sneak.
His use of a shortsword also means he will not gain any penalty from the narrow hallway.

If the characters enter with the Torpor spell cast, or rings of asphyxiation, they will be able to pass the Pious Ghast and the Golden Priests invisibly, so long as they do not linger too close to any of them.

\npc{\M\D}{The Pious Ghast}

\person{3}% STRENGTH
  {2}% DEXTERITY
  {0}% SPEED
  {{0}% INTELLIGENCE
  {0}% WITS
  {-5}}% CHARISMA
  {2}% DR
  {3}% COMBAT
  {Academics~1, Stealth~2, Tactics~2, Vigilance~1}% SKILLS
  {\shortsword, \completeplate, 200sp worth of jewellery}% EQUIPMENT
  {}

\mapentry[underPriests]{Hall of the Golden Priests}

\textbf{Background:}
A long time ago, onlookers came here to gawk at the splendour of a glorious afterlife.
The priests of Qualm\"{e} who gained the highest honours of the temple would remain here to guard it forever.
Each is decked in golden jewellery.

\begin{boxtext}

  Five dead men, mummified and covered in golden jewels, stand in each of five enclaves at the side of the room.
  You notice head wounds and missing limbs upon some of the bodies.

  Little spears litter the area, as if a battle has taken place here.

\end{boxtext}

\boxPair[t]{
  \demilich[\npc{\D\F}{Demilich Delilah}]
}{
  \demilich[\npc{\D\M}{Demilich Jonah}]
}

\demilich[\npc{\D\F}{Demilich Tamar}]

\paragraph{Once the PCs enter,}
the demiliches cast hostile spells at them.

\mapentry[underDoors]{Exposed Doors}

\textbf{Background:}
These openings once had doors to hide the secret items of the Temple of Ohta, but the doors have since shattered and broken from centuries of warfare.
Primitive spears litter the area.

\mapentry[underSideRoom]{Side Room}

\textbf{Background:}
The natural tunnel was further excavated in order to horde the priests' treasures.
At present, this room contains \lootMedium, \lootBig, and \lootMagic, all held in a small chest at the side of the room.%
\iftoggle{aif}{%
  \footnote{The coins date the room to the year of Rex Hunter.
  See \textit{Fenestra} \autopageref{r_hunter}.}
}{}

The room also contains 14 undead hobgoblins, standing ready to kill any nura who come up from the tunnel, or just anyone who enters.

\undeadhobgoblin[\npc{\T[14]\N\D}{14 Undead Hobgoblins}]

\mapentry[underSpiral]{Downward Spiral}

\paragraph{If the troupe venture down the hallway,}
they find four more undead hobgoblins in the tunnel.
Their job is to open the door and jump on any nura climbing up, but they will attack anyone on sight.

\undeadhobgoblin[\npc{\T\N\D}{4 Undead Hobgoblins}]

\begin{exampletext}

  Throughout the long years, the golden priests directed the stupid undead to dig and dig into the hillside.
  They looped round, and dug into the tunnel, then fashioned a door from the broken wood from previous battles.

Any nura digging upwards can now be attacked by the undead, who hurl themselves down to attach anything climbing up, before one of the golden priests shut the door again.
While several nura would normally burst through any door, opening a stuck door while climbing can be almost impossible.

\end{exampletext}

\mapentry[Abyss]{Abyss}

Characters entering this area feel a warm breeze coming from the abyss, and intense magical energies.
The abyss itself is a third-level mana lake, constantly radiating magical energy.%
\iftoggle{aif}{%
  \footnote{See \autopageref{mana_lake} for more on mana lakes.}
}{}

\paragraph{If they somehow get down the hole,}
they will find a skull embedded in the wall -- the head of a devout warrior.
His head -- carved with the story of his victories -- functions as a permanently-open magical doorway.

%! Removed The Portal Summoner

\paragraph{Destroying the Portal Summoner}
is easy if one can get close enough.
However, getting close to it will almost certainly alert the nura below, in their citadel.
At this point, heavily armoured hobgoblins will attempt to climb up to stop the vandalism.

\hobgoblin

\begin{exampletext}

  The priests of Ohta who opened the door to the Realm of Darkness and Fire%
  \iftoggle{aif}{%
    \footnote{See page \pageref{darknessandfire} for more on that realm.}
  }{}%
 were not stupid.
 They created the door through a magical artefact and placed that artefact down a deep chasm, so they could better defend their location in case of attack.
  What they had not counted on was the sheer number of nura who could climb up from below.
  They clamoured up in their hundreds, with spiked gloves to climb more easily up the soft cavernous walls.

  When the nura turned against the city, the two portals spewed hobgoblins, ogres, and the like, at the same time.
  The priests argued, and considered caving the entire temple in.
  They prayed to Ohta, and saw a vision of the temple's destruction, with a hole which sucked in more and more earth, getting wider and wider, until nura filled the land.

  At this point, the priests of Qualm\"e decided to take over, and remain, guarding the temple forever.
  They had their servants wrap them in cloth while warriors fought against the nura.
  They each took vows to never eat, open their eyes, or listen to music again.
  All five became demiliches, sworn to serve their temple forever.

  As the nura came up, the golden priests killed them with magic, then raised the bodies from the dead to fight against more nura.
  This has left the situation in a stalemate.
  Nura come up from the tunnel, they die, the golden priests raise them as ghouls to fight until all nura lie dead.
  This repeats every few years.

  So far, two of the golden priests have met their final end.
  Their remains have been placed back in their chambers by the others.

  Throughout these centuries, barely a dozen goblins have managed to escape and get into the lands above, and almost all died of starvation before finding anywhere above.

\end{exampletext}

\paragraph{If the troupe attempt to understand this area,}
they will no doubt find it confusing -- as well they should.
They will not find communication with the gold priests easy.
However, the demiliches are willing to communicate, and one can send telepathic thoughts with the Enchantment sphere.

\paragraph{If two of the golden priests die,}
the nura kill the other and break out.
Raise the local nura rating by 3.


