\section[Shadow Vault]{The Shadow Vault}
\label{shadowVault}

\begin{multicols}{2}

\histEvent{350}{11}{The Vore Gate opens in Sixshadow}

\begin{exampletext}
  \noindent
  Sixshadow began as an unremarkable settlement, but with a research outpost for alchemists.
  The small \glspl{village} sustained the alchemists' tower, and the tower let them keep all the dangerous things far away, and over the Sixshadow mountains.

  Dangerous \glspl{artefact} stayed underground, in `the Shadow Vault', and dangerous alchemists stayed in the tower, away from any large populations.

  The greatest and maddest of the alchemists at the time created a long-range spell to detect powerful mana flows, deep underground, and then created a gateway -- a tear in space -- which allowed him to step through, and explore the tunnels of the \gls{deep}, in order to find powerful \glspl{ingredient}.
  Unfortunately, the gateway opened to nothing but dark, rocky, tunnels, and those who explored a little way in never returned.

  As a by-product of the spell, the gateway itself became sentient (meaning it was \pgls{artefact}).
  And like most \glspl{artefact}, it became obsessed with the one task which the alchemist created it to do -- to have people enter through it.
  So the \gls{artefact} gateway would open again, whenever it had the energy, in the hopes of tempting someone inside.
  The alchemists named it, `the Vore Gate'.
\end{exampletext}

\subsection[Downstairs]{\glssymbol{yonder}~Downstairs~\glssymbol{abderian}}

The air smells of dampness and coagulated mana.
In the darkness, the characters may year \gls{yonder}'s bell ring below.

\mapentry[shadowStairs]{Muddy Steps}

\begin{exampletext}
  Water and mud have slid down the stairs of the Shadow Vault for the last couple of centuries.
  Over most of \pgls{cycle}, the temple has deep water.
  However, during the cold seasons, the water freezes over.
\end{exampletext}

Traversing the stairs demands a \roll{Dexterity}{Caving} roll (\tn[8]).
Failure inflicts $1D3$ Damage.

\mapentry[entranceCullis]{Gnome-Crafted Fence}

\histEvent{340}{01}{Six \glsentrytext{deep} gnomes emerge from the Vore Gate}

\begin{exampletext}
  The alchemists build a stone wall around the Vore Gate, with a cast-iron fence, under lock-and-key, to make sure nothing dangerous would come out.
  What actually emerged were half a dozen young gnomes with large, bulbous eyes.

  They stayed in the fenced-off area for days, starving, until the alchemists took them in out of a mixture of pity and curiosity.
  They remained at the Shadow Vault, sweeping up, keeping extensive records, and soon learnt to speak the common language which they referred to as `Up'.

  The six gnomes of Sixshadow had persistent criticism of the metal fence in front of the Vore Gate, and soon created their own from a special steel mix based on charcoal derived from \glsentrytext{digger} infants.
\end{exampletext}

The metal fence looks black in the darkness, but with enough light a subtle blue tint gleams.
Destroying it requires 20 Damage (in a single hit), or someone can try to pick the lock with an \roll{Intelligence}{Larceny} roll at \tn[16].

\shadowVaultMap

\mapentry[shadowVore]{Vore Gate}

\begin{exampletext}
  A young alchemist descended into the Shadow Vault to cast inquisitive spells through the Vore Gate.
  However, the moment he unlocked the gate, something assaulted him, removing his fingernails then face before bolting into the night.

  The alchemists concluded that the Vore Gate had let the creature in earlier, but nobody could see it by looking through the gate for danger.
  Since then, the alchemists had two peep-holes commissioned at either side of the Vore Gate's enclosure, so they could check if anything hid within the enclosure before unlocking the gate.
\end{exampletext}

\begin{boxtext}
  Past the ornate gate, you see a room, and in the far darkness, two ivory pillars frame a dark glimmer\ldots of some description.
\end{boxtext}

If the \glspl{pc} manage to speak with the Vore Gate,%
\footnote{Perhaps with a \textit{Delicate Audience} spell.}
it tells them enthusiastically, and emphatically, that they have to see the other side.
It may lie, or entice them with mystery, but most of all it says how much it loves having people entering.

\setcounter{statDots}{0}
Last night, a \superWierdzi\ emerged from the Vore Gate, and it still wants to escape the enclosure.
The \superWierdzi\ looks like a sickly-yellow mound of Van-Gogh playdough, but whenever someone looks at it, the \superWierdzi\ begins to look a little like them.

\mphlg

The \superWierdzi\ can reduce any Body Attribute by up to 3 in order to immitate someone's appearance, but cannot make any noises.

The acidic oozes will not react to the \superWierdzi\ in any way.

\paragraph{If the troupe look into the Vore Gate enclosure,}
the \superWierdzi\ tries to hide.
If it cannot (because all peep-holes have someone looking through them), it stares back through a peep-hole, absorbing \pgls{pc}'s face.

\paragraph{If it escapes,}
the \superWierdzi\ flees, then stops steps away, and stares at the troupe, imitating each of them in turn.
Ranged weapons will make it flee, and then hide outside, waiting to stare at the troupe again.

\paragraph{Letting it observe for 3 rounds}
allows it to duplicate someone's full bodily appearance exactly, at which point it attempts to return through the Vore Gate (but may wander outside for a few \glspl{cycle} if the Vore Gate is not available).

\artefact[16]{Vore Gate}% Name
  {Two ivory pillars, with a beckoning empty space between}% Body
  {1}% Intelligence
  {0}% Wits
  {0}% Charisma
  {to get people into itself}% Mission
  {Fae Door}% Base Spell
  {
    \setcounter{Fire}{3}
    \setcounter{Earth}{2}
    \setcounter{Fate}{1}
    \setcounter{Water}{1}
    \setcounter{Academics}{2}
    \setcounter{Cultivation}{1}
  }% Spheres

\showStdSpells[
  \setcounter{diceNo}{0}
]

\mapentry[shadowPot]{Sticky Pot}

\begin{exampletext}
  After one too many expensive accidents, the alchemists decided to shut down the Shadow Vault, so the \glspl{server} of the \gls{templeOfPoison} bought the building, and the basement.
  They also inherited all of the \glspl{artefact} which were not considered dangerous, or easy to move.

  Within \pgls{cycle}, the Shadow Vault began to see rich visitors, coming just to observe the strange \glspl{artefact}.
  They didn't have any magical defences, so they bought some safety precautions from the alchemists.
\end{exampletext}

The only alchemical safety device remaining is a ceramic pot, with an acidic ooze placed inside, under a \textit{Preservation} spell.
At the bottom of the pot rests a pile of 30~\glspl{gp} (clearly visible).

The ooze which inhabits the deeper water \vpageref{shadowJelly} occasionally comes out, and leaves acidic mucus all over the pot, presumably sensing the contents.

Inside, rests a tiny little ooze-creature, under a \textit{Preservation} spell, ensuring it will not age so long as it remains still.
The temple \glspl{server} placed it here `in case of emergency', with a vague plan to arm the establishment's guard in case anything large entered.

%\swarm[\gloopy]{Ooze-in-a-Jar}% Name
%{6}% HP
%{3}% Attack
%{3}% Speed
%{2}% Wits

\jelly[1]

\mapentry[shadowOaths]{Lady of Oaths}

\begin{exampletext}
  `Wouldn't it be great if people could just keep their promises?', is the kind of thing that people say when they don't think about the consequences.
  One alchemist, who thought he was being clever, created \pgls{artefact} which would listen to people's oaths, promises, or intentions, and then bind their mind to that sole task, until they complete it.
  The \gls{artefact} still thinks people should keep their word, but like any \gls{artefact}, has no concept of context or priorities.

  So whenever anyone states their intentions next to her, she casts an \textit{Oath} spell, ensuring that they will keep their word.
  If they try, but struggle, she may well aid them with more spells, but if they try to avoid doing what they said, she turns to unfriendly spells.

  The creator abandoned the \gls{artefact} in frustration shortly after making it, then sold it to the Shadow Vault, where it stood as a visitors' attraction.
  People who were angry with their spouse would often drag their spouse along to swear an oath in front of the statue.
  Other tried to remain silent when passing it.
\end{exampletext}

\artefact[6]{Lady of Oaths}% Name
  {A statue of a brass woman with blue eyes, looking up and listening earnestly.
  The sign beneath says `tell me your intentions'}% Body
  {2}% Intelligence
  {0}% Wits
  {3}% Charisma
  {to ensure everyone finishes their stated intentions}% Mission
  {Doom Study}% Base Spell
  {
    \setcounter{Fate}{2}
    \setcounter{Water}{2}
    \setcounter{Empathy}{2}
    \setcounter{Cultivation}{1}
  }% Spheres

\showStdSpells[
  \setcounter{diceNo}{0}
]

\mapentry[watchfullOrb]{Watchful Orb}

\begin{exampletext}
  Having two alchemical gateways in the building means trouble, so the \gls{templeOfPoison} commissioned yet another \gls{artefact}.
  The Watchful Orb consists of an illusion of a rainbow bubble, with a \textit{Witness} spell layered on top so the illusion spell can react to nearby changes in spells of the Force Sphere.
  Learning the language of the orb takes a long time.

  Grabbing the orb hurts it, and it loses \pgls{mp} at every contact.
  All contact should be gentle.
\end{exampletext}

\begin{boxtext}
  Within the murky water, at the bottom, something glows.
\end{boxtext}

\artefact[1]{Watchful Orb}% Name
  {A rainbow bubble, slowly swirling and changing, always glowing}% Body
  {0}% Intelligence
  {-1}% Wits
  {2}% Charisma
  {to inform everyone of any Force spells, and how fun they are}% Mission
  {Find Gateway}% Base Spell
  {
    \setcounter{Air}{3}
    \setcounter{Fire}{3}
    \setcounter{Earth}{1}
    \setcounter{Academics}{2}
    \setcounter{Xenomology}{2}
  }% Spheres

\showStdSpells[
  \setcounter{enc}{2}
  \findGatewaySpell
]

\paragraph{If the \glspl{pc} feel about,}
they will find the Orb.
They can take it away, and it will continue to show the state of any Force spells in range.
It watches them with fascination, and refuses to do anything else.

Right now, it sits underwater, deprived of air and \glsentrylongpl{mp}.
And it desperately wants to know what is happening with the Shadow Gate (room \vref{shadowGate}).

\mapentry[shadowJelly]{Uncommon Ooze}

\histEvent{300}{01}{The six gnomes who work at the Shadow Vault build an alchemical gateway to Archwarp called `the Shadow Gate', and instruct it to open for anyone who can answer its riddles}

\begin{exampletext}
  The servants of the \gls{wheatGuild} did not receive many customers at the Shadow Vault, despite their marvellous attractions, because Sixshadow sat out of the way, with a long walk to its South, and the six mountains to its North.
  The six gnomes who worked there proposed a plan -- they required some \glspl{ingredient}, and would create another magical gateway.
  Unlike the Vore Gate, this one would transport things to a human settlement -- Archwarp.

  Once the gnomes completed the `Shadow Gate', people could stand in the sand-pits at either side, and the Shadow Gate would arrange the sand to write a riddle.
  The requester then wrote the answer in the sand, and the gateway would open on a correct answer.

  The \gls{wheatGuild} did not feel terribly pleased about the situation, but the gnomes now had a source of permanent employment as `riddle-speakers', as the Shadow Gate only knew the riddles they had taught it.
\end{exampletext}

An acidic ooze sits down here, minding its own business.
Occasionally \pgls{woodspy} or \gls{basilisk} enters to look around or hibernate, and the ooze begins to eat it.
And sometimes that creature escapes, and sometimes it does not.
The ooze has existed here for a long time, and has grown to the size of a room.

\uncommonlyLargeJelly

The \glspl{pc} may find the ooze a serious problem.
It's deadly, humongous, hungry, and fearless.
But like any ooze, it's stupid, so the \glspl{pc} can lead it out of the temple, and the only way it can find its way back is to feel its own slime-trail.

\boxPair[b]{
 \humanthief[\NPC{\M\Hu}{Goutscrape}{morose}{`Oh~well' \ldots \textit{sigh\ldots}}{to confirm every disappointment}]
  }{
  \humanmaid[\NPC{\F\Hu}{Fleasinge}{tubby, and smiley}{giggles}{to return to a lost home}]
  }

\paragraph{During cold seasons,}
the ooze rests under the ice.

\paragraph{If the large ooze makes contact with the smaller ooze,}
(area \vref{shadowJelly})
they envelop each other, then mate for \pgls{interval}, then the larger one explodes into six copies of the smaller ooze.

Disturbing the pair during this time results in  both attacking at once.

\mapentry[shadowStaff]{Staff Door}

\begin{exampletext}
  The gnomes who worked here restricted access to one direction, so the staff could port people though the attractions, one at a time.
  They began with a one-way door, and then started work on a `janitor's closet'.

  The gnomes had always resented the \gls{wheatGuild} holding the only keys to the Vore Gate's enclosure.
  They did not want to return to their home in the \gls{deep}, but also didn't want anyone else making that decision for them.
  So with some slow, quiet work, over \pgls{cycle}, they created a passage from their little `janitor's closet' to the Vore Gate enclosure.
\end{exampletext}

This hallway remains dry.
The door only opens towards the stairs leading out, never the other way.
Pulling it out requires a \roll{Dexterity}{Crafts} roll (\tn[11]).

A little hole in the wall leads into the little Gnomish room, where they kept brushes and ledgers.
A black cloth covers the entrance, and in the darkness it simply looks like the back of the alcove.
A small, painted, vase completes the illusion, by explaining why the alcove exists.
The lines on the vase, look decorative, but they show the relative locations of the tunnels through the Vore Gate.
However, nobody knows its value, beyond its obvious beauty.

\mapentry[janitorCloset]{Janitor's Closet}

Inside this little room sit various cleaning-clothes, brushes, and records of income.
A single wall has wood over the front, leading to room \vref{shadowVore}.

\mapentry[shadowGate]{Shadow Gate}

\histEvent{270}{4}{%
  The \glsentrytext{warden} of \glsentrytext{town} asks a gnome who works at the Shadow Vault, to destroy the Shadow Gate.
  Instead, he ask the Shadow Gate an impossible riddle, and the Shadow Gate stops working until it has solved the riddle.
  In return, the \gls{warden} allows the \gls{deep} gnomes to open a warren, just outside of \glsentrytext{town}}

\begin{exampletext}
  While the Shadow Gate ferried people from Sixshadow to Archwarp, it competed with \glsentrytext{town} for transportation access.
  \Glsentrytext{town}'s \glsentrytext{warden} hated having to lower his prices to compete, so he made a deal with the gnomes who worked at the Shadow Vault.
  He would let them build a warren right next to \glsentrytext{town}, and in return, they would break the Shadow Gate.

  Gnomes, as a general rule, dislike `breaking' things, especially \glsfmtplural{artefact}, but they found a way to halt the gateway.
  One night, after the humans had gone to bed, one of the gnomes approached the Shadow Gate, and asked for the entry riddle.
  But instead of entering, the gnome asked for another riddle to solve.
  After solving many more, he began to insult the Shadow Gate, saying its riddles were too easy, and it should `make a challenge, not small-talk'.

  So the gnome offered it a new riddle -- something that would create a real challenge\ldots
\end{exampletext}

The entrance to the Shadow Gate stands about as tall as a human, so seeing the passage will be a challenge if the troupe get here by swimming atop the slimy water, and over the cold seasons, the troupe have even bigger problems -- they won't even see the entrance to the Shadow Gate.

The \glspl{pc} can roll \roll{Strength}{Crafts} (\tn[10]) to smash through this ice, which of course releases the ooze.

Two staircases, on either side of the causeway lead to two raised platforms, still blessedly dry.
Each has little walls to contain all the sand.
When the Shadow Gate wishes to speak, it uses a \textit{Warp Detailed Earth} spell to solidify and soften different parts of the sand, which spells out a riddle.

The Shadow Gate's last riddle remains on the floor, but this riddle has multiple stages, so once someone begins to answer the riddle, the sand will start to shift about.

\artefact[16]{Shadow Gate}% Name
  {A massive stone doorway to nothing, with a doormat placed in front}% Body
  {3}% Intelligence
  {0}% Wits
  {1}% Charisma
  {to ask devious riddles}% Mission
  {Fae Door}% Base Spell
  {
    \setcounter{Fire}{2}
    \setcounter{Earth}{3}
    \setcounter{Water}{1}
    \setcounter{Academics}{2}
    \setcounter{Caving}{1}
    \setcounter{Xenomology}{1}
  }% Spheres

\showStdSpells[
  \setcounter{diceNo}{0}
]

The Shadow Gate then proceeds to ask its devious riddle.%
\footnote{The real challenge here is to find the gnome who created this riddle. Players cannot solve it, unless you assume they can't solve it, in which case, they will.}

\hardestRiddleEver

\mapentry[shadowDesk]{Empty Desks}

\begin{exampletext}
  This room once had a steep stone staircase up to it.
  But with the guardians (\vpageref[above]{shadowGuardian}) sleeping peacefully, the proprietor didn't want leave them unprotected (even with the gate present).
  So he ordered men to build a wall over the entrance, leaving a gap at the top so they could squeeze out.
\end{exampletext}

The entrance stands at almost double the height of a human, and only wide enough to crawl through, so not many will spot it.
(and anyone with a Strength of +4 will have to remove their armour to fit through)

With \roll{Wits}{Crafts} (\tn[14]) they might notice the bricks are a little different around that wall.
With \roll{Wits}{Vigilance} (\tn[12]) they see the small entrance, above.
Once the \glspl{pc} pass by the entrance, they can make a single roll for both checks.

\paragraph{Rifling through the desk}
uncovers records of riddles\ldots hundreds of riddles.

\mapentry[memoryOrb]{Nuisance Orb}

\begin{exampletext}
  The Nuisance Orb once sat on display for everyone to see, as it would entertain people with illusions.
  But after one too many accidents%
  \footnote{One rampaging \gls{basilisk}, a false accusation of theft, and three accounts of indecent nudity.}
  the staff became irritated by it, and eventually locked it away.
\end{exampletext}

The Nuisance Rock cannot imitate people when locked in the little cell, and craves freedom more than anything.
It can spend 2~\glspl{mp} to `see' around itself, 2~\glspl{mp} to make a convincing illusion of something, or 2~\glspl{mp} to create convincing sound.

\artefact[2]{Nuisance Rock}% Name
  {A grey rock, covered by the illusion of a massive diamond.}% Body
  {-1}% Intelligence
  {1}% Wits
  {2}% Charisma
  {to create the most convincing illusions}% Mission
  {Phantasm}% Base Spell
  {
    \setcounter{Fire}{2}
    \setcounter{Earth}{2}
    \setcounter{Air}{2}
    \setcounter{Academics}{2}
    \setcounter{Wyldcrafting}{1}
  }% Spheres

\showStdSpells[
  \setcounter{diceNo}{0}
]

\paragraph{If anyone passes by,}
it creates an illusion of Goutscrape (area \vref{shadowGuardian}), begging to be let out of the cell.
The illusion will create realistic sound, and then the Nuisance Orb will have no further \glsentrylongpl{mp} to see anything, so it will make the illusion of Goutscrape lie down and sob inconsolably.

\Glspl{pc} can see through the illusion standing in the darkness with a \roll{Wits}{Vigilance} roll
(\tn[12]).

The Rock will do anything it can to leave this room, although its abilities remain limited (it must wait 2 more \glspl{interval} before it can cast another illusion, or make a voice).

\paragraph{If the \glspl{pc} take it with them,}
the Rock will cause constant chaos and misunderstandings.
If it ever speaks, it does so by making an illusion of someone giving advice about what to do with the Rock.

\mapentry[shadowGuardian]{Sleeping Guardians}

\begin{exampletext}
  When the magical gateway shut, the local \gls{warden} tasked \pgls{doula} to resolve the situation.
  Like anyone in her position, she told an enchanting story about success, then delegated the task.

  Finding three people too young to know better, she arranged for some fancy clothes, and declared them `Masters of the Riddle', and `Keepers of the Gate'.
  Then she sent them into an enchanted state, preserving their bodies, and setting their minds to focus, without failure, on the single task of solving the riddle which had vexed all of \gls{valley}.
  And `within a year or so', she claimed, one of them would solve the riddle.

  The first thought about how much he hated Maths.
  The second thought about visualizing the riddle as a series of rivers, each one representing a possible set of questions and answers.
  The third had already gone to sleep to dream of sex with his attractive neighbour.

  With everyone's attention focussed on the `Masters of the Riddle', the \gls{doula} went home, and died peacefully some years later.

  The three rest in this hidden room to this day, still focussed on the same thoughts.
\end{exampletext}

\paragraph{Picking the lock}
of the rusted steel gate requires a \roll{Strength}{Larceny} roll (\tn[16]).

\paragraph{If the \glspl{pc} awaken any of the sleepers,}
that one wakes the others up before doing anything else, then they unlock the fence in front of their room.

The three `Keepers of the Gate' can explain everything concerning the reality of \gls{lostcity}, how the Shadow Gate once worked, and what happened, but they won't\ldots at least not for a while.
They take a long time to come to terms with their situation.

\Person{\NPC{\M\Hu}{Embersnatch}{lost in the clouds, still thinking about the last thing said}{\textit{`ya know, like?'}}{to return to a lost home}}%
  {{1}{0}{0}}% BODY
  {{-3}{-2}{0}}% MIND
  {%
    \set{Melee}{1}
    \set{Empathy}{2}
    \Dagger
  }% SKILLS
  {}% KNACKS
  {\lootSmall}% EQUIPMENT
  {}% ABILITIES

\null
\begin{speechtext}
  ``What's Ratpelt doing these days?''

  \textit{\raggedleft\adforn{15} She's dead.}

  ``Her mum won't be happy about that''

  \textit{\raggedleft\adforn{15} Her mum's dead.}

  ``Blimey. Her too?''

  \textit{\raggedleft\adforn{15} It's been a \underline{hundred} \glspl{cycle}.}

  ``Do I get birthday presents?
  My mum always gets me a whole chicken for my birthday, ya know?''
\end{speechtext}

The \glspl{pc} have a serious decision ahead: the three will need food, and to return to civilization.

\end{multicols}

