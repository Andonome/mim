\thread[town,Roads]{Rising Titles}
\label{risingTitles}

\noindent
The \gls{whiteBandits} and \glspl{diggers} finally start pushing their plans to destabilise the area.
And how these \glspl{segment} play out, depends heavily on the \glspl{pc} political positions.
They may support some of these actions, or push back against them.
They may attempt to be `apolotical', and just get on with their jobs as the carnage rolls out.

\segment[\gls{vlg}]{Roads}% AREA
{Dry Inkwells}% NAME
{\Glsfmtplural{guard} defect to the \glsfmtplural{whiteBandits}, leaving the \glsentrytext{broch} with a dead \glsentrytext{jotter}}% SUMMARY

\begin{exampletext}
  The \gls{whiteBandits} have finally pulled off their plan to have every \gls{jotter} in \gls{valley} killed in a single night.
  Each \gls{jotter} has records of where all the \glspl{guard} go, including other \glspl{jotter}.
  \Gls{traitor} compiled the information, and made a plan.
\end{exampletext}

The troupe find \pgls{broch} with a murdered \gls{jotter}, and with some ingenuity may figure out what happened.

\begin{enumerate}
  \item
  Half a dozen bandits arrived in secret
  \hint{footprints throughout the forest show they went offroad just before the \gls{broch}, so nobody would see their approach}
  \item
  They shot at the \gls{jotter} through a window
  \hint{arrows still stand around the shutter}
  \item
  and wounded the \gls{jotter}
  \hint{blood spatters mark her desk}
  \item
  then fought with the \glspl{guard}
  \hint{blood and sword-marks remain around the tree border, just beyond the \gls{broch}}
  \item
  but ultimately, the \glspl{whiteBandits} made peace with the \glspl{guard}
  \hint{when one died, they burried him respectfully, digging a grave with some rocks on top, and leaving him money, to trick \gls{wrecan} into not taking him}%
  \footnote{\Gls{wrecan} normally takes the souls of those who die through hatred.  People sometimes try to make amends after death by leaving gifts with the dead, so they can show \gls{wrecan} their gifts.  Where the dead go after that, nobody really thinks about.}
  \item
  and so the \glspl{guard} left with the \glspl{whiteBandits}, becoming part of their group.
 \hint{A dozen tracks lead away, while the \gls{jotter} lies dead}
\end{enumerate}

The players may conclude any number of things from these events, but the tracks leading away should unambiguously tell them where complete answers lie.

\paragraph{If the troupe want to track down the culprits,}
they can roll \roll{Speed}{Survival} (\tn[8]).
Success indicates that they locate the bandits just before Sundown, while they go off-road to wash the blood from their clothes (so none are wearing armour).
A tie indicates that they must either travel overnight, or make another roll the next day at \tn[10].

\paragraph{Following the bandits back to their hide-out}
requires a \roll{Strength}{Stealth} roll, at \tn[10], +1 per day of travel.
See \vpageref{Dyson_Logos/bandit_camp} for details.

\paragraph{Confronting the bandits immediately}
results in a political conversation.
\Gls{sewerthief} and the \glspl{guard} who have just now turned to banditry don't like being forced to live outside of civilization, or how people see them.
They want better pay and working conditions, and they want to know why the \glspl{pc} don't want the same thing.

However, \gls{sewerthief} will only let them come back with the group to the \nameref{banditLair} if he thinks they're sincere about wanting a violent revolution, based on their earlier conversations.

\paragraph{Wherever the troupe go next,}
they will hear about the total absence of \glspl{jotter}, and will not receive any more orders or missions, unless they encounter some higher-ranking member of the \gls{guard}; but even then, the ranking member will almost certainly not have the time to keep track of what the \glspl{pc} get up to.

\segment{town}% AREA
{Guards, Guards!}% NAME
{With so many commanders dead, \glsentrytext{guard} vagabonds run amok in town}% SUMMARY

\begin{exampletext}
  The entire market has been chatting about the recent murder of all the \gls{guard} \glspl{jotter}, and how many \glspl{guard} have come to town, to drink and cause trouble.

  Three of them have recently taken to stealing at the market, though they don't have the subtlety to do anything but snatch-and-run.
\end{exampletext}

After the gossip, three of the \gls{guard} (who snuck into town by hiding in a wagon) dart past the troupe, followed by Mousebark of the \gls{weaversGuild}.
She's out of breath, so she stops to hollar at the troupe.

The three \gls{guard} will recognise the \glspl{pc}, and hide behind them, laughing like children, while Mousebark shouts at them all.

\humanmaid[\NPC{\F\Hu}{Mousebark}{young, red-faced, and angry}{stamps feed}{to retain her pride}]

\paragraph{If the \glspl{pc} do not manage to return the clothing,}
Mousebark informs the \gls{sunGuard}, who begin searching for the lot of them.

The group should roll \roll{Intelligence}{Stealth} (\tn[8]).

\paragraph{If the \glspl{pc} try to take the clothes,}
the knavish \glspl{guard} turn violent quickly.

\humanthief[\npc{\T[3]\M\F\Hu}{\arabic{noAppearing} Knaves}]

\segment{town}% AREA
{Underground Assassins}% NAME
{The bandits in the sewer cut \glsentrytext{captain}'s throat}% SUMMARY

\begin{exampletext}
  \Gls{captain} announced the previous day that he would begin harsh raids against the people living under \gls{town}.
  The \glspl{diggers} didn't like that, so they've organized an assassination.

  Five men walked casually behind him as he returned home from \gls{whitehorse} with his wife, then closed in so \gls{sewerthief} could stab him in the neck.
\end{exampletext}

If \gls{sewerthief} has died, replace him with \gls{sewerking}, or some other appropriate character.

\begin{boxtext}
  You hear guards shouting ``after them!'', in the distance, and quickly scurrying feet, as a woman shouts for someone to help her wounded husband.
\end{boxtext}

\paragraph{If the characters stay to help the wounded man,}
they find \gls{captain} next to his wife, blood spilling from his neck.
Saving him needs an \roll{Intelligence}{Medicine} roll (\tn[9]).

\paragraph{If they run after the thieves,}
the \glspl{pc} make a \roll{Speed}{Athletics} roll.

Remember that whoever's trying to patch up \gls{captain}'s bleeding neck won't be able to join the chase.

\begin{nametable}{Running Results}
  12 & \textit{``Giving chase, you catch up to four men running from the scene of the crime.''} \\
  11 & \textit{``You run round an alley, and see them entering \gls{town}'s underground.''} \\
  9 & \textit{``You run in hot pursuit, but the attackers have disappeared down a street, into thin air.''} \\
  7 & \textit{``The attackers sprint away, leaving you running in the dark.''} \\
\end{nametable}

\sewerthief

\boxPair{
  \newGhast[\npc{\T[7]\D}{\arabic{noAppearing} Ghasts}]
}{
  \newGhast[\npc{\T[7]\D}{\arabic{noAppearing} Ghasts}]
}

\humanthief[\npc{\T[2]\F\Hu}{\composeHumanName\ \& \composeHumanName}]

\paragraph{If the party follow the assassins underground,}
they run to the nearest entrance -- perhaps the \gls{doulaShop}
(area \vref{greyDoulaShop})
or \gls{pig} (\vpageref{sewers}).
Otherwise, this incident may remain a mystery.

\humanthief[\npc{\T[2]\M\Hu}{\composeHumanName\ \& \composeHumanName}]

\segment{town}% AREA
{Bandits Caught}% NAME
{The \glsentrytext{sunGuard} have captured \glsentrytext{banditking}}% SUMMARY

\label{banditsCaught}

The \glspl{sunGuard} have captured \gls{banditking}, along with four of the bandits.
Two once worked in the \gls{guard}, two more came from \gls{redfall}, and signed up with the \glspl{whiteBandits} to avoid starvation.

\begin{speechtext}
  Hear, ye!  Hear all!

  Bandits who roamed the highways, lead by a man known as `\gls{banditking}', have been apprehended.
  The leader shall be drawn and quartered by week's end, and his companions hanged that night.

  The \gls{templeOfPoison} invites any charismatic men or women to aid the festivities, as playwrites and attractive actors are required for the upcoming festivities (not you, \gls{forestpriest}!).
\end{speechtext}

Later this night, \gls{sewerking} will come to save him.
The \glspl{diggers} recently discovered an underground tunnel which leads from the \nameref{stationDungeon}'s waste-shaft (\vpageref{dunGrate}), to the \nameref{sewers} (area \vref{sewerWaterHall}).

\Gls{sewerking} and some of the \glspl{diggers} will wrap dry wood in cheese-cloth, swim up to the grate, and light a fire.
Once the \nameref{stationDungeon} fills with smoke, the \glspl{sunGuard} will have to evacuate, leaving the \glspl{diggers} a moment to jump up and start opening doors.

\paragraph{If the \glspl{diggers} still have the \gls{bskulls},}
it will cast \textit{Soul Specks} on them, allowing them to deal with the smoke.
Otherwise, \gls{forestpriest} will create \pgls{talisman} for them which creates a \textit{Bubble} spell, to save a lock-smith \gls{diggers} from the smoke.

\segment{town}% AREA
{The Dead Rise}% NAME
{\Glsfmttext{sewerking} releases ghouls upon \glsentrytext{citadel}, killing \glsentrytext{townmaster}}% SUMMARY

If the \glspl{pc} have already stolen the \gls{bskulls} or somehow disbanded the \gls{diggers} movement, nothing happens here.
All remains well in \gls{town}.

Otherwise, not a single soul in the city will sleep this night, as the \glspl{diggers} enact their plan.

\begin{exampletext}
  The \glspl{diggers} finally finished digging the tunnel underneath \gls{citadel}, where \gls{townmaster} lives.%
  \footnote{See \gls{area} \vref{citadelTunnel} for the underground tunnel, and \gls{town}'s \gls{area}~\vref{citadel} for \glsentrytext{citadel}.}
  They waited for the \gls{bskulls} to curse them with a touch of undead (so the ghasts would not attack them), then laid out wretched kindling for \gls{citadel} -- dry twigs, green branches, rotten meats, and bone.

  Then the \glspl{diggers} barred all the doors and hatches leading to their living space, and released the all-too-lively undead army, with only one route to take -- right into \gls{citadel}.

  With the fire at the base lit, smoke begins to flood \gls{citadel}.
\end{exampletext}

\begin{boxtext}
  One of the \gls{sunGuard} darts into an alley, and desperately tries to remove his uniform, then darts out, with a helmet and his hands covering his face.

  Shocked screams cry out, and hands point towards \gls{citadel}.
  A young hands from a window-ledge with only one hand; his other has a lantern.
  More hands come from the shadowy window, trying to grapple him back in and he throws the \gls{torch} and jumps to the ground but they already have his wrist.
  The dark hands pull the screaming man back into the dark citadel.
\end{boxtext}

\paragraph{If the troupe try to help,}
they find \gls{sunGuard} fleeing (and trying not to be seen) and about 100 ghasts in \gls{citadel}, devouring the face of anyone they encounter.

Move through the areas, rapidly.
Everywhere in \gls{citadel} remains dark unless a screaming, fleeing, servant is present.
A few of the \gls{sunGuard} remain, but most flee.

\humansoldier[\npc{\T[5]\E\Hu}{Palace Guard}]

\paragraph{If \gls{townmaster} dies,}
check the next part.
Otherwise, discard it.

\segment{town}% AREA
{Same Old New Management}% NAME
{The \glsentrytext{sunGuard} request the troupe clear the dead-infested \glsentrytext{citadel}, as the old order return}% SUMMARY

If \gls{townmaster} dies, his eldest surviving son, Constance, takes over leadership (with the help of \gls{captain}).
\index{Grey Family!Constance}

The \gls{sunGuard} barricade \gls{citadel}, and \gls{overseer} Cronblight%
\footnote{Find him \vpageref{cronblight}.}
at \gls{lochside} designates new \glspl{jotter}, who then order the \glspl{pc} to clear \gls{citadel} of the dead, along with five other \glspl{guard}.%
\footnote{The \gls{sunGuard} explain that, while the ghasts exist within \gls{citadel}, \gls{citadel} counts as `beyond the \gls{edge}', and is therefore not their jurisdiction.}

Twenty ghasts remain inside \gls{citadel}, while another thirty remain in the catacombs.
Both groups of ghasts will attack en mass, rather than splitting up.

