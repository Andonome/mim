\chapter{Protagonists \& Plots}
  \epigraph{He attacked everything in life with a mix of extraordinary genius and na\"ive incompetence, and it was often difficult to tell which was which.}{Douglas Adams}
\label{sideQuestIntro}

\noindent
Each day the \glspl{pc} will awaken to new plots unfolding.
Most days, they will hear rumours, on other days they will see the consequences of the plots.

The \glspl{pc} most likely work as members of the \glspl{guard}, but whatever they want to do, they will have their own tasks to attend to, and the little scenes here may seem like little distractions at first -- just a bar-fight here, and a tall story there.
But the plots of our various factions will soon begin to interfere with the lives of everyone in the \gls{valley} -- humans, elves, and gnomes alike.

Resolution will demand hunting down the right locations, but the locations will not make any demands on the players or \glspl{pc}.
They will remain free to ignore any `call to adventure', which will leave the various actors below free to resolve their plots.


\printglossary[
  type=people,
  style=mcolindex,
]

\label{Irina/greylands}
\mapPic[\large]{t}{Dyson_Logos/map_circles}{
  \rule[.3em]{.16\textwidth}{2pt}/86/04,
  \Large 10 miles/86/08,
  \N\normalsize\glsentrytext{redfall}/40/52,
  \N\nameref{lostcity}/32/96,
  \Glsentrytext{town}/54/67,
  \D\nameref{necromancers_lair}/13/22,
  \rotatebox{55}{\normalsize\nameref{lakeside}}/85/49,
  \rotatebox{10}{\normalsize crossroads}/70/67,
}

\section{Side Quest Summaries}
\label{sqSummaries}

\begin{multicols}{2}

\noindent
Each \gls{interval}, check which Side~Quests below are ready in the current area.%
\exRef{judgement}{Judgement}{sidequests}
Scan down the list below for any events which begin with the `\gls{sqr}` symbol, and once the scene plays out, mark the Side~Quest as done with an `\textit{X}', and mark the next scene as ready with a `\gls{sqr}' tick.

You might strictly run the next available Side~Quest, or smash two scenes together to create a chaotic event where multiple events pull at the \glspl{pc} attention.
Or, just pick whichever seems most appropriate for the moment.

Most of the threads in these various plots skips about, from one location to another.
The \glspl{pc} may hear about a rumour of bandits in town, find the bandits on the road, then stumble upon some clue in the outer forest.
Most of these events can plausibly occur at any point, so you can wait for the \glspl{pc} to enter the right location before running them.

\subsubsection{Cartography}

Take a look at the map \vpageref{Irina/greylands}.
Outside, a dense, primordial forest surrounds everything.
Travelling beyond here leads nowhere for several days.

Brave, little \glspl{village} protrude a little into the wilderness, and every few nights some new creature emerges from the forest to crawl over their rooves, and claw at the doors.

Some of the outer buildings only show a single tower.
These are the \glspl{bothy}, where traders rest overnight, and where the new \glspl{guard} often stay.
In fact, new \glspl{guard} may not enter towns or \glspl{village}, so they must remain here until they increase their rank.%
\exRef{judgement}{Judgement}{fodder}

A little further inside, quiet hamlets rest peacefully.
Any beasts which emerge from the forests would have wandered towards the noise of the outer \glspl{village} or \glspl{bothy}.
\iftoggle{judgement}{
  These little areas have a `civilization rating' of 7 to 12, depending on their proximity to town, so the \glspl{pc} will receive few dangerous encounters while staying there.
}{}

\end{multicols}

\foreach \x in {Roads, Town, Forest}{
  \center\subsection*{\x}
  \printcontents[\x]{l}{2}{\setcounter{tocdepth}{3}}
}

