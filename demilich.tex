\sidequest[Roads,Town]{The Necromancer's Friends}

\begin{exampletext}
  \noindent
  \Gls{necromancer} swore to protect his temple forever, but never said what he would protect it from.
  After he died, he continued his sole task while undead (as so many in the old temples of \gls{eldren} did back in the day).
  He always enjoyed bird-watching, which he could continue to do from his lonely tower.

  When the nearby \glspl{village} fell to ruin, he continued.
  When the people in his own town and temple left, he remained, staring quietly at the world, and saw his temple crumble around him.

  After time lost its meaning, bandits came to rest within his temple.
  He summoned a mist to suffocate them in their sleep, drained their souls, tore them apart, stuffed their bladders into their own skulls out of contempt, then stood to admire what he had created.

  The \gls{artefact} at once stood as an insult to the robbers, and a beautiful mark of change -- something \gls{necromancer} had not seen in a long time.
  He spent the energy, gained from the robbers, on a knot of spells.
  First, a Preservation spell ensured the item would not degrade.
  Then, he twisted the little preservation spell into a real mind.
  He spoke to the new creation, though it could not understand much of anything, yet.
  He indicated (silently, without moving his physical body) to observe the tiny flecks of souls which dart around the trees, and explained that these were birds.
  The \textit{Bladder Skull} could not speak, or move, but \gls{necromancer} could see some response in the simple, silent, language of the dead.

  Sometimes, \gls{necromancer} would teach the skull about magical Spheres, which it slowly learned, or teach it to detect the subtle, little souls of insects when they try to eat your feet.
  And since the Bladder Skull was not undead (despite its body of three bladders, inflated eternally, and stuffed inside three skulls), it could absorb the magical energies of the world.

  \Gls{necromancer} and the Skull Bladder Philosophized, and watched the birds together, and forgot that time existed.

  But time persisted, and one day brought two souls to the temple.
  The dead can see a sentient soul in the far distance, like a shining light in the darkness.
  In fact, they can see very little else.
  \Gls{necromancer} had not seen a living soul in such a long time that he felt a tiny touch of fear, but for someone who spends centuries watching the soul-specks of birds and bee-hives, the tiniest touch of fear felt like a horse galloping towards him.
  He moved to the top tower, and picked up his ivory bow (preserved with another spell).

  Drawing an arrow back, he saw the two lights fleeing his temple, and felt horror.
  Worse than his broken oath to protect the temple, centuries of solitude had given him a phobia of living people.
  And to make matters worse, those little souls had picked up the Bladder Skull!

  He shot an arrow, wounding one, then missed with several others.
  \Gls{necromancer} stood like only the dead can stand, and silently mourned his broken oath, and lost friend.

\end{exampletext}

\noindent
\Gls{necromancer} has a plan.
He wants an army of ghouls surrounding his temple, to ensure no more \emph{people} approach his temple.
He wants armour, to protect himself.
Since movement drains his energy, he will need a steady supply of unconscious people, so he can drain their souls (he feels brave enough to approach people when they're not awake).

Unfortunately, doing all this will make \gls{necromancer} paranoid about people coming to end his existence, which drives him to seek a larger army, and more living prisoners.

\subsubsection{Searching for \Glsentrytext{necromancer}'s Lair}
\label{huntingNecro}
The \glspl{pc} have a number of options, all of them bad.

\Gls{town}'s \gls{paperGuild}%
\footnote{See \autopageref{paperGuild}.}
has records of the original temple, but they could only research about him if they have something to go on.
A successful \roll{Intelligence}{Academics} roll (\tn[12]) reveals the map in the handouts, and reaching \tn[13] reveals the location.

Following the tracks will take a long time, although a horde of hundreds will be rather easy to follow.
The \glspl{pc} can roll \roll{Wits}{Wyldcrafting} at \tn[5], with +2 to the \gls{tn} each day thereafter, and +3 on any day it rains or snows.

\subsubsection{Returning the Bladder Skull}

This resolution is extremely unlikely, but it does indeed resolve the issue.
\Gls{necromancer} will take his pet back, and live peacefully, until his temple crumbles to nothing, and the local stirges eat his body.

\begin{boxtext}
  Crackling sticks indicate someone walks close by, and a moment later indicates a full procession walking somewhere close by.
  But you wait, and no voices come out -- only crackling sticks.
\end{boxtext}

\sqpart{Roads}% AREA
{\gls{vlg}~The Undead Horde}% NAME
{A botched spell leaves \glsentrytext{necromancer}'s ghouls wandering alone}% SUMMARY

\begin{exampletext}
  \Gls{necromancer}'s first attempt at raiding \pgls{village} did not go according to plan.
  His foul fog made everyone sick.
  Most of the \gls{village} scattered, and he entered to kill the few remaining inside, and consume their souls.
  He stuffed dead spirits inside their empty shells, to irse and consume the living.

  Each of the new ghouls fled in a different direction, chasing the living.
  He spent a little while trying to round them up, but they had gone beyond the range of his spells.

  Eventually, he gave up and returned to his temple, while the dead stood scattered, from river to hill.
\end{exampletext}

Run this part when the troupe approach \pgls{village} which \gls{necromancer} might travel to.
His temple is marked with a `\gls{D}' on the map (\vpageref{Irina/greylands}), and lies upstream of a number of \glspl{village} and hamlets.
He can comfortably travel downriver, since he has no need to breathe.%
\Gls{necromancer} will never return via roads -- instead, he uses the light which comes off \glspl{village} like sailors use starlight.
\footnote{He can also cast Air spells to float a little.}

\paragraph{Initially}
you should ease slowly into a mounting sense of dread.

\begin{boxtext}
  Half the houses give off weak smoke, but every door lies open, as if trying to show you the empty darkness inside.
  The only sound comes from upset sheep, pleading for food.

  Behind you, a single farmer has returned to the \gls{village}.
  He approaches you, with a sense of urgency\ldots
\end{boxtext}

\ghoul[\npc{\D}{Ghoulish Farmer}]

\paragraph{In part 2,}
the dead pile on bit by bit.

\ghoul[\npc{\T[3]\D}{More Ghouls}]

\ghoul[\npc{\T[7]\D}{Still More Ghouls}]

Over 100 ghouls head towards them, so keep adding a larger amount each time, until the \glspl{pc} run (or die, with bravery, honour, and through stupidity).

\paragraph{Once the troupe realize they need to flee}
they can add \pgls{fatigue}, and march 2 miles away, and will probably out-pace the ghouls.

Ask them `where to next?', and continue as if the scene has ended.
They may head back along the road, or try to go around the ghoulish \gls{village}, by taking a short detour through the forest.

Once the scene settles, have them roll \roll{Intelligence}{Stealth} (\tn[12]).
Success means that they understand how far the dead can see, and how to hide the light of their souls.%
\footnote{Thick clothing helps, as long as it covers everything. Hills and rivers also helps, as they block some of the light of a soul, just like any other light.}
Failure means they ghouls catch up, this time travelling together, as an army.

\paragraph{If they flee to \pgls{bothy},}
they will find safety, but risk starvation.
Roll encounters as usual -- the ghouls will attack any traders or woodspies which approach, but soon enough return to the \gls{bothy} unless the troupe manage to hide themselves well enough.

\paragraph{If they flee to \pgls{village},}
then the archers there can help pick off the dead, but they will run out of arrows before the ghouls all drop down\ldots immobile.

\paragraph{If they encounter a basilisk,}
then they have been saved!
(basilisks have no problem eating ghouls, and the ghouls will not find killing it an easy task)

\paragraph{Tracking down \gls{necromancer}}
presents a thousand challenges.
Have the players roll \roll{Wits}{Wyldcrafting} (\tn[14] or higher) if they figure out that someone started the massacre, and attempt to track them.

\paragraph{Whatever happens,}
you will have to think on your feet to interpret the clashing of plans with events, as will the players.

\sqpart{Town}% AREA
{\gls{vlg}~\squash Rumours of a Breach}% NAME
{\Glsentrytext{necromancer} destroys \pgls{village} at the \Glsentrytext{edge}}% SUMMARY

\Gls{necromancer} has destroyed yet another \gls{village}, and now has a horde of \arabic{sqNo}00 ghouls, along with a few ghasts.
\Glspl{guard} found the \gls{village} two days ago, and now all of \gls{town} is talking about it.

By this time, \gls{necromancer} has gathered an army of \arabic{sqNo}00 undead.

\sqpart{Roads}% AREA
{\gls{vlg}~The Survivors}% NAME
{A few farmers flee a massacre}% SUMMARY

\begin{exampletext}
  Last night, \gls{necromancer} attacked \pgls{village}, successfully created some ghouls, then enchanted them to follow him.
  \Gls{necromancer} noted that they carried a necklage with the symbol of \gls{eldren} -- `\gls{sickness}' -- and instructed the ghouls to spare them.
\end{exampletext}

Run this part only when somewhere close the \gls{necromancer}'s temple, or somewhere downriver, West of \gls{town}.
Remember to roll encounters as usual, and combine that encounter with the fleeing farmers.

\begin{boxtext}
  On the horizon, four humanoid silhouettes stumble forward silently.
  Once they see you, they start running towards you.
\end{boxtext}

The farmers arrive famished, and request all the food the troupe can spare.

One of the survivors -- Fangkrist -- has been affected by a \textit{Soul Specks} spell.
\Gls{necromancer} cast it on him in order to communicate, and continuously ask where his `beloved Bladder Skull' went.

Fangkrist will tell everyone what he saw, and ask how to break the curse, before explaining the strange things he can see.
The curse wears off by the end of the day, as long as he keeps moving.

\begin{speechtext}
  I can see your inner lights.
  They feel blinding.
  Especially \emph{yours}%
  \footnote{He directs this at whoever has the most \glspl{mp}, assuming anyone has any.}
  What did that creature in the darkness mean?
  He kept on saying it.

  ``My beloved Bladder Skull''

  ``Where did they take my Bladder Skull?''
\end{speechtext}

\paragraph{If the characters try to track the ghoulish army,}
they find only one challenge -- if they approach too close, the ghouls may turn, and attack.
An \roll{Intelligence}{Wyldcrafting} roll (\tn[10]) lets them keep the proper distance.

The \nameref{necromancers_lair} now hosts \arabic{sqNo}00 ghouls, a few ghasts, and an undead basilisk.%
\footnote{See \nameref{antBasilisk}, \autopageref{antBasilisk}.}
The farmers will not accompany the troupe when pursuing the undead horde.

Mark off another \gls{village} on the \gls{valley} map.

\ghoul[\npc{\T[7]\D}{200 Ghouls}]

\sqpart{Town}% AREA
{\squash Armourless}% NAME
{\Glsentrytext{necromancer} kills a trading wagon for their armour}% SUMMARY

\begin{exampletext}
  \Gls{necromancer}'s ghasts journeyed out on a mission -- to sieze armour for him.
  They found a trading caravan, with one wagon of armour from \gls{southDale} -- leather, chain, helmets, brigadine, the elf-stuff -- she had the lot.
\end{exampletext}

The prices of armour in all of \gls{valley} doubles after this event, for the next month.
That single wagon may not have carried all \emph{that much}, but when supply problems appear, people panic-buy.

\Gls{necromancer} now has a horde of \arabic{sqNo}00 ghouls, along with a few ghasts, and an undead, armour-plated basilisk.
If the troupe try to track him down, remember to check \vpageref{huntingNecro}.

\begin{figure*}[t]
\begin{nametable}{Dead Tactics}

  \textbf{Round} & \textbf{Event} \\\hline

  1 & 200 ghouls assault the gate. \\

  2 & 4 ghasts each attack a quarter of the \gls{village} each (one attempts to open the gate). \\

  3 & \Gls{necromancer} casts offensive spells on anyone at the gate. \\

  4 & The ghasts have formed a ghastly pyramid at the gate, and 5 climb over. \\

  5 & 10 more ghasts climb over the gate. \\

  6 & 15 more ghasts climb over the gate, and the \gls{village} falls. \\

\end{nametable}
\label{necroTactics}
\end{figure*}

\begin{boxtext}
  The \gls{village} falls quiet at night, except for shuffling feet as people try to be quiet going out to the toilet at night, or chattering about local town gossip.

  A man in the distance tells his child off harshly for going out at night, past the \gls{edge}, with his friends.
\end{boxtext}

\sqpart{Roads}% AREA
{The Dead Devour}% NAME
{The necromancer makes a full-on assault on \pgls{village}}% SUMMARY

\Gls{necromancer} has gathered an army of \arabic{sqNo}00 ghouls.
He takes only 200 ghouls and the 4 ghasts out for the raid, leaving the rest behind (ghouls can become unmanageable).
Whichever \gls{village} the troupe have arrived at (or near), is the one he assaults.

He arrives at night, and begins some miles away with his \textit{Marshweed} spell (the spell cannot be cast at close range).
The \gls{village} will know he's coming, but they will not be able to do much about it.

\paragraph{The Events}
occur, round by round, as above \vpageref{necroTactics}.

\paragraph{Survival}
depends upon a single roll, but with many factors.

The troupe should roll \roll{Intelligence}{Tactics} to see how they fare overall.
Don't mention the \gls{tn} -- just record the result, and adjust it as time goes on.

\begin{nametable}{Bonuses}

  \textbf{Bonus} & \textbf{Condition} \\\hline

  1 & Per ghast felled. \\

  1 & Per 10 ghouls felled. \\

  2 & Organizing the \gls{village} with \roll{Strength}{Tactics} (\tn[10]). \\

  3 & Per clever plan the \glspl{pc} come up with. \\

  4 & Somehow killing \gls{necromancer}. \\

\end{nametable}

\noindent
Reaching \tn[16] means the battle has been won, as the farmers fight well enough to push the dead back, and \gls{necromancer} flees, along with any ghasts still functional.
If the \glspl{pc} have not surpassed the roll by this point, the \gls{village} has been destroyed.

For tactical purpose, divide the \gls{village} into four quadrants -- perhaps `the well', `the hallway', `the fields', and `the barn', or whatever fits the \gls{village} the characters have ended in.
Each quadrant is attacked by 50 ghouls, while the ghasts stay around the outside, picking off anyone who tries to escape.

\Gls{necromancer} sends another arrow into the \gls{village} when he can see someone.

\paragraph{Spotting \gls{necromancer}}
won't be easy, but the \glspl{pc} may notice the arrows come from well outside the \gls{village}, and that they cannot target anyone properly (they go up and out).

Despite this large clue, finding \gls{necromancer} in the dark requires an \roll{Intelligence}{Projectiles} roll (\tn[12]).
He stands 30 \glspl{step} away, in the dark.

\paragraph{If the dead take the day,}
the \glspl{pc} may be able to flee (the dead will eat well, and don't need them).

\paragraph{If the farmers win,}
\gls{necromancer} leaves with the remaining ghasts.
The \glspl{pc} may take this time to hunt him down, but he still has potent spells at his call.

\thenecromancer

\showStdSpells[
  \deathStormSpell
]

\ghast[\npc{\T[2]\D}{\arabic{noAppearing} Ghasts}]

\ghast[\npc{\T[2]\D}{\arabic{noAppearing} Ghasts}]

\ghoul[\npc{\T[9]\D}{200 Ghouls}]

