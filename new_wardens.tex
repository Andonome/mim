\sidequest[Town,Roads]{Rising Titles}
\label{risingTitles}

The \gls{whiteBandits} and \glspl{digger} finally start pushing their plans to destabilise the area.

\sqpart{Roads}% AREA
{\gls{vlg} Dry Inkwells}% NAME
{When every \glsfmtplural{jotter} lies dead, the troupe discover freedom}% SUMMARY

\begin{exampletext}
  The \gls{whiteBandits} have finally pulled off their plan to have every \gls{jotter} in \gls{valley} killed in a single night.
  Each \gls{jotter} has records of where all the \glspl{guard} go, including other \glspl{jotter}.
  \Gls{traitor} compiled the information, and made a plan.
\end{exampletext}

The troupe find \pgls{bothy} with a murdered \gls{jotter}, and with some ingenuity may figure out what happened.


\begin{enumerate}
  \item
  Half a dozen bandits arrived in secret
  \hint{footprints throughout the forest show they went offroad just before the \gls{bothy}}.
  \item
  They shot at the \gls{jotter} through a window
  \hint{arrows still lie around the opening}
  \item
  and wounded the \gls{jotter}
  \hint{blood spatters mark her desk}
  \item
  then fought with the \glspl{guard}
  \hint{blood and sword-marks remain around the tree border, just beyond the \gls{bothy}}
  \item
  but ultimately, the \glspl{whiteBandits} made peace with the \glspl{guard}
  \hint{when one died, they burried him respectfully, digging a grave with some rocks on top, and leaving him money, to trick \gls{wrecan} into not taking him}
  \item
  and so the \glspl{guard} left with the \glspl{whiteBandits}, becoming part of their group.
 \hint{A dozen tracks lead away, while the \gls{jotter} lies dead}
\end{enumerate}

The players may conclude any number of things from these events, but the tracks leading away should unambiguously tell them where complete answers lie.

\paragraph{If the troupe want to track down the culprits,}
they can roll \roll{Speed}{Wyldcrafting} (\tn[8]).
Success indicates that they locate the bandits just before Sundown, while they go off-road to wash the blood from their clothes (so none are wearing armour).
A tie indicates that they must either travel overnight, or make another roll the next day at \tn[10].

\paragraph{Following the bandits back to their hide-out}
requires a \roll{Strength}{Stealth} roll, at \tn[10], +1 per day of travel.
See \vpageref{Dyson_Logos/bandit_camp} for details.

\paragraph{Confronting the bandits immediately}
results in a political conversation.
\Gls{sewerthief} and the \glspl{guard} who have just now turned to banditry don't like being forced to live outside of civilization, or how people see them.
They want better pay and working conditions, and they want to know why the \glspl{pc} don't want the same thing.

\paragraph{If \gls{sewerthief} has any reason to think the \glspl{pc} are insincere,}
he will not show them the bandit hide-out.
The \glspl{pc} must make a \textit{Group Roll} of \roll{Charisma}{Empathy} (\tn[8]).
They can get a +2 Bonus for every time any of them recall supporting \gls{sewerking}'s political beliefs.

\sqpart{Town}% AREA
{Guards, Guards!}% NAME
{The \glspl{guard} run amuck in \gls{town}}% SUMMARY

\begin{exampletext}
  The entire market has been chatting about the recent murder of all the \gls{guard} \glspl{jotter}, and how many \glspl{guard} have come to town, to drink and cause trouble.

  Three of them have recently taken to stealing at the market, though they don't have the subtlety to do anything but snatch-and-run.
\end{exampletext}

After the gossip, three of the \gls{guard} (who snuck into town by hiding in a wagon) dart past the troupe, followed by Mousebark of the \gls{weaversGuild}.
She's out of breath, so she stops to hollar at the troupe.

The three \gls{guard} will recognise the \glspl{pc}, and hide behind them, laughing like children, while Mousebark shouts at them all.

\humanmaid[\NPC{\F\Hu}{Mousebark}{young, red-faced, and angry}{stamps feed}{to retain her pride}]

\paragraph{If the \glspl{pc} do not manage to return the clothing,}
Mousebark informs the \gls{sunGuard}, who begin searching for the lot of them.

The group should roll \roll{Intelligence}{Stealth} (\tn[8]).

\paragraph{If the \glspl{pc} try to take the clothes,}
the knavish \glspl{guard} turn violent quickly.

\humanthief[\npc{\T[3]\M\F\Hu}{\arabic{noAppearing} Knaves}]

\sqpart{Town}% AREA
{Underground Assassins}% NAME
{The bandits in the sewer cut \Glsentrytext{captain}'s throat}% SUMMARY

\textbf{Background:}
\Gls{captain} has been pushing the investigation into the undead and nura sightings in town, so \gls{sewerking} ordered a hit on him, lead by \gls{sewerthief}.
Five men walked casually about him as he returned home \gls{whitehorse} with his wife, then closed in so one could stab him in the neck.

\begin{boxtext}
  You hear guards shouting ``After them!'', in the distance, and quickly scurrying feet, as a woman shouts for someone to help her wounded husband.
\end{boxtext}

\paragraph{If the characters stay to help the wounded man,}
they find \gls{captain} with a knife-wound, next to his wife.
The roll is \roll{Wits}{Medicine}, at \tn[9] to save his life.

\paragraph{If they run after the thieves,}
the \glspl{pc} make a Group Roll of Speed + Athletics.%
\iftoggle{core}%
  {\footnote{See the core rules, page \pageref{grouproll}, for Group Rolls.}}%
{}%
Remember that whoever's trying to patch up \gls{captain}'s bleeding neck won't be able to join the chase.

\begin{tcolorbox}[tabularx={cX},top=10pt,bottom=10pt]

  Roll & Result \\\hline
  12 & \textit{``Giving chase, you catch up to four men running from the scene of the crime.''} \\
  11 & \textit{``You run round an alley, and find a drain cover clanking. The assassins have jumped underground.''} \\
  9 & \textit{``You run in hot pursuit, but the attackers have disappeared down a street, into thin air.''} \\
  7 & \textit{``The attackers sprint away, leaving you running in the dark.''} \\

\end{tcolorbox}

\sewerthief

\humanthief[\npc{\T[4]\E\Hu}{Four of the Sewer Thieves}]

\paragraph{If the party follow the assassins underground,}
they run to the nearest entrance -- perhaps the butchers or \gls{pig}.
Go to page \pageref{sewers}.
Otherwise, this incident will remain a mystery.

If \gls{captain} survives, he has little idea of what's happening, although a little investigation could reveal what he's been asking about recently (\roll{Intelligence}{Vigilance} Teamwork Roll, \tn[10]).

\sqpart{Town}% AREA
{Bandits Caught}% NAME
{The bandits who plagued the countryside have been imprisoned}% SUMMARY

\label{banditsCaught}

\textbf{Background:}
The \glspl{pc} have captured \gls{banditking} alive.

\begin{boxtext}

  Hear, ye!  Hear all!

  Bandits who roamed the highways, lead by a man known as \gls{banditking}, have been apprehended.  The leader shall be drawn and quartered by week's end, and his companions hanged that night.

  Bakers are henceforth forbidden from purchasing the flour of the Quennome region, and any found doing so will be charged with consorting with elves.

  The temple of Alass\"e invites any charismatic men or women to aid the festivities, as playwrites and attractive actors are required for the upcoming festivities (not you, Margaret!).

\end{boxtext}

\Gls{banditking} will not be killed by law enforcement.
Those in \gls{pig} will inform \gls{sewerking} long before, and the rescue will commence as the bandits in the sewer storm the guards' holding.
Meanwhile, if \gls{necromancer}'s lair survives, the other bandits await instructions there.

\humansoldier[\npc{\T[10]\Hu}{10 Bandits}]

The only way for the characters to secure \gls{banditking}'s demise it to watch the guards' station all night.
If they do so, ten bandits stage an attack during the night.

\paragraph{The Distraction} starts by using ogre dust in three places in town to distract the local guards.

The attack begins once there are only ten guards left in the station.
\Glsentrytext{sewerking}'s men arrive, ready to break everyone out underground and take off into the nearest entrance to the sewer.

\sqpart{Town}% AREA
{The Dead Rise}% NAME
{\Glsfmttext{sewerking} releases ghouls upon the citadel}% SUMMARY

\textbf{Background:}
\Gls{sewerking} undead have finally finished digging the tunnel
\footnote{See page \pageref{citadelTunnel} for the underground tunnel.}
underneath \gls{town}'s citadel,
\footnote{See page \pageref{citadel} for the town's Citadel.}
where \gls{townmaster} lives.
He releases the undead, and guides them upstairs, through the dark halls.
As the \gls{guard} torches cast light on undead faces, each one makes a decision to stay and do their duty, or flee and live.

By this point, the undead horde numbers over a hundred, and the citadel has not prepared for anything like this.
It has perhaps twenty guards, and most of them work there because they seemed loyal, rather than capable.

\Gls{sewerking} leads his horde to the palace's main door with two ghasts to stop anyone exiting, and leaves the horde to slowly work its way up the stairs and rooms, killing anyone they come across.

Play the next available encounter first, then interrupt it with news of the citadel's attack.
Everyone can hear the screams from half-way across \gls{town}.

\paragraph{If the \glspl{pc} intervene,}
they find \gls{sewerking} at the door, with much of his mana depleted already from fights with the guard.
He will flee to the sewers the moment he feels unsafe.

\ghast

\ghast

\sewerking

\humansoldier[\npc{\T[6]\E\Hu}{Palace Guard}]

\paragraph{If the \glspl{pc} pursue \gls{sewerking},}
he already has a cave-in prepared to block his escape, so he waits -- holding a piece of rope, tied around a beam.
As the \glspl{pc} round the corner, they can roll Wits + Crafts (TN 10) to notice the trap and retreat.
If they fail (or run through the trap anyway), they can roll Speed + Athletics (TN 10) to make it through anyway.

If both rolls fail, the roof collapses, inflicting 8 Damage, and 8 \glspl{fatigue}.
Digging someone out from under the rubble requires an Intelligence + Crafts roll (TN 12), and they don't have long before the person under the rubble dies from asphyxiation (no resting actions allowed).

\paragraph{If the \glspl{pc} want to run up the stairs and save \gls{townmaster},}
they have 8 rounds before the ghouls devour him, and will need to travel across 20 squares before reaching \gls{townmaster}'s room on the second floor.

\paragraph{If \gls{townmaster} dies,}
then his sons most likely die with him.
Leadership of \gls{town} `temporarily' goes to \Gls{captain} (or \gls{traitor} if \gls{captain} already died).


