\subsection*{Overview}

\begin{multicols}{2}

This book evolved like a weird-looking mushroom.
It has a few patterns designed to fit worlds which no longer exist.
And sometimes the shape makes you wonder where one part ends and a segment begins, but only where those distinctions don't matter.
And like the best mushrooms, it should be consumed, in an almost literal sense.

\textit{Missions in Maitavale} is \emph{anti-collectible}.
It has nearly-zero images besides the maps.
And it will age \emph{terribly}; this copy was created on \today, and which means that in \gls{fenestra}, they are experiencing the \showTemperature\ season of \showSeason, where this story begins.
Next month here means a new season there, so this copy will become irrelevant soon.

\randomdozen
Every man, woman, and spider in the book has a statblock, with \glsentrylongpl{hp} boxes, like this: \setcounter{wounds}{3} \boxStat{r12}, designed to score through.
When the \glspl{pc} discover a map, you will have to tear that handout from the middle of the book.

And like strange mushrooms, it's tougher than it looks.
These stories accept player involvement and tangents as part of their natural cycle.
They evolved over many iterations so block people together into packs, so the death of \pgls{npc} only means a twisted plot with a scar; not a broken one.

Entire locations will vanish, and others appear.
And whether the \glspl{npc} live or die, the start of these stories will tell you where they want to go, but you will have to figure out how they grow from whatever position the players forge or fall into.

You're now the caretaker to a weird mushroom.

\subsubsection*{Acknowledgements and Thanks}

\subsubsection*{Irina}
for the map around \glsentrytext{town}, page \pageref{Irina/greylands}.

\subsubsection*{Daniel Walthall}
for the prison, page \pageref{Daniel_Walthall/prison}.

\subsubsection*{Dyson Logos}

for the
campsite, page \pageref{Dyson_Logos/forgotten_city};
the Green Tower, page \pageref{Dyson_Logos/green_tower};
Redfall Keep, page \pageref{Dyson_Logos/redfall_keep};
Redfall Village, page \pageref{Dyson_Logos/redfall};
the town map, page \pageref{Dyson_Logos/town};
\glsentrytext{pig}, \pageref{Dyson_Logos/mincing_pig};
the forgotten temple, \pageref{Dyson_Logos/qualme_temple};
the sewers beneath the town, page \pageref{Dyson_Logos/sewer};
the ruined village, page \pageref{Dyson_Logos/ruined_village};
forest gate temple, page \pageref{Dyson_Logos/shadow_gate},
lakeside village, page \pageref{Dyson_Logos/lakeside}.

(find them at {\tt www.dysonlogos.com})

\subsubsection*{Decky}

For the necromancer, page \pageref{Decky/necromancer}.

\subsubsection*{\ldots and of course, the myriad playtesters.}
Almost every credit in every other book has involved one iteration of this campaign.

\end{multicols}

\subsection*{Licence}

BIND is open source, and available under the {\tt GNU General Public License 3} or (at your option) any later version.

You have full access to all the source files, including art, and the right to change anything and share those changes with others.
BIND will never have any `house rules', because anyone can place their alterations directly into the book and make their rules official.

\pagebreak
