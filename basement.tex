\subsection{The Old Alchemy Basement}
\label{old_alchemy_basement}

\begin{exampletext}
  Like all advanced alchemists, Inkparch felt himself above others, and like most alchemists, he had poor social skills and didn't understand the link between these two facts.

  His grandest tantrum so far began two centuries ago.
  At the time, a magical gateway connected Archwarp's alchemical research post with Sixshade's temple, and Inkparch wanted to take over the alchemical laboratories, then control the portal.

  Inkparch cast his final living spell -- \textit{Soul Specks} -- then answered the standard gateway-riddle in Sixshade.
  He marched straight past the traders, guards, and apprentice alchemists, then put a binding spell on the entrance door, sealing everyone inside.
  With all inhabitants locked in with him, he began a brutal, magical assault on everyone inside.

  Unfortunately for Inkparch, he had neglected one important detail;%
  \footnote{This is always the case when a spellcaster feels over-confident.}
  a guard stabbed him in the teeth.

  Death would not slow him down -- the \textit{Soul Specks} spell meant his soul remained tied to his body, making him permanently undead.
  However, spitting half his teeth and tongue on the ground meant he could not speak the words to open the magical gateway.
  The simple inability to say `\textit{open to trade}' left him trapped.

  The remainder of his plan also went awry.
  He planned to trap the entrance with the threat of destroying the gateway.
  With long lengths of chains, he tied the dead to the pillars in front of the gateway, then animated the bodies.
  The ghouls would stand, motionless while they couldn't see anyone, but the moment the gateway opened, they would pull on the chains, threatening to collapse the pillars, destroy the alchemical research station, and seal the gateway forever.

  However, this plan also failed.
  The gateway remained closed\ldots nobody emerged to barter with him or accept his victory.

  Inkparch killed the remaining people in the basement, and drank their souls to charge himself with the last mana he would feel in centuries.
  And over those centuries, after the door became barred from the \emph{outside}, as flood-water slowly filled the basement, he attempted to teach the simple spirits which inhabit corpses to say the one phrase he needed to emerge safely.

  The ghouls here still make noises with what remains of their larynx, but not a single one has managed to properly utter the required sentence.
  As a result, the undead within the basement still chant the opening words of the portal, but they can never get it right.
\end{exampletext}

The entire basement of the old magical laboratory is sodden with water, resting knee-height to a human.

\begin{boxtext}
  Looking down stairs, you see a hallway built for gnomes, lined with pillars, and with a black floor, then you correct yourself.
  Black water sits along the entire hallway, which looks like a floor.

  In the far darkness, you can just about see great double doors along the hall, to the right, before the rest of the hall goes black.

\end{boxtext}

\mapPic{b}{Dyson_Logos/under_lost_city}{
  \Large A/18/37,
  \Large B/42/37,
  \Large C/30/18,
  \huge S/355/18,
  \huge\rotatebox{90}{S}/73/71,
  \huge S/64/86,
  \ref{alchemyHall}/18/84,
  \nameref{alchemyHall}/18/93,
  \rotatebox{45}{\nameref{alchemyEquipment}}/14/61,
  \ref{alchemyEquipment}/20/59,
  \rotatebox{25}{\nameref{alchemyLibrary}}/46/76,
  \ref{alchemyLibrary}/52/72,
  \rotatebox{45}{\nameref{alchemyRooms}}/29/41,
  \ref{alchemyRooms}/31/37,
  \rotatebox{-45}{\nameref{alchemySecret}}/76/88,
  \ref{alchemySecret}/71/83,
  \rotatebox{45}{\nameref{alchemyGift}}/66/31,
  \ref{alchemyGift}/60/39,
  \rotatebox{-90}{\nameref{summoningRoom}}/93/58,
  \ref{summoningRoom}/87/58,
}

\paragraph{Wading through the water}
is difficult.
All movement rates are halved.

\paragraph{Darkness}
means someone must carry a torch for the group to see, and any torches dropped in the water instantly extinguish.

\paragraph{The cold water and foetid air}
inflicts 2 \glspl{fatigue} per \gls{interval}.
Allowing the barrow to air out for \pgls{interval} means it only inflicts 1 \gls{fatigue} per \gls{interval}.

\paragraph{All doors}
have swollen due to the many years of water-logging.
Opening them requires a \roll{Strength}{Crafts} roll (\tn[7]).
As usual, each roll remains as-is, so if someone fails to open a door they will need to find a way to gather more strength, or give up opening that door.

\paragraph{Narrow hallways}
make wielding long weapons challenging.%
\exRef{core}{Core Rules}{enclosedcombat}

\paragraph{The Dead Chant} when not in combat.
If they stand at the other end of a hallway, they chant.
If the characters lock them in a room, the dead stand outside and chant while clawing at the door.

\paragraph{Cave-ins} present a real danger here.  If the ceiling ever collapses while the characters are inside, the falling rocks from above at first deals $1D6-2$ Damage to everyone in the room, then $1D6-1$, and so on, increasing by 2 each round, until everyone exits or dies.

\mapentry[alchemyHall]{Dead in the Water}

When the troupe enter, they move over an uneven floor, and some of the debris below them are ghouls.

The ghouls' stiff bodies move slowly, and by the time the characters have all moved into the hall, the dead rise between them, separating members of the troupe.

Five ghouls rest at the start of the hallway, and another five later on.
A further five at the other end of the hallway begin walking towards them immediately.
The entire situation makes for the perfect ambush, though the dead have not planned for it.

\paragraph{If the party attempt to fell any pillars,}
have them roll \roll{Strength}{Crafts}, \tn[13] (or less, if they use the right equipment, such as rope).
Once the pillars fall, the entire area collapses within two rounds.

\paragraph{If anyone searches the bodies,}
they find eight of the ghouls which were once \glspl{guard}.
The bodies have a total of \lootMedium\ between them.

\needspace{30em}
\ghoul[\npc{\T[6]\D}{\arabic{noAppearing} \gls{guard} Ghouls}]

\ghoul[\npc{\T[9]\D}{\arabic{noAppearing} Ancient Ghouls}]

\begin{boxtext}
  The shelves in this wide room are full of smashed and broken equipment, but it looks generally alchemical.
  Beakers, jars with rotten mouthdigger babies suspended in unknown fluids, pots with mould leaching out, alcohol-burners, and brass beakers litter the shelves.
\end{boxtext}

\mapentry[alchemyEquipment]{Supplies}

\index[mana]{Fate!Rubies}
Almost all of the various \glspl{ingredient} alchemists rely on have gone rotten over the centuries, but a couple remain useable.
In particular, two pouches of red ruby dust function as Fire \glspl{boon}.

\paragraph{Characters who scour the room,}
can find valuable remains with a \roll{Dexterity}{Academics} roll.

\begin{nametable}{\nameref{alchemyEquipment} Investigation}
  \textbf{Roll} & \textbf{Result} \\
  \hline
      6         &   \lootMedium   \\
      8         &   \lootBig      \\
      10        &   Ruby-dust pouch (Fire \gls{boon}) \\
      12        &   Ruby-dust pouch (Fire \gls{boon}) \\
      12        &   \lootBig      \\
\end{nametable}

\begin{boxtext}

  These two doors stand locked and refusing to budge.  A simple brass lock stands rusted on the front.

\end{boxtext}

\begin{boxtext}
  The doors throw inwards, revealing row upon row of rotten books.
\end{boxtext}

\bookInvestigationChart

\mapentry[alchemyLibrary]{Library}
\index{Library!Ancient}

Anyone searching joins the Group Roll of \roll{Dexterity}{Academics}.
Compare the results to the table \vpageref{libraryTable}.

Perusing the contents of a book requires \pgls{interval}, and a \roll{Dexterity}{Academics} roll (\tn[10]).
Failure means the book falls apart while the character tries to read it.

\begin{boxtext}
  These doors swing open effortlessly, showing a new room with three more doors; right, left and centre.
\end{boxtext}
\mapentry[alchemyRooms]{Dark Pit}

\begin{exampletext}
  Three rooms here used to house the various masters of alchemy.
  Stairs reached down to a lower floor, then back up.

  When Inkparch entered, the alchemists locked themselves in their rooms, so he cast \textit{Soul Specks} to make partially undead.
  Soon enough they died of starvation (rather than fighting him) and now remain here, with the knowledge that using their bodies means the spell will degrade, and their souls will wander to some unknown afterlife.

  Over the centuries, they have spoken in the unfathomable language of the dead, having a long, bitter conversation, full of resentment, angst, and occasionally playing chess by memorising a theoretical gaming-board.
  Eventually, one of the alchemists suicided.

  Since then, the room has flooded, leaving Inkparch standing deep in the depression between the three rooms.
\end{exampletext}

\paragraph{If anyone steps into the water,}
they need to pass a \roll{Dexterity}{Caving} check (\tn[10]).
Failure means they stumble, then fall into the water, where Inkparch wrestles with them -- they count as \textit{Prone}.%
\exRef{core}{Core Rules}{prone}

\setcounter{wounds}{4}

\npc{\M\D}{Inkparch}
\person{1}% STRENGTH
  {1}% DEXTERITY
  {0}% SPEED
  {{3}% INTELLIGENCE
  {0}% WITS
  {2}}% CHARISMA
  {0}% DR
  {1}% COMBAT
  {
    Aggression 2, Academics 2, Xenomology 3, Medicine 1
  }% SKILLS
  {\longsword}% EQUIPMENT
  {
    \addtocounter{xpbonus}{3}
    \setcounter{Brawl}{2}
    \setcounter{Academics}{3}
    \setcounter{Xenomology}{2}
    \setcounter{Medicine}{1}
    \setcounter{Stealth}{1}
    \setcounter{Air}{3}
    \setcounter{Earth}{2}
    \setcounter{Fire}{2}
    \undead
  }

\showStdSpells

\demilich[\npc{\D\M}{Cainrush}]

\showStdSpells

\paragraph{Room A} used to house the master of Force, who built the portal in room \ref{summoningRoom}.

\paragraph{Room B} houses nothing but broken furniture, sludge, and the remains of a ghast who smashed his own skull against the wall until permanent death.

The last room, however, is different\ldots

\begin{boxtext}
  The heavy door creaks open to an attractive room, like an expensive upstairs room in a tavern, complete with a bed, a study, and a freshly cooked breakfast on the table.
\end{boxtext}
\paragraph{Room C} used to house an illusionist, and his spells are still going ever since he died.
Instead of cleaning his room, he would simply cast an illusion of cleanliness.
The room looks immaculate, and full of light.

Within the room, under the comfortable-looking (but filthy) bed, is a hidden little tunnel, which leads up to a secret room.

\begin{boxtext}
  The stairs reach up, and finally you step your muddy boots out of the water and along a cold, but dry corridor.
\end{boxtext}

\mapentry[alchemySecret]{Secret Study}

Up the stairs the area remains dry, safe and eventually leads to a regular door (no roll required to open it).
Inside, the room contains tables with extremely old scrolls, dust, and a series of very out-of-date books on alchemical theory.

The \glspl{pc} can make another roll on the old books table \vpageref{libraryTable} (with a +2 Bonus if they have already rolled on it).

\paragraph{Resting here}
causes no \glspl{fatigue}, as the place is not full of cold water.

A \roll{Wits}{Crafts} check, \tn[10], reveals a loose wooden board in the ceiling.
It used to be an exit to the ground floor of the town above, \vpageref{basementTrapdoor}, but now the upper floor is just the ground outside\ldots after a lot of digging upwards.

\begin{boxtext}

  The bricks fall away easily, revealing a full new room.  Two skeletons rest on a table, each clutching a book.

\end{boxtext}

\mapentry[alchemyGift]{Giftschrank}

\begin{exampletext}
  The two skeletons on the table have aged worse than the other corpses, as they were never preserved in the peat-water.
  They died of hunger rather than facing the dead they knew to be outside.
  One holds a book of poetry, and the other holds a book of Enchantment spells, which she never managed to understand before dying.
\end{exampletext}

The book on Enchantment spells contains various excellent spells.
It is worth 6~\glspl{gp}, while the book of poems is worth 4~\glspl{sp}.

\paragraph{Finding the words which unlock the portal}
requires an \roll{Intelligence}{Vigilance} roll, \tn[9].
It's hidden among a dozen rather dull books on proper etiquette with alchemy, and accountancy books concerning what the guild brings in and what is can produce.

The door to the Summoning Room is only blocked by a crude wooden panel, so exiting only requires a kick

\paragraph{Spotting the hidden door from the outside}
requires a \roll{Wits}{Vigilance} (\tn[8]).

\begin{boxtext}
  The massive double doors slowly swing inwards, and the torchlight reveals a flooded hallway of six stone pillars, two enclaves, and a stairway leading up to a stage.
  The stage shows a grand stone arch, like a doorway, leading to darkness.
  You can see an writing across the top.
\end{boxtext}
\mapentry[summoningRoom]{Summoning Room}

%! More portal riddle details

\begin{exampletext}
  This is where the magic happened.
  When anyone said `open to trade', the portal came to life and allowed tradesmen to go in and out.

  After Inkparch became resentful of the idea of someone killing him and using this place for their own benefit.
  He took a measure of chains from the equipment storage, and used them to tie up a number of ghouls to the pillars.
\end{exampletext}

\paragraph{If any of the players say the words out loud,}
so do their characters; the portal opens and the ghouls in the room begin chanting along with them in unison.%
\footnote{As usual, speech costs 1~\gls{ap}, so if the ghouls are in combat once the words are spoken, the party should enjoy the unexpected advantage they get.}

\paragraph{If the \glspl{pc} remove the gemstones in the portal,}
it breaks forever.
The gems are worth 30gp in total.

If undiscovered, the dead stand and begin their chant, then slowly walk towards the characters.

\paragraph{After 2 rounds,}
the dead stand up.

\begin{boxtext}
  You look behind, and note two-dozen dead men standing from the water and staring at you.
  Their skin has gone brown with age, and they look barely able to move.
  Each drags a chain behind it, tied around one of the entrance pillars.
  They pull together towards you, each uttering the same strange, chanting moan, and then stop as the chains go tight around the stone pillars.
\end{boxtext}

\paragraph{If a pillar has six or more ghouls pulling at it,}
then it collapses after 3 rounds.

\paragraph{If a group of ghouls grab a character,}
they stop pulling at their chains and focus on attacking that character.
