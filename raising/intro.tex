The next \glspl{thread} expand on the previous ones.

In \nameref{troubleAle}, \pgls{alemaster} starts a fight with \pgls{southCook} of the \gls{wolfhead}, but the \glspl{pc} only witness the echoes (and screams) of the covert fights.
Investigating the crimes may lead them to understand the local smuggling ring, or \gls{pig}.
This \gls{thread} dominates the action in \gls{town} for some time.
It forms a `\gls{segment}-stopper', just to slow things down a little, so later \glspl{thread} won't advance too far until most of the dangling \glspl{segment} from \autoref{threadsI} reach a conclusion.

In \nameref{littleprince}, the \glspl{pc} meet some elves, and may learn more about \gls{spiderqueen}.

In \nameref{sewerking}, the \glspl{pc} meet \gls{sewerking}, who offers to sell stolen goods.
Soon after, there will be hints of the growing vault of \glspl{ghast} under \gls{town}, but the \glspl{pc} will not understand the link between \gls{sewerking} and the \glspl{ghast} unless they investigate.

By this point, the players should have begun to investigate \gls{valley}, and begun thinking about its history (or at least they should have noticed that it \emph{has} history to be found).

These \glspl{thread} have an assumed start-date of \showCycle,
\ifnum\thecycle<4%
  in the year \thefenestraYear,
\fi%
but other \glspl{thread} from \autoref{threadsI} and \autoref{threadsIII} will overlap with these \glspl{thread}, so you should make some adjustments to account for \gls{snow}, \glspl{heatwave}, or \pgls{storm}.
