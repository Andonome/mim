\subsection[Forgotten Temple]{The Forgotten Temple}
\label{necromancers_lair}
\index{Forgotten Temple}

\mapPic{t}{Dyson_Logos/qualme_temple}{
  \ref{antGrounds}/17/10,
  \Large S/28/46,
  \ref{antHall}/34/53,
  \ref{antBasilisk}/50/53,
  \ref{antPrison}/34/89,
  \ref{necroGhasts}/23/79,
  \ref{antOgre}/82/30,
  \ref{antTower}/85/67,
  \ref{antStudy}/17/44,
}

Priests of \gls{eldren} lived here at the side of \pgls{village} some centuries ago, when \gls{lostcity} still stood.
See \vpageref{necroStory} for the full background.

All over the temple grounds, ghouls wander, or simply stand and stare.
The number depends on how many Side~Quest parts have played out in `\nameref{necroStory}' -- in part 1, he has 100 ghouls, in part 2 he has 200, and so on.

These ghouls will see the troupe from a long way off, as the dead can see human souls in the far distance.

\begin{boxtext}
  In the far distance, through the dense foliage, you see the top of a single crumbling tower, and you can just about make out a statue, or possibly a motionless man, standing watch.
  Before the tower, something moves through the forest -- many footsteps, like a herd of aurochs, moving silently.%
  %\exRef{judgement}{Judgement}{auroch}
\end{boxtext}

If the troupe have no magic to make them less visible to the dead, then \gls{necromancer} will spot them, and begin casting offensive spells (check his abilities \vpageref{antTower}).

Once the horde of the dead arrive in front of the troupe, they have a problem -- they can outrun the dead for a while, but will soon grow tired.
This is the same problem as part \vref{necroHorde}.

\paragraph{If the \glspl{pc} try to circle round the ghouls,}
they should roll \roll{Intelligence}{Athletics} (\tn[8]) to figure out the ground they need to cover to out-pace the dead.
Success means they accrue 2~\glspl{fatigue} from the harsh marching, but gain \pgls{interval} before the ghouls catch up to them.
A tie indicates they gain 6~\glspl{fatigue}, or simply do not lose the ghouls (players' choice).

\mapentry[antGrounds]{The Grounds}

The garden lies untended, but paradoxically has more life than it ever did when humans tended to the grounds.
The grass has grown long, the trees' fruit comes and goes according to the seasons, and the local area has become populated with a lot of apple trees.

A hundred broken arrow parts lie littered around the area, as \gls{necromancer} practices with his bow whenever he has the energy.

\ghoul[\npc{\T[10]\D}{Ghouls}]

\stirgeSwarm

A swarm of stirges%
\exRef{judgement}{Judgement}{stirge}
has moved into the area, but continuously make the mistake of trying to feed from the dead.
As a result, they have become carriers for a nasty disease -- Breathrot.%
\exRef{judgement}{Judgement}{Breathrot}
Anyone stung by the stirges contracts Breathrot, and becoming less able to breathe until they find a cure, or death.

\mapentry[antHall]{The Hallway}

The shrine at the side of the hallway is composed of five skulls standing on top of each other, all raised on a pillar.

\paragraph{Reading the archaic writing}
requires an Intelligence + Academics check, TN 10.

Each skull has a female name, and a message thanking all those below for life.
Anyone saying a prayer of gratitude to their matrilineal lineage in front of the item gains $1D6$ \glspl{fp}.%
\footnote{The \gls{artefact} has 6 MP in total, and spends 2 MP to cast the spell.}

Further along, two empty pedestals show where old treasures once stood.
These were pilfered by \gls{sewerking} -- see \vpageref{skullCrown}.

\mapentry[antBasilisk]{the Preserved Basilisk}

\Gls{necromancer} felled a basilisk recently, then raised it as a ghoul.
If he has already raided the carriage of armour (in part \vref{necroArmour}), then the basilisk also has partial plate armour, cobbled together from bits and pieces of human armour.

This armour is not well-made, so it falls apart once hit.
Each hit removes the armour \textit{on that \gls{natural}}, so if someone hits it while rolling a `10', then any more rolls of `10' will remove the basilisk's armour.

The plate armour is only \textit{Partial}, so any rolls which hit 3 steps above the \gls{dr} can bypass this armour.
However, it still has its natural armour, and the \gls{dr} from undeath.%
\exRef{core}{Core Rules}{stackingarmour}

\undeadBasilisk

\boxPair[t]{
  \ghast[\npc{\T[2]\D}{\arabic{noAppearing} Ghasts}]
}{
  \ghast[\npc{\T[2]\D}{\arabic{noAppearing} Ghasts}]
}

\mapentry[antPrison]{The Prison}

This room once housed people making important decisions.
\Gls{necromancer} now uses it to house prisoners so he can feed off their souls.%
\footnote{Returning a living person to the area comes with complications, as the ghouls always want to `eat' them.
The solution lies in first making them partially undead, and then having the ghasts carry them back.}
Currently, it contains four traumatized and terrified farmers.
Each have a -2 penalty to everything, due to the starvation.

The door has a massive, brass lock, with the ornate markings of a Dwarvish artisan.
It also stopped functioning about a century ago.

\Gls{necromancer} relies on fear, which works better than locks.
As the prisoners enter, the door shuts and the hungry dead outside can no longer see them, so they leave the doors alone (ghouls are mostly blind, and entirely incurious).

\mapentry[necroGhasts]{Ghast Spotters}

The part \ref{necroRumours} of `\nameref{necroStory}' has already played out, then two ghasts stay here, watching the birds.
Since the undead cannot hear, they will not notice the \glspl{pc} unless \gls{necromancer} calls to them, or they see the \glspl{pc} approach.

\mapentry[antOgre]{two Ghast Guards}

The two ghasts stationed here can see outside just as well as the ghouls can, but will not move from their spot.
The \glspl{pc} will easily notice the two watchmen standing unnaturally still, and observing them from afar.

\mapentry[antTower]{The Watchtower}

\begin{exampletext}
This tower once hosted a call to prayer the nearby \gls{village} could hear.
Now \gls{necromancer} uses it to watch the world, silently.
\end{exampletext}

A large stockpile of 200 arrows sits at the side of the little room, along with a small pile of 4 Water \glspl{boon}, made from Woodspy Beaks.

From here, \gls{necromancer} can see any living person approaching, and has spells which can affect them.

\thenecromancer

\showStdSpells[
  \deathStormSpell
]

\mapentry[antStudy]{The Secret Study}

\begin{exampletext}
  The old \gls{healersGuild} collected tributes from the people, which generally just piled up around the room.
  When people abandoned the local town, none dared remove any of the treasures, as they knew that \gls{necromancer} had sworn an oath to protect the temple, including all its valuable assets.
\end{exampletext}

\noindent
A single, very thin, slab of rock was set on hinges in the inner side.
From the outside, it looks like just another slab of stone, comprising the hallway.

\paragraph{Noticing the secret entrance}
requires a \roll{Wits}{Crafts} roll (\tn[12]).

\begin{boxtext}
  The stone slab in the wall swings open at a push.
  Hinges on the other side squeal with centuries of rust.
  The tiniest touch of light sneaking inside the void tells you this area could fit \pgls{warden}'s banquet hall.
\end{boxtext}

\begin{itemize}
  \item
  Many degraded canvases -- once ornate paintings.
  \item
  Shelves with scrolls (all prayers to one's ancestors).
  \item
  A desk with various communications, including praise to the portals, and complaints about how nobody visits this area, as they have no portal.
  \item
  A workshop bench with various pieces of armour on it, including a complete suite of chain (suitable for someone of Strength +1).
  \item
  Degraded coats, now as thin and full of holes as a spiderweb.
  \item
  Three small chests containing \lootMedium, \lootBig, and \lootBig\ respectively.
  \item
  Four ornate wooden boxes%(worth 1~\gls{sp} each) containing \lootJewellery, \lootJewellery, \lootJewellery, and \lootJewellery.
\end{itemize}

The dead cannot see into this room, so the troupe can hide from ghouls, or any non-sentient dead.
The ghasts will locate them if they saw them enter this area, but \gls{necromancer} has told the ghasts never to enter this room, so they will not.
