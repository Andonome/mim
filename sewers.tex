\subsection[the Lost Library]{The Lost Library}
\label{sewers}
\index{Library!Ancient}

The old temple of Qualm\"{e} stretched deep underground, and soon after it was built, a library was commissioned by local alchemists.
The two shared much of the space for some time.
The place held students, priests, alchemists, and a grand library.
However, once the nearby city was destroyed (now \gls{lostcity}), there was no longer enough money, pilgrims, or students hoping to one day see the great university of the city to sustain the underground library, or the temple.

A century later, the underground flooded, and everything below sat alone in darkness.

When the Whiteplains nobles, bereft of a home and desperate, took refuge in the mincing pig, they began exploring the tunnels below, and found they could divert the waters which had drowned the library.%
\footnote{See the `\nameref{runoff}' location, page \pageref{runoff}'.}
Once the waters began to recede into the porous earth, many of the artificial tunnels had collapsed, while some new tunnels had been dug out.

They dug their way down as quickly as they could, and cemented a stream down into an unending underground hole (area 12).
Water flows down from \gls{pig}, and they can easily dispose of any dirt dug up.

\paragraph{The doors} are all locked by a single key type of key.
Lock-picking them requires an Intelligence + Larceny Group Roll, TN 10 if the bar's up, and TN 14 otherwise.

\paragraph{The narrow hallways}
make longer weapons difficult to use.
The \textit{Enclosure Rating} here is 4, so anyone using a weapon which requires 6 Initiative will suffer a -2 penalty to attack.%
\iftoggle{core}{
  \footnote{See the core rules for more on Enclosure Ratings, page \pageref{enclosedcombat}.}
}{%
  \footnote{See the core rulebook for more on Enclosure Ratings.}
}

\mapPic{t}{Dyson_Logos/sewer}{
  \ref{butcher_exit}/23/95,
  \rotatebox{60}{\normalsize\nameref{butcher_exit}}/14/85,
  {\large\D}\ref{sewerCells}/21/75,
  {\large\D}\ref{sewerCells}/52/06,
  \D\ref{sewerLockup}/07/41,
  \D\ref{sewerLockup}/37/03,
  \ref{oldlibrary}/28/49,
  \ref{sewerArtefacts}/86/63,
  \large S/83/65,
  \Large\ref{sewerKingRoom}/71/51,
  \Large\ref{sewerDust}/70/31,
  \ref{sewerWaterHall}/48/38,
  \rotatebox{-23}{\small\nameref{sewerWaterHall}}/43/32,
  \D\ref{sewerLockup}/61/42,
  {\large\D}\ref{sewerCells}/21/75,
  \Large\ref{sewerFood}/70/69,
  \ref{underHall}/77/27,
  {\normalsize\rotatebox{-90}{\nameref{underHall}}}/87/27,
  \ref{citadelTunnel}/11/18,
  {\normalsize\nameref{citadelTunnel}}/05/09,
  {\large\D}\ref{sewerCells}/21/16,
  \ref{sewerPigWalk}/84/82,
  {\small\rotatebox{45}{\nameref{sewerPigWalk}}}/90/81,
  \ref{sewerPig}/92/71,
  {\normalsize\rotatebox{-90}{\nameref{sewerPig}}}/97/74,
  \ref{farmExit}/71/91,
  {\normalsize\nameref{farmExit}}/71/98,
  \Large\ref{sewerDust}/49/76,
  \Large\ref{underGuard}/63/865,
  {\large\D}\ref{sewerCells}/43/79,
  \ref{sewerShrine}/56/77,
  \Large\ref{sewerDog}/33/67,
}

\mapentry{Stairway to the Butchers}\label{butcher_exit}

This stairway has been dug upwards to a drain just outside of a butchers.
The bandits enter and exit through here.

\mapentry[sewerCells]{Dead Cells}

\textbf{Background:}
\Gls{sewerking} doesn't have enough space to store his undead pets, so he has his ghouls make these little alcoves.
Once he has four ghouls, he packs them all together and hammers wooden bars to keep them wandering out.
He tried making a larger alcove, but the pressure created by random movements broke the bars open; the ghouls can leave any time they have sufficient motivation, the bars just stop them wandering.

The Immortal Bandits wander in and out safely by using their rings of asphyxiation.%
\footnote{See \textit{Fenestra}\iftoggle{aif}{\autopageref{ring_asphyxiation}}{for more on these rings}.}
If many need to come in or out, two go through, and one carries two rings back for the next, so one by one a dozen can enter slowly.

\paragraph{When the PCs enter,}
give the first in the line a Wits + Vigilance roll, TN 16.
Every 2 margins of failure means the ghouls from a cell burst open (up to the maximum of 7), so rolling a `10' implies that 3 cells have burst open, containing a total of 12 ghouls.

\ghoul[\npc{\T[4]\D}{4 Ghouls}]

\paragraph{If the PCs attempt to simply rush past,}
have them roll Speed + Athletics (TN 12).
Again, every 2 margins of failure mean that they have missed the end of the tunnel-section by one cell, so the dead exit and grab them.

\mapentry[sewerLockup]{Dead Rooms}

\textbf{Background:}
\Gls{sewerking} uses these larger alcoves to store individual ghouls before he has enough to stuff into one of the cells (see above). 
He also keeps the his more dangerous undead experimentations here -- ghasts.
He can't control them all, so he simply stuffs them in, waiting to use them later.

\begin{boxtext}
  The hallway ahead has doors on the left and right, each shut with a simple bar across the front.
  The doors have a tight seal, and no window to give a clue about the contents.
\end{boxtext}

\paragraph{Anyone knocking off the bar}
will probably find a nasty surprise.
Consult the chart (no matter which door the PCs open first, that's the first door).

\begin{boxtable}
  Door & Contents \\\hline
  1st & 1 ghast hides behind the door. \\
  2nd & 3 ghouls jump out. \\
  3rd & \gls{sewerthief} is taking a nap while he should be working. \\
  4th & 2 ghasts jump out, and one flees towards \gls{town}, above. \\
  5th+ & 1 ghast jumps out. \\
\end{boxtable}

\paragraph{If the PCs listen at the door,}
they find the dead are silent when not active, and don't respond to noise.

\ghast

\ghoul

\mapentry{The Old Library }\label{oldlibrary}
\index{Library!Ancient}

\begin{boxtext}

  Empty stone shelves show where an expansive library once provided the entire city knowledge, but not a scrap of paper remains.
  Three stone pillars divide the room, each with a brazier hanging in front of them by a chain.
  The central pillar's brazier is made from a human skull.

\end{boxtext}

\paragraph{Investigating the brazier,}
shows it contains incense, ready to be lit.

\paragraph{If anyone dies in the room,}
the skull resurrects them as a ghoul.

%! Remove brazier of ghoul calling.

\mapentry[sewerArtefacts]{Magical Item Storage}

\begin{boxtext}

  The boarded up wall pulls open -- the entire thing was a door made to look like a blocked entrance.  The rings of shelves show a strange assortment of items -- jars filled with human teeth, an old brazier, dried snowdrops, and a vial of blood.

\end{boxtext}

\Glsentrytext{sewerking} stashes most of his prizes in this room on a simple series of shelves.
Each is cast with Intelligence +1 and Wits +1.

\begin{enumerate}

  \item
  A vial of lamb's blood which makes the user invisible to the dead and immune to fatigue, marked ``Dead Wine'' (as per the Torpor spell).
  \item
  An old scroll, proclaiming elves the friends of humans, and seven reasons not to worry about nobles being assassinated.
  \item
  \textit{The Secrets of Elven Witchcraft: Attain Great Magic within the Week}.
  This book rambles about commanding secret spirits who live in reflections, and hints at lost treasure hidden under \gls{town}.
  The book speaks lies from start to end.%
  \exRef{aif}{\textit{Fenestra}}{warOfLies}
  \item
  The Assassination Dagger, which inflicts an additional $1D3+1$ HP Damage during the round's first attack (ignoring all FP).
  This ability can be used once per \gls{interval}.
  \item
  Magic Mushrooms, enchanted with Saurecanta level 2 to decrease the user's Intelligence and Charisma by 3 and increase Speed by the same amount.
  \item
  Foul alcohol in a bottle, which makes the imbiber regenerate fatigue if they eat, and otherwise inflicts hunger paints, as per Saurecanta level 1 (see \textit{Fenestra}, \iftoggle{aif}{\autopageref{saurecantaone}, }{} for more).
\end{enumerate}

\mapentry{\Glsentrytext{sewerking}'s Room}
\label{sewerKingRoom}

\begin{boxtext}
  The door opens to a noble's room, bearing a striking contrast to the dungeon around.  The bed's well made, the sheets are silk, and various books sit on shelves.  On the table sits various maps.
\end{boxtext}

\begin{itemize}

  \item
  The city map shows every entry point the bandits can enter the city above, including the theoretical passage the bandits think could be found again under \glsentrytext{townmaster}'s Citadel.

  \item
  A map of the area, outlining \glsentrytext{lostcity}, the portal by \glsentrytext{redfall}, and \gls{necromancer}'s lair.

  \item
  A complete map of the current location

\end{itemize}

The books are variously written on history (real and imagined), \textit{The Art of Lies} (by an elvish author -- `Erende'), and instructions on hosting a dinner party.

\mapentry[sewerDust]{Ogre Dust Trap}

A thin wire was stretched across the floor, leading up to a small stretch of leather, holding snowdrops.  Anyone failing a Wits + Vigilance roll, TN 12 in the twilight, feels the petals fall down.  A moment later, the character's afflicted as per Saurecanta, level 2, and gains +4 Strength at the cost of -4 Charisma.

\begin{boxtext}

  A little thread pushes against your face, like a steel spiderweb.
  A second later, something flutters around your head.
  The falling debris feels annoying beyond words, and it's difficult to say why -- you simply feel incredibly irritated, and hungry.
  \emph{Extremely} hungry.

\end{boxtext}

\mapentry[sewerFood]{Food Storage}

\begin{boxtext}

  Barrel after barrel fill the room, along with the smell of wine, apples, and vinegar.  A little basket of choice snacks sits on top.

\end{boxtext}

The room is normal, except for the basket of choice snacks, which is poisoned with an intense laxative.
\Gls{sewerking} suspects one of his men steals food when returning from business in \gls{town}, so he's left a basket of poisoned food.
Someone can tell it's poisoned with a Wits + Medicine roll, TN 8.
Failure means the character will have a bad night, and gain 3 \glspl{fatigue} each \gls{interval} for the rest of the day.%
\footnote{Feel free to roll for the characters so they're not aware there's a problem.}

\mapentry[sewerWaterHall]{Drowned Hallway}

This area recently suffered a little flood.  Most of the water has dissipated, but this lower portion of the tunnels remains flooded.  The undead hobgoblins remain locked in their cells underground.


\begin{boxtext}
  Going down the stairs you feel your feet hitting cold water.  It's not clear how far the water goes down, but it's cold.
\end{boxtext}

The water goes up to the ceiling by the last step, and for four squares after.  Each ghoul-stuffed room the characters pass the dead will lash out, with TN 12 to escape the grabbing hands, assuming the characters aren't Keeping Edgy, and have been blinded by the dark waters.

\mapentry[underHall]{Grand Hall}

The thieves feast, plan, and sometimes wrestle here.

\begin{boxtext}
  Opening the door, soft lantern-light trickles into your eyes.
  The grand hall has a massive feasting table in the centre, currently occupied by a dozen rushing towards you with weapons drawn.

  ``The dead are lose!'', they cry, as one man prepares some kind of spell.
\end{boxtext}

\paragraph{As the PCs enter,}
\gls{sewerking} begins casting a \textit{Potent Sickness} spell.
After 3 rounds, he deals 1D6 HP damage to the member of the troupe with the highest Strength Bonus.

The other bandits rush at them immediately.

\sewerking

\humanthief[\npc{\T[12]\E\Hu}{12 Sewer Bandits}]

\paragraph{Four rounds later,}
the cutthroats from room \ref{underGuard} arrive (assuming they weren't killed already).

\mapentry[citadelTunnel]{Tunnel to Citadel}

\begin{boxtext}
  A scraping sound comes from the tunnel ahead.
  You slowly come to a halt, but whatever makes that kind of sound doesn't seem to care about you; it just keeps on scraping at something.
\end{boxtext}
 
\begin{exampletext}
  \Gls{sewerking} set this ghoul to dig into the citadel's basement.
  It dug until its rotten hands fell off, then pushed its stumps into the dirt until the arms wore away.
  Biting into the dirt destroyed its teeth, so now it gnaws at the earth with its jaw-bone.
\end{exampletext}

\paragraph{If the PCs attack,}
they will kill it easily -- it can do nothing but attempt to dig.

\mapentry[sewerPig]{Sewer Entrance}

This artificial stream loops round from \gls{pig}, above.
The stream continues downwards to an underground abyss.

Hidden in the darkness above, a straight shaft leads upwards to an iron grate at the base of \nameref{stationDungeon}.
Waste and filth slides down through the grate regularly, followed by the sound of an ogre roaring out of sheer boredom.
\footnote{\nameref{dunGrate}, room \ref{dunGrate}, \autopageref{dunGrate}.}

\paragraph{Anyone venturing down-river gets swept along,}
and dies in the unending blackness, unless they make a Dexterity + Caving roll, TN 10, to cling onto the sides.
PCs can spend 5 FP instead to make the roll automatically.

\mapentry[sewerPigWalk]{Entrance to \glsentrytext{pig}}

Up these stairs, the troupe can reach the bowels of \gls{pig}.

\Gls{pigowner} understands what trouble she's in, and immediately accuses the characters of theft and calls on everyone in the room to kill them.

Calming down the patrons requires a Charisma + Empathy roll (TN 10).

\paragraph{If the PCs arrest the patrons of \gls{pig},}
everyone in \gls{pig} will deny any knowledge of the deeper tunnels, and the fact that bandits lived down there.

\mapentry[farmExit]{Exit to the External Farm}

\textbf{Background:}
Outside \gls{town}'s walls the tunnel ends in a farmhouse.
The Immortal Bandits use this place to move beyond the walls without worrying about any inspections.

\begin{boxtext}

  The stairs go upwards for some time, and eventually arrive at a room filled with barrels of food, and a trapdoor above.
  You can hear a deep snore coming from just beyond the trapdoor.

\end{boxtext}

Angus' house above has several rooms, as he's done rather well for himself, and hopes that once the revolution comes he'll be in an even better position.

\humanfarmer[\npc{\M\Hu}{Angus}]

\mapentry[underGuard]{Guardroom}

Here four of the thieves sit and play simple dice games to pass the time, or occasionally sleep in the foetid straw.

\paragraph{If the PCs fight,}
the cutthroats try to ram them out the door, and run to alert the others in room \ref{underHall}.

\humanthief[\npc{\T[4]\E\Hu}{Four Cutthroats}]

\mapentry[sewerShrine]{The Old Shrine to Qualm\"{e}}

\begin{boxtext}
  Letters in Elvish above the doors state ``We bones await yours''.
\end{boxtext}

There are twelve pillars in total in the room, and each one was formed by members of a family, over the course of generations, donating money to the Temple of Qualm\"{e}.  Those who died in its service had their skulls added to the tower.  It could take two hundred generations to create some of these towers.  Once the tower is completed, the top skull has the family's name carved into the forehead.  Family members

\begin{boxtext}

  The massive room has a strange lack of smell.  Towers of skulls stand in neat piles, each resting in a small pillar.  Some are as tall as a man, others reach nearly to the ceiling.  Each one has writing upon the top skull.

  At the far side of the room is another exit.

\end{boxtext}

There are twelve pillars in total in the room, and each one was formed by members of a family, over the course of generations, donating money to the Temple of Qualm\"{e}.
Those who died in its service had their skulls added to the tower.
It could take a dozen generations to create some of these towers.
The top skull has the family's name carved into the forehead, and speaking the family name along with religious incantations to Qualm\"{e} activates some magical effect.

Each pillar can be used once a day.
They require a short prayer (which takes some time to complete), and understanding which prayer aligns with which skull requires an Intelligence + Academics Group Roll (the TN depends upon the type of pillar).
The character requires a Margin of 2 to know what will happen before casting the spell.

\paragraph{Smaller Pillars: (TN 8)}

\begin{enumerate}

  \item{(2) Regenerate $1D6$ FP.}
  \item{The target loses $1D6$ FP (this family stopped paying temple dues).}
\end{enumerate}

\paragraph{Larger Pillars: (TN 10)}

\begin{enumerate}

  \item{The room is filled with sweet-smelling mist, as per Aldaron level 2.}
  \item{The target sees a vision of the future, as per the Fate spell Augury.}
  \item{Target regenerates $1D6+2$ FP.}
\end{enumerate}

\paragraph{Towering Pillars: (TN 12)}

\begin{enumerate}

  \item{This one pillar is topped by a trapped spirit, rather than being a simple magical item.  The spirit, once presented with any appropriate token of the dead, will enchant it to be a useful magical item.  The enchantment will come from the Necromancy sphere, and is cast with Intelligence +2 and Wits +2.  The enchantment lasts for a day.}
  \item{The target regains $1D6+3$ FP.}
  \item{Any recently deceased target is returned to life, as per Fate level 5.}

\end{enumerate}

\mapentry[sewerDog]{The Guard Dog}

\textbf{Background:}
The Immortal Bandits keep one of farmer Angus' dogs chained here at all times (rotating, so they don't get bored).

\begin{boxtext}

  Just ahead, you can see a dog, lying on the ground.
  It instantly stands up and begins to bark.

\end{boxtext}

\paragraph{If the characters approach,}
Rover hears them and begins barking.
If the group remained silent, they can make a Dexterity + Stealth Group Roll (TN 9) to move silently.

\paragraph{If Rover hears them,}
he begins barking and four of the bandits in room \ref{underHall} (\nameref{underHall}) come to investigate.

They won't poke about long, so the PCs simply need to pick a reasonable hiding place or roll Intelligence + Stealth (TN 8).

\paragraph{If a bandit or a ghast walks by,}
Rover does nothing -- he got used to the bandits, and even random undead wandering about, and the undead have no interest in animals.

\huntingdog[\npc{\A}{Rover}]

\vfill\null


