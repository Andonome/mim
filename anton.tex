\sidequest[Forest,Town]{The Necromancer's Friends}

\noindent
\Gls{necromancer} has lived as the lone priest and guardian of his temple for centuries, and has grown increasingly paranoid, so he wants to gather an army of ghouls to guarantee his safety.
\Gls{banditking} has helped by bringing corpses to his keep, so now he has enough to assault full-on \glspl{village} and expand his army.

If the troupe ever decide to track \gls{necromancer} down, they will need to make an Intelligence + Academics roll, TN 15 (or TN 10 with access to the library in town).

\sqpart{Roads}% AREA
{Body Bags}% NAME
{\Glsentrytext{banditking} transports the bodies of his victims to \glsentrytext{necromancer}}% SUMMARY

\textbf{Background:}
\Gls{banditking} killed a caravan of half a dozen traders, stuffed the bodies into one of the carts, and now wants to transport them all back to \gls{necromancer}'s lair for them to be raised from the dead.

\begin{boxtext}
  Three wagons approach ahead of you.
  As they get closer, you notice well-armed traders giving you a friendly wave, and carrying the stink of rotten corpses with them.
\end{boxtext}

\Gls{banditking} speaks with the \glspl{pc} happily, but only if they stop him.

\paragraph{If the \glspl{pc} ask about the smell,}
\gls{banditking} tells them he has killed some bandits on the road, and wants to take them in for a reward.

\paragraph{If anyone mentions that this is not the right way to \gls{town},}
or points out some other problem with his story, he gives very reasonable answers, concerning drop-off points, proper checks, or contacts he knows who can help embalm the corpses before a longer journey.

\paragraph{Spotting the lies}
requires a Wits + Empathy roll.
The basic TN is 12, but you should decrease it by 1 for every oddity the \glspl{pc} can point out (roll in secret, and keep the roll).

\begin{itemize}
  \item
  ``The blood? It's from the fight of course!''
  \item
  ``Yea, I know they don't look like warriors.
  It's because they aren't.
  Desperate times push people to desperate actions, but we still have to defend ourselves.
  I know she's still now, but she was as dangerous as a cornered badger, honestly.''
  \item
  ``Where are \emph{their} weapons?
  They were pretty rudimentary -- just crude spears, so we threw them away.
  They're about half a day's walk back, if you really want some new spears.''
  \item
  ``Why do we have bows?
  To stay safe!
  Do you have any idea how dangerous travelling is in these parts?''
\end{itemize}

\banditking

\humanarcher[\npc{\T[6]\Hu}{6 Immortal Bandits}]

\paragraph{If the \glspl{pc} follow the bandits back to \gls{necromancer}'s lair,}
have them roll Wits + Wyldcrafting, TN 10.

\sqpart{Forest}% AREA
{The Undead Horde}% NAME
{Hundreds of ghouls have become lost and now wander the forest}% SUMMARY

\textbf{Background:}
\Gls{necromancer} is not a precise creature, and has misplaced twenty of his ghouls.
Once they wandered away from the rest of the herd, they broke into the forest to wander some more.

\begin{boxtext}

  Crackling sticks indicate someone walks close by, and a moment later indicates a full procession walking somewhere close by.
  But you wait, and no voices come out -- only crackling sticks.

\end{boxtext}

The troupe makes a \roll{Wits}{Vigilance} roll (\tn[6]), to notice the undead while they are still 10 squares away.
Each roll on the margin indicates an additional 10 squares to notice the horde, so rolling 10 means 50 squares' distance.

The ghouls spot the \glspl{pc} from a long distance.
Such is the nature of undead vision.%
\exRef{judgement}{Judgement}{undead}

While these ones have run away, the necromancer has gathered a full army of 100 ghouls.

\paragraph{If the \glspl{pc} flee,}
they have little trouble, but they should be aware that the horde will eventually find living people to attack.

Whatever they do, the question remains -- where did the dead come from?

\sqpart{Roads}% AREA
{The Survivors}% NAME
{A village has been ransacked by ghouls, and only followers of Qualm\"{e} have been spared}% SUMMARY

\textbf{Background:}
Last night, \gls{necromancer} guided his horde quietly past the well-guarded, walled \glspl{village} around the \gls{edge}, and into a less protected settlement, then let the dead tear apart the living, and rose them from the dead to grow his army.

However, four people were evidently followers of Qualm\"{e}, god of death, as they wore medallions made to show respect to Qualm\"{e}, and hopes that He would guide their ancestors to a peaceful realm in the afterlife.

Rupert, Jake, Sarah and Eliza cannot explain why they were spared -- they only know the undead never touched them while they ate the rest of the village.

\begin{boxtext}

  On the horizon, four humanoid silhouettes stumble forward silently.  Once they see you, they start running towards you.

\end{boxtext}

\paragraph{If the characters try to track the ghoulish army,}
they find it rains en route, making tracking more difficult.
However, a Wits + Wyldcrafting roll, TN 10, will get them there.
Alternatively, researching previous old temples to Qualm\"{e} in \gls{town}'s library will give them a rough idea of where they need to search.
The research requires an Intelligence + Academics roll (TN 10), and hunting the exact location of \gls{necromancer}'s crumbling temple requires a Wits + Wyldcrafting roll (TN 10 again).

By this time, \gls{necromancer} has gathered an army of 200 undead.

\sqpart{Town}% AREA
{\squash Rumours of a Breach}% NAME
{\Glsentrytext{necromancer} destroys \pgls{village} at the \Glsentrytext{edge}}% SUMMARY

\textbf{Background:}
\Gls{banditking} worries about having a run-in with the \gls{guard}, so he's asked \gls{necromancer} to assault the walled \glspl{village} around the \gls{edge}.
Once these go, the \gls{guard} have to rush in to protect the breach, and rebuild the fortifications.
This means less interference from the \gls{guard}, which leaves the Immortal Bandits free to roam, plunder, and loot.

\paragraph{Once the \glspl{pc} enter \gls{town},}
everyone around is discussing the recent breach, and whether or not \gls{town} will soon be devoured as the walled \glspl{village} fall, and the \gls{edge} closes in around them.

\paragraph{For the next three random encounters,}
roll for encounters on the roads as if the \glspl{pc} were in the forest.

\sqpart{Roads}% AREA
{The Dead Devour a Village}% NAME
{The necromancer makes a full-on assault on \pgls{village}}% SUMMARY

\begin{figure*}[t]
\begin{boxtable}

  Roll & Result \\\hline

  $14$ & The troupe somehow spot the dead in the distance, and have 10 rounds to organize. \\

  $10$ & The troupe notice the dead almost too late, and have 3 rounds before the dead engage the village in combat. \\

  $9$ & The troupe have only 2 rounds before the dead engage the village in combat. \\

  $8$ & The dead surround the village entirely before being spotted, and the troupe first hear of the dead when they enter the village and begin tearing a house apart. \\

  $<8$ & The cries of war are mistaken for a normal scuffle, and the moment the troupe investigate, they are engaged in combat. \\

\end{boxtable}
\end{figure*}

\textbf{Background:}
\glsfirst{necromancer} has gathered an army of a full 400 undead, and has decided to take most of them to a nearby village, and kill everyone inside.
Whichever \gls{village} the troupe have arrived at (or near), is the one he assaults.

\begin{boxtext}
  The village falls quiet at night, except for shuffling feet as people try to be quiet going out to the toilet at night, or chattering about local town gossip.

  A man in the distance tells his child off harshly for going out at night into the forest with his friends.
\end{boxtext}

His tactics are to create a full ring of undead around the village, and pull it slowly tighter until the entire village is surrounded.
Once there is nowhere to go, he releases the undead to attack.

The troupe can make a \roll{Wits}{Vigilance} roll (\tn[10]) to spot the dead before they attack.

\paragraph{Preparation} depends upon a single Tactics roll.
If the troupe see the dead before the dead arrive, they can make an Intelligence + Tactics roll, TN 6.
Otherwise, they can make a Wits + Tactics roll, TN 10.
Each margin on the roll decreases the number of round the troupe must fight before the dead retreat.
If the margin is 0 or less, the dead attack for 6 rounds before more raise.

\begin{boxtext}

  The screams outside get louder and louder.
  Outside the window you can see some enemy, rising from the river.
  They're pouring out like a reverse waterfall, with every part of the bank filled end to end.

\end{boxtext}

For tactical purpose, divide the village into four quadrants -- perhaps `the well', `the hallway', `the fields', and `the shrine', or whatever fits the village the characters have ended in.

Each quadrant is attacked by 50 ghouls, while the necromancer stays around the outside, picking off anyone who tries to escape.  He starts with hit hunting bow -- his undead sight allows him to spot people in the dark with ease.  After that, he uses magic, starting with curses, then invocation magic.

The entire village attack together.
While many may attack each round, each character only has to face the number of ghouls attacking them personally.
Use the following as a guide:

\begin{boxtable}

  Round 1 & 4 ghouls attack. \\

  Round 2 & 5 ghouls attack. \\

  Round 3 & 5 ghouls attack. \\

  Round 4 & Nearby houses go down, torn apart by the dead. \\

  Round 5 & All houses a broken, and the dead invade.
  3 more ghouls attack. \\

  Round 6 & Those already dead rise again as the necromancer completes a spell. \\

  Round 7 & 2 ghouls attack. \\

  Round 8 & The dead retreat at the necromancer's order, and any dead villagers return with him. \\

\end{boxtable}

Anyone carrying obvious trinkets displaying an allegiance to Qualm\"e may be spared by \gls{necromancer}'s curses, so long as they are not actively attacking the dead.

\paragraph{Cleanup} depends upon the battle's outcome.
Once the battle's over, the village will most likely be mostly destroyed, and \gls{necromancer}'s army much larger.  The villagers will shake, huddle together, and most consider moving.

The encounter repeats every second Village-encounter until the \glspl{pc} intervene.

\ghoul[\npc{\T[10]\D}{200 Ghouls}]

