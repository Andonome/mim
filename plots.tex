\chapter{Protagonists \& Plots}
  \epigraph{He attacked everything in life with a mix of extraordinary genius and na\"ive incompetence, and it was often difficult to tell which was which.}{Douglas Adams}
\label{sideQuestIntro}


\noindent
Each day the \glspl{pc} will awaken to new plots unfolding.
Most days, they will hear rumours, on other days they will see the consequences of the plots.

The \glspl{pc} most likely work as members of the \glspl{guard}, but whatever they want to do, they will have their own tasks to attend to, and the little scenes here may seem like little distractions at first -- just a bar-fight here, and a tall story there.
But the plots of our various factions will soon begin to interfere with the lives of everyone in \gls{valley} -- humans, elves, and gnomes alike.

Resolution will demand hunting down the right locations, but the locations will not make any demands on the players or \glspl{pc}.
They will remain free to ignore any `call to adventure', which will leave the various actors below free to resolve their plots.

\printglossary[
  type=people,
  style=mcolindex,
]

\renewcommand\csComments{
  \label{Irina/greylands}
  \mapCircle{-4,3.4}{2}{Dyson_Logos/forgotten_city}
  \mapCircle{-5.6,-2.8}{2}{Dyson_Logos/qualme_temple}
  \mapCircle[4]{1.1,0.1}{2.5}{Dyson_Logos/town}
  \mapCircle{6.1,-3.6}{1.7}{Dyson_Logos/shadow_gate}
  \mapCircle{-1,-0.4}{2}{Dyson_Logos/redfall}
  \mapCircle{6.3,-0.1}{1.7}{Dyson_Logos/lakeside}
  \mapCircle{-6.3,2.3}{1.7}{Dyson_Logos/bandit_camp}
}

\widePic[t]{Irina/greylands}

\begin{multicols}{2}

\subsection{Cartography}

Take a look at the map \vpageref{Irina/greylands}.
Note how long a journey between two \glspl{village} might take.
Travellers going West to East can go downriver, which decreases their travel times significantly, while other journeys will go around walking speed.

If the smallest of the \glspl{pc} have 8~\glspl{hp}, they can travel 16 miles before gaining \gls{fatigue} Penalties.
Of course, that would push them to their limits, which invites dangerous situations.
Alternatively, they may travel for 8~miles (gaining 4~\glspl{fatigue} when walking on the roads), but then they would have to travel in the darkness of the evenings, which also invites danger\ldots

Some \glspl{sq} \glspl{segment} have a `\gls{vlg}' symbol in their summary, indicating that the scene may involve destroying \pgls{village}.
When this happens, mark the \gls{village} off the map.
If the \glspl{pc} find \pgls{village} destroyed, this often presents a dangerous situation.
They will find themselves outside, at night, without the protection of any civilization.

\subsubsection{Primordial Forests}
Outside, a dense, primordial forest surrounds everything.
Travelling beyond here leads nowhere for a week or more, and navigation could not be more difficult.

\subsubsection{Outer Civilization}
Brave, little \glspl{village} protrude a little into the wilderness, and every few nights some new creature emerges from the forest to crawl over their rooves, and claw at the doors.

Some of the outer buildings only show a single tower.
These are the \glspl{bothy}, where traders rest overnight, and where the new \glspl{guard} often stay.
In fact, new \glspl{guard} may not enter towns or \glspl{village}, so they must remain here until they increase their rank.%
\exRef{judgement}{Judgement}{fodder}

\subsubsection{Inner Hamlets}
A little further inside, quiet hamlets rest peacefully.
Any beasts which emerge from the forests would have wandered towards the noise of the outer \glspl{village} or \glspl{bothy}.

These little areas have a `civilization rating' of 15 to 17, depending on their proximity to town, so you can replace any encounters which roll that number or lower with traders.%
\exRef{judgement}{Judgement}{encounters}

\end{multicols}

\section{\Glsfmttext{sq} Summaries}
\label{sqSummaries}

\begin{multicols}{2}

\noindent
BIND has its own way of structuring stories.
They have about three to six \glspl{segment}.
The first few \glspl{segment} of each \glspl{sq} usually introduce a major character or situation, often through foreshadowing.

Each \gls{segment} of \pgls{sq} has an region where you can run it, and each \gls{segment} should make sense to run almost anywhere within that region.
So when the \glsentrylongpl{pc} go beyond the \gls{edge} and into the surrounding forest, you can look down the list of \gls{sq} \glspl{segment} which take place in `Forest', and select the first one available.
Once the \glspl{pc} have encountered \gls{segment} I, you can make \gls{segment} II available.
If the next \gls{segment} only runs in `Town', then the rest of that \gls{sq} will not run until the \glspl{pc} reach \gls{town}.

Having stories cut into \glspl{segment} lets the players experience many at once, although they might not notice at first.
Players often think of one plot-hook as `the main quest', which works out very well for all the little foreshadowing scenes.
They can blend quietly into the background, leaving the players with a half-memory and a vague sense of recognition when the next \gls{sq} ramps up.

Scan down the list \vpageref{sq:Roads} for the Roads, and use the first \gls{segment} with a `\gls{sqr}', then mark it done (`\textit{X}'), and place a `\gls{sqr}' on the next \gls{segment} of that story.

Think about which \glspl{segment} will play out if the \glspl{pc} go from \textit{Road to Forest to Road, then Town}, then think about what happens if they just wander the roads, without ever entering the Forest region.



\subsubsection{Getting Started}

You might begin with the standards \gls{guard} duties,%
\exRef{judgement}{Judgement}{NGmissions}
such as guarding traders, fixing \glspl{bothy}, and protecting nearby \glspl{village}.
Once the \glspl{jotter} send them this way and that, they will instantly find \glspl{sq} assaulting them daily, and the players will eventually investigate at least one lead.

Throughout this time, \glspl{pc} should receive all the standard encounter rolls.
Mixing the lot -- encounters and plots, \glspl{sq} and \gls{guard} missions -- creates a maelstrom of twists and turns.
This can be a lot of fun, but it also threatens the players' focus, so it's best to remember to pull that focus back to whatever the players think of a their primary mission.
This may change half way through the mission, or they may disagree; and both of those results are fine.
However, if the players ever feel \emph{confused} about what their characters should be doing, they will quickly disconnect from \gls{fenestra}, and pull out their phones.

Chaos, and focus should flow like breathing in and out.

\subsubsection{Let Players `Ruin' the Mission}

The \glspl{sq} below can withstand a lot of punishment from the players.
If \gls{segment} I of a \gls{sq} goes haywire, \gls{segment} II can usually take place unscathed, or with minor adjustments.
But you should let the players change the world, otherwise what is the point of this world?

When \glspl{pc} actions threaten to derail \pgls{sq}, you have a few options.

\begin{description}
  \item[If the \glspl{pc} kill a major character]
  use another.
  It's no accident that two brothers lead the \gls{whiteBandits}.
  If \gls{sewerking} dies, \gls{banditking} can take over most of his duties.
  Similarly, \gls{alemaster} can also take over any of \gls{pigowner}'s \glspl{segment}, as they're both members of the \gls{wheatGuild}.
  \item[If the group decide to join the \gls{whiteBandits}]
  adjust!
  So they don't want to be in the \gls{guard} any more -- that's quite understandable.
  So they will have to listen to \gls{sewerking} instead of \gls{traitor}.
  \Gls{sewerking} still has plenty of missions for them.
  \item[When all else fails]
  just cut out the \gls{sq}.
  If the \glspl{pc} somehow kill \gls{spiderqueen} during Part 1, the troupe only lose 6 scenes.
  The plot has plenty of other scenes!
\end{description}

\end{multicols}

\foreach \x in {Roads, Forest, Town}{
  \center\subsection*{\x}
  \label{sq:\x}
  \printcontents[\x]{l}{2}{\setcounter{tocdepth}{3}}
}

