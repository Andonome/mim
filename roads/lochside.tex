
\mapPic{t}{Dyson_Logos/lochside}{
  \rotatebox{-40}{\nameref{lochSquare}}/51/45,
  \ref{lochSquare}/46/43,
  \rotatebox{-40}{\nameref{lochTemple}}/78/81,
  \ref{lochTemple}/70/79,
  \rotatebox{40}{\nameref{lochDocks}}/16/68,
  \ref{lochDocks}/21/63,
}

\section[\Glsfmttext{lochside}]{\glssymbol{paik}~\Glsfmttext{lochside}~\glssymbol{sylf}}
\label{lochside}

\begin{multicols}{2}

\begin{exampletext}
  \noindent
  \Gls{lochside} began as \pgls{broch} on cursed soil.
  Plants here cannot thrive, and even the trees grow crooked and pale.
  It has recently grown into \pgls{village}, and sustains itself with fishing, trade, and import taxes.
  \Gls{lochWarden} functions effectively as \pgls{warden} to the two-hundred souls who live here.%
  \footnote{All \glspl{village} begin like this, as do their \glspl{warden}, at least in theory.
  In practice, \gls{lochWarden} grew up as \pgls{warden}'s child, in a nearby \gls{village}, and receive fast promotions.}
\end{exampletext}

Tall stone walls reach 4 metres into the air, and each house has stone walls and slate rooves.
But its bridges are made of wood, and ready to burn at a moment's notice if either side of the \gls{village} falls to some enemy.

\Gls{lochside} has an unusually large number of well-armed young soldiers, ready to fight for their little patch of the world at a moment's notice.

The river and loch here provides important trade to the surrounding areas, and has the only source of raw iron to the town, so if the \gls{village} ever perishes, the price of weapons will double.

\mapentry[lochSquare]{Central Square}

The town's central square contains a statue of \gls{townmaster}'s father looking lordly.

\mapentry[lochTemple]{Big \Glsfmttext{broch}}

The \glspl{pc} can buy just about any weapon they can name here.

\paragraph{Hiring services}
has a default \gls{tn} of 5, since the \gls{village} has reliable and excellent guides, cartographers, and soldiers.
However, the small \gls{village} does not offer many services.

\mapentry[lochDocks]{Docks}

Here, boats from \gls{southDale} enter, carrying dwarvish spirits.

\paragraph{Newcomers at the Docks}
arrive regularly.
Roll $1D6$ to find out who has arrived.

\begin{dlist}
  \item
  \Gls{southCook} has come to collect barrels of dwarvish spirits.%
  \footnote{See `\nameref{troubleAle}', \vpageref{troubleAle}, for more on the plot.}
  \item
  \Gls{southSeeker} has come to collect barrels on behalf of \gls{southCook}.
  \item
  \Gls{sewerking} has arrived to sell art, stolen from local \glspl{warden}.
  \item
  \Gls{warningbard} arrives to pick up a new lute (the \gls{sunGuard} destroyed his old one).
  \item
  A shipment of \glspl{ingredient} of every type have arrived.
  \Glspl{pc} can select any to buy, and each type costs $1D6$~\glspl{sp} (+1 per purchase as the price rises).
  The seller has $2D6$ of each type of \gls{ingredient}.
  For example, if \pgls{pc} wants a Fire \gls{ingredient}, a roll could show the price is 5~\glspl{sp}, then the next costs 6~\glspl{sp}, and so on.
  \item
  A dwarvish jeweller from \gls{southDale} has arrived, selling fancy trinkets of every type.
  If the \glspl{pc} want to sell him stolen goods, they can try with a \roll{Charisma}{Empathy} roll (\tn[12]).
  If successful, he will buy them for half their value.
\end{dlist}

If any of the \glspl{wolfhead} have arrived, they journey back to \glsentrytext{town}, with a cart, to \gls{traitor}'s house, where he can smuggle it inside, tax-free.

\lochWarden[\label{cronblight}\npcQuote{Oh that? He just caught it fishing.  You find lots of things in this old loch when the\ldots chaffinch!\ldots when the waters are still}]

\humansoldier[\npc{\T[4]\E\Hu}{\Glsfmttext{lochside} Soldiers}]

\end{multicols}
