\mapArea[poison]{shadowVault}{Vault of the Shadow Gate}

\begin{exampletext}
  Sixshadow began as a tiny settlement, but when `the Vault' gained its gateway to Archwarp, the community flourished.
  People in \gls{valley} called it `the Shadow Vault`.

  The local \gls{warden}, tasked to look after the area, made some extra money by making it an attraction, with various treasures, and even a magical \gls{artefact}.

  When the Shadow Gate closed forever, the keepers locked the door, hoping to return soon after.
  A generation later, the place lay forgotten, and the door began to degrade.

  \ldots an acidic ooze has since eaten the door, and moved in.
\end{exampletext}

Water and mud have slid down the stairs of the Shadow Vault for the last couple of centuries.
Most of the year, the temple has deep water.
However, during the cold seasons, the water freezes over.

\mapentry[entranceCullis]{Rusted Portcullis}

Gnomes crafted this portcullis from a special blend of steel and mouthdigger infants.
Destroying it requires 20 Damage, and picking the lock requires an \roll{Intelligence}{Larceny} roll at \tn[15].

\mapPic{t}{Dyson_Logos/shadow_gate}{
  S/82/47,
  S/28/26,
  {\normalsize}\ref{entranceCullis}/09/47,
  \ref{shadowPot}/11/25,
  \ref{shadowOaths}/23/90,
  \ref{shadowVore}/21/47,
  \ref{shadowJelly}/77/47,
  \ref{shadowGate}/55/47,
  \ref{shadowDesk}/88/47,
  \ref{shadowGuardian}/88/82,
  \rotatebox{90}{S}/33/18,
}

\mapentry[shadowVore]{Guardian \Glsentrytext{artefact}}

\begin{exampletext}
  While other \glspl{artefact} cause problems, due to their singular focus and complete lack of empathy or perspective, the \textit{Vore Gate} had a well-reasoned creator, who created the \gls{artefact} specifically to serve the purpose he had in mind.

  ``Let people enter, when they want'', said the alchemist.

  This solved the first problem which \glspl{artefact} have, and only this problem.%
  \footnote{If someone in your campaign ever casts a high-level, long-standing spell which has no unintended consequences, you are doing BIND wrong.}
  The alchemist only had two problems left with his otherwise genius (and certainly inspired) plan.
  The first was that the Vore Gate went to the wrong place (a lightless \gls{deep}, populated by unpredictable horrors).
  The second was that it was extremely enthusiastic about people travelling through it, with a very broad definition of `people'.

  The alchemist solved these problems by moving on with his life.
  The owner of the establishment solved them by building a wall around the gate, with the hope of capitalizing on it later.
  And the staff capitalized on the issue by building a secret entrance through the wall to trade through the portal in secret.
\end{exampletext}

\begin{boxtext}
  Past the ornate gate, you see a room, and in the far darkness, two pillars surround a gleam\ldots of some description.
\end{boxtext}

If the \glspl{pc} manage to speak with the Vore Gate, it tells them enthusiastically, and emphatically, that they have to see the other side.
It may lie, or embellish the riches on the other side, but most of all it says how much it loves having people entering.

\noindent
\begin{exampletext}
  When a group of `brave' warriors entered this room to explore the other side of the Vore Gate, a creature struck them dead before they could enter.
  This strange creature came from the other side of the Vore Gate the previous night, and simply remained at one side of the room, out of view, until someone opened the door.

  Since that time, the proprieter commissioned two peep-holes, so that people could see everywhere inside the room before he would open the gate.
\end{exampletext}

\paragraph{If the \glspl{pc} do not observe one peep-hole each,}
(and at the same time)
they will not notice the strange, ectoplasmic, creature floating patiently, and waiting to enter \gls{fenestra}.

Some rare Gnomish books have mentioned this creature, which they call `the archmage', due to its incredible command of magical spheres.
It has a brain, a few stubby tentacles, and nothing more.
Air spells allows it to fly, and Force spells allow it to hunt.

\paragraph{If the archmage escapes,}
it flies off, to some unknown location, probably destroying \pgls{village} along the way.
It will not harm anyone except as a means to escape.

\archmage

\showStdSpells

\mapentry[shadowPot]{Sticky Pot}

This ceramic pot has a thick seal of ropes and wax, and will not open easily.
The ooze which inhabits the deeper water \vpageref{shadowJelly} occasionally comes out, and leaves acidic mucus all over the pot, presumably sensing the contents.

Inside, rests a tiny little ooze-creature, under a \textit{Preservation} spell, ensuring it will not age so long as it remains still.
The temple \glspl{server} placed it here `in case of emergency', with a vague plan to make the entrance extremely dangerous if anything large entered.

\jelly[\npc{\A}{Ooze-in-a-Jar}\setcounter{r4}{2}]

\mapentry[shadowOaths]{Lady of Oaths}

\begin{exampletext}
  `Wouldn't it be great if people could just keep their promises?', is the kind of thing that people say when they don't think about the consequences.
  One alchemist, who thought he was being clever, created \pgls{artefact} which would listen to people's oaths, promises, or intentions.
  The \gls{artefact} still thinks people should keep their word, but like any \gls{artefact}, has no concept of context or priorities.

  So whenever anyone states their intentions next to her, she casts an \textit{Oath} spell, ensuring that they will keep their word.
  If they try, but struggle, she may well aid them with more spells, but if they try to avoid doing what they said, she turns to unfriendly spells.

  The creator abandoned the \gls{artefact} in frustration shortly after making it, then sold it to the Shadow Vault, where it stood as a visitors' attraction.
\end{exampletext}

\artefact[6]{Lady of Oaths}% Name
  {A statue of a brass woman, looking up and listening earnestly.
  The sign beneath says `tell me your intentions'}% Body
  {2}% Intelligence
  {0}% Wits
  {3}% Charisma
  {to ensure everyone finishes their mission, by casting an \textit{Oath} spell}% Mission
  {Doom Study}% Base Spell
  {
    \setcounter{Fate}{2}
    \setcounter{Water}{2}
    \setcounter{Empathy}{2}
    \setcounter{Wyldcrafting}{1}
  }% Spheres

\showStdSpells[
  \setcounter{diceNo}{0}
]

\mapentry[peepingOrb]{Sunken Light}

\begin{exampletext}
  The Peeping Orb loves to spy on Force spells, especially magical gateways.
  In fact, it was designed to do just this, in order to peep at the riddles that the Shadow Gate states, so the keepers could write them down.

  And once every three years, during the eclipse of Qualmea,%
  \exRef{judgement}{Judgement}{Qualmea}
  if an earthquake hits, if someone releases Fire and Earth \glspl{boon} around the Orb, it would show other images, with strange creatures that sit by distance gateways.
\end{exampletext}

\setcounter{wounds}{9}

\artefact[1]{Peeping Orb}% Name
  {A perfect, black sphere, made of cooled magma, the size of a head (but very light)}% Body
  {0}% Intelligence
  {-1}% Wits
  {2}% Charisma
  {to inform everyone of nearby Force spells}% Mission
  {Float}% Base Spell
  {
    \setcounter{Air}{3}
    \setcounter{Fire}{3}
    \setcounter{Earth}{1}
    \setcounter{Academics}{2}
    \setcounter{Xenomology}{2}
  }% Spheres

\setcounter{wounds}{0}

\showStdSpells[
  \setcounter{enc}{2}
  \findGatewaySpell
]

\paragraph{If the \glspl{pc} feel about,}
they will find the Orb.
They can take it away, and it will continue to show the state of any Force spells in range.
It watches them with fascination, and refuses to do anything else.

Right now, it sits underwater, access to any \glsentrylongpl{mp}, as it sits underwater, deprived of air.
And it desperately wants to know what is happening with the Shadow Gate (room \vref{shadowGate}).

\mapentry[shadowJelly]{Lowly Ooze}

An acidic ooze sits down here, minding its own business.
Occasionally a woodspy or basilisk enters to look around or hibernate, and the ooze begins to eat it.
And sometimes that creature escapes, and sometimes it does not.

The ooze has existed here for a long time, and has grown very large.

\jelly[\npc{\A}{Uncommonly Large Ooze}\setcounter{r4}{5}]

The \glspl{pc} may find the ooze a serious problem.
It's deadly, humongous, hungry, and fearless.
But like any ooze, it's stupid, so the \glspl{pc} can lead it out of the temple, and the only way it can find its way back is to feel its own slime-trail.

\boxPair[b]{
 \humanthief[\NPC{\M\Hu}{Goutscrape}{morose}{`Oh~well' \ldots \textit{sigh\ldots}}{to confirm every disappointment}]
  }{
  \humanmaid[\NPC{\F\Hu}{Fleasinge}{tubby, and smiley}{giggles}{to return to a lost home}]
  }

\paragraph{During cold seasons,}
the ooze rests under the ice.

\paragraph{If the large ooze makes contact with the smaller ooze,}
(area \vref{shadowJelly})
they envelop each other, then mate for \pgls{interval}, then the larger one explodes into ten copies of the smaller ooze.

Disturbing the pair during this time results in  both attacking at once.

\mapentry[shadowGate]{Shadow Gate}

The entrance to the Shadow Gate (\vpageref{shadowGate}) stands about as tall as a human, under the ice.
\Glspl{pc} won't see the entrance easily, and will have to smash the ice open to gain access -- \roll{Strength}{Crafts} (\tn[10])\ldots which of course releases the ooze.

Stairs along the causeway lead to two raised platforms, still blessedly dry.
Each has little walls to contain all the sand.
When the Shadow Gate wishes to speak, it uses a \textit{Warp Detailed Earth} spell to solidify and soften different parts of the sand, which spells out a riddle.

The Shadow Gate's last riddle remains on the floor, but this riddle has multiple stages, so once someone begins to answer the riddle, the sand will start to shift about.

\artefact[16]{Shadow Gate}% Name
  {A massive stone doorway to nothing, with a doormat placed in front}% Body
  {3}% Intelligence
  {0}% Wits
  {1}% Charisma
  {to ask devious riddles}% Mission
  {Fae Door}% Base Spell
  {
    \setcounter{Fire}{3}
    \setcounter{Earth}{2}
    \setcounter{Fate}{1}
    \setcounter{Water}{1}
    \setcounter{Academics}{2}
    \setcounter{Wyldcrafting}{1}
  }% Spheres

\showStdSpells[
  \setcounter{diceNo}{0}
]

\mapentry[shadowGuardian]{Sleeping Guardians}

\begin{exampletext}
  When the magical gateway shut, the local \gls{warden} tasked \pgls{doula} to resolve the situation.
  Like anyone in her position, she told an enchanting story about success, then delegated the task.

  Finding three people too young to know better, she arranged for some very fancy clothes, and declared them `Masters of the Riddle', and `Keepers of the Gate'.
  Then she sent them into an enchanted state, preserving their bodies, and setting their minds to focus, without failure, on the single task of solving the riddle which had vexed all of \gls{valley}.

  The first thought about how much he hated Maths.
  The second thought about visualizing the riddle as a series of rivers, each one representing a possible set of questions and answers.
  The third had already gone to sleep to dream of sex with his attractive neighbour.

  With everyone's attention focussed on the `Masters of the Riddle', the \gls{doula} went home, and died peacefully shortly after.

  The three rest in this hidden room to this day.
\end{exampletext}

\paragraph{Picking the lock}
of the rusted steel gate requires a \roll{Strength}{Larceny} roll (\tn[16]).

\paragraph{If the \glspl{pc} awaken any of the sleepers,}
that one wakes the others up before doing anything else.

The three `Keepers of the Gate' can explain everything concerning the reality of \gls{lostcity}, how the Shadow Gate once worked, and what happened, but they won't\ldots at least not for a while.
They take a long time to come to terms with their situation.

\NPC{\M\Hu}{Embersnatch}{lost in the clouds, still thinking about the last thing said}{\textit{`ya know, like?'}}{to return to a lost home}
\setcounter{Intelligence}{-2}
\person{1}% STRENGTH
{0}% DEXTERITY 
{0}% SPEED
{{-3}% INTELLIGENCE
{-2}% WITS
{0}}% CHARISMA
{0}% DR
{1}% COMBAT
{Crafts 1, Empathy 2}% SKILLS
{\Dagger}% EQUIPMENT
{}

\begin{speechtext}
  ``What's Ratpelt doing these days?''

  \textit{\raggedleft\adforn{15} She's dead.}

  ``Her mum won't be happy about that''

  \textit{\raggedleft\adforn{15} Her mum's dead.}

  ``Blimey. Her too?''

  \textit{\raggedleft\adforn{15} It's been two \underline{hundred} years.}

  ``Do I get birthday presents?
  My mum always gets me a whole chicken for my birthday, ya know?''
\end{speechtext}

The \glspl{pc} have a serious decision ahead: the three will need food, and to return to civilization.

\mapentry[shadowDesk]{Empty Desks}

\begin{exampletext}
  This room once had a steep stone staircase up to it.
  But with the guardians (\vpageref[above]{shadowGuardian}) sleeping peacefully, the proprietor didn't want leave them unprotected (even with the gate present).
  So he ordered men to build a wall over the entrance, leaving a gap at the top so they could squeeze out.
\end{exampletext}

The entrance stands at almost double the height of a human, and only wide enough to crawl through, so not many will spot it.
(and anyone with a Strength of +4 will have to remove their armour to fit through)

Once the \glspl{pc} pass by the entrance, they can make a single roll for both checks.

With \roll{Wits}{Crafts} (\tn[14]) they might notice the bricks are a little different around that wall.

With \roll{Wits}{Vigilance} (\tn[12]) they see the small entrance, above.

\paragraph{Rifling through the desk}
uncovers records of riddles\ldots hundreds of riddles.

\mapentry[memoryOrb]{Empty Desks}

\begin{exampletext}
  The Nuisance Orb once sat on display for everyone to see, as it would entertain people with various illusions.
  But after one too many accidents%
  \footnote{One rampaging basilisk, a false accusation of theft, and three accounts of indecent nudity.}
  the staff became irritated by it, and eventually locked it away.
\end{exampletext}

\setcounter{wounds}{2}
\artefact[2]{Nuisance Rock}% Name
  {A grey rock, covered by the illusion of a massive diamond.}% Body
  {-1}% Intelligence
  {1}% Wits
  {2}% Charisma
  {to create the most convincing illusions}% Mission
  {Phantasm}% Base Spell
  {
    \setcounter{Fire}{3}
    \setcounter{Earth}{2}
    \setcounter{Fate}{1}
    \setcounter{Water}{1}
    \setcounter{Academics}{2}
    \setcounter{Wyldcrafting}{1}
  }% Spheres

\showStdSpells[
  \setcounter{diceNo}{0}
]

\mapentry[shadowGuardian]{Sleeping Guardians}

\begin{exampletext}
  When the magical gateway shut, the local \gls{warden} tasked \pgls{doula} to resolve the situation.
  Like anyone in her position, she told an enchanting story about success, then delegated the task.

  Finding three people too young to know better, she arranged for some very fancy clothes, and declared them `Masters of the Riddle', and `Keepers of the Gate'.
  Then she sent them into an enchanted state, preserving their bodies, and setting their minds to focus, without failure, on the single task of solving the riddle which had vexed all of \gls{valley}.

  The first thought about how much he hated Maths.
  The second thought about visualizing the riddle as a series of rivers, each one representing a possible set of questions and answers.
  The third had already gone to sleep to dream of sex with his attractive neighbour.

  With everyone's attention focussed on the `Masters of the Riddle', the \gls{doula} went home, and died peacefully shortly after.

  The three rest in this hidden room to this day.
\end{exampletext}

\paragraph{Picking the lock}
of the rusted steel gate requires a \roll{Strength}{Larceny} roll (\tn[16]).

\paragraph{If the \glspl{pc} awaken any of the sleepers,}
that one wakes the others up before doing anything else.

The three `Keepers of the Gate' can explain everything concerning the reality of \gls{lostcity}, how the Shadow Gate once worked, and what happened, but they won't\ldots at least not for a while.
They take a long time to come to terms with their situation.

\NPC{\M\Hu}{Embersnatch}{lost in the clouds, still thinking about the last thing said}{\textit{`ya know, like?'}}{to return to a lost home}
\setcounter{Intelligence}{-2}
\person{1}% STRENGTH
{0}% DEXTERITY 
{0}% SPEED
{{-3}% INTELLIGENCE
{-2}% WITS
{0}}% CHARISMA
{0}% DR
{1}% COMBAT
{Crafts 1, Empathy 2}% SKILLS
{\Dagger}% EQUIPMENT
{}

\begin{speechtext}
  ``What's Ratpelt doing these days?''

  \textit{\raggedleft\adforn{15} She's dead.}

  ``Her mum won't be happy about that''

  \textit{\raggedleft\adforn{15} Her mum's dead.}

  ``Blimey. Her too?''

  \textit{\raggedleft\adforn{15} It's been two \underline{hundred} years.}

  ``Do I get birthday presents?
  My mum always gets me a whole chicken for my birthday, ya know?''
\end{speechtext}

The \glspl{pc} have a serious decision ahead: the three will need food, and to return to civilization.

\mapentry[shadowDesk]{Empty Desks}

\begin{exampletext}
  This room once had a steep stone staircase up to it.
  But with the guardians (\vpageref[above]{shadowGuardian}) sleeping peacefully, the proprietor didn't want leave them unprotected (even with the gate present).
  So he ordered men to build a wall over the entrance, leaving a gap at the top so they could squeeze out.
\end{exampletext}

The entrance stands at almost double the height of a human, and only wide enough to crawl through, so not many will spot it.
(and anyone with a Strength of +4 will have to remove their armour to fit through)

Once the \glspl{pc} pass by the entrance, they can make a single roll for both checks.

With \roll{Wits}{Crafts} (\tn[14]) they might notice the bricks are a little different around that wall.

With \roll{Wits}{Vigilance} (\tn[12]) they see the small entrance, above.

\paragraph{Rifling through the desk}
uncovers records of riddles\ldots hundreds of riddles.

\mapentry[memoryOrb]{Empty Desks}

\begin{exampletext}
  The Nuisance Orb once sat on display for everyone to see, as it would entertain people with various illusions.
  But after one too many accidents%
  \footnote{One rampaging basilisk, a false accusation of theft, and three accounts of indecent nudity.}
  the staff became irritated by it, and eventually locked it away.
\end{exampletext}

\setcounter{wounds}{2}
  {
    \setcounter{Fire}{2}
    \setcounter{Air}{2}
    \setcounter{Academics}{2}
    \setcounter{Crafts}{1}
    \setcounter{Empathy}{2}
    \setcounter{Deceit}{2}
  }% Spheres

\showStdSpells[
  \setcounter{diceNo}{0}
]

The Nuisance Rock cannot imitate people when locked in the little cell, and craves freedom more than anything.
It can spend 2~\glspl{mp} to `see' around itself, 2~\glspl{mp} to make a convincing illusion of something, or 2~\glspl{mp} to create convincing sound.

\paragraph{If anyone passes by,}
it creates an illusion of Goutscrape (area \vref{shadowGuardian}), begging to be let out of the cell.
The illusion will create realistic sound, and then the Nuisance Orb will have no further \glsentrylongpl{mp} to see anything, so it will make the illusion of Goutscrape lie down and sob inconsolably.

\Glspl{pc} can see through the illusion standing in the darkness with a \roll{Wits}{Vigilance} roll
(\tn[12]).

The Rock will do anything it can to leave this room, although its abilities remain limited (it must wait 2 more \glspl{interval} before it can cast another illusion, or make a voice).

\paragraph{If the \glspl{pc} take it with them,}
the Rock will cause constant chaos and misunderstandings.
If it ever speaks, it does so by making an illusion of someone giving advice about what to do with the Rock.

% the inner sanctum goes somewhere else, to a long dungeon.
% But won't the sleeping guardians give the game away?
