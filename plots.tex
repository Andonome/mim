\chapter{Protagonists \& Plots}
\label{sideQuestIntro}

\section{A Tapestry of Karma}
\label{sqSummaries}

\begin{multicols}{2}

% Theme: Karma

\noindent
On one level, this \gls{campaign} presents a farce where some people argue about the tax on a river, and monsters start eating them, because they're too busy arguing to notice.
On another level, the \gls{campaign} weaves together nine stories with one theme: the subtle line between karma and vengeance.

`\nameref{spiderSong}' begins with \gls{spiderqueen} killing people with her monstrous pets, then she feeds an entire hamlet to them.
She seems too powerful for anyone to kill her, but once she makes too many enemies, someone volunteers to help the \glspl{pc} kill her.
So the \glspl{pc} may become instruments of karmic vengeance.

While that thread is still in development, `\nameref{wolfHeads}' begins, and the \glspl{pc} meet a series of suspicious characters.
They may argue with one, become friends with another, or just ignore every troublesome character they meet.
Soon after, the \glspl{wolfhead} go out together and find the \glspl{pc} are in trouble; and so the \glspl{wolfhead} decide what counts as karma.
And long after the \glspl{wolfhead} help, hinder, or mock the \glspl{pc}, the situation reverses when the \glspl{pc} find the \glspl{wolfhead} suffering an attack on the road.
Vengeance on either side can lead to escalation.

But before either of those two threads resolve, the \gls{campaign} has another, and another.
Because whenever the \glspl{pc} don't head towards the next part of the story, BIND simply starts a new story.

% Map + regions

\subsubsection{The Three \Glsfmtplural{region}}
share some parts of their stories, as \glspl{segment} dart between \gls{sq}.
But each one contains a looming threat, which appears gradually, as a consequence of multiple \glspl{segment} coming together.


%%%%%%%%%%%%%%%%%%%% Map %%%%%%%%%%%%%%%%%%%%

\renewcommand\csComments{
  \mapCircle{16}{76}{1.7}{Dyson_Logos/bandit_camp}
  \mapCircle{35}{88}{2}{Dyson_Logos/forgotten_city}
  \mapCircle{27}{09}{2}{Dyson_Logos/qualme_temple}
  \mapCircle[4]{56}{52}{2.5}{Dyson_Logos/town}
  \mapCircle{44}{41}{2}{Dyson_Logos/redfall}
  \mapCircle{83}{09}{1.7}{Dyson_Logos/shadow_gate}
  \mapCircle{86}{45}{1.7}{Dyson_Logos/lochside}
  \draw[very thick,white] (11,0.6) -- (12,0.6) node[anchor=north]{\outline{10 Miles}} -- (13,0.6) ;
}

\mapNotes{
  \nameref{banditLair}/16/73,
  \nameref{lostcity}/35/86,
  \glssymbol{yonder}/35/94,
  \glssymbol{paik}/56/61,
  \Glsentrytext{town}/56/50,
  \Large\glssymbol{paik}/44/49,
  \glsentrytext{redfall}/44/41,
  \glssymbol{abderian}/83/13,
  \nameref{shadowVault}/83/05,
  \normalsize\nameref{silentGorge}/83/30,
  \glssymbol{sylf}/86/49,
  \gls{lochside}/86/45,
  \D/27/15,
  \glsfmttext{oldTemple}/25/08,
}

\widePic[t]{Irina/greylands}

\end{multicols}

\printAllSideQuests{Forest}

\begin{multicols}{2}
\noindent
The forest holds two active \glspl{fiend} who want to eat the outer \glspl{village}.
\Gls{necromancer} leaves his crumbling temple to kill people and grow his undead army.
\Gls{spiderqueen} emerges from her mobile fortress of webs, and wants to feed farmers to her \glspl{crawler} pets.
When \pgls{segment} shows this symbol -- \gls{vlg} -- it means \pgls{village} or \gls{broch} may fall to the forest, and the \gls{edge} advances another step inwards.

And if enough outer \glspl{broch} and \glspl{village} fall, all the \glspl{monster} of the forest will flood into the unprotected, interior farmlands.
\Gls{valley} would then become uninhabitable, and turn into a flood of refugees, all trying to reach some other settlement before \gls{cTwo} brings \glspl{snow}.

Making matters worse, the \glspl{wolfhead} want to poke around the forest in search of ancient alchemical gateways.
They, or the \glspl{pc}, will eventually find the remains of \gls{archwarp} and \gls{sixshadow} from \gls{lostcity}, which bring all new complications.

\end{multicols}


\printAllSideQuests{Roads}
\label{sqList}

\begin{multicols}{2}
\noindent
Each time the \glspl{pc} go to \gls{town} or the forest, they must pass through this middling \gls{region}.

\Gls{banditking}, the would-be \gls{warden} to \gls{valley}, hides in the \nameref{banditLair}, just past the \gls{edge}, and slowly gathers a small army of malcontents, mostly from the \glspl{guard}.
His messenger will also approach \glspl{pc}, who will have to decide whether or not to join his violent revolution.

Take a look at the thin crust of protective structures surrounding delicate fields on the map \vpageref{Irina/greylands}.

\begin{itemize}
  \item
  The little houses on the inside are scattered hamlets, each controlling large amounts of farmland, full of grazing livestock.
  \item
  The outer structures with multiple rooves are \glspl{village}, with tall walls to defend themselves from the constant probes from \gls{sylf}'s children.
  \item
  The structures with a single tower are \glspl{broch}.
  Each morning and evening, the \glspl{guard} of these tall towers play loud pipes to attract the forest's \glspl{monster} towards them, and away from the inner farmlands.
\end{itemize}

When \pgls{fiend} destroys any \gls{broch} or \gls{village}, you should take a pen, and score out the settlement.

\end{multicols}

\printAllSideQuests{Town}

\begin{multicols}{2}
\noindent
\Gls{town} will provide the \glspl{pc} somewhere to buy every kind of supply (check the handouts at \gls{town}'s entrance, \autopageref{townStart}), but a violent revolution is brewing from the bowels.

The \glspl{pc} may stop the \glspl{diggers}' plan to destroy \gls{town}'s \glspl{keeper}, or ignore them, or even help the \glspl{diggers}, and start to consider how \gls{town} could remain stable after the death of its \gls{warden}.
\end{multicols}

\bigLine

%%%%%%%%%%%%%%%%%%%%

% How to Run - starting base (link), missions, open-ended.

\begin{multicols}{2}
\subsubsection{Wandering Monsters}
should keep wandering throughout the \gls{campaign}, because their presence provides the \glspl{pc} with a reason to go out and hunt, and because their presence is the crux of the matter -- they are the doom which comes if \gls{valley} tears itself apart.

Most \glspl{segment} require insight and negotiation, rather than combat.
This helps the \glspl{sq} run alongside random encounters, without making dangerous combat-encounters repetitive.

The players may take a while to fully understand the grander political schemes threatening \gls{valley}, and even longer to come up with counter-schemes.
Bog-standard \gls{guard} missions can help kick-start journeys, as they often involve entering the forest.
\iftoggle{judgement}{%
  You can find some random tables to generate missions in the \textit{Book of Judgement}, \autopageref{NGmissions}.
}{}

% Factions, death, letting go.
\subsubsection{Player Interference}
is good.
\Glspl{segment} won't ask anyone's permission to \textit{start}, but once started, the \glspl{pc} actions' should have repercussions.
Every \gls{segment} could change.
For example, if \pgls{pc} somehow kills the \gls{spiderqueen} during the second \gls{segment} of her \gls{sq}, the rest of the \glspl{segment} would disappear instantly.
However, most \glspl{sq} will not fade away so easily, because most of the protagonists come from one of five political factions.
If \gls{banditking} dies, \gls{traitor} (or even his brother \gls{sewerking}) can take his place.

%!
Since the \glspl{pc} begin in the \gls{guard}, that makes them part of the establishment at the \gls{campaign}'s start.
Of course, this may change soon, as \gls{sewerthief} will gently inquire how the \glspl{pc} feel about joining the \glspl{whiteBandits}, or they may find themselves making concrete plans with the \glspl{wolfhead}.

\end{multicols}

\setglossarypreamble[people]{
  These are the movers and shakers, the doers and thinkers.
  You can find each of them in the \glspl{sq}, sometimes as primary actors, other times wandering into the background of multiple threads and \glspl{segment}.
}

\printglossary[
  type=people,
  style=topicmcols,
  title={Places \& Politics},
]

