\sidequest[town,Roads]{Desperate Measures}
\label{desperatemeasures}

\histEvent{0}{5}{\Glsfmttext{banditking} curses \glsfmttext{nurabaron} with strength and hunger}

\begin{exampletext}
  \noindent
  \Gls{banditking} approached \gls{nurabaron} and asked him to join `the new \glspl{warden}', with dramatically decreased taxes.
  \Gls{redfall} receives none of the benefits, as it doesn't sit along the primary river, so \gls{nurabaron} seemed like he was in the perfect position to be receptive.

  Unfortunately for \gls{banditking}, \gls{nurabaron} took offense to the idea of deposing \gls{townmaster}, and began threatening \gls{banditking}.
  \Gls{banditking} responded with a curse.

  \begin{speechtext}
    Eat all you can, but this place will not sustain you.

    The \gls{redfall} family will starve!

  \end{speechtext}

  This \glsentrytext{invocation} made \gls{nurabaron} and his family massive, muscular, angry, and clumsy.
  \Gls{banditking} took that moment to flee, leaving the \gls{redfall} \gls{warden} family cursed.

  Initially, \gls{nurabaron} enjoyed his `curse' -- he felt strong, and soon adjusted to his new size, and felt less clumsy.
  His family too, adjusted.%
  \footnote{The spell even affected his horse, which has caused no end of trouble among the guards, as the nightmarish creature can now barely fit inside its own stable, and breaks out any time it feels hungry.}

  \null
  The curse came with an extra-large appetite, which seemed fine at first.
  But a family of six, with a horse, eating four times the normal amount, means they consume the food of twenty-one extra people.
  And worse, the \gls{redfall} family began to enjoy food far more than they once did.
  In a small \gls{village}, the extra consumption quickly started to put pressure on the locals.

  \Gls{nurabaron} then ran into an additional complication -- his massive proportions had made him monstrous.
  He could not show his face to other \glspl{warden}, and found it difficult to write letters.
  All correspondence had to go through a single, well-lettered servant.
  \Gls{redfall} has been mismanaged for some time.

  The staff won't reveal their \gls{warden}'s secret.
  If he becomes deposed, all the \glspl{sunGuard} who serve him would lose their position, and would most likely join the \gls{guard}.
  All the inhabitants of \gls{redfall} know something has changed, but they have no idea that the \gls{warden} family have become monsters.
\end{exampletext}

The \gls{sq} \glspl{segment} below show the echoes of a mismanaged, isolated property, where the inhabitants starve, and the staff must keep their \gls{warden}'s secrets.

\sqpart{Roads}% AREA
{Bad Bandits}% NAME
{Starvation drives \glsfmttext{redfall} farmers to banditry and they attempt to rob the troupe}% SUMMARY

\Gls{redfall}'s starving inhabitants have taken to banditry, though they don't have the experience or the strength to do it properly.
They demand silver, or at least copper, but then quickly settle for any \glspl{ration} the characters might have.

\begin{boxtext}
  A single arrow hits the road ten feet in front of you with a dull thud.
  A man stands up from the bushes nearby saying ``Stand forth, and deliver!''.
\end{boxtext}

\paragraph{If the characters refuse,}
the bandits might shoot, but they're easily intimidated.
If the characters attack, the bandits flee.

\Person{\npc{\T[4]\Hu}{Emaciated ``Bandits''}\addtocounter{weight}{6}}%
  {{2}{-1}{-1}}% BODY
  {{0}{-1}{0}}% MIND
  {%
    \set{Crafts}{1}
    \set{Survival}{1}
    \Dagger
  }% SKILLS
  {}% KNACKS
  {(three have shortbows)}% EQUIPMENT
  {}% ABILITIES

\paragraph{If the \glspl{pc} investigate further,}
they may well end up at \gls{redfall} Keep.
In that case, \gls{seneschal} Thorn will ask for their help, as per \gls{segment} number \vref{nonstarter}.

\sqpart{Roads}% AREA
{Wrong Direction Chickens}% NAME
{The \glsentrytext{redfall} \glsfmtplural{warden} eat so much that they import chickens \emph{from} \glsentrytext{town}}% SUMMARY

\Gls{redfall} needs a lot of food to keep \gls{nurabaron}'s family fed, so they have started ordering more food.
Normally, \glspl{village} feed the towns, but in this case the town is feeding the \gls{village}.

\begin{boxtext}
  Light rain speckles the road, traders pass.
  All of them come from \gls{town}, so most trundle by with empty wagons, though one has a full cart of chickens in cages.
  The rain lets off just as the Sun sets, leaving everyone damp.
\end{boxtext}

Slip in the fact that a trader is travelling with chickens away from \gls{town} casually.
If the troupe notice, they can ask, and he'll tell them he's going to \gls{redfall} because he was paid a lot to do so.
Otherwise, just leave the clue dangling.

\sqpart{Roads}% AREA
{The Search for \Glsentrytext{forestpriest}}% NAME
{\Glsentrytext{nurabaron}'s \Glsentrytext{seneschal} asks the \glsfmtplural{pc} to find \glsfmttext{forestpriest} for secret mission}% SUMMARY
\label{nonstarter}

\begin{exampletext}
  Thorn has left to complete a trade-deal for pottery, and has to hurry back to \gls{redfall} to make sure the estate does not land itself in any trouble.
  He's heard that the renowned \gls{doula} -- \gls{forestpriest} -- sometimes sells her wares in \gls{town}, and he thinks she can help, but has no time to fetch her.
\end{exampletext}

Thorn sees the \gls{pc}-troupe on the road while travelling with a caravan, and asks them to deliver a letter to \gls{forestpriest}, in \gls{town} urgently.
He pays them 5~\glspl{sp} upfront, and tells them they will receive 10 more if they deliver it successfully, and accompany her.

He refuses to give his name.
The sealed letter reads as follows:

\null
\begin{speechtext}
  Renowned \pgls{forestpriest},

  I respectfully request your expedient services at \gls{redfall} estate.
  Some dark curse has transformed the family and one horse into ravenous beasts.
  We can pay whatever remains in the power of this diminished estate for your herbal cures, sufficient for a man, and (if immediately available) for four children and \gls{nurabaron}'s wife.

  The horse is also afflicted, and all present a danger, with their increases size and appetite.

  The family must be harmed \underline{in no way}.
\end{speechtext}

\thornSeneschal

\label{thorn}

\paragraph{If the troupe go to \gls{town},}
run the first available \gls{segment} as usual (because plans never go smoothly) but remember that \emph{this} \gls{segment} does not end until the troupe complete Thorn's mission, or abandon it.

\paragraph{Once in \gls{town},}
the troupe find \pgls{forestpriest} in her shop (\vpageref{greyDoulaShop}), ready to leave on a journey (her tea-leaves told her about a journey, but she doesn't know where she's going).
She has a lot of spare \glspl{ration} for the road.

\paragraph{Once at \gls{redfall} with \gls{forestpriest}}
(\vpageref{redfallKeep})
she insists on walking slowly, behind the \glspl{pc}, so she can avoid danger.
Once she sees the state of the \gls{redfall} \glspl{warden}, she plans to put them into an enchanted sleep.

\begin{speechtext}
  The curse affecting them will not lift until their bellies empty -- but that will take time.
  I need to find them calm, sleeping, and I can make them sleep for a while.

  You could tire them, hound them (but don't hurt them), until they become too tired to continue with any fight.
  Or perhaps fetch me some \glspl{ingredient}, and we could retreat, then return with a spell which will make everyone in the keep sleep.
\end{speechtext}

\Gls{forestpriest} remains open to any plans the \glspl{pc} come up with, but if they agree to find \glspl{ingredient} for her, she needs a Water and a Fate \gls{ingredient} to make a sleeping spell which will keep everyone in \gls{redfall} sleeping.
The spell must be cast while they are already asleep, but it works from far away, so the troupe simply need to wait until late at night.
Once she completes her \gls{casting}, the keep sleep soundly, and the \glspl{pc} can selectively wake the guards later, instructing them to be silent.

Of course, the entire keep now has the problem of keeping the noise level down, but that's Thorn's problem, and he can handle it fine.

\sqpart[\gls{vlg}]{Roads}% AREA
{Rumours of the Beast}% NAME
{People on the road chatter about `the beast' who rides at night by \glsentrytext{redfall}}% SUMMARY
On the road, people discuss sightings of the `giant rider'.
In fact people have seen \gls{nurabaron} riding his massive horse at night (he just rides for fun).

\paragraph{If the troupe kill or cure the \gls{redfall} \glspl{warden},}
skip the next \gls{segment}, and perhaps add some news about the aftermath (the players should hear about the consequences of whatever happened).

\sqpart[\gls{vlg}]{town}% AREA
{The Master's Bounty}% NAME
{\Glsfmttext{townmaster} discovers \glsfmttext{nurabaron} has become a beast, and puts a bounty on his head}% SUMMARY
\label{mastersBounty}

A guard at \gls{redfall} has fled, and informed the whole town that \gls{nurabaron} has turned into a monster.
\Gls{townmaster} immediately declares him an illegal entity.

\begin{speechtext}
    Hear ye! Hear ye!

    Oi! I said ``Bloody well listen!''

    \begin{itemize}
      \item
      The current price of dwarvish coin is to be lowered by a tenth of the current value.
      \item
      \Glspl{sunGuard} no longer allowed to urinate\ldots in public.
      Guards caught urinating in public, may be reported, to the local {\footnotesize guard station}.
      \item
      Honest work is to be found digging fortifications for the Newarp \gls{village}.
      \index{Newarp}
    \end{itemize}

    It can wait till I'm bloody-well finished, Sootboil.
    Shut it!

    Listen good to this one!
    \begin{itemize}
      \item
      \Gls{nurabaron} of \gls{redfall} Keep has turned-evil, become-a-depraved-monster, and is to be \underline{\large killed -- on -- sight}.
      \item
      His last known whereabouts is his own keep.
      Within this establishment, his own staff may be \emph{killed} on the basis that they harbour a criminal.
      \item
      All goods found therein are considered legal property by the finder.
    \end{itemize}
\end{speechtext}

Arkblow the crier knows nothing more than he's said.  A number of townsfolk quickly decide to take up arms and slay the local monster, hoping the ransack his house and loot anything of value.

Of course, the only way to put a stop to this is for the characters to find \gls{forestpriest} and bring her to the keep before the angry mob arrive, convince the mob that they have already cured \gls{nurabaron}, or somehow rush the \gls{redfall} family to safety.

\paragraph{Stalling the impending trouble}
before it starts could involve extolling laws (\roll{Strength}{Academics}), pleading (\roll{Charisma}{Empathy}), or anything else.
However they approach the problem, the \gls{tn} is 12, but they should get at least three rolls for different actions before the townsfolk stop listening.

\paragraph{Journeying to \gls{redfall} before the crowd}
won't be a problem if they have horses.
If no horses are present, a few traders will arrive before them, at the very least.

