\subsection{Archwarp Fallen}
\label{lostcity}
\mapPic{t}{Dyson_Logos/forgotten_city}{
  \ref{fallen_tower}/36/78,
  \ref{basementTrapdoor}/22/70,
  \ref{campfire}/28/70,
  \ref{lostGriffins}/73/80,
  \ref{shiningLake}/66/49,
  \ref{lostDoor}/56/35,
}

\begin{exampletext}
  Some centuries ago, a town known as Archwarp rose to prominence and wealth due to the magical portal inside, making it part of the Shattered City -- a location created by tying various portals into a single area.

  This particular spot merely held a town, and the entrance to the portal, seated inside an alchemical laboratory, and organization which would later become the \gls{paperGuild}.
\end{exampletext}

\begin{boxtext}
  The trees and vines here cling to blocks of stone, as if trying to crush the last remnants of civilization.
  You sometimes wander for so long without seeing a single brick that you forget you're anywhere but a normal forest, but then another errant stone, or the marble hand of a statue juts up to remind you of the ghosts resting in this place.
\end{boxtext}

\begin{exampletext}
  \noindent
  \Pgls{southSeeker} of the \gls{wolfhead} came here in search of the old alchemy library, and found it.
  However, he didn't dare enter the old door to the alchemy basement's library he found, so he hired a few of the \gls{guard} to enter, and check the barrow for safety.
  He planned to have them `split the loot', and then help them sell whatever valuables remained inside, and hoped they didn't see (or didn't understand) the portal.
\end{exampletext}

\begin{enumerate}
  \item
  This old, fallen tower supports the last roof (made of stone) in all of Archwarp.
  \Gls{southSeeker}'s bedding remains here.
  \label{fallen_tower}
  \item
  This hidden doorway has so much topsoil grown over it that nobody could imagine it was there.
  However, anyone in the basement below, in room \vref{alchemySecret}, may notice the underside, and dig their way up.
  \label{basementTrapdoor}
  \item
  The campfire shows evidence of a dozen people.
  The \glspl{pc} can roll \roll{Intelligence}{Vigilance} to understand what the various debris around the campfire means.
  \begin{nametable}{Campfire Investigation}
  \textbf{Roll} & \textbf{Result} \\
    7 & A dozen men were here. \\
    8 & They ate a griffin. \\
    9 & \glspl{guard}. \\
    10 & Only one person left the area. \\
    11 & Someone prepared griffin feathers to make \pgls{boon} over the camp fire. \\
  \end{nametable}
  \label{campfire}
  \item
  These blossoming fruit trees hold a trap: mouthdiggers.%
  \exRef{judgement}{Judgement}{mouthdigger}
  The amount of fruit depends upon the season.
  \begin{description}
    \item[Cold]
    seasons mean nothing but snow on the trees.
    \item[Hot]
    seasons see the tree nearly bursting with food.
    The trees hold 6 days' worth of food (which goes rotten within 2 days)
    and the mouthdiggers here have two litters of 6 sprogs each.%
    \footnote{Mouthdigger children can be used as \glspl{ingredient}.}
    \item[Mild]
    seasons mean the trees contain 4 days' worth of rations.

  \end{description}
  \mouthdigger[\npc{\T[3]\A}{\arabic{noAppearing} Mouthdiggers}]
  \label{lostGriffins}
  \item
  This glistening lake holds an acidic ooze.%
  \exRef{judgement}{Judgement}{ooze}
  However, during cold seasons, the rivers and lakes freeze over, leaving the ooze in a silent and pure contemplation.

  \jelly
  \label{shiningLake}
  \item
  This old basement door lay under the mud for some centuries, but it became a good deal more obvious once \gls{southSeeker} opened it.
  The \glspl{pc} can roll \roll{Wits}{Vigilance} (\tn[8]) to locate it.
  \label{lostDoor}
\end{enumerate}

