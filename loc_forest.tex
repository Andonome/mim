\section{Forest Locations}

\begin{multicols}{2}

\subsection[Forgotten Temple]{The Forgotten Temple}\label{necromancers_lair}
\index{Forgotten Temple}

\mapPic{t}{Dyson_Logos/qualme_temple}{
  \ref{antGrounds}/17/10,
  \Large S/28/46,
  \ref{antHall}/34/53,
  \ref{antBasilisk}/50/53,
  \ref{antPrison}/34/89,
  \ref{antCamp}/23/79,
  \ref{antOgre}/82/30,
  \ref{antTower}/85/67,
  \ref{antStudy}/17/44,
}

\paragraph{Background:}
Priests lived here at the side of a village, at the time acting as advisors to the nobility.
\Gls{necromancer} enjoyed his work, and studied hard.

Now the garden lies untended, but paradoxically has more life than it ever did when humans tended to the grounds.
The grass has grown long, the trees' fruit comes and goes according to the seasons, and the local area has become populated with a lot of apple trees.

\mapentry[antGrounds]{The Grounds}

All over the temple grounds, ghouls wander, or simply stand and stare
(depending upon which encounter the characters are on, this could be 50 -- 400).

A hundred broken arrow parts lie littered around the area, as \gls{necromancer} practices with his bow daily.

One tree has a large iron trap built for birds -- they can enter the two gates easily, but the little gates can only be pushed open one way.
\Gls{necromancer} uses these birds to see.%
\footnote{The undead cannot see by normal light, but can see by the light cast by souls, invisible to most people.}

\boxPair{
  \ghoul[\npc{\T[10]\D}{Ghouls}]
}{
  \humanfarmer[\NPC{\M}{Laith}{Pessimistic}{Mouth breather}{Acquisition}]
}

\mapentry[antHall]{The Hallway}

The shrine at the side of the hallway is composed of five skulls standing on top of each other, all raised on a pillar.

\paragraph{Reading the archaic writing}
requires an Intelligence + Academics check, TN 10.

Each skull has a female name, and a message thanking all those below for life.
Anyone saying a prayer of gratitude to their matrilineal lineage in front of the item gains $1D6+3$ FP.%
\footnote{The item has 9 MP in total, and spends 2 MP to cast the spell.}

\mapentry[antBasilisk]{The Preserved Basilisk}

\paragraph{Background:}
\Gls{necromancer} felled a basilisk recently.
He worries about raising it as a ghoul, as such a larger creature might cause havoc, and he does not know if his prayers are up to the task.

It remains here, with a \textit{Preservation} spell to ensure the corpse does not rot.
He is slowly building armour around it, to create an unstoppable ghoul, but has not completed the armour (it will fall off after taking 2 hits if the creature wakes up).

\mapentry[antStudy]{The Secret Study}

When the bandits are away, \gls{necromancer} sneaks back to his private study, through a stone door, balanced on massive iron hinges.
It contains three cages with birds, scattered throughout the room.

The books and scrolls consist of:

\begin{itemize}

  \item
  Prayers to the dead (lots)
  \item
  A song-magic item (i.e. a magical song) which calls all the local woodland creatures to attack the singer.%
  \iftoggle{aif}{\footnote{See \autopageref{medalofheroism}.}}{}
  \item
  Old letters praising all the new food passing through the area, as `our under-cousins' do business.
  \item
  Letters noting the best routes from one village to another.
  These are a clue as to the original layout of \gls{lostcity}, and also confirm that the city never stretched across the entire area, but simply contained a number of villages.
  \item
  A workshop bench with various pieces of armour on it, including a complete suite of chain.
  \item
  A small chest containing \lootMedium.

\end{itemize}

\mapentry[antPrison]{The Prison}

This room once housed people making important decisions.  \Gls{necromancer} now uses it to house prisoners so he can feed off their souls.
Currently, it contains one terrified farmer called Laith.
He's starving, and petrified, as every day all he can hear are the shambling dead, who sometimes come to grope at the locked door.
\Gls{necromancer}, or course, holds the key.

Picking the lock requires an Intelligence + Larceny roll, TN 7.

Laith can join the characters if given a weapon, but he won't be terribly effective.

\mapentry[antCamp]{The Abandoned Camp}

\paragraph{Background:}
The Immortal Bandits stayed here for a while, as \gls{banditking} made arrangements with \gls{necromancer}.
They had to leave once the place became too full of undead, but their bedding and old firepit remain.

\paragraph{Investigating the bedding}
shows a pair of gloves, bedding, and other standard items which distinctly come from Whiteplains, although understanding what this means requires and Intelligence + Crafts roll (TN 12).

\mapentry[antOgre]{The Undead Ogre}

\Gls{necromancer}'s prized specimen -- an undead ogre.
\Gls{necromancer} killed the ogre some time ago, and pulled the body back to undeath.
Since then, \gls{necromancer} has cobbled together leather armour to glad the oversized ghoul in.
Now he stands humongous and impenetrable.

\npc{\D\N}{Undead Ogre}

\person{6}% STRENGTH
{0}% DEXTERITY
{0}% SPEED
{{0}% INTELLIGENCE
{-4}% WITS
{-5}}% CHARISMA
{2}% DR
{2}% AGGRESSION
{Deceit 1}% SKILLS
{\greatclub, \completeleather}% ABILITIES
{}%

\mapentry[antTower]{The Watchtower}

This tower once hosted a call to prayer the nearby village could hear.
Now \gls{necromancer} uses it to practice his longbow daily, or shoot at intruders.

A large stockpile of arrows sits at the side of the little room.

\pic{Decky/necromancer}

\thenecromancer

\subsection{\Glsentrytext{forestpriest}'s Camp}
\label{lostcity}
\mapPic{t}{Dyson_Logos/forgotten_city}{
  \ref{fallen_tower}/33/73,
  \ref{lostGriffins}/77/75,
  \ref{lostManaLake}/70/35,
  \ref{lostDoor}/63/21,
}

\paragraph{Background:}
A city once stood in this area, evidenced by the stone ruins all around.
In this particular spot, a small group of alchemists gathered study material and crafted experiments; it was the prototypical \gls{college}.

\Glsentryfirst{forestpriest} has wondered for a long time about how \gls{lostcity} fell -- did elves destroy it out of malice, or did they destroy it because humans in the area began opening portals to the dangerous nura realms?
He has found the area by scouting in bird form for many months, all around the area, and determined its function.

One barrier to \gls{forestpriest}'s question remains; the alchemical basement swarms with undead.

\Gls{forestpriest} first tried to trick some of the Woodspy Bandits inside to find out the truth (he told them the place contained a lot of treasures).
Unfortunately, they never came out again.

\begin{boxtext}

  The trees and vines here cling to blocks of stone, as if trying to crush the last remnants of civilization.
  You sometimes wander for so long without seeing a single brick that you forget you're anywhere but a normal forest, but then another errant stone, or the marble hand of a statue juts up to remind you of the ghosts resting in this place.

  In the distance, a little plume of smoke rises, suggesting a camp fire.

\end{boxtext}

\paragraph{If the PCs head towards the smoke}
anyone with MP will feel it recharging by the second.

\mapentry[fallen_tower]{The Fallen Tower}

\Gls{forestpriest} camps here alone, in the abandoned tower.
It still has a single room, with a stone ceiling standing strong, where a small tapestry showing a prayer to Laiqu\"e displays on the wall.
This simple tapestry with a children's rhyme (saying thanks to all the animals -- even the dangerous ones) functions as a shrine to Laiqu\"e.

\Gls{forestpriest} keeps spare torches at the side.

The nearby hidden trapdoor underground cannot be discovered from above -- in fact it's so tightly covered with rock, then earth, then well-made roofing, that it's nearly waterproof.

\begin{speechtext}

  The humans say that the elves tore the city down because the elves disliked humans growing so powerful.

  The elves say that the humans used nasty magics to open portals to the nura realm, and goblins spilled out.
  They say the city was already mostly ruined, and they had to save the area through magic.

  I feel the answer lies close.

\end{speechtext}

\paragraph{Once the PCs approach,}
\gls{forestpriest} replies with a friendly greeting, sizes the party up.
He won't tell them the plan, or how to find the hidden entrance to the alchemical laboratory unless he trusts them.

\paragraph{If the PCs encounter trouble with any beasts,}
\gls{forestpriest} can almost certainly deal with them using his knowledge of Aldaron.

\paragraph{If the PCs wish to donate money to the shrine,}
those following Laiqu\"e can gain XP as usual, and \gls{forestpriest} will take the money to local villagers.

\mapentry[lostGriffins]{Griffin Grove}

Two griffin couples live in this bountiful grove, and are tending to their nests (assuming the current season is not cold).
The trees are full of large, ripe fruits.

\griffin[\npc{\T[4]\A}{Four Griffins}]

\paragraph{If the PCs approach during daylight hours,}
the griffins perk their heads up, but they can be threatened with a Strength + Wyldcrafting Team Roll (TN 10) by shouting and banging weapons together).

\paragraph{If the PCs enter to get the food,}
the griffins attack together.

\mapentry[lostManaLake]{The Glistening Mana Lake}

This lake blossoms with magical energies, and regenerates 2 MP per turn to anyone nearby, and 4 MP to anyone touching it.%
\iftoggle{aif}{%
  \footnote{See \textit{Fenestra}, \autopageref{mana_lake} for more on mana lakes.}
}{}

Four woodspies rest just under the water's surface, camouflaged to look exactly like the base of the pool.

\paragraph{Anyone getting close to the lake}
must make a Wits + Vigilance roll, TN 10, to spot them.
Failure, of course, means being dragged under water by the first.
Anyone helping must roll or be dragged underwater by the second.

\paragraph{If \gls{forestpriest} walks with the characters,}
he can use his Aldaron magic to immediately put \emph{one} Woodspy to sleep by spending 2 MP, but he won't waste further mana on the task.

\paragraph{If the party want to rush past the woodspies without fighting,}
get a Speed + Athletics Group Roll (TN 8, plus 1 per person after the first).

\Gls{forestpriest} warns them to shout when they want to exit, so he can make sure the door is clear, otherwise the woodspies will be waiting for them the moment they exit.

\mapentry[lostDoor]{The Old Door}

The old door has brass braces and such solid oak that it has survived through centuries of rot and nastiness.
Despite the Woodspy Bandits recently opening the door, it has such a covering of mud that spotting it requires a Wits + Vigilance Team Roll (TN 10).

\begin{speechtext}
  This old door leads to the basement of some old alchemical research building.
  If a portal to the nura realm exists, it exists here.
  I need to know if it really does contain such a portal.

  If you could go down, we could know for certain what the answer is.
\end{speechtext}

\paragraph{If the party ask for payment,}
he says he's already paid all the money he can to local bandits to find out what's in the well, and they never returned.%
\footnote{Specifically, these were the Woodspy Bandits.}
Therefore, there should be plenty of silver on their bodies, if the party can only find the corpses.

\paragraph{If the party ask him to come with them,}
he'll only go down with the party if he has to, and if he trusts them, but will not allow himself to be surrounded by people he doesn't trust and could stab him in an instant.

\forestpriest

\paragraph{If the party ask him to clarify the plan,}
he explains that if he can get witnesses to spread the word about what really happened in \gls{lostcity}, he will have the political backing to put a stop to expansions into this area.

If men return here to uncover the ruins containing portals, they may repeat their past mistakes.
If the elves have destroyed the area in the past, they may do so again.

\paragraph{If asked about turning people into animals,}
he refuses to speak on the subject until after the contents of the old wooden door have been thoroughly inspected.

\subsection{The Old Alchemy Basement}
\label{old_alchemy_basement}

The entire basement of the old magical laboratory is sodden with water, resting knee-height to a human.

\begin{boxtext}
  Descending the stairs, you find a low ceiling, and a moment later correct yourself.  It's not a low ceiling -- black, stagnant, water has flooded the entire hall.

  The torch picks up a great stone pillar in the distance, and another a little farther along.
  Great double doors along the hall, to the right.

\end{boxtext}

\paragraph{Wading through the water}
is difficult.
All movement is limited by 2 squares minus the character's Strength Bonus (minimum of 2), so some will receive no penalty, while those with Strength -2 receive a 4 square penalty to movement each round.
For many, this will mean they cannot move at all without spending a full round pushing forward, or simply swimming.

Remember to note who has torches when underground, and that carrying a torch in one hand means the character is effectively duel-wielding.

\paragraph{The cold water and foetid air}
inflicts 2 Fatigue Points per scene.

\paragraph{All doors}
have swollen due to the many years of water-logging.
Opening them requires a Strength + Crafts Team Roll (TN 7).

As usual, each roll remains as-is, so if someone fails to open a door they will need to find a way to gather more strength, or give up opening that door.

\paragraph{Narrow hallways}
make wielding long weapons challenging.
The \textit{Enclosure Rating} for this place is 5, so any weapon which requires 6 Initiative to wield takes a -1 penalty to Strike.

\paragraph{The Dead Chant} when not in combat.
If they stand at the other end of a hallway, they chant.
If the characters lock them in a room, the dead stand outside and chant while clawing at the door.

This strange behaviour could vex any necromancer.
The simple spirits which inhabit and animate ghouls do not usually speak.
Their strange behaviour is the result of a powerful necromancer living in the catacombs.
The undead necromancer%
\footnote{See room \ref{undead_ogre}.}
felt the words engraved on top of the magical portal which can open a portal to the Realm of Darkness and Fire.
There was only one problem: the creature's tongue was too rotten to speak the words properly.
Its body has only been preserved due to the high content of peat in the water, but that was not enough to allow it to speak properly, so it tried teaching the dead to chant.

The words themselves simply mean `open to trade', but the characters will hear only ``Opena trei, opena trei, opena trei!''.

\paragraph{Cave-ins} present a real danger here.  If the ceiling ever collapses while the characters are inside, the falling rocks from above at first deals $1D6-2$ Damage to everyone in the room, then $1D6$, and so on, increasing by 2 each round, until it's unliveable.

\mapentry[alchemyHall]{Drowned Hallway}

\paragraph{Background:}
When \gls{lostcity} was still burning from the nura attack some centuries ago, one necromancer raised a powerful undead spirit into the body of an ogre.
That ogre demi-lich then used the corpses around him to raise a regiment of ghouls.
The door was sealed during that time, peat-filled water flooded in, and the dead rested there, perfectly embalmed and perfectly still.

\paragraph{When the party enter,}
they move over an uneven floor, and some of the debris below them are ghouls.

The ghouls' stiff bodies move slowly, and by the time the characters have all moved into the hall, the dead rise between them, separating members of the troupe.

Five ghouls rest at the start of the hallway, and another five later on.
A further five at the other end of the hallway begin walking towards them immediately.
The entire situation makes for the perfect ambush, though the dead have not planned for it.

\ghoul[\npc{\T[15]\D}{15 Ghouls}]

\paragraph{If the party attempt to fell any pillars,}
have them roll Strength + Crafts, TN 13 (or less, if they use the right equipment, such as rope).
Once the pillars fall, the entire area collapses within two rounds.

\paragraph{If anyone searches the bodies,}
they find one of the ghouls which were once Woodspy Bandits has \gls{forestpriest}'s payment of \lootMedium.

\mapPic{b}{Dyson_Logos/under_lost_city}{
  \Large A/18/37,
  \Large B/42/37,
  \Large C/30/18,
  \huge S/355/18,
  \huge\rotatebox{90}{S}/73/71,
  \huge S/64/86,
  \ref{alchemyHall}/18/84,
  \nameref{alchemyHall}/18/93,
  \rotatebox{45}{\nameref{alchemyEquipment}}/14/61,
  \ref{alchemyEquipment}/20/59,
  \rotatebox{25}{\nameref{alchemyLibrary}}/46/76,
  \ref{alchemyLibrary}/52/72,
  \rotatebox{45}{\nameref{alchemyRooms}}/29/41,
  \ref{alchemyRooms}/31/37,
  \rotatebox{-45}{\nameref{alchemySecret}}/76/88,
  \ref{alchemySecret}/71/83,
  \rotatebox{45}{\nameref{alchemyGift}}/66/31,
  \ref{alchemyGift}/60/39,
  \rotatebox{-90}{\nameref{summoningRoom}}/93/58,
  \ref{summoningRoom}/87/58,
}

\mapentry[alchemyEquipment]{Equipment}

\paragraph{Background:}

The standard alchemist's equipment -- gold dust, rubies, beechwood, chitin, and black soil -- have mostly been removed from the area during the panic when people fled.

Some of those panicked people returned and were dragged back into the portal, only to return as ogres.
Those ogres were resurrected as undead, along with everyone else.

\begin{boxtext}

  The shelves in this wide room are full of smashed and broken equipment, but it looks generally alchemical.

\end{boxtext}

\begin{boxtext}

  A powerful force grabs your ankle, and squeezes.
  A creature, taller than any man, stands up and turns you upside down, then pulls you in towards his teeth.

\end{boxtext}

\npc{\T[3]\N\D}{3 Undead Ogres}
\animal{6}{-3}{0}{-3}{3}{2}{}{Death Sight}{}

\paragraph{Characters who scour the room,}
can find rare gems on the ground, although wading through all the sludge will not be pleasant.
This requires a Dexterity + Academics roll (TN 6) to correctly identify coinage, and what items might be useful, by feel.
Each marginal point means 15sp worth of items has been found.

\mapentry[alchemyLibrary]{Library}
\index{Library!Ancient}

\begin{boxtext}

  These two doors stand locked and refusing to budge.  A simple brass lock stands rusted on the front.

\end{boxtext}

Opening \emph{this} door requires a Strength + Crafts roll, TN 10.

\begin{boxtext}
  The doors throw inwards, revealing row upon row of rotten books.
\end{boxtext}

\paragraph{Careful perusal of the books,}
allows a few items to be discovered.
Anyone searching joins the Group Roll of Dexterity + Academics, TN 6.
Each margin allows a particular book to be carefully extracted, but destroyed the book above, so rolling `9' means the valuable book on Invocation is found, but the letter is destroyed.
Destroyed books simply fall apart due to rotten spines and damp pages, but if taken back and cared for, some could be preserved.
The party can make any number of rolls, but if they want to use a resting action, each new roll requires a new scene, and the cold water around them will make them fatigued.

\begin{enumerate}

\setcounter{enumi}{6}
  \item
  A book detailing a nearby treasure, guarded by a dragon and her two children.
  The mother leaves for a holiday every Alassea.
  A greedy dwarf penned this complete fabrication during the War of Lies.%
  \exRef{aif}{\textit{Fenestra}}{warOfLies}
  \item
  An ancient city map, detailing sites of interest such as a Temple of Qualm\"{e}, which holds beautifully decorated, and unaging corpses (see page \pageref{green_tower}).
  \item
  A valuable book on Invocation, worth 20sp.
  \item
  A letter stating that a portal to an unknown labyrinth realm has been opened, and that trade has opened with various dwarves in exchange for food.  It also states which word will activate this portal.
  \item
  Letters of complaint from the Dean of Conjuration, stating that the Dean of Illusion must tidy his room, and that the rats he's brought in have become so bad that he's ordering no food to be permitted in the area, under any circumstances.
  \item
  A valuable book on late-stage Conjuration, worth 100sp, covering the fifth level of the Conjuration sphere.
  \item
  Threatening letters from elves saying to be wary of opening portals.
  \item
  Hidden behind some other books: a book on Nuramancy.  This is highly illegal, but allows anyone to gain up to a single level in Nuramancy with a little study, and the right (or wrong) attitude.
  \item
  A book on opening portals to distant places using a clever combination of the Force and Conjuration spheres.
  \item
  A letter granting permission to open a portal to the Realm of Darkness and Fire, with the hopes of trading magical items for food.
  This proves the humans were opening dangerous portals!

\end{enumerate}

\mapentry[alchemyRooms]{Dark Pit}

\begin{boxtext}
  These doors swing open effortlessly, showing a new room with three more doors; right, left and centre.

\end{boxtext}

\paragraph{Background:}
Three rooms here used to house the various masters of alchemy.
Stairs reached down to a lower floor, then back up.
Since then it flooded, though the water will not show that.

One of the ogres in this area was raised as a ghast, and the spirit inhabiting that body knows powerful necromantic magic.
It tried to teach the dead how to say the password to open the door, and resurrect the Woodspy Bandits who came in earlier.

Mostly, it simply sits at the bottom of the pool.

\paragraph{If anyone steps into the water,}
they need to pass a Dexterity + Caving check (TN 10).

Success means the ghastly ogre reaches out, and attempts to grab them.
Failure means they fall into the water, \emph{then} get grabbed (with a -2 penalty to resist the attack).

\npc{\M\N\D}{Ghastly Ogre Mage}
\label{undead_ogre}

\person{6}% STRENGTH
  {-3}% DEXTERITY
  {0}% SPEED
  {{2}% INTELLIGENCE
  {0}% WITS
  {-5}}% CHARISMA
  {3}% DR
  {1}% COMBAT
  {Aggression 2, Academics 2, Xenomology 3, Medicine 1
  \knacks{\nuraCaster}
  \Path{\necromancy~4, \saurecanta~3}}% SKILLS
  {\longsword}% EQUIPMENT
  {\addtocounter{xpbonus}{3}}

\paragraph{Room A} used to house the master of Conjuration, who built the portal in room \ref{summoningRoom}.
The ogre has kept him around for his own amusement, as a ghoul.
He still has \lootBig\ in his pockets.

\paragraph{Room B} houses nothing but broken furniture and sludge.
The last room, however, is different.

\paragraph{Room C} used to house an illusionist, and his spells are still going ever since he died.
Instead of cleaning his room, he would simply cast an illusion of cleanliness.
The room looks immaculate, and full of light.

\begin{boxtext}
  The heavy door creaks open to an attractive room, like an expensive upstairs room in a tavern, complete with a bed, a study, and a freshly cooked breakfast on the table.
\end{boxtext}

Within the room, under the comfortable-looking (but filthy) bed, is a hidden little tunnel, which leads up to a secret room.
The ogrish mage cannot follow the characters here, even if he wanted to put up with the irritating light, because he is far too large to fit through the narrow opening.

\mapentry[alchemySecret]{Secret Study}

Up the stairs the area remains dry, safe and eventually leads to a regular door (no roll required to open it).
Inside, the room contains tables with extremely old scrolls, dust, and a series of very out-of-date books on alchemical theory.

\begin{boxtext}

  The stairs reach up, and finally you step your muddy boots out of the water and along a cold, but dry corridor.

\end{boxtext}

\paragraph{Resting here}
causes no Fatigue Points, as the place is not full of cold water.

\paragraph{Reading the old language}
requires an Intelligence + Academics roll (TN 10).
The scrolls say this:

\begin{exampletext}

  I shall see you by Laiquea.  Have the portal completed.  We have no funds.  Five lands mapped.

\end{exampletext}

\begin{exampletext}

  Some funding came through.  They want mutton, beef, bread and soup.  Everything must be prepared before sending, except the meat.

  Prepare the food.  Destroy this letter.

\end{exampletext}

\begin{exampletext}

  The portal has been established.  Negotiations are going well, but please have more guards available than last time.  Excuses aside, we can't have a repeat of the last incident.  Three women.  It doesn't sound good in song.

  Of course if you want my advice we would put every bard in the kingdom to the sword and be done with the matter.

\end{exampletext}

A Wits + Crafts check, TN 10, reveals a loose wooden board in the ceiling.
It used to be an exit to the ground floor of the Citadel above,%
\footnote{See area \ref{fallen_tower} in `\nameref{lostcity}', page \pageref{fallen_tower}.}
but now the upper floor is just the ground outside\ldots after a lot of digging upwards.

\mapentry[alchemyGift]{Giftschrank}

\paragraph{Background:}
This bare room used to store various books, including the words which open the portal.
It's flooded, like every other room on its level.

The two skeletons on the table have aged worse than the other corpses, as they were never preserved in the peat-water.
They died of hunger rather than facing the dead they knew to be outside.
One holds a book of poetry, and the other holds a book of conjuration which she never managed to understand before dying.

\begin{boxtext}

  The bricks fall away easily, revealing a full new room.  Two skeletons rest on a table, each clutching a book.

\end{boxtext}

\paragraph{The book of conjuration}
is outdated, but still worth at least 20gp to \gls{alchemists}.

\paragraph{The book of poetry is pleasant,}
and hides one spell-song -- a poem which still functions to stop the user fearing any type of problem and regenerates 1D6+1 FP (it holds 3 mana, and costs 2 to cast).

\paragraph{Finding the words which unlock the portal}
requires an Intelligence + Vigilance roll, TN 9.
It's hidden among a dozen rather dull books on proper etiquette with alchemy, and accountancy books concerning what the guild brings in and what is can produce.

The door to the Summoning Room is only blocked by a crude wooden panel, so exiting only requires a kick

\paragraph{Spotting the hidden door from the outside}
requires a Wits + Vigilance (TN 8).

\mapentry[summoningRoom]{Summoning Room}

\paragraph{Background:}
This is where the magic happened.
When anyone said `open to trade', the portal came to life and allowed a trade of foodstuffs for magical items and knowledge.

The language is old but an Intelligence + Academics roll, TN 9, will allow anyone to understand it.

The dungeon's necromancer (in room \ref{undead_ogre}) has laid a trap for anyone entering this room.
He chained ten ghouls to each of the front pillars.


\begin{boxtext}

  The massive double doors slowly swing inwards, and the torchlight reveals a flooded hallway of six stone pillars, two enclaves, and a stairway leading up to a stage.
  The stage shows a grand stone arch, like a doorway, leading to darkness.
  You can see an writing across the top.

\end{boxtext}

\paragraph{If any of the players say the words out loud,}
so do their characters; the portal opens and the ghouls in the room begin chanting along with them in unison.%
\footnote{As usual, speech costs 2 Initiative points, so if the ghouls are in combat once the words are spoken, the party should enjoy the unexpected advantage they get.}

\paragraph{If the PCs remove the gemstones in the portal,}
it breaks forever.
The gems are worth 30gp in total.

If undiscovered, the dead stand and begin their chant, then slowly walk towards the characters.

\paragraph{After 2 rounds,}
the dead stand up.

\begin{boxtext}

  You look behind, and note two-dozen dead men standing from the water and staring at you.
  Their skin has gone brown with age, and they look barely able to move.
  Each drags a chain behind it, tied around one of the entrance pillars.
  They pull together towards you, each uttering the same strange, chanting moan, and then stop as the chains go tight around the stone pillars.

\end{boxtext}

\paragraph{If a pillar has six or more ghouls pulling at it,}
then it collapses after 3 rounds.

\paragraph{If a group of ghouls grab a character,}
they stop pulling at their chains and focus on attacking that character.

\paragraph{If the characters open the portal,}
they see a dark room, with a distant light.
What might be less obvious is that the portal opens on the \emph{ceiling} of a room in the Realm of Darkness and Fire.
Anyone throwing an item in notices it flies, then `sticks' to the far `wall' (meaning, the ground).
Characters may notice the discrepancy from the odd appearance of the doors on the other side, with a Wits + Crafts roll, TN 10.

Ten hobgoblins immediately arrive with a ladder and start making their way up, into the dungeon.
They know the portal can open, and they know they need a password.
They fight, but try to keep the characters alive so that they can learn the magic word which opens the portal (they cannot read).

\paragraph{If the characters drop through the portal,}
you're on your own.

Perhaps they will survive a while in the nura realm.
Perhaps they will make it back through a different portal.

But probably not.

\subsection{\Glsfmttext{greentower}}
\label{green_tower}

\mapPic[\Large]{t}{Dyson_Logos/green_tower}{
  \Huge\ref{greenGrounds}/62/55,
  \ref{greenConservatory}/26/35,
  \ref{green1}/10/35,
  \ref{green2}/23/85,
  \ref{green3}/84/84,
  \ref{greenTop}/90/24,
  \ref{underGreenTower}/01/35,
  \ref{underWatcher}/44/02,
  \ref{underPriests}/50/02,
  \ref{underDoors}/58/02,
  \ref{underSpiral}/68/07,
  \ref{Abyss}/65/00,
}

A Temple of Ohta once stood here, but was destroyed with the rest of \gls{lostcity}.
Now \gls{townmaster} has sent masons (members of the Woodspy Bandits), to build a base of operations for him to begin rebuilding the old human city.

However, everyone building here is unaware that the lower parts of the temple are still active.
While \gls{lostcity} still stood, the priests of Ohta wanted to make their own trade-routes with the nura realm, but eventually fell to war, just like everyone else.
The priests of Qualm\"e (who were close with the temple's priests) volunteered to remain, and suppress the horde which came up from below.

The priests still fight -- every few years the nura make some attempt to journey up through the eternally-open portal, and the golden priests of Qualm\"e push them back, to protect a civilization they have no idea has fallen.

\mapentry[greenGrounds]{Outer Grounds}

The area around the tower contains piles of rock which labourers have collected from the surrounding ruins.

\paragraph{If anyone in the party casts spells,}
they noticed immediately that there is a magical hum of energy in the area.
Everyone here regenerates 3 MP per scene.

The source is a mana lake, slightly underground.
Any spell-caster with depleted mana can easily locate where the energy is strongest by simply wandering about, and eventually identifying a large rock which blocks the path underground (and has done, for decades).
The rock has a Weight Rating of 8, so a total Strength of +4 is required to lift it.%
\footnote{Of course another option is a lot of time, digging, or the proper use of a lever with an Intelligence + Crafts roll (TN 8).}

\humanfarmer[\npc{\T[6]\M\Hu}{6 Masons}]

\mapentry[greenConservatory]{Conservatory}

Overnight, the labouring equipment rests here.
The lock is a simple knot tied on the inside, and anyone slipping a knife inside can get in (Intelligence + Larceny, TN 5).

\mapentry[green1]{First Floor}

Basic straw-stuffed beds, clothes, and some bows lie here.

\mapentry[green2]{Second Floor}

The men sleep here, though it's eventually planned as a station for lower-level archers.

\mapentry[green3]{Third Floor}

The top floor provides a place for \gls{traitor}, overseer of the operation, to get a good look at the surrounding area.
This is also where he keeps a stockpile of weapons:

\begin{itemize}

  \item{20 longbows}
  \item{50 longswords}
  \item{50 suits of partial leather armour}

\end{itemize}

\mapentry[greenTop]{Top Floor}

This green roof meshes with the local trees for some distance around, but like any other camouflage, it seems obvious once you look at it.

From the top of the roof, anyone can see for miles around.
They can't see people or creatures sneaking about, but they can see basilisks and groups of men without fail, even by moonlight.

\mapPic{b}{Dyson_Logos/green_secret}{
  \rotatebox{15}{\nameref{underGreenTower}}/09/55,
  \ref{underGreenTower}/17/43,
  \rotatebox{-15}{\nameref{underWatcher}}/09/02,
  \ref{underWatcher}/18/09,
  \rotatebox{15}{\nameref{underPriests}}/20/78,
  \ref{underPriests}/28/57,
  \rotatebox{0}{\nameref{underDoors}}/40/02,
  \ref{underDoors}/375/31,
  \rotatebox{-15}{\nameref{underSideRoom}}/68/85,
  \ref{underSideRoom}/56/73,
  \rotatebox{-90}{\nameref{underSpiral}}/90/43,
  \ref{underSpiral}/81/40,
  \rotatebox{25}{\nameref{Abyss}}/67/49,
  \ref{Abyss}/68/36,
}

\mapentry[underGreenTower]{The Stairs}

\paragraph{Background:}
Years of growth and soil-spillage have left a thin layer of mucus on the stairs.

\paragraph{Anyone descending}
must make a Dexterity + Athletics roll, TN 8, or fall down one staircase, taking $1D6-2$ Damage.
Each of the three staircases require a different roll.
Characters can get a bonus for proper equipment, such as rope.

The air down in the tomb has become so dry and foetid, that anyone spending time there gains 2 Fatigue Points per scene.
Of course, this can be offset with sufficient rest (or just leaving the place to air out for a few days).

The stairs and all hallways here are rather narrow, making weapons difficult to swing.%
\footnote{%
  \iftoggle{core}%
  {\nameref{enclosedcombat}.}%
  {See the Core rules on Enclosed Spaces.}%
}

\mapentry[underWatcher]{The Watcher}

\paragraph{Background:}
The tomb is guarded by a Pious Ghast who stands in a side-cupboard, sworn to guard the golden priests.
He has spent a long time down here in the dry air, and his body so seized up that he cannot move.
A little time, however, will allow him to regain the use of his limbs, and open the door.

\paragraph{Opening the door}
requires a Strength + Crafts roll at TN 14, as the door is bolted shut from the inside, and will have to be busted in.

\paragraph{If a player insists on `picking the lock'}
(despite the lack of lock) they roll Intelligence + Larceny, TN 16.

\begin{boxtext}

  You swing the door open to find a highly decorated corpse with a pendant to Qualm\"e, made of gold, with bone strung along the copper thread.
  The clothes look like they were once silk, and the helmet's leather covering seems to be a human face, stretched out and tanned.

\end{boxtext}

\paragraph{If the PCs stab the corpse,}
they can kill it.

\paragraph{If they pass the door,}
the Pious Ghast animates and comes to kill them.
He is intelligent enough to know to sneak.
His use of a shortsword also means he will not gain any penalty from the narrow hallway.

If the characters enter with the Torpor spell cast, or rings of asphyxiation, they will be able to pass the Pious Ghast and the Golden Priests invisibly, so long as they do not linger too close to any of them.

\npc{\M\D}{The Pious Ghast}

\person{3}% STRENGTH
  {2}% DEXTERITY
  {0}% SPEED
  {{0}% INTELLIGENCE
  {0}% WITS
  {-5}}% CHARISMA
  {2}% DR
  {3}% COMBAT
  {Academics~1, Stealth~2, Tactics~2, Vigilance~1}% SKILLS
  {\shortsword, \completeplate, 200sp worth of jewellery}% EQUIPMENT
  {}

\mapentry[underPriests]{Hall of the Golden Priests}

\paragraph{Background:}
A long time ago, onlookers came here to gawk at the splendour of a glorious afterlife.
The priests of Qualm\"{e} who gained the highest honours of the temple would remain here to guard it forever.
Each is decked in golden jewellery.

\begin{boxtext}

  Five dead men, mummified and covered in golden jewels, stand in each of five enclaves at the side of the room.
  You notice head wounds and missing limbs upon some of the bodies.

  Little spears litter the area, as if a battle has taken place here.

\end{boxtext}

\boxPair[t]{
  \demilich[\npc{\D\F}{Demilich Delilah}]
}{
  \demilich[\npc{\D\M}{Demilich Jonah}]
}

\demilich[\npc{\D\F}{Demilich Tamar}]

\paragraph{Once the PCs enter,}
the demiliches cast hostile spells at them.

\mapentry[underDoors]{Exposed Doors}

\paragraph{Background:}
These openings once had doors to hide the secret items of the Temple of Ohta, but the doors have since shattered and broken from centuries of warfare.
Primitive spears litter the area.

\mapentry[underSideRoom]{Side Room}

\paragraph{Background:}
The natural tunnel was further excavated in order to horde the priests' treasures.
At present, this room contains \lootMedium, \lootBig, and \lootMagic, all held in a small chest at the side of the room.%
\iftoggle{aif}{%
  \footnote{The coins date the room to the year of Rex Hunter.
  See \textit{Fenestra} \autopageref{r_hunter}.}
}{}

The room also contains 14 undead hobgoblins, standing ready to kill any nura who come up from the tunnel, or just anyone who enters.

\undeadhobgoblin[\npc{\T[14]\N\D}{14 Undead Hobgoblins}]

\mapentry[underSpiral]{Downward Spiral}

\paragraph{If the troupe venture down the hallway,}
they find four more undead hobgoblins in the tunnel.
Their job is to open the door and jump on any nura climbing up, but they will attack anyone on sight.

\undeadhobgoblin[\npc{\T\N\D}{4 Undead Hobgoblins}]

\begin{exampletext}

  Throughout the long years, the golden priests directed the stupid undead to dig and dig into the hillside.
  They looped round, and dug into the tunnel, then fashioned a door from the broken wood from previous battles.

Any nura digging upwards can now be attacked by the undead, who hurl themselves down to attach anything climbing up, before one of the golden priests shut the door again.
While several nura would normally burst through any door, opening a stuck door while climbing can be almost impossible.

\end{exampletext}

\mapentry[Abyss]{Abyss}

Characters entering this area feel a warm breeze coming from the abyss, and intense magical energies.
The abyss itself is a third-level mana lake, constantly radiating magical energy.%
\iftoggle{aif}{%
  \footnote{See \autopageref{mana_lake} for more on mana lakes.}
}{}

\paragraph{If they somehow get down the hole,}
they will find a skull embedded in the wall -- the head of a devout warrior.
His head -- carved with the story of his victories -- functions as a permanently-open magical doorway.

\magicitem{The Portal Summoner}% NAME
  {Gate}% SPELL
  {Devotion (Ohta)}% PATH
  {continuous}% DURATION
  {Artefact}% TYPE
  {3}% Potency
  {5}% MP

\paragraph{Destroying the Portal Summoner}
is easy if one can get close enough.
However, getting close to it will almost certainly alert the nura below, in their citadel.
At this point, heavily armoured hobgoblins will attempt to climb up to stop the vandalism.

\hobgoblin

\begin{exampletext}

  The priests of Ohta who opened the door to the Realm of Darkness and Fire%
  \iftoggle{aif}{%
    \footnote{See page \pageref{darknessandfire} for more on that realm.}
  }{}%
 were not stupid.
 They created the door through a magical artefact and placed that artefact down a deep chasm, so they could better defend their location in case of attack.
  What they had not counted on was the sheer number of nura who could climb up from below.
  They clamoured up in their hundreds, with spiked gloves to climb more easily up the soft cavernous walls.

  When the nura turned against the city, the two portals spewed hobgoblins, ogres, and the like, at the same time.
  The priests argued, and considered caving the entire temple in.
  They prayed to Ohta, and saw a vision of the temple's destruction, with a hole which sucked in more and more earth, getting wider and wider, until nura filled the land.

  At this point, the priests of Qualm\"e decided to take over, and remain, guarding the temple forever.
  They had their servants wrap them in cloth while warriors fought against the nura.
  They each took vows to never eat, open their eyes, or listen to music again.
  All five became demiliches, sworn to serve their temple forever.

  As the nura came up, the golden priests killed them with magic, then raised the bodies from the dead to fight against more nura.
  This has left the situation in a stalemate.
  Nura come up from the tunnel, they die, the golden priests raise them as ghouls to fight until all nura lie dead.
  This repeats every few years.

  So far, two of the golden priests have met their final end.
  Their remains have been placed back in their chambers by the others.

  Throughout these centuries, barely a dozen goblins have managed to escape and get into the lands above, and almost all died of starvation before finding anywhere above.

\end{exampletext}

\paragraph{If the troupe attempt to understand this area,}
they will no doubt find it confusing -- as well they should.
They will not find communication with the gold priests easy.
However, the demiliches are willing to communicate, and one can send telepathic thoughts with the Enchantment sphere.

\paragraph{If two of the golden priests die,}
the nura kill the other and break out.
Raise the local nura rating by 3.

\end{multicols}
