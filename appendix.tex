\glsunsetall

\section*{Parting Note}

There is no ending here.

I've run this campaign, start-to-finish, three times over the years.
Each time it changes a little, and always has a different ending.

By the time the players have made their decisions, and their characters have a reputation among the factions moving across the \glspl{sq}, they could be anywhere, so I'm afraid I can't help you there, except to suggest taking stock of who's left alive, what resources they have, and what they want.

Perhaps \gls{banditking} has no troupe left and has to make a last, desperate attempt at lowering the taxes in \gls{whiteplains} by murdering \gls{townmaster}'s heir as a warning to anyone who would overburden \gls{whiteplains} again.

Perhaps \gls{banditking} arrives with a small army, and offers to clear the citadel of ghasts in order to gain the favour of everyone in \gls{town}, then begins to sweet-talk \gls{town}'s guild leaders before making a play for leadership.

Perhaps the \gls{wolfhead} and \gls{whiteBandits} join forces and both get what they want.

Or perhaps all the primary characters died at the hands of the \gls{guard}, and this campaign comes to a halt far before the last \glspl{segment} can start.

However it turns out, happy gaming.

\section{Books}
\label{bookAppendix}

\begin{multicols}{2}

\subsection{A Complete Explanation of the \Glsfmttext{deep}}
\label{bookOfDeep}
\index{Book!\Glsfmttext{deep}}

This massive tome's disorganized chapters, unreferenced references to other books, and dense diagrams make it a pain to read.
However, it promises a complete account of \gls{deep}.

The entire book tells lies from start to finish.

If players ask what's inside, tell them it holds a lot of information, and they should simply ask.
If their question ends in a vowel, say `yes', and if their question ends in a consonant, say `no'.

\begin{description}
  \itshape
  \item[Do goblins live in the \gls{deep}?]
  No, the book states that the goblins have their own realm, unrelated to the \gls{deep}.
  \item[Do the creatures down there just carry lights all the time?]
  Yes -- the book explains that all creatures have some source of light, or carry torches.
\end{description}

Every two questions will take an entire \gls{interval} to answer.
And if answers contradict each other, so be it.

Little by little, the troupe can create a complete image of how the \gls{deep} works, all of it fictional.

\subsection{An Account of the Fingers}
\label{bookOfFingers}
\index{Fingers, The}
\index{Book!Fingers}

`The Fingers' can be found inside the Vore Gate (area \vref{shadowVore}).
These brickwork tunnels show signs of advanced ageing.
The first tunnel extends for five minutes (the book measures distance in minutes' walked), which then splits left and right.

Once someone goes beyond this split, they sometimes return, and sometimes do not.
Those that return, never take any light with them; therefore whatever danger lurks in the tunnel must come towards light.
And everyone who returns walks silently, which means that the think which lurks in the fingers comes towards any noise.

Through long trials, recording every success and failure, people mapped out tunnels.

Some have returned with precious metals -- items made of platinum, or steel.
Nobody can tell if the creator intended to make the items for some purpose, or simply as art.
However, a great deal of metallic items, stacked on shelves, in the dark, threatens to make a noise; any sliding, or drop will create loud bangs.
So taking anything from this `treasure room', usually results in death.

Everyone who enters puts a single, silent, finger on the walls, and walks forwards, following the tunnel laid out for them.
Hence the name, `the Fingers'.

\end{multicols}

\indexprologue{%
  \noindent
  By the Gnomish calendar, the current \gls{cycle}%
  \exRef{judgement}{Judgement}{astronomy}
  is 6071.
}

\printindex[history]

\printindex[talismans]

\printindex

