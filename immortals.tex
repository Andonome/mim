\resumecontents[Town]

\stopcontents[sq]

\sidequest{Immortal Bandits}

\stopcontents[Town]

\startcontents[sq]

\sqminitoc

\noindent
The Immortal Bandits are a bandit ring with a difference.
Their magic rings make them semi-undead, and they sometimes use undead horses, which cannot gallop, but never tire, so they can raid for many miles around their haven.%
\footnote{See page \pageref{necromancers_lair} for more on their foetid living conditions.}

\sqpart{Villages}% AREA
{The Deal}% NAME
{The Immortal Bandits request help}% SUMMARY

\textbf{Background:}
The Immortal Bandits have gotten in a fight with the \gls{guard}, and both sides have left in a bad state.
The Woodspy Bandits heard the battle, came to investigate, then started following them.
They don't like the idea of two bandit rings in the area, so they plan to follow the Immortal Bandits quietly, and kill them in their sleep.

\begin{boxtext}

  As you top the next hill, a group of six men wheel round and ready their bows at you, before their commander calms them.

\end{boxtext}

\Gls{sewerking} has noted he's being followed, so when he sees the party, he acts friendly, pretends to be a member of the Weapons Guild from Whiteplains, and hopes they can travel together until the danger has passed.
He explains the situation while telling them not to look behind (``the bandits might notice you've spotted them!''), and asks for their help in formulating a plan.

\paragraph{If the \glspl{pc} ask more about his background,}
he tells them his current business is top-secret.

\humanarcher[\npc{\T[6]\Hu}{6 Immortal Bandits}]

\sewerking

\humansoldier[\npc{\T[12]\Hu}{12 Woodspy Bandits}]

\paragraph{If the party agree to accompany the Immortal Bandits,}
then \gls{sewerking} stays with them for a day, but makes sure they do not follow him all the way to their secret home, and avoids town at all cost in case someone recognizes one of his men.

\paragraph{If the party push for a fight,}
they will have one.

\iftoggle{core}%
  {To easily resolve fights between \glspl{npc}, see the \textit{Judgement} book, \autopageref{npcfights}.}
  {}

\sqpart{Villages}% AREA
{Fallen Traders}% NAME
{A number of traders have fallen to bandits}% SUMMARY

\begin{boxtext}

  A dead caravan lies ahead, with dead horses in front, and dead men at the side.  Every wagon, person and horse has been filled with arrows.

\end{boxtext}

Here the bandits have been again, and removed all the goods they could -- food, beer, clothing, and some swords.

The Night Guard still hunt for the Immortal Bandits, but the bandits appear randomly, and then disappear just as quickly into the depths of the forest.
This time they've been careless with their tracks.
The party can follow them with a Strength + Wyldcrafting roll, TN 8, if they have the time and supplies for a journey of many days.

\sqpart{Villages}% AREA
{Hidden Eyes}% NAME
{Bandits watch the characters from the side of the road silently}% SUMMARY

\Gls{banditking} and twenty men hide in the dense trees on a hill, a mile away from the road.
They watch, quietly, for traders.
A single man, Inkbane, sits closer to the road, waiting to call like a pigeon if someone rich wanders onto the road, or call like a crow if an armoured troop wander along.

\begin{boxtext}

  The road through over to the hamlet is quiet, with only rustling trees, and a crow cawing in the distance.

\end{boxtext}

Nobody will bother the characters, unless they look both rich and unarmed.

Spotting the fake call requires a Wits + Vigilance roll, TN 10.

\paragraph{If the characters capture Inkbane,}
he claims to be a hunter, but spotting the lie is only TN 8 on a Wits + Empathy roll.
If they come for the rest of the bandits, they flee on horseback.

\NPC{\M}{Inkbane}{Brash}{Licks lips}{warm soup}

\person{2}% STRENGTH
{0}% DEXTERITY 
{1}% SPEED
{{0}% INTELLIGENCE
{-1}% WITS
{-1}}% CHARISMA
{1}% DR
{2}% COMBAT
{Deceit~1, Wyldcrafting~2}% SKILLS
{\shortsword, dagger, \partialleather, ring of asphyxiation \iftoggle{aif}{(see \textit{Fenestra}, page \pageref{ring_asphyxiation}).}{}}% EQUIPMENT
{}

\sqpart{Villages}% AREA
{The Feast is Cancelled}% NAME
{The Immortal Bandits have raided \pgls{village} during a wedding}% SUMMARY

\textbf{Background:}
Jameson has announced that he will track down and personally behead the Immortal Bandits.
He works in the \gls{guard} as a scout, and enjoys the patronage of his father -- \pgls{village} master.
Unfortunately the Immortal Bandits heard of his plans, and know where he lives.

\begin{boxtext}

  The village looks like some kind of inverted funeral.
  A woman stands in a beautiful beige dress at a wooden alter to V\'{e}r\"{e}, consoled by a man wearing the traditional bright wedding hat.
  Three dead men lie at the altar's feet, including a priest.
  A crow attempts to land on one of the dead men, but a local man lunges at it, then gives chase, as if trying to chase the crow into the sky.

\end{boxtext}

The bandits are ten miles away, and can still be tracked with a Wits + Wyldcrafting roll at TN 10.
If the \glspl{pc} catch up to them, the bandits run away.

\paragraph{If anyone asks about why this happened,}
the villagers all tell them about Jameson's recent bravado.

\sqpart{Villages}% AREA
{The Dead Tracker}% NAME
{A member of the \glsentrytext{guard} sent to track down where the dead have come from has returned as a ghoul}% SUMMARY

\textbf{Background:}
Anderson -- a scout in the \gls{guard} -- was hired to track down the bandits.
However, \gls{sewerking} (on one of his rare trips to the temple in the forest) spotted him, cast a spell to turn his sword into stone, then stabbed him. 
Finally, he raised the corpse from the dead, but this is where things went wrong.
He failed to control the spirit in the body, so it fled into the empty night.

\ghoul[\npc{\D}{Ghouled Tracker}]

\paragraph{If they try to follow the ghoul's tracks back to where it came from,}
have them roll Wits + Wyldcrafting (TN 11).

\paragraph{If the party enquire with the \gls{guard},}
they will be told where Anderson was wandering to search for \gls{necromancer}, which should help them narrow the necromancer's location down.

\sqpart{Villages}% AREA
{The Showdown}% NAME
{The \glspl{pc} catch the Immortal Bandits resting}% SUMMARY

\textbf{Background:}
The Immortal Bandits killed yet another trader, taking his supply of meats, and his corpse.
But this time, they got sloppy, and left some evidence of their crimes.
Then they wandered beyond the \gls{edge}, into the deep forest.
When the light vanished, they thought it was night and lit a fire.
But outside the forest, the Sun is still setting, so the smoke from their fire can be seen easily from the road.

Have the party roll Wits + Vigilance.

\paragraph{If they roll 10 or more,}
they spot a horse's hoof in the bushes, in the distance, a short walk from the road.
It leads them to the rest of the horse, and a blood-spattered cart, which once carried fresh meats and bread.

The bandits tidied up the tracks well.

\paragraph{If they roll 8 or more,}
they spot the bandits' resting-spot.

\begin{boxtext}
  By the light of the setting Sun, you see smoke coming out of the forest beyond the \gls{edge}.
  It looks like someone has a campfire.
\end{boxtext}

\paragraph{If the \glspl{pc} follow,}
it will take them over an hour to walk, but locating the bandits won't be much trouble as long as they don't lose their baring in the dark.
Request a Wits + Wyldcrafting roll (\gls{tn} 7) to track the bandits, and a Dexterity + Stealth roll (\gls{tn} 8) to remain unnoticed.

\banditking

\humansoldier[\npc{\T[6]\Hu}{6 Bandits}]

\humanarcher[\npc{\T[6]\Hu}{6 Immortal Bandits}]

Play the next encounter only if the \glspl{pc} take \gls{banditking} in alive.

\resumecontents[Town]

\sqpart{Town}% AREA
{Bandits Caught}% NAME
{The bandits who plagued the countryside have been imprisoned}% SUMMARY

\stopcontents[Town]

\textbf{Background:}
The \glspl{pc} have captured \gls{banditking} alive.

\begin{boxtext}

  Hear, ye!  Hear all!

  Bandits who roamed the highways, lead by a man known as \gls{banditking}, have been apprehended.  The leader shall be drawn and quartered by week's end, and his companions hanged that night.

  Bakers are henceforth forbidden from purchasing the flour of the Quennome region, and any found doing so will be charged with consorting with elves.

  The temple of Alass\"e invites any charismatic men or women to aid the festivities, as playwrites and attractive actors are required for the upcoming festivities (not you, Margaret!).

\end{boxtext}

\Gls{banditking} will not be killed by law enforcement.
Those in \gls{pig} will inform \gls{sewerking} long before, and the rescue will commence as the bandits in the sewer storm the guards' holding.
Meanwhile, if \gls{necromancer}'s lair survives, the other bandits await instructions there.

\humansoldier[\npc{\T[10]\Hu}{10 Bandits}]

The only way for the characters to secure \gls{banditking}'s demise it to watch the guards' station all night.
If they do so, ten bandits stage an attack during the night.

\paragraph{The Distraction} starts by using ogre dust in three places in town to distract the local guards%
\iftoggle{aif}{(see \textit{Fenestra}, \autopageref{ogredust} for Ogre Dust)}{}.

The attack begins once there are only ten guards left in the station.
\Glsentrytext{sewerking}'s men arrive, ready to break everyone out underground and take off into the nearest entrance to the sewer.

\stopcontents[sq]


