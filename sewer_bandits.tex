\sidequest{Sewer Bandits}
\label{sewerking}

\startcontents[sq]

\sqminitoc

\noindent
\Gls{sewerking} of Whiteplains has built an underground lair, and is using it to build a team of bandits.
Unfortunately, the nura influence below starts to spill upwards.

\sqpart{Town}% AREA
{\squash\N Bad Water}% NAME
{The town's main spring smells disgusting}% SUMMARY

Deep underground, all the changes made in the secret sewer lair have unsettled the earth.
As a result, \gls{town}'s water has gone foul.

\begin{boxtext}

  You stop at a nearby fountain, as everyone does in \gls{town}, but the water taste's suddenly foul.
  The rest of the night, others make the same complaints about the rotten taste.

\end{boxtext}

Raise the local Nura Rating by 1.
Play this encounter at the same time as the next Side Quest part.

\sqpart{Town}% AREA
{\N The Nura Child}% NAME
{A street urchin transforms into a nura creature}% SUMMARY

\Gls{sewerthief} was delivering nura-enchanted food to \gls{townmaster}, so that \gls{townmaster} could use it to poison enemies in the future.
En route, a street-child stole a piece of dried meat, and turned into an ogre.

\begin{boxtext}

  Screams erupt nearby and people flee.
  Around the tavern's corner, a monster walks out.
  It stoops to pick up a piece of beef one of the people had dropped, then stares at it with large, innocent eyes.

\end{boxtext}

Characters can make a Wits + Medicine roll, TN 9, to notice that this monster is in fact a kid who's simply been afflicted by dark magic, and turned into a small ogre.

\paragraph{Curing the child won't be easy,}
but sufficient research at a Temple to C\'{a}l\"{e} shows that it can be done through starvation (Intelligence + Academics, TN 8).%
\iftoggle{aif}{
  \footnote{See \textit{Fenestra}, \autoref{nura} for more.}
}{}

\paragraph{If the characters want to ask the child about where he stole the food on the spot,}
they can make an Intelligence + Empathy roll, TN 11, to find which direction \gls{sewerthief} went.

Raise the local Nura Rating by 1.

\npc{\M\N\Hu}{James, Street-Child}
\person{3}% STRENGTH
{1}% DEXTERITY 
{2}% SPEED
{{-3}% INTELLIGENCE
{0}% WITS
{-3}}% CHARISMA
{0}% DR
{0}% COMBAT
{Projectiles 1, Larceny 1}% SKILLS
{Whatever he can pick up and throw at you}% EQUIPMENT
{}

\paragraph{If taken alive,}
James will be able to describe \gls{sewerthief}' face, but couldn't say who he is.
James has only been a nura for a couple of hours, so if the party know to starve him, he can return to being a normal child before long.

\sqpart{Town}% AREA
{\N Streetbrawl}% NAME
{The local alcoholics are on the street, and fights are breaking out}% SUMMARY
\label{ghastEscape}

\Gls{sewerking} recently learned how to summon more intelligent and powerful undead creatures, but did not manage to control the spirit he pulled into the corpse.
Havoc broke out, and \gls{pig} is closed until further notice.
\footnote{See page \pageref{pig_pantry} for details on what happened.}

As a result, all the alcoholics who have been chucked out of every area in the city are out in full swing.
The rest of the night is full of random people brawling.

\begin{boxtext}

A drunken man with uncomprehending eyes stares at you, then shouts.

\begin{quotation}

  Oi! Pointy-eared freak.  The freakshow left last week.  They decide to leave you behind?

\end{quotation}

\end{boxtext}

Trev's not happy, and he's taking it out on the group.
Pete doesn't know the group, but he's decided he's going to speak up for them and kick the crap out of Trev.
However the PCs react, a fight will break out around them.
Once the fight's ended, the characters might think they're out of harm's way, but the streets are rampant with trouble.

\humanfarmer[\npc{\T[4]\Hu}{Irate Alcoholics}]

The Whiteplains thieves in the sewers have a real fight on their hands, as the ghast freed some of the less intelligent undead, and they don't have enough rings to remain safe from the undead.

\paragraph{If the characters approach \gls{pig},}
give them a Wits + Vigilance check  to notice the distant sounds of a fight (TN 10 to notice, TN 12 to figure out the sounds come from below the Private Room).

\paragraph{If the PCs come to the rescue,}
\gls{pigowner} and the rest immediately lie about the situation:

\begin{speechtext}
  We always used the start of that tunnel to flush away waste underground, and now we find the dead are coming up!

  The underbelly of \gls{town} has been flooded for as long as anyone can remember, so perhaps something nasty lurks down there.

  I'm going to barricade the tunnel, then the door!
  I won't have this happening again, I can tell you!
\end{speechtext}

\paragraph{If the PCs venture down the tunnel,}
\gls{pigowner} invites them, but conspires to have some of the Immortal Bandits ambush them.

\sqpart{Town}% AREA
{\N\N Unexpected Ghouls}% NAME
{Hobgoblin ghouls from below emerge and attack the town}% SUMMARY

Twenty of the undead from the sewer have escaped through the butcher's place%
\footnote{See area 5 on \gls{town}'s map.}
because the bandits who live down there were simply not careful enough.
If the characters confront the dead head-on, they will have a bad time, six of the town guard are only a few streets away.

\begin{boxtext}

  Screams erupt next door.
  Feet move quickly, and you see three men being pulled to the ground by a silent mob of massive, naked men.
  The mob pulls the men inside but makes no sound, and then the screaming stops.

  Another steps closer, and you can see these men and women have died and rotted some time ago.
  Their white eyes look at you with intense interest, then start to walk towards you.

  In the far distance, the town guard can be heard shouting to keep the noise down.

\end{boxtext}

Any character can summon the nearby guard with a Strength + Empathy roll, TN 7, to shout out loud.
The guards take 5 rounds to arrive, but every Margin on the roll reduces that time by 1 round, to a minimum of 2 rounds.

\ghoul[\npc{\T[20]\D}{20 Ghouls}]

\paragraph{At the end of the day,}
the ghouls will be put to the sword one way or another, but the question remains; what brought them here?

\paragraph{If the party track the ghouls,}
they won't find the city streets help much, and the bandits in the sewers will start to clear up the butchers as soon as the horde have gone.
The roll is Wits + Vigilance (TN 14).

Increase the local nura rating by 1.

\sqpart{Town}% AREA
{\N Underground Assassins}% NAME
{The bandits in the sewer cut \Glsentrytext{captain}'s throat}% SUMMARY

\textbf{Background:}
\Gls{captain} has been pushing the investigation into the undead and nura sightings in town, so \gls{sewerking} ordered a hit on him, lead by \gls{sewerthief}.
Five men walked casually about him as he returned home \gls{whitehorse} with his wife, then closed in so one could stab him in the neck.

\begin{boxtext}

  You hear guards shouting ``After them!'', in the distance, and quickly scurrying feet, as a woman shouts for someone to help her wounded husband.

\end{boxtext}

\paragraph{If the characters stay to help the wounded man,}
they find \gls{captain} with a knife-wound, next to his wife.
The roll is Wits + Medicine, at TN 9 to save his life.

\paragraph{If they run after the thieves,}
the PCs make a Group Roll of Speed + Athletics.%
\iftoggle{core}%
  {\footnote{See the core rules, page \pageref{grouproll}, for Group Rolls.}}%
{}%
Remember that whoever's trying to patch up \gls{captain}'s bleeding neck won't be able to join the chase.

\begin{tcolorbox}[tabularx={cX},top=10pt,bottom=10pt]

  Roll & Result \\\hline
  12 & \textit{``Giving chase, you catch up to four men running from the scene of the crime.''} \\
  11 & \textit{``You run round an alley, and find a drain cover clanking. The assassins have jumped underground.''} \\
  9 & \textit{``You run in hot pursuit, but the attackers have disappeared down a street, into thin air.''} \\
  7 & \textit{``The attackers sprint away, leaving you running in the dark.''} \\

\end{tcolorbox}

\sewerthief

\humanthief[\npc{\T[4]\E\Hu}{Four of the Sewer Thieves}]

\paragraph{If the party follow the assassins underground,}
they run to the nearest entrance -- perhaps the butchers or \gls{pig}.
Go to page \pageref{sewers}.
Otherwise, this incident will remain a mystery.

If \gls{captain} survives, he has little idea of what's happening, although a little investigation could reveal what he's been asking about recently (Intelligence + Vigilance Teamwork Roll, TN 10).
Raise the local Nura Rating by 1.

\sqpart{Town}% AREA
{\N\N\squash Rise of the Immortal Bandits}% NAME
{\Glsfmttext{sewerking} releases ghouls upon the citadel}% SUMMARY

\textbf{Background:}
\Gls{sewerking} undead have finally finished digging the tunnel
\footnote{See page \pageref{citadelTunnel} for the underground tunnel.}
underneath \gls{town}'s citadel,
\footnote{See page \pageref{citadel} for the town's Citadel.}
where \gls{townmaster} lives.
He releases the undead, and guides them upstairs, through the dark halls.
As the \gls{guard} torches cast light on undead faces, each one makes a decision to stay and do their duty, or flee and live.

By this point, the undead horde numbers over a hundred, and the citadel has not prepared for anything like this.
It has perhaps twenty guards, and most of them work there because they seemed loyal, rather than capable.

\Gls{sewerking} leads his horde to the palace's main door with two ghasts to stop anyone exiting, and leaves the horde to slowly work its way up the stairs and rooms, killing anyone they come across.

Play the next available encounter first, then interrupt it with news of the citadel's attack.
Everyone can hear the screams from half-way across \gls{town}.

\paragraph{If the PCs intervene,}
they find \gls{sewerking} at the door, with much of his mana depleted already from fights with the guard.
He will flee to the sewers the moment he feels unsafe.

\ghast

\ghast

\sewerking

\humansoldier[\npc{\T[6]\E\Hu}{Palace Guard}]

\paragraph{If the PCs pursue \gls{sewerking},}
he already has a cave-in prepared to block his escape, so he waits -- holding a piece of rope, tied around a beam.
As the PCs round the corner, they can roll Wits + Crafts (TN 10) to notice the trap and retreat.
If they fail (or run through the trap anyway), they can roll Speed + Athletics (TN 10) to make it through anyway.

If both rolls fail, the roof collapses, inflicting 8 Damage, and 8 Fatigue Points.
Digging someone out from under the rubble requires an Intelligence + Crafts roll (TN 12), and they don't have long before the person under the rubble dies from asphyxiation (no resting actions allowed).

\paragraph{If the PCs want to run up the stairs and save \gls{townmaster},}
they have 8 rounds before the ghouls devour him, and will need to travel across 20 squares before reaching \gls{townmaster}'s room on the second floor.

\paragraph{If \gls{townmaster} dies,}
then his sons most likely die with him.
Leadership of \gls{town} `temporarily' goes to \Gls{captain} (or \gls{traitor} if \gls{captain} already died).

\stopcontents[sq]


