\sidequest{Sewer Bandits}
\label{sewerking}

\noindent
\Gls{sewerking} has ingraciated himself with the down-and-outs of \gls{town}, and built up a small following.
He collects prisoners to lock in his underground kingdom, and then turns them into sentient, ravenous, undead.%
\exRef{judgement}{Judgement}{ghoul}
He plans to dig a tunnel under the citadel, and release an army of ghouls into it, from the ground up, killing \gls{warden} and his entire family.

\sqpart{Town}% AREA
  {Missing People}% NAME
  {The \glsentrytext{healersGuild} have noticed aging \glsfmtplural{warden} going missing}% SUMMARY

\Gls{sewerking} and \gls{sewerthief} have wriggled through the hidden entrance from the \gls{pig} (area \vref{pigShaft}) to the \gls{healersGuild}, and tricked an aging \gls{warden} into coming down to the sewers with them.

\sqpart{Town}% AREA
{Illegal Songs}% NAME
{A bard is caught singing about the wonders of \glsentrytext{lostcity}}% SUMMARY

\Gls{warningbard}, arrives to sing of the wonders of \gls{lostcity}, and how the `golden priests' in the \gls{healersGuild} would bring green-glowing mushrooms of great magical power up from the ground, to feed the people.%
\footnote{See \nameref{green_tower}, \vpageref{green_tower}, for more on these `priests'.}

\Gls{townmaster}, wandering by, overhears this and quietly asks nearby \gls{sunGuard} to arrest him for:

\begin{itemize}
  \item
  Dissent (talking about `priests' as if that were better than simple guilds).
  \item
  Singing out of key.
  \item
  Radical ideas.
  \item
  Performing without a licence.%
  \footnote{Performance licences aren't a thing.}
\end{itemize}

\paragraph{For the rest of the day,}
all of \gls{town} is full of talk about the golden priests of \gls{lostcity}.

\paragraph{If the troupe do not intervene,}
\gls{warningbard} will end up in the cells for six months.
\footnote{For the cells under the \glsentrytext{guard} station, see \nameref{guardstation}, \vpageref{guardstation}.}

\warningbard

\sqpart{Town}% AREA
{Unexpected Ghasts}% NAME
{The dead run through the streets}% SUMMARY

\Gls{sewerthief} forgot to lock a door properly, and a room of ghasts escaped via \nameref{slum_exit} (area \vref{slum_exit}).

\ghast[\npc{\D\Hu}{Old Ghast}]

\ghast[\npc{\D\Hu}{Young Ghast}]

\ghast[\npc{\T[3]\D\Hu}{Other Ghast}]

Any character can summon the nearby \gls{sunGuard} with a \roll{Strength}{Empathy} roll (\tn[7]), to shout out loud.
The guards take 5 rounds to arrive, but every Margin on the roll reduces that time by 1 round, to a minimum of 2 rounds.

\paragraph{At the end of the day,}
the ghasts will be put to the sword one way or another, but the question remains; what brought them here?

\paragraph{If the party track the ghouls,}
they won't find the city streets help much, and the bandits in the sewers will start to clear up the butchers as soon as the horde have gone.
The roll is \roll{Wits}{Vigilance} (\tn[14]).

\sqpart{Town}% AREA
{Underground Assassins}% NAME
{The bandits in the sewer cut \Glsentrytext{captain}'s throat}% SUMMARY

\textbf{Background:}
\Gls{captain} has been pushing the investigation into the undead and nura sightings in town, so \gls{sewerking} ordered a hit on him, lead by \gls{sewerthief}.
Five men walked casually about him as he returned home \gls{whitehorse} with his wife, then closed in so one could stab him in the neck.

\begin{boxtext}

  You hear guards shouting ``After them!'', in the distance, and quickly scurrying feet, as a woman shouts for someone to help her wounded husband.

\end{boxtext}

\paragraph{If the characters stay to help the wounded man,}
they find \gls{captain} with a knife-wound, next to his wife.
The roll is \roll{Wits}{Medicine}, at \tn[9] to save his life.

\paragraph{If they run after the thieves,}
the \glspl{pc} make a Group Roll of Speed + Athletics.%
\iftoggle{core}%
  {\footnote{See the core rules, page \pageref{grouproll}, for Group Rolls.}}%
{}%
Remember that whoever's trying to patch up \gls{captain}'s bleeding neck won't be able to join the chase.

\begin{tcolorbox}[tabularx={cX},top=10pt,bottom=10pt]

  Roll & Result \\\hline
  12 & \textit{``Giving chase, you catch up to four men running from the scene of the crime.''} \\
  11 & \textit{``You run round an alley, and find a drain cover clanking. The assassins have jumped underground.''} \\
  9 & \textit{``You run in hot pursuit, but the attackers have disappeared down a street, into thin air.''} \\
  7 & \textit{``The attackers sprint away, leaving you running in the dark.''} \\

\end{tcolorbox}

\sewerthief

\humanthief[\npc{\T[4]\E\Hu}{Four of the Sewer Thieves}]

\paragraph{If the party follow the assassins underground,}
they run to the nearest entrance -- perhaps the butchers or \gls{pig}.
Go to page \pageref{sewers}.
Otherwise, this incident will remain a mystery.

If \gls{captain} survives, he has little idea of what's happening, although a little investigation could reveal what he's been asking about recently (\roll{Intelligence}{Vigilance} Teamwork Roll, \tn[10]).

\sqpart{Town}% AREA
{The Dead Rise}% NAME
{\Glsfmttext{sewerking} releases ghouls upon the citadel}% SUMMARY

\textbf{Background:}
\Gls{sewerking} undead have finally finished digging the tunnel
\footnote{See page \pageref{citadelTunnel} for the underground tunnel.}
underneath \gls{town}'s citadel,
\footnote{See page \pageref{citadel} for the town's Citadel.}
where \gls{townmaster} lives.
He releases the undead, and guides them upstairs, through the dark halls.
As the \gls{guard} torches cast light on undead faces, each one makes a decision to stay and do their duty, or flee and live.

By this point, the undead horde numbers over a hundred, and the citadel has not prepared for anything like this.
It has perhaps twenty guards, and most of them work there because they seemed loyal, rather than capable.

\Gls{sewerking} leads his horde to the palace's main door with two ghasts to stop anyone exiting, and leaves the horde to slowly work its way up the stairs and rooms, killing anyone they come across.

Play the next available encounter first, then interrupt it with news of the citadel's attack.
Everyone can hear the screams from half-way across \gls{town}.

\paragraph{If the \glspl{pc} intervene,}
they find \gls{sewerking} at the door, with much of his mana depleted already from fights with the guard.
He will flee to the sewers the moment he feels unsafe.

\ghast

\ghast

\sewerking

\humansoldier[\npc{\T[6]\E\Hu}{Palace Guard}]

\paragraph{If the \glspl{pc} pursue \gls{sewerking},}
he already has a cave-in prepared to block his escape, so he waits -- holding a piece of rope, tied around a beam.
As the \glspl{pc} round the corner, they can roll Wits + Crafts (TN 10) to notice the trap and retreat.
If they fail (or run through the trap anyway), they can roll Speed + Athletics (TN 10) to make it through anyway.

If both rolls fail, the roof collapses, inflicting 8 Damage, and 8 \glspl{fatigue}.
Digging someone out from under the rubble requires an Intelligence + Crafts roll (TN 12), and they don't have long before the person under the rubble dies from asphyxiation (no resting actions allowed).

\paragraph{If the \glspl{pc} want to run up the stairs and save \gls{townmaster},}
they have 8 rounds before the ghouls devour him, and will need to travel across 20 squares before reaching \gls{townmaster}'s room on the second floor.

\paragraph{If \gls{townmaster} dies,}
then his sons most likely die with him.
Leadership of \gls{town} `temporarily' goes to \Gls{captain} (or \gls{traitor} if \gls{captain} already died).

\stopcontents[sq]
