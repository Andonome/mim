\subsection[Smuggler's Library]{The Smuggler's Library}
\label{sewers}

A century ago, the old \glsentrytext{warden} of \glsentrytext{town} commissioned tunnels to transport goods from the outer gates to the citadel.
The \gls{healersGuild} also used the tunnels to place the bodies of the dead, then it expanded into a storage area for the citadel and \gls{healersGuild} (who receive artistic gifts from patrons trying to assure a bed for themselves when their time comes).

During the earthquakes of the season of Otsea%
\exRef{judgement}{Judgement}{Otsea}
it collapsed, and everyone left the area.

Soon after, someone noticed an opening in an old drain pipe, and entered.
Others followed, and now a small population of people live, who have nowhere else to go.

\paragraph{The narrow hallways}
make longer weapons difficult to use.%
\exRef{core}{Core Rules}{enclosedcombat}

\paragraph{Once the \gls{sunGuard} find trouble,}
they ignore it, until \gls{captain} finds out, and tells them to plug the hole.

The \nameref{sewers} have entrances in areas \ref{farmExit}, \ref{sewerWaterHall}, \ref{sewerPigWalk}, \ref{sewerPig}, \ref{slum_exit}

\mapPic{t}{Dyson_Logos/sewer}{
  \ref{farmExit}/23/95,
  \rotatebox{60}{\normalsize\nameref{farmExit}}/14/85,
  %{\large\D}\ref{sewerCells}/21/75,
  %{\large\D}\ref{sewerCells}/52/06,
  \D\ref{sewerLockup}/07/41,
  \D\ref{sewerLockup}/37/03,
  \ref{oldlibrary}/28/49,
  \ref{sewerArtefacts}/86/63,
  \large S/83/65,
  \Large\ref{sewerKingRoom}/71/51,
  \Large\ref{wireTrap}/70/31,
  \ref{sewerWaterHall}/48/38,
  \rotatebox{-23}{\small\nameref{sewerWaterHall}}/43/32,
  \D\ref{sewerLockup}/61/42,
  %{\large\D}\ref{sewerCells}/21/75,
  \Large\ref{sewerFood}/70/69,
  \ref{underHall}/77/27,
  {\normalsize\rotatebox{-90}{\nameref{underHall}}}/87/27,
  \ref{citadelTunnel}/11/18,
  {\normalsize\nameref{citadelTunnel}}/05/09,
  %{\large\D}\ref{sewerCells}/21/16,
  \ref{sewerPigWalk}/84/82,
  {\small\rotatebox{45}{\nameref{sewerPigWalk}}}/90/81,
  \ref{sewerPig}/92/71,
  {\normalsize\rotatebox{-90}{\nameref{sewerPig}}}/97/74,
  \ref{slum_exit}/71/91,
  {\normalsize\nameref{slum_exit}}/71/98,
  \Large\ref{wireTrap}/49/76,
  \Large\ref{underGuard}/63/865,
  %{\large\D}\ref{sewerCells}/43/79,
  \ref{sewerShrine}/56/77,
  \Large\ref{sewerDog}/33/67,
}

\mapentry[farmExit]{Exit to the \Glsentrytext{jotter}}

This tunnel connects the \nameref{sewers} to \gls{traitor}'s house, just outside \gls{town}'s walls.
The \gls{whiteBandits} smuggle goods through here for \gls{pigowner}, and use it to travel, without moving through any official entrances to \gls{town}.

\Gls{traitor}'s house above has several rooms, as he's done rather well for himself, and hopes that once the revolution comes he'll be in an even better position.

\mapentry[sewerLockup]{Dead Rooms}

\begin{exampletext}
  \Gls{banditking} stole \pgls{artefact} from \gls{necromancer}.
  This collection of human offal and skulls has sentience, and can cast spells.
  It detects the little bits of death we all carry with us -- injury and tiredness -- and use the feeling of their position to curse them with a state of semi-undeath.
  If the target dies while the spell remains in effect, they become ghasts -- sentient undead.%
  \exRef{judgement}{Judgement}{ghast}

  \Gls{sewerking} has been using it to create an army of the dead, by locking people in various rooms (originally used for storage) and letting the \gls{artefact} curse them.
\end{exampletext}

\noindent
Each storage cell has a bar across the front, and a few ghasts, stuffed in tightly.
Each was created by the Bladder Skull (\vpageref{The Bladder Skull}).

\paragraph{Anyone knocking off the bar}
will probably find a nasty surprise.
Consult the chart (no matter which door the \glspl{pc} open first, that's the first door).

\begin{nametable}{Cell Contents}
  Door & Contents \\\hline
  1st & 3 ghasts jump out. \\
  2nd & 7 ghasts jump out. \\
  3rd & String on the door indicates \gls{sewerthief} is taking a nap while he should be working. \\
  4th & 2 ghasts jump out, and one flees towards \nameref{farmExit}, above. \\
  5th+ & 5 ghasts jumps out. \\
\end{nametable}

\Gls{sewerthief} sleeps secretly by tying a piece of string round the door's bar, pushing the string inside the room, then balances the bar on top of the door.
When he closes the door, the bar falls into place, leaving only a small strip of string as evidence of the trickery.
The troupe can spot this, with a \roll{Wits}{Vigilance} roll (\tn[13]).

\paragraph{If the \glspl{pc} listen at the door,}
they find the dead are silent when not active, and don't respond to noise, as they are also deaf.%
\exRef{judgement}{Judgement}{undead_senses}

\ghast[\npc{\T[7]\D}{\arabic{noAppearing} Ghasts}]

\begin{boxtext}

  Empty stone shelves show where an expansive library once provided the entire city knowledge, but not a scrap of paper remains.
  Three stone pillars divide the room, each with a brazier hanging in front of them by a chain.
  The central pillar's brazier is made from three human skulls.

\end{boxtext}

\mapentry[oldlibrary]{the Old Library}
\index{Library!Ancient}

\begin{exampletext}
  The old library once held a treasure of autobiographies, dictated by the dying.
  This includes outdated maps of areas in \gls{valley}.
\end{exampletext}

\sidebox{
  \begin{boxtable}
    \textbf{\gls{tn}} & \textbf{Item Found} \\
    8 & Alchemical Basement \\
    10 & \Gls{necromancer}'s Temple \\
  \end{boxtable}
}
\noindent
Searching the books for meaningful information requires a Group Roll of \roll{Wits}{Academics}.
Success indicates that they have found an ancient map.
Find the maps in the handouts.

The skull-filled brazier above is \pgls{artefact} which slowly turns those in the cells undead.
If anyone manages to communicate with it, the Bladder Skull wants to return to \gls{necromancer}.

\artefact{The Bladder Skull}% Name
  {Three human skulls, tied together with their guts, each stuffed with their own inflated bladder.
  The Preservation spell cast upon it stops the bladders deflating.}% Body
  {2}% Intelligence
  {0}% Wits
  {3}% Charisma
  {to make everyone see as the dead see}% Mission
  {
    \setcounter{Fate}{2}
    \setcounter{Air}{2}
    \setcounter{Water}{1}
  }% Spheres
  {Projectiles~2, Academics~1, Crafts~1, Vigilance~2}% Skills
  {
  }% Spells

\showStdSpells[
    \revelationSpell
    \stepcounter{enc}
]

\begin{boxtext}
  Just ahead, you can see a dog, lying on the ground.
  It instantly stands up and begins to bark.
\end{boxtext}

\mapentry[sewerDog]{The Guard Dog}

\begin{exampletext}
  The \gls{whiteBandits} keep one of \gls{traitor}'s dogs chained here at all times (in rotation, so they don't get bored).
\end{exampletext}

\paragraph{If the characters approach,}
Rover hears them and begins barking.
If the group remained silent, they can make a \roll{Dexterity}{Stealth} Group Roll (\tn[9]) to move silently past the pup.

\paragraph{If Rover hears them,}
he begins barking and four of the bandits in room \ref{underHall} (\nameref{underHall}) come to investigate.

They won't poke about long, so the \glspl{pc} simply need to pick a reasonable hiding place or roll \roll{Intelligence}{Stealth} (\tn[8]).

\paragraph{If a bandit walks by,}
Rover does nothing -- he knows them all by scent alone.

\huntingdog[\npc{\A}{Rover}]

\begin{boxtext}
  Going down the stairs you feel your feet hitting cold water.
  It's not clear how far the water goes down, but it's cold.
\end{boxtext}

\mapentry[sewerWaterHall]{Drowned Hallway}

All the waste and filth from \nameref{stationDungeon} flows down a grate, and lands here.
The frigid water inflicts \pgls{fatigue}, and the filth demands a \roll{Strength}{Medicine} roll (\tn[10]) to avoid contracting a disease which slowly paralyses them.%
\exRef{judgement}{Judgement}{Corpse Hands}
The disease will not take effect for $1D6$ \glspl{interval}.

The roof lowers to meet the water, forcing heads to duck under.

\paragraph{Halfway through the icy slime,}
a little space to rise, and breathe, emerges.
And above, \emph{a roar echoes down}.

A tunnel leads straight upwards, and up to the grate sitting in room \nameref{stationDungeon}'s \vref{dunGrate}.

Neither \gls{sewerking}, nor anyone else, have any idea this tunnel exists.

\begin{boxtext}
  The boarded up wall pulls open -- the entire thing was a door made to look like a blocked entrance.
  The rings of shelves show a strange assortment of items -- jars filled with human teeth, an old brazier, dried snowdrops, and a vial of blood.
\end{boxtext}

\mapentry[sewerArtefacts]{Magical Item Storage}

\Glsentrytext{sewerking} stashes most of his prizes in this room on a simple series of shelves.
The shelves contain a bunch of books, \lootMagic, and a Wolf-Run potion.

\begin{enumerate}
  \item
  An old scroll, proclaiming elves the friends of humans, and seven reasons not to worry about nobles being assassinated.
  \item
  \textit{The Secrets of Elven Witchcraft: Attain Great Magic within the Week}.%
  \footnote{This book rambles about commanding secret spirits who live in reflections, and hints at lost treasure hidden under \gls{town}.
  The book speaks lies from start to end.}
  \item
  Basilisk-hide armour (\gls{dr}~4, \gls{weight}~2).
\end{enumerate}

\showTalisman

\talisman{Wolf-Run}% Name
  {detailed}% Enhancements
  {Wax}% Action
  {Air, Water}% Spheres
  {current \glspl{hp} plus \gls{dr}}% Resistance
  {Drinking the muddy potion increases the target's Speed by +\arabic{spellPlusOne}, while giving them a wolf-like outer layer of fur.
  The spell also inflicts a -\arabic{spellCost} Penalty to Intelligence, and forces the target to eat an extra \arabic{spellCost} meals per day, or suffer \glspl{fatigue} from starvation}% Summary
  {
  The effects endure until the target takes \gls{fatigue} Penalties due to starvation}% Details
\showTalisman

\begin{boxtext}
  The door opens to a noble's room, bearing a striking contrast to the dungeon around.  The bed's well made, the sheets are silk, and various books sit on shelves.
  On the table sit various maps.
\end{boxtext}

\mapentry{\Glsentrytext{sewerking}'s Room}
\label{sewerKingRoom}

During the night, \gls{sewerking} sleeps here.

\paragraph{Picking the lock}
requires an \roll{Intelligence}{Larceny} roll (\tn[9]).

Check the handouts for the maps.

\begin{itemize}
  \item
  The city map shows every entry point the bandits can enter the city above, including the theoretical passage the bandits think could be found again under \glsentrytext{townmaster}'s Citadel.
  \item
  A map of the area.
  \item
  A complete map of the current location.
  \item
  Many books.
\end{itemize}

The books are variously written on history (real and imagined), \textit{The Art of Lies} (by an elvish author -- `Erend\"e'), and instructions on hosting a dinner party.%
\footnote{It takes 280 pages to say `use seasoning`, and `have an anecdote'.}

\mapentry[wireTrap]{Wire Trap}

Both areas are barred from the outside (i.e. the lighter side).
Opening the doors from the anterior side requires an \roll{Intelligence}{Larceny} roll (\tn[10]).

A thin wire, nailed across the rock walls, will pull the shins of anyone crossing it.
Seeing the wire in the dark requires a \roll{Wits}{Athletics} roll (\tn[12] by light, or 16 in darkness).

Failing while going uphill means the character is Prone.%
\exRef{core}{Core Rules}{prone}
Failing while going downhill also inflicts $1D6$ Damage.

\begin{boxtext}

  Barrel after barrel fill the room, along with the smell of wine, apples, and vinegar.
  A little basket of choice snacks sits on top.

\end{boxtext}

\mapentry[sewerFood]{Food Storage}

\begin{exampletext}
  The basket of food in the basket contains a cursed apple, which contains an intense laxative.
  \Gls{sewerking} suspects one of the people who live down here in the \nameref{sewers} steals food, so he left that tantalizing basket with a poisoned apple, injected with a laxative.
\end{exampletext}

The \glspl{pc} roll \roll{Wits}{Medicine} (\tn[11]) to notice the poison on the apple, if they try to eat it.%
\footnote{Feel free to roll for the characters so they're not aware there's a problem.}

\paragraph{Failing the roll}
means that \pgls{interval} later, they receive 3~\glspl{fatigue}, then 2~\glspl{fatigue} on the next, then 1.

\begin{boxtext}
  Around the next corner, soft lantern-light trickles into your eyes.
  Myriad voices murmur, as if planning, quickly.
\end{boxtext}

\mapentry[underHall]{Grand Hall}

Here, the down-and-outs of \gls{town} gather, sleep, drink, and discuss ways to make money.
Permanent residents sleep in the alcoves to the side.
Theft is nearly impossible, as the \nameref{underHall} never sits empty.

At night, this place contains 30 people, and the air is foetid.
During the day, it contains 20.

\begin{description}
  \item[Doorcane] -- a woman with an abusive husband, who's fled to the safety of the \nameref{sewers}.
  \item[Chowdirge] -- an apprentice thief, learning from \gls{sewerthief}.
  \item[`Shitrat'] -- (real name `Starvale') the youngest son of \gls{townmaster}, here to find out what `real life is like'.%
  \footnote{Of course, \gls{townmaster} has no idea he comes here, and the undertown residents have no idea who his father is.  If a player makes note of the strange name, tell them that the boy sounds a bit strange (he's putting on a fake lower-class accent).}
\end{description}

\paragraph{If the \glspl{pc} approach from the lower doors,}
this shocks the crowd, and they run.
\Gls{sewerking} already warned them never to touch those doors, and to flee if anything exits except for him.

Once the crowd begins to run, they run up the stairs all at once (area \vref{slum_exit}).
Once half the crowd reach the slippery stairs, a couple at the top slip, and fall, creating a heavy cascade of bruised flesh and broken bones.

If the troupe follow the crowd, the first receives $1D6$ Damage, and gains $1D6$ \glspl{fatigue}, the second $1D3$ of each, and so on.

\paragraph{If the \glspl{pc} approach from the top door,}
(which leads along to room \ref{sewerKingRoom}) everyone will assume they're \glspl{guard}, trying to run from their duties.
The crowd will poke fun at them, then welcome them.

\begin{speechtext}

  ``Had enough monsters eh?
  Careful, this one's a proper monster!''

  ``\Glspl{bothy} toilets not good enough fer yer arse?''

  ``\Glspl{sunGuard} scaring yous?
  Go have a sit, I'll draw you a nice warm bath\ldots''
\end{speechtext}

\paragraph{If the \glspl{pc} corner the crowd and attack,}
they find a dozen regularly fight with the \gls{whiteBandits}.

\sewerking

\humanthief[\npc{\T[6]\E\Hu}{6 Sewer Bandits}]

\humanthief[\npc{\T[6]\E\Hu}{6 Sewer Bandits}]

\paragraph{Three rounds later,}
the cutthroats from room \ref{underGuard} arrive (assuming they weren't killed already).

\begin{boxtext}
  A scraping sound comes from the tunnel ahead.
  You slowly come to a halt, but whatever makes that kind of sound doesn't seem to care about you; it just keeps on scraping at something.
\end{boxtext}
 
\mapentry[citadelTunnel]{Tunnel to Citadel}

\begin{exampletext}
  \Gls{sewerking} set this ghoul to dig into the citadel's basement.
  It dug until its rotten hands fell off, then pushed its stumps into the dirt until the arms wore away.
  Biting into the dirt destroyed its teeth, so now it gnaws at the earth with its jaw-bone.
\end{exampletext}

\paragraph{If the \glspl{pc} attack,}
they will kill it easily -- it can do nothing but attempt to dig.

\begin{boxtext}

  The stairs go upwards for some time, and eventually arrive at a room filled with barrels of food, and a trapdoor above.
  You can hear a deep snore coming from just beyond the trapdoor.

\end{boxtext}
\mapentry[sewerPig]{Sewer Entrance}

This artificial stream loops round from \gls{pig}, above.
The stream continues downwards to an underground abyss.

\paragraph{Anyone venturing down-river gets swept along,}
and dies in the unending blackness, unless they make a \roll{Dexterity}{Caving} roll, \tn[10], to cling onto the sides.
\Glspl{pc} can spend 5~\glspl{fp} instead to make the roll automatically.

\mapentry[sewerPigWalk]{Entrance to \glsentrytext{pig}}

Up these stairs, the troupe can reach the bowels of \gls{pig}.

\Gls{pigowner} understands what trouble she's in, and immediately accuses the characters of theft and calls on everyone in the room to kill them.

Calming down the patrons requires a \roll{Charisma}{Empathy} roll (\tn[10]).

\paragraph{If the \glspl{pc} arrest the patrons of \gls{pig},}
everyone in \gls{pig} will deny any knowledge of the deeper tunnels, and the fact that bandits lived down there.

\mapentry[underGuard]{Guardroom}

Here four of the thieves sit and play simple dice games to pass the time, or occasionally sleep in the foetid straw.

\paragraph{If the \glspl{pc} fight,}
the cutthroats try to ram them out the door, and run to alert the others in room \ref{underHall}.

\humanthief[\npc{\T[4]\E\Hu}{Four Cutthroats}]

\mapentry[sewerShrine]{Old Shrines}

There are twelve pillars in total in the room, and each one was formed by members of a family, over the course of generations, donating money to the \gls{healersGuild}.

\begin{boxtext}
  Letters in Elvish above the doors state ``We bones await yours''.
\end{boxtext}

Those who died in its service had their skulls added to the tower.
It could take two hundred generations to create some of these towers.
Once the tower is completed, the top skull has the family's name carved into the forehead.

\begin{boxtext}

  The massive room has a strange lack of smell.  Towers of skulls stand in neat piles, each resting in a small pillar.  Some are as tall as a man, others reach nearly to the ceiling.  Each one has writing upon the top skull.

  At the far side of the room is another exit.

\end{boxtext}

Each pillar can be used once a day.
They require a short prayer (which takes some time to complete), and understanding which prayer aligns with which skull requires an \roll{Intelligence}{Academics} Group Roll (the \gls{tn} depends upon the type of pillar).
The character requires a Margin of 2 to know what will happen before casting the spell.%

\textbf{Smaller Pillars:} (TN 8)

\begin{enumerate}
  \item{(2) Regenerate $1D6$ \glspl{fp}.}
  \item{The target loses $1D6$ \glspl{fp} (this family stopped paying temple dues).}
\end{enumerate}

\textbf{Larger Pillars:} (\tn[10])

\begin{enumerate}
  \item
  The room is filled with sweet-smelling mist.
  \item
  The target feels light and detached, and takes no \gls{fatigue} penalties for 3 \glspl{interval}.
  \item
  Target regenerates $1D6+2$ \glspl{fp}.
\end{enumerate}

\textbf{Towering Pillars:} (\tn[12])

\begin{enumerate}
  \item
  The target regains $1D6+3$ \glspl{fp}.
  \item
  The closest hostile person loses $1D6$ \glspl{hp} (\glspl{fp} are ignored).
\end{enumerate}

\mapentry{Stairway to the Slums}
\label{slum_exit}

An old drain leads out to the city streets.
%! Where does it go? Doula's house?  Centre of rough street?


