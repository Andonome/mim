\chapter{\Glsfmttext{town}}
\epigraph{A criminal is a person with predatory instincts who has not sufficient capital to form a corporation. Most government is by the rich for the rich.
Government comprises a large part of the organized injustice in any society, ancient or modern.
Civil government, insofar as it is instituted for the security of property, is in reality instituted for the defence of the rich against the poor, and for the defence of those who have property against those who have none.}%
{Adam Smith}

\label{townChapter}

\noindent
In \gls{town}'s bowels, a violent revolution brews, as the \glspl{diggers} gather stores of undead to inflict upon \gls{citadel}.
The \glspl{pc} may help or hinder the revolution, but until that happens, \gls{town} provides the only place to buy rare equipment.

\widePic[b]{extracted/town_wide}

\printThreadsInRegion{town}

\section{On the Ground}

\begin{multicols}{2}

\subsection{Spells in Town}

The wind only brings so many \glsfmtlongpl{mp}, so multiple \glspl{witch} cannot occupy the same area happily.
\Gls{alchemist} casts spells all day long at the citadel, so if \pgls{pc} casts a spell in town, the ambient \glsfmtlongpl{mp}, will either go towards him, or the \gls{pc}, depending on who has the most empty \gls{mp}-slots.
If the \gls{alchemist} takes the \glspl{mp}, the \gls{pc} will not regenerate \glspl{mp} (which the character will, of course, notice); but if the \gls{pc} takes those \glspl{mp}, then \gls{alchemist} will know about them, and will begin hunting for them, through spells and the \gls{sunGuard}.

The first time \gls{alchemist} catches someone \gls{casting}, he will ask them to never do that in town again.
The second time, he won't ask -- he will sue them in the \gls{court}.%
\footnote{Find both \vpageref{citadel}.}

\townMap

\subsection{Sight Seeing}

Whenever the \glspl{pc} enter \gls{town}, take a look at where they would enter from, and where they pass through.

\mapentry[spyGrate]{Spy Grate}

This grate begins only a little above the water-level, and descends to the bottom of the river.
It exists to stop \glspl{woodspy}%
\exRef{judgement}{Judgement}{woodspy}
slithering into \gls{town}.

\mapentry[docks]{Docks}

The \gls{whiteplains} Docks on the West receive goods and travellers coming from \gls{whiteplains} and every \gls{village} to the west of \gls{town}.
Anyone going upriver will travel around the same speed as walking, as a horse (or person) has to pull the boat upstream by walking on a path beside the river and towing it.
Heavy rainfall will block upriver travel for a while, as the river's speed increases.

The Big Docks is where people travel downriver to the Eastern \glspl{village}.
Boats float quickly down to \gls{lochside} within 2~\glspl{interval}.
Fast rivers from heavy rains will reduce this to just 1~\gls{interval}, if anyone feels desperate enough to attempt the journey (\roll{Strength}{Seafaring}, \tn[12] to avoid capsizing).

Both docks sometimes go upriver to the South, which means half the people who want to travel go to the wrong dock.
Half of the \gls{town} inhabitants have firm opinions on the correct solution, and none of them have ever managed to agree on one.

\mapentry[greyGraveyard]{Graveyard}

The \gls{templeOfSickness} place the bodies of the honoured dead -- those taken by \gls{eldren} -- in the upper graveyard.
Most receive nothing, but if someone had enough riches, their family might pay for a cenotaph in the lower graveyard.

\mapentry[greyMarket]{Market}
\index{Market!\glsentrytext{town}}

Every morning, wagons full of vegetables pour into \gls{town} with food from \glspl{village} and hamlets, and enter the market.
Servants of the \glspl{warden} and guild leaders arrive to buy large baskets.

The food sellers then take their coin, and move to stalls where the guilds sell their wares -- beer, armour, blessings, clothes, carts, oil, and anything else \pgls{village} might not produce.
Despite its remote location, one can buy almost anything in \gls{town}.

Next, the \glspl{sunGuard} take their share.
They may be few, but they buy a large portion with their heavy taxes on the guilds.

``No price is too high for safety'', they tell everyone inside \gls{town}'s walls.

Lower-ranking guild members then pay for the last of the good food at the market, and return home with what they can get.
Their employment alone keeps them out of the \gls{guard}.
When servants and lower-ranking guild members misbehave, they find themselves with a quick job outside the walls.
And the worse the deed, the closer they may come to the \gls{edge}.

The last group live in a state of constant questioning.
Will they find a little work, or steal today?
If someone catches them, will it be the gallows or the guard?
If someone makes them work in the \gls{guard}, what kind of creature will eat them?

Take a look at the handouts for the \gls{town} market.

\mapentry[paperGuild]{The \Glsfmttext{paperGuild}~\glssymbol{yonder}}
\index{Library!\glsentrytext{paperGuild}}

This dominating building, the size of a battlefield, holds an orchestra of hums and murmurs.
From \gls{cFour} to \gls{cSix}, bee-hives on the rooftop let out a constant drone throughout the day.
Small doors on the bottom lead into candle and soap shops, where people come to gossip at night, just for the light.

Up the great stairway, through the double-doors, the cloakroom with a hundred hooks echoes with voices reading aloud in the main library, as \pgls{scribe} insists people remove cloaks, backpacks, or anything else which might help them steal a book or scroll.

Once inside the main hall, the maze of books sits below.
Readers can ask about the right place to find what they want from this high vantage point, and try to remember where they're going by the time they get down and inside the maze.
Since most humans read aloud, often to a small audience, the readers try to avoid the same pitch as others, so people can distinguish them over others.

\label{greyLibrarian}
\Person{\NPC{\glsentrysymbol{greyLibrarian}}{\Glsfmtname{greyLibrarian}}{long sleeves smeared across an eternally running nose}{sings `\textit{Nobody knows the trouble I've seen\ldots}'}{to find that missing cat}}%
  {{0}{0}{-1}}% BODY
  {{2}{-2}{0}}% MIND
  {
    \setcounter{Academics}{3}
    \setcounter{Crafts}{1}
    \setcounter{Empathy}{1}
  }% SKILLS
  {%
    \specialist{Valley History}
  }% KNACKS
  {writing equipment, bag of dried trout-strips}% EQUIPMENT
  {}% ABILITIES

Inside, the massive library contains a wealth of books and scrolls in piles, stacks, shelves, and occasionally used as doorstops (when \gls{greyLibrarian}, the sorting-librarian, disapproves of the book).

While a number of \glspl{seeker} within the \gls{paperGuild} outrank \gls{greyLibrarian}, she commands respect within the temple after sorting all of the books; only she knows how to find them.
She sorts books by:

\index{Books!(Sorting methods)}
\begin{multicols}{2}
\begin{enumerate}
  \item
  Height
  \item
  Width
  \item
  Colour
  \item
  Language
  \item
  Quality (she scores each from 1 to 12)
  \item
  Handwriting quality (scored 1 to 6)
  \item
  Author
\end{enumerate}
\end{multicols}

\paragraph{Researching at the Library}
only grants a +1~Bonus, due to the incomprehensible sorting algorithm.
However, if \gls{greyLibrarian} helps, the Bonus rises to~+3.

\ldots but she will not help anyone, as she's too busy sorting, reading, copying, and worrying about a thousand problems.

She can't find one of her cats -- `Blob', the fat black cat (he's gone to the local \gls{healersGuild}, \vpageref{townHealersGuild}),
her brother's in jail (room  \vref{spyBroth} in the \nameref{stationDungeon}), and \gls{alchemist} hasn't returned his book on \textit{Riddles for Simpletons} (area \vref{citadel_alchemist}).
\label{paperCat}

\index{Books!\Gls{valley}}
\index{Maps!\Gls{necromancer}'s Temple}
\index{Books!Spellbooks}
\index{Books!Fingers}
\index{Maps!Elvish}
\index{Maps!Vault Advert}

\begin{itemize}
  \item
  Researching the local area might bring up a map of the whole of \gls{valley} (\roll{Intelligence}{Survival}, \tn[14]).
  \item
  Researching old temples of \gls{eldren} might bring up the map of \gls{necromancer}'s temple, from before it crumbled (\roll{Intelligence}{Empathy}, \tn[10]).
  \item
  Asking the young people who frequent the library about what they are doing reveals a young man who has mapped all of \gls{town} (\roll{Intelligence}{Empathy}, \tn[10]).
  \item
  Spell books never have `Magic' on the cover: instead, spells come as part of a larger work, making them difficult to find.
  Nevertheless, an \roll{Intelligence}{Empathy} roll (\tn[9]) turns up a spell-book with $1D6$ spells (the first is level 1, the second level 2, et c.).

  Each roll increases the \gls{tn} by +1.
  \item
  Researching \gls{lostcity} yields a map of an old \gls{paperGuild} building in \gls{archwarp} (\roll{Wits}{Academics}, \tn[18]).
  \item
  Researching old elvish settlements yields a map.
  The format makes little sense, the cartographer seems to name random places, and age has changed everything which might have been remotely useful.
  However, if you look closely, with an elvish eye, this map matches \gls{valley}.
  It also shows the location of two alchemical gateways, marked `gate place' (pointing to \gls{archwarp} and \gls{sixshadow}).
  \item
  Researching \nameref{shadowVault} reveals a related book called `\nameref{bookOfFingers}' (see \vpageref{bookOfFingers}, \autoref{bookAppendix}.) and an old leaflet, which was an advert for \gls{shadowVault} (see the handout).
\end{itemize}

You can find these maps between the numbered pages, and slice them out as required.

\Glspl{pc} can take a week to search if they want to use \pgls{restingaction}, otherwise, research requires \pgls{interval}.

\paragraph{Stealing books}
will prove more challenging than one might expect.
The library closes outside daylight hours, and the massive windows admit a lot of Sunlight.
Its multi-level structure, and the fact that piles of books form a king of pyramid shape mean that anyone on the upper levels has an excellent view of everyone below them.

The \glspl{scribe} use secret whistles to alert each other to a theft, at which point one of them will lock the front door (by the cloak-room) and hide the key in a random coat-pocket.

\paragraph{Brief events}
play out, whenever \pgls{pc} enters the \gls{paperGuild}:

\begin{enumerate}
  \item
  Two Philosophers argue over whether a single magical gateway can exist.
  \begin{speechtext}
    \raggedright
    It must be two gateways, or none!
    One creates the other\ldots or they create each other, like \underline{being} and \underline{non-being}.
    Or did you somehow put on a \underline{trouser} this morning?

    \bigLine
    \flushright\raggedleft
    A gateway is a gateway, like a tunnel is a tunnel.
    Two sides does not mean two objects.
    Have you never seen a door!?
  \end{speechtext}
  \item
  \Gls{warningbard} enters to buy writing equipment, then asks a thousand questions about what the \gls{pc} wants in the \gls{paperGuild}.
  \item
  \Gls{southSeeker} argues angrily with \gls{greyLibrarian}, because he cannot find old maps, from the time of \gls{lostcity}.
  If someone helps resolve their argument, he finds a copy of the old Elvish map of \gls{valley}, but will lie about the contents (`just a map of somewhere near \gls{southDale}! Nothing important').
  \item
  Three men debate picking up a book.
  On the one hand, \gls{townmaster} has requested they fetch it; but on the other hand, a grey cat (Fagin) is sleeping on top of it, and \gls{greyLibrarian} never approves of people disturbing her cats.

  The book tells the tale of the \gls{town} \gls{warden} family, including a promise made to \gls{keras} (details \vpageref{shadowGate}).
\end{enumerate}

\paragraph{Behind the building,}
\gls{cThree} to \gls{cSix}, a half-tended garden hosts a rainbow of flowers, and gently buzzing bees.

\boxPair{
  \townmaster
}{
\Person{\NPC{\glsentrysymbol{paik}\F\Hu}{Roidspike}{Elegant, harlequin-style make-up, and very short, with a massive bag}{Pulls out miniature drum-set to indicate that a punchline has been stated}{to get a little god-damned respect!}}%
  {{-1}{-1}{0}}% BODY
  {{0}{0}{-1}}% MIND
  {
      \set{Academics}{1}
      \set{Empathy}{2}
      \set{Deceit}{1}
      \set{Larceny}{2}
      \npcQuote{My hound hath no nose\ldots}
  }% SKILLS
  {}% KNACKS
  {Large bag with small drum-set, \lootJewellery}% EQUIPMENT
  {}% ABILITIES
}

\mapentry[armourHall]{\Glsfmttext{armourHall}~\glssymbol{wrecan}}

Here, the angry and insolent people, who have not yet committed the kinds of crimes worthy of a place in the \gls{guard} learn to focus their anger on their craft.
They sell armour of every price to anyone, although the biggest client will always remain the \gls{sunGuard}.

\mapentry[greyDoulaShop]{\Glsfmttext{doulaShop}~\glssymbol{nulla}}

Here, \gls{forestpriest} sells tea, blessings, and advice.
Sometimes she sells \glspl{boon} (check the handouts), and will buy \glspl{ingredient} for 10~\glspl{sp} each.

Inside her shop, sitting under a different mess of old laundry every week, a trapdoor leads down to the \glspl{diggers}' tunnels, below (area \vref{slum_exit}, \vpageref{slum_exit}).

The shop has ten different secret compartments, each with 10~\glspl{sp} (inside books, in the rafter-shadows, nestled in her under-garments, and so on).

\mapentry[citadel]{The Citadel~\glssymbol{paik}}

\histEvent{260}{3}{As people began to migrate from \gls{archwarp}, \glsentrytext{warden} Klept Grey orders his \glsfmtplural{sunGuard} to take \glsentrytext{archwarp}'s stone, to improve \glsentrytext{citadel}}

\begin{exampletext}
  When the town upriver -- \gls{archwarp} -- lost its alchemical gateway, it started to fall into economic collapse.
  The \Gls{warden} Klept Grey (\gls{townmaster}'s ancestor) took the opportunity to start liberating the rocks which composed \gls{archwarp} the moment they left.

  \Gls{town}'s citadel still looks a little brighter than the rest of the town, due to the limestone rocks, pilfered from \gls{archwarp} to build it.
\end{exampletext}

Sparse lighting divides the citadel sharply into day and night.
During the day, the tall building's tall windows let Sunlight stream in, while leaving deep pockets of shadow in the corner of every room.
At night, the only light comes from servants wandering with lanterns.

\Gls{citadel} open doors, beckoning people into the \gls{court}, where \gls{townmaster} makes his verdicts.


%!
\null
\begin{description}
  \item[\Glsfmttext{court}:]
  where the public enter.
  \begin{itemize}
    \item
    The viewing area, where people buy snacks (see the town market handouts).
    \item
    \Gls{townmaster}'s platform, with stairs up to the Guard Room, above.
  \end{itemize}
  \item[Ground Floor:]
  where outsiders and servants come and go.
    \begin{description}
      \item[Left Wing:]
      with a long hall.
      \begin{itemize}
        \item
        Ballroom, with stuffed griffin centre-piece.
        \item
        Guardroom, with stairs down to the \gls{court}.
      \end{itemize}
      \item[Right Wing:]
      full of tapestries.
      \begin{itemize}
        \item
        Dining Room, table supported by stuffed \gls{basilisk}.
        \item
        Servants' Quarters.
        \item
        Kitchen.
      \end{itemize}
    \end{description}
  \item[First Floor:]
  where the important people stay.
    \begin{description}
      \item[Left Wing:]
      bare walls make echoes.
      \begin{itemize}
        \item
        Office.
        \item
        Library.\index{Library!Citadel}
        \item
        Guest Rooms.
      \end{itemize}
      \item[Right Wing:]
      paintings of \glspl{guard} on every wall, battling \glspl{monster}.
      \begin{itemize}
        \item
        \Glsentrytext{townmaster}'s Sons' 9 quarters (a nearby tree stands tall enough to access one room).
        \item
        Secret stairway up to the floor above, behind a door that looks like a wooden panel.
        \item
        Winery.
      \end{itemize}
    \end{description}
  \item[Second Floor:]
  with important people and valuable items.
    \begin{description}
      \item[Left Wing:]
      plain white walls.
      \begin{itemize}
        \item
        \gls{alchemist}, the \gls{seeker}'s Study, with one \lootTalisman, and one \lootTalisman (descriptions \vpageref{talismanIndex}).
        \item
        \Glsentrytext{townmaster}'s close servants' quarters.
      \end{itemize}
      \item[Right Wing:]
      where ornate weapons line the walls.
      \begin{itemize}
        \item
        \Glsentrytext{townmaster}'s room.
        \item
        Treasury, with 30 items, each worth $1D6\times 4$~\glspl{gp}, including a statue of an arachnid-like crab (\gls{weight}~3), made of a red crystal, which serves as 3 Fire \glspl{ingredient}.
        This statue came from the Fingers some centuries ago (covered \vpageref{bookOfFingers}).
      \end{itemize}
    \end{description}
\end{description}

The lower floor holds fifteen guards in each wing.

\humansoldier[\npc{\T[5]\Hu~\glssymbol{paik}}{Citadel Guards}]

\paragraph{If trouble emerges in the Citadel,}
all the guards rush to the source of the noise, ready to prove themselves.

\citadelAlchemist
\label{citadel_alchemist}

\showStdSpells

\paragraph{If \gls{alchemist} sees lawbreakers,}
he threatens the most awesome magic imaginable, even if his real powers leave much to be desired.

\paragraph{If the troupe ask \gls{alchemist} for the \gls{paperGuild}'s book back,}
he will return it only after composing a letter to \gls{greyLibrarian}, so the troupe can deliver the letter to her.
He begins by arranging two books on seducing women, and one on romance, then takes an entire \gls{interval} to actually write the thing. 

The book details a fiendish fiddly riddle, called `the Riddle of \gls{shadowGate}', but does not explain any context.
\index{Books!Riddles of the Gate}

\histEvent{270}{5}{All of \glsentrytext{valley} tries and fails to solve `the Riddle of the Gate'}

\label{hardestRiddle}
\hardestRiddleEver

\mapPic{t}{Dyson_Logos/white_horse_2}{
  \ref{horseHall}/18/45,
  \ref{horseKitch}/68/53,
  \ref{horseYard}/88/57,
}

\humandiplomat[\npc{\T[9]\M\Hu}{\Glsentrytext{townmaster}'s Nine Sons}]
\index{Grey Family!Older Sons}

\paragraph{If \gls{townmaster}'s sons find intruders,}
they talk big, then surrender before the fight has begun, reminding the intruders that their father will pay handsomely.

\mapentry[townHealersGuild]{\Glsfmttext{healersGuild}~\glssymbol{eldren}}

The temple houses priests who write biographies (mostly for the rich), handle pensions (repaid on a per-family basis), practice starvation (so others can have more), and sing prayers for their ancestors to stay somewhere nice in the afterlife.

While modern \glspl{healersGuild} ensure maximum accessibility, people created this building some centuries ago, when the guild was not a guild, and things were different.
It has a lower level, with long, thin, stairs, reserved for the most senior \glspl{helper} to store the bodies of the most worthy dead.
Since the catacombs under \gls{town} collapsed, the dead only stay in this room temporarily, before transportation to the graveyard outside.

\begin{exampletext}
  When \gls{healerLeader} saw one-to-many people in serious need of help just lying in \gls{town}'s streets, and waiting to be flung into the \gls{guard} or the \gls{court}, he decided to do something.
  After finding unsatisfying answers for some years, \gls{forestpriest} eventually asked him to become part of the solution.

  The \gls{diggers} give him poisonous mushrooms which paralyse people,
  \footnote{See area \vref{sewerMushrooms} in the \nameref{sewers}.}
  and he feeds them to the rich who take up the \gls{healersGuild}'s beds, instead of those who really need them.
  Once they freeze up enough, he sends them downstairs, and signs them off as dead.
  Finally, he has the `body' taken downstairs, where the \glspl{diggers} replace it with a body-shaped bag, stuffed with rocks, dirt, and cloth-padding.
\end{exampletext}

\Pgls{healerLeader} likes to keep track of things, and feels terribly annoyed that a fat, black cat keeps sneaking into the building, and he can't figure out how.
The ground floor's windows only let in a little Sunlight through wooden slits, but cannot open.
The only door out opens and closes only when the attentive staff open them with a key.

\healerLeader

\Gls{healerLeader} will not manage to keep the fat, black cat out, as he enters through a secret tunnel in \gls{pig} (area \vref{pigTemple}).
The \glspl{pc} won't find the cat either, as he's already left to annoy the horses in the \nameref{guardstation} (area \vref{stationStables}).

\mapentry[whitehorse]{\Glsentrytext{whitehorse}~\glssymbol{abderian}}
\index{Taverns!\Glsentrytext{whitehorse}}
The food inside smells rich, and spicy.
Barkwind stands at the front door, ensuring good people are well met, and scruffy low-lifes don't enter.

\boxPair{
  \smolMapPic{Dyson_Logos/white_horse_1}{
    \ref{horseUpstairs}/55/27,
    \ref{horseSideRoom}/55/80,
    \ref{wolfRoom}/28/27,
    \ref{horseCupboard}/85/08,
  }
}{
  Roll two dice to find the current activities, and use both results.

  \begin{dlist}
    \item
    \Glsfmtname{alemaster} complains bitterly and loudly about the current ale prices, and the price of staff.
    \item
    \Glsfmtname{greyLibrarian} sits in a corner, reading a book about \gls{lostcity}.
    \item
    \Glsfmtname{townmaster} plays ridiculous games outside.
    \item
    \Glsfmtname{alchemist} plays a game of go with \pgls{whiteplains} cattle trader.
    \item
    \Glsfmtname{keras} plays a game of go with another gnome, from \gls{oolery}.
    \item
    \Glsfmtname{captain} and \glsfmtname{investigator} argue about which of them must take responsibility for bandit raids close to \gls{town}.
  \end{dlist}
}

Entry through deception requires a roll against Barkwind's \roll{Wits}{Vigilance} (\tn).
A tie indicates that he forgets them soon, and they can try again within a week.

\Gls{whitehorse} has no locks as nobody expects thieves to enter.
 
%!
%\null
\begin{enumerate}
  \item
  The drinking hall hosts \gls{village} \glspl{warden} playing games, and half a dozen \glspl{sunGuard} (sometimes including \gls{captain}).
  \label{horseHall}
  \item
  \label{horseKitch}
  The staff sleep in the kitchen here during long shifts.
  The lack of proper ventilation makes the air difficult to breathe.
  \item
  \label{horseYard}
  The courtyard usually contains a couple of carriages, and nobles playing ridiculous games.
  \item
  \label{horseUpstairs}
  The bookshelves contain rather a lot of history books, most focussing upon anti-elven propaganda, such as the time they destroyed \gls{lostcity}.
  \index{Books!Elven War on \gls{lostcity}}

  A map on the wall shows all of \gls{valley}.
  Show your players the map in the handouts, but don't give it to them unless the \glspl{pc} want to steal the map.
  \index{Maps!\Glsentrytext{valley}}
  \item
  \label{horseCupboard}
  This cupboard contains cleaning supplies, two Fire \glspl{ingredient} (bear hearts), and one Water \gls{ingredient} (\pgls{woodspy} beak), in case someone becomes sick.
  \item
  \label{horseSideRoom}
  Tavern equipment, bookshelves for less popular books, and beds for the favoured servants, lie in semi-organized heaps and stacks.
  \item
  \label{wolfRoom}
  Guest room.
\end{enumerate}

\begin{boxtext}
  \Gls{townmaster} is running away from an entourage of chuckling men with his hands tied behind his back.
   A chicken runs out in front of him with a little paper hat.
   He lunges forwards and grabs the chicken in his teeth, then shakes it like a mad dog until it stops squawking.
   He gives a triumphant grin as the crowd clap and another man steps forward to have his hands tied.
\end{boxtext}

\humandiplomat[\NPC{\glsentrysymbol{abderian}\M\Hu}{Barkwind}{Proud}{Curling moustache}{ensure everything is `proper'}\npcQuote{This is a nice place for nice people.  There's nothing for \underline{you} here!}]

\mapentry[smugglerHouse]{\Glsfmttext{traitor}'s House~\glssymbol{sylf}}

This two-story wooden house has every bit of finery normally owned by \glspl{warden}, except for servants.
It is \emph{filthy}, but otherwise lavish.

\begin{boxtext}
  A ragged \gls{dawnDolly} sits by the door, despite the building's safe location.
  Dishes and plates with fancy pictures painted on them sit on and under every table.
  Books and notebooks lay in piles on the side of chairs, and tapestries depicting ancient heroes droop down the walls.
\end{boxtext}

\begin{itemize}
  \item
  A secret trapdoor in the kitchen leads down to a basement.
  \item
  A further trapdoor in the basement leads down to the \nameref{sewers}, area \vref{farmExit}.
  \item
  Some nights, the \glspl{wolfhead} arrive with barrels of spirits for \gls{southCook} (though they have never seen the tunnels).%
  \footnote{See `\nameref{troubleAle}', \vpageref{troubleAle} for details.}
  \item
  Other nights, \glspl{whiteBandits} take items from \gls{town} to \glspl{village} with rich \glspl{warden} who don't like paying taxes.
\end{itemize}

\boxPair{
  \smolMapPic[\Large]{Dyson_Logos/guard_station}{
    \ref{stationStables}/42/15,
    \ref{stationStorage}/81/427,
    \ref{stationStorage}/54/11,
    \ref{stationToilet}/63/11,
    \ref{stationCaptainToilet}/895/427,
    \ref{stationCaptainRoom}/85/59,
    \ref{stationSleep}/86/14,
    \ref{stationDressing}/30/11,
    \ref{stationLecture}/86/81,
    \ref{stationRecords}/37/41,
    \ref{stationInterrogation}/285/57,
    \ref{stationShrine}/44/65,
    \ref{stationStairs}/53/41,
  }
}{
  \humansoldier[\npc{\T[9]\glssymbol{paik}\Hu}{30 \glspl{sunGuard}}]
}

\mapentry[guardstation]{Guard Station~\glssymbol{paik}}

A minimum of five \glspl{sunGuard} patrol the grounds at a time.
\Gls{captain} has an obsession with guards constantly rotating around the premise.
As a result, they've hidden a stash of whiskey in the bushes at the back, and sometimes have `rounds', while they do the rounds.

The wooden buildings tacked into the outer wall have thin rooves which constantly bend and creak -- walking silently across them is impossible for anything with a total \gls{weight} of 4 or more.

%!
\null
\begin{enumerate}
  \item
  Dressing room, with armour.
  \label{stationDressing}
  \item
  Stables\label{stationStables} (the horses are terrified of a fat, black cat darting between them, before running off to \gls{pig}).
  \item
  Storeroom for food, such as \rations, \rations, and \rations.
  \label{stationStorage}
  \item
  Toilet.
  \label{stationToilet}
  \item
  Sleeping Quarters.
  \label{stationSleep}
  \item
  Captain's Toilet.
  \label{stationCaptainToilet}
  \item
  \gls{captain}'s Room.
  \label{stationCaptainRoom}
  \item
  Lecture Hall (though mostly used as a drinking hall).
  \label{stationLecture}
  \item
  Records Room, containing lists of fugitives, laws, tax records (a copy is kept in \gls{townmaster}'s treasury), and valuable paintings of local nobles.
  \label{stationRecords}
  \item
  Interrogation room.
  \label{stationInterrogation}
  \item
    Simulacra of \pgls{woodspy}, \gls{crawler}, griffin, (small) \gls{basilisk}, and bandit, made from wood and leather.
  The guards use these to explain combat tactics.
  \label{stationShrine}
  \item
  Stairway down to the dungeons.
  \label{stationStairs}
\end{enumerate}

\end{multicols}

