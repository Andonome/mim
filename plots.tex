\chapter{Protagonists \& Plots}
  \epigraph{He attacked everything in life with a mix of extraordinary genius and na\"ive incompetence, and it was often difficult to tell which was which.}{Douglas Adams}
\label{sideQuestIntro}

\section{How to Weave}
\label{sqSummaries}

\begin{multicols}{2}

\noindent
\Gls{valley} is larger than most human settlements, but the cost came in poor defences.
This tapestry of small stories form a morality-tale of geography, and the dangers of thin defences, although the players will not find this obvious for some time.
They will begin by engaging with some people, then notice the scene.
They will come to understand that \gls{segment}, before they notice the \gls{sq} that the \glspl{segment} are forming.

Some \glspl{sq} \glspl{segment} have a `\gls{vlg}' symbol in their summary, indicating that something may destroy \pgls{village} during, or before, this \gls{segment}.
If this happens, mark the \gls{village} off the map.
If the \glspl{pc} find \pgls{village} destroyed, this often presents a dangerous situation.
They will find themselves outside, at night, without the protection of any civilization.

\subsubsection{Your Roadmap}
sits below, with each \gls{segment} arranged by \gls{region}.
Have a look at the \textit{Wolf Heads} story.
Follow the \glspl{segment}, in order, through the three \glspl{region}.
Notice how the \glspl{pc} will see the shadows come and go, as they hear rumours, and meet the \glspl{wolfhead} one at a time.

%%%%%%%%%%%%%%%%%%%% Map %%%%%%%%%%%%%%%%%%%%

\renewcommand\csComments{
  \mapCircle{16}{76}{1.7}{Dyson_Logos/bandit_camp}
  \mapCircle{35}{88}{2}{Dyson_Logos/forgotten_city}
  \mapCircle{27}{09}{2}{Dyson_Logos/qualme_temple}
  \mapCircle[4]{56}{52}{2.5}{Dyson_Logos/town}
  \mapCircle{44}{41}{2}{Dyson_Logos/redfall}
  \mapCircle{83}{09}{1.7}{Dyson_Logos/shadow_gate}
  \mapCircle{86}{45}{1.7}{Dyson_Logos/lochside}
  \draw[very thick,white] (11,0.6) -- (12,0.6) node[anchor=north]{\outline{10 Miles}} -- (13,0.6) ;
}

\mapNotes{
  \nameref{banditLair}/16/73,
  \nameref{lostcity}/35/86,
  \glssymbol{yonder}/35/94,
  \glssymbol{paik}/56/61,
  \Glsentrytext{town}/56/50,
  \Large\glssymbol{paik}/44/49,
  \glsentrytext{redfall}/44/41,
  \glssymbol{abderian}/83/13,
  \nameref{shadowVault}/83/05,
  \normalsize\nameref{silentGorge}/83/30,
  \glssymbol{sylf}/86/49,
  \gls{lochside}/86/45,
  \D/27/15,
  \nameref{necromancers_lair}/25/08,
}

\widePic[t]{Irina/greylands}

\end{multicols}

\label{sqList}
\printAllSideQuests{Roads, Forest, Town}

%%%%%%%%%%%%%%%%%%%%
\needspace{12em}
\begin{multicols}{2}

\subsection{Getting Started}

Every session goes the same way:

\input{config/start.tex}

The first session should follow the same pattern, including players explaining to \pgls{jotter} how they ended up in the \gls{guard}.
%! Write up a starting broch
The first scene should begin in \pgls{broch}, somewhere on the `Roads'.
This area covers everywhere outside of \gls{town}, up to the forest, so it constitutes most of the map \vpageref{Irina/greylands}.

Standard \gls{guard} duties%
\exRef{judgement}{Judgement}{NGmissions}
include transporting trader caravans on their ways between \gls{town} and the outer \glspl{village}.
The pleasant areas, closer to \gls{town}, see more traders and travellers than monsters, while the distance \glspl{village} have roads piercing the forest, and must push back massive predators almost daily.

Once the \glspl{jotter} send them this way and that, they will instantly find \glspl{sq} assaulting them daily, or weekly, or at any pace you like, and the players will eventually investigate at least one lead.

Throughout this time, \glspl{pc} should receive all the standard encounter rolls.
Mixing the lot -- encounters and plots, \glspl{sq} and \gls{guard} missions -- creates a maelstrom of twists and turns.
This can be a lot of fun, but it also threatens the players' focus, so it's best to remember to pull that focus back to whatever the players think of as their primary mission.
This may change half way through the mission, or they may disagree about what's important; and both of those results are fine.
However, if the players ever feel \emph{confused} about what their characters should be doing, they will quickly disconnect from \gls{fenestra}, and start staring into their rectangles.

Chaos, and focus should flow like breathing in and out.

\subsection{Cartography}

Take a look at the map \vpageref{Irina/greylands}.
Note how long a journey between two \glspl{village} might take while travelling 10~miles a day.

Travellers going West to East can go downriver, which decreases their travel times significantly, while other journeys will go around walking speed.

\subsubsection{Primordial Forests}
surround everything.
Travelling beyond this \gls{edge} leads nowhere except for those few locations marked on the map; the rest of civilization lies far away.

The troupe should remain tactful when out in the forests -- making noise or fire has a 1-in-6 chance of attracting the attention of some predator, so each \gls{interval} with a noise should prompt another roll.

\subsubsection{Outer Civilization}
protrudes a little into the wilderness, and every few nights some new creature emerges from the forest to crawl over their rooves, and claw at the doors.

Some of the outer buildings only show a single tower.
These are the \glspl{broch}, where traders rest overnight, and where the new \glspl{guard} often stay.
In fact, new \glspl{guard} may not enter towns or \glspl{village}, so they must remain here until they increase their rank.%
\exRef{judgement}{Judgement}{fodder}

Once the troupe enters \pgls{village}, tear out the price handouts.
Each handout has an individual list of items, services, and prices.

\subsubsection{Inner Hamlets}
jest peacefully.
Any beasts which emerge from the forests would have wandered towards the noise of the outer \glspl{village} or \glspl{broch}.

These little areas have a `civilization rating' of 12 to 14, depending on their proximity to town, so you can replace any encounters which roll that number or lower with traders.%
\exRef{judgement}{Judgement}{encounters}

\subsubsection{Let Players `Ruin' the Mission}
because the \glspl{sq} below can withstand a lot of punishment from the players.
If \gls{segment}~1 of a \gls{sq} goes haywire, \gls{segment}~2 can usually take place unscathed, or with minor adjustments.
But you should let the players change the world, otherwise what is the point of this world?

When \glspl{pc}' actions threaten to derail \pgls{sq}, you have options.

\begin{description}
  \item[If the \glspl{pc} kill a major character]
  use another.
  It's no accident that two brothers lead the \gls{whiteBandits}.
  If \gls{sewerking} dies, \gls{banditking} can take over most of his duties.
  Similarly, \gls{pigowner} can also take over any of \gls{alemaster}'s \glspl{segment}, as they're both members of the \gls{wheatGuild}.
  \item[If the group decide to join the \gls{whiteBandits}]
  adjust!
  So they don't want to be in the \gls{guard} any more -- that's quite understandable.
  So they will have to listen to \gls{banditking} instead of \gls{traitor}.
  \Gls{sewerking} still has plenty of missions for them.
  \item[When all else fails]
  just cut out the \gls{sq}.
  If the \glspl{pc} somehow kill \gls{spiderqueen} during part 1, the campaign only loses a few \glspl{segment}.
  The campaign has plenty more to give!
\end{description}

\end{multicols}

\setglossarypreamble[people]{
  These are the movers and shakers, the doers and thinkers.
  You can find each of them in the \glspl{sq}, sometimes as primary actors, other times wandering into the background of multiple threads and \glspl{segment}.
}

\printglossary[
  type=people,
  style=topicmcols,
]

