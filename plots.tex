\chapter{Protagonists \& Plots}
  \epigraph{He attacked everything in life with a mix of extraordinary genius and na\"ive incompetence, and it was often difficult to tell which was which.}{Douglas Adams}
\label{sideQuestIntro}

\section{A Tapestry of Karma}
\label{sqSummaries}

\begin{multicols}{2}

% Theme: Karma

\noindent
Each of the nine stories revolves around a central theme: the subtle line between karma and vengeance.
In \nameref{spiderSong}, the \gls{spiderqueen} kills people, then slays entire hamlets, until her enemies mount up; eventually, the \glspl{pc} will find someone willing to help kill her, potentially making the \glspl{pc} an instrument of karmic vengeance.
While that thread develops, the \nameref{wolfHeads} story begins, and the \glspl{pc} will a series of suspicious characters, but cannot demonstrate any definitive harm or crime they have committed.
The two groups will have to cement their attitude towards each other before the end, as they stumble upon the \glspl{pc} just as the \glspl{pc} deal with a dangerous situation.
Later the situation reverses; the \glspl{pc} find the \glspl{wolfhead} wounded, and close to death.
They may end up repaying a debt, or grab at the chance for vengeance, depending on the troupes' relationship to each other.

% SQ.

These `\glspl{sq}' use a special structure, where each part is an independent \gls{segment}.
Some start with \pgls{segment} simply for foreshadowing -- something to give the players a feint memory to hang onto and a hint of things to come.
And each one should work with a changeable table, where some players have to miss a week or two, and new players arrive mid-way through the \gls{campaign}, because the heart of the narrative does not arrive during any particular \gls{segment} or \gls{sq} -- the `main plot' comes out of the collection of threads, which combine into a central narrative.

% Map + regions

\subsubsection{Geography}
shapes the \gls{campaign} in two ways.
First, the \glspl{sq} split into \glspl{segment}, and each \gls{segment} lives on one of the main areas on the map, waiting patiently to ambush the \glspl{pc}.

\paragraph{The forest}
surrounds the \gls{valley}, so the moment the \glspl{pc} step foot past the \gls{edge}, you can look for the next available \gls{segment} which takes place in the forest.
Every forest \gls{segment} should fit into your game, regardless of where on the map the \glspl{pc} have gone.

\paragraph{The roads}
lie between the forest and town, so each time the \glspl{pc} go from one to the other, they must pass through the roads which link all the \glspl{village} in the \gls{valley} together.
Some \glspl{segment} might make sense when the \glspl{pc} awaken in \pgls{bothy}, or if they stay in \pgls{village}; but if not, then all of them will make sense when the \glspl{pc} travel along the roads.%
\footnote{Of course, this has already been covered, in the glossary, \vpageref{glosPreamble}, which you must have read already.
Otherwise that would make you a mad person, who skips past random spots in their books, attempting to make sense of them in random order.}

While the forest presents grand, solitary, \glspl{segment}, the road has some \glspl{segment} with a `\squash' symbol next to them.
This indicates that you should run the next available \gls{segment} immediately, tying both of them together.
This might mean double-trouble for the \glspl{pc}, who end up fighting two enemies at the same time, or it might mean a conversation interrupted by combat.

\paragraph{The town}
forms the nexus of news for the \gls{valley}, much of it bad.
The forest holds \glspl{fiend} -- \gls{spiderqueen}, \gls{necromancer}, and \gls{whiteBandits} -- who can destroy entire \glspl{broch} and \glspl{village}.
If this happens, \gls{valley}'s geography will change rapidly.

Check out the map \vpageref{Irina/greylands}.
Notice the protective barrier of \glspl{broch} and \glspl{village} forming a thin, protective barrier against the \gls{edge}.
If the outer protection falls, \glspl{monster} will become far more common inside the defenceless, inner farmland, where hamlets raise all the livestock.
Once meat production goes down and danger goes up, traders will refuse to use a road, then nearby \glspl{village} will not have enough provisions to survive the cold seasons, and will have to move.

One by one, by \gls{village} and \gls{broch}, \gls{valley} would collapse, leaving the \glspl{fiend} to consume the last, weakened, inhabitants.

\subsubsection{Trek a Path}
through the \gls{campaign} with your finger, to see how it might unfold.
If the \glspl{pc} go to town quickly, you would begin the \gls{sq} `\nameref{troubleAle}'.
But if they pass through the first two \glspl{segment} of the \nameref{wolfHeads} \gls{sq} first, then the \glspl{pc} will hear what the \gls{wolfhead} troupe have been up to, and may meet \gls{southRogue} the very moment they leave \gls{town}.

If all these little frayed ends seem confusing, just skip to \nameref{spiderSong} \vpageref{spiderSong} -- every \gls{sq} has a short, easy plot when considered alone.

%%%%%%%%%%%%%%%%%%%%


%%%%%%%%%%%%%%%%%%%% Map %%%%%%%%%%%%%%%%%%%%

\renewcommand\csComments{
  \mapCircle{16}{76}{1.7}{Dyson_Logos/bandit_camp}
  \mapCircle{35}{88}{2}{Dyson_Logos/forgotten_city}
  \mapCircle{27}{09}{2}{Dyson_Logos/qualme_temple}
  \mapCircle[4]{56}{52}{2.5}{Dyson_Logos/town}
  \mapCircle{44}{41}{2}{Dyson_Logos/redfall}
  \mapCircle{83}{09}{1.7}{Dyson_Logos/shadow_gate}
  \mapCircle{86}{45}{1.7}{Dyson_Logos/lochside}
  \draw[very thick,white] (11,0.6) -- (12,0.6) node[anchor=north]{\outline{10 Miles}} -- (13,0.6) ;
}

\mapNotes{
  \nameref{banditLair}/16/73,
  \nameref{lostcity}/35/86,
  \glssymbol{yonder}/35/94,
  \glssymbol{paik}/56/61,
  \Glsentrytext{town}/56/50,
  \Large\glssymbol{paik}/44/49,
  \glsentrytext{redfall}/44/41,
  \glssymbol{abderian}/83/13,
  \nameref{shadowVault}/83/05,
  \normalsize\nameref{silentGorge}/83/30,
  \glssymbol{sylf}/86/49,
  \gls{lochside}/86/45,
  \D/27/15,
  \nameref{necromancers_lair}/25/08,
}

\widePic[t]{Irina/greylands}

\end{multicols}

\label{sqList}
\printAllSideQuests{Roads, Forest, Town}

%%%%%%%%%%%%%%%%%%%%
\needspace{12em}
\begin{multicols}{2}

% How to Run - starting base (link), missions, open-ended.

\subsubsection{Motivational Prompts}
ensure the \glspl{pc} always want (or need) to go somewhere, which then activates another \gls{segment}, throwing yet more plot at the players.
The players may take a while to fully understand the grander political schemes threatening the \gls{valley}, and even longer to come up with counter-schemes.
Bog-standard \gls{guard} missions can help kick-start \glspl{journey}, as they often involve entering the forest.
\iftoggle{judgement}{%
  You can find some random table to generate missions in the \textit{Book of Judgement}, \autopageref{NGmissions}.
}{}

Most \glspl{segment} require insight and negotiation, rather than combat.
This helps the \glspl{sq} run alongside random encounters, without making dangerous combat-encounters repetitive.

% Factions, death, letting go.
\subsubsection{Player Interference}
should be considered desirable.
Every \gls{segment} could change.
In fact, if \pgls{pc} somehow kills the \gls{spiderqueen} during the second \gls{segment}, the rest of that \gls{sq} would disappear instantly.

However, most \glspl{sq} will not fade away so easily, because most of the protagonists come from one of five political factions.
So if one of the \gls{whiteplains} brothers dies, the other can take over for him.

Since the \glspl{pc} begin in the \gls{guard}, that makes them part of the \glsentrytext{establishment} at the \gls{campaign}'s start.
Of course, this may change soon, as \gls{sewerthief} will gently inquire how the \glspl{pc} feel about joining the \glspl{whiteBandits}, or they may find themselves making concrete plans with the \glspl{wolfhead}.

\end{multicols}

\setglossarypreamble[people]{
  These are the movers and shakers, the doers and thinkers.
  You can find each of them in the \glspl{sq}, sometimes as primary actors, other times wandering into the background of multiple threads and \glspl{segment}.
}

\printglossary[
  type=people,
  style=topicmcols,
]

