\chapter{Protagonists \& Plots}
  \epigraph{He attacked everything in life with a mix of extraordinary genius and na\"ive incompetence, and it was often difficult to tell which was which.}{Douglas Adams}
\label{sideQuestIntro}

\section{A Tapestry of Karma}
\label{sqSummaries}

\begin{multicols}{2}

% Theme: Karma

\noindent
Each of the nine stories revolves around a central theme: the subtle line between karma and vengeance.
In \nameref{spiderSong}, the \gls{spiderqueen} kills people, then slays entire hamlets, until her enemies mount up; eventually, the \glspl{pc} will find someone willing to help kill her, potentially making the \glspl{pc} an instrument of karmic vengeance.
While that thread develops, the \nameref{wolfHeads} story begins, and the \glspl{pc} meet a series of suspicious characters, but cannot demonstrate any definitive harm or crime they have committed.
The two groups will have to cement their attitude towards each other before the end, as the \glspl{wolfhead} stumble upon the \glspl{pc} just as the \glspl{pc} deal with a dangerous situation.
Later the situation reverses; the \glspl{pc} find the \glspl{wolfhead} wounded, and close to death.
They may end up repaying a debt, or grab at the chance for vengeance, depending on the troupes' relationship to each other.

% Map + regions

\subsubsection{The Three \Glsfmtplural{region}}
share smaller tales, as the \gls{sq} \glspl{segment} dart between them.
But each \gls{region} contains a looming threat, which appears gradually, as a consequence of multiple \glspl{segment} coming together.


%%%%%%%%%%%%%%%%%%%% Map %%%%%%%%%%%%%%%%%%%%

\renewcommand\csComments{
  \mapCircle{16}{76}{1.7}{Dyson_Logos/bandit_camp}
  \mapCircle{35}{88}{2}{Dyson_Logos/forgotten_city}
  \mapCircle{27}{09}{2}{Dyson_Logos/qualme_temple}
  \mapCircle[4]{56}{52}{2.5}{Dyson_Logos/town}
  \mapCircle{44}{41}{2}{Dyson_Logos/redfall}
  \mapCircle{83}{09}{1.7}{Dyson_Logos/shadow_gate}
  \mapCircle{86}{45}{1.7}{Dyson_Logos/lochside}
  \draw[very thick,white] (11,0.6) -- (12,0.6) node[anchor=north]{\outline{10 Miles}} -- (13,0.6) ;
}

\mapNotes{
  \nameref{banditLair}/16/73,
  \nameref{lostcity}/35/86,
  \glssymbol{yonder}/35/94,
  \glssymbol{paik}/56/61,
  \Glsentrytext{town}/56/50,
  \Large\glssymbol{paik}/44/49,
  \glsentrytext{redfall}/44/41,
  \glssymbol{abderian}/83/13,
  \nameref{shadowVault}/83/05,
  \normalsize\nameref{silentGorge}/83/30,
  \glssymbol{sylf}/86/49,
  \gls{lochside}/86/45,
  \D/27/15,
  \nameref{necromancers_lair}/25/08,
}

\widePic[t]{Irina/greylands}

\end{multicols}

\printAllSideQuests{Roads}
\label{sqList}

\begin{multicols}{2}
\noindent
Each time the \glspl{pc} go to \gls{town} or the forest, they must pass through this middling \gls{region}.

\Gls{banditking}, the would-be \gls{warden} to \gls{valley}, hides in the \nameref{banditLair}, just past the \gls{edge}, and slowly gathers a small army of malcontents, mostly from the \glspl{guard}.
His messenger will also approach \glspl{pc}, who will have to decide whether or not to join his violent revolution.

Take a look at the thin crust of protective structures surrounding delicate fields on the map \vpageref{Irina/greylands}.

\begin{itemize}
  \item
  The little houses on the inside are scattered hamlets, each controlling large amounts of farmland, full of grazing livestock.
  \item
  The outer structures with multiple rooves are \glspl{village}, with tall walls to defend themselves from the constant probes from \gls{sylf}'s children.
  \item
  The structures with a single tower are \glspl{broch}.
  Each morning and evening, the \glspl{guard} of these tall towers play loud pipes to attract the forest's \glspl{monster} towards them, and away from the inner farmlands.
\end{itemize}

When \pgls{fiend} destroys any \gls{broch} or \gls{village}, you should take a pen, and score out the settlement.

\end{multicols}

\printAllSideQuests{Forest}

\begin{multicols}{2}
\noindent
The forest holds two active \glspl{fiend} who want to eat the outer \glspl{village}.
\Gls{necromancer} emerges from \nameref{necromancers_lair} to kill people to grow his undead army.
\Gls{spiderqueen} stays in a mobile fortress of webs, and wants to feed farmers to her \glspl{crawler} pets.

And if the outer \glspl{broch} and \glspl{village} fall, all the \glspl{monster} of the forest will flood into the unprotected, interior farmlands.
\Gls{valley} would then become uninhabitable, and turn into a flood of refugees, all trying to reach some other civilization before \gls{cTwo} brings \glspl{snow}.

\end{multicols}

Making matters worse, the \glspl{wolfhead} want to poke around the forest in search of ancient alchemical gateways.
They, or the \glspl{pc}, will eventually find the remains of \gls{archwarp} and \gls{sixshadow} from \gls{lostcity}, which bring all new complications.

\printAllSideQuests{Town}

\begin{multicols}{2}
\noindent
\Gls{town} will provide the \glspl{pc} somewhere to buy every kind of supply (check the handouts at \gls{town}'s entrance, \autopageref{townStart}), but a violent revolution is brewing from the bowels.

The \glspl{pc} may stop the \glspl{diggers}' plan to destroy \gls{town}'s \glspl{keeper}, or ignore them, or even help the \glspl{diggers}, and start to consider how \gls{town} could remain stable after the death of its \gls{warden}.
\end{multicols}

%%%%%%%%%%%%%%%%%%%%

% How to Run - starting base (link), missions, open-ended.

\begin{multicols}{2}
\subsubsection{Wandering Monsters}
should keep wandering throughout the \gls{campaign}, because their presence provides the \glspl{pc} with a reason to go out and hunt, and because their presence is the crux of the matter -- they are the doom which comes if \gls{valley} tears itself apart.

Most \glspl{segment} require insight and negotiation, rather than combat.
This helps the \glspl{sq} run alongside random encounters, without making dangerous combat-encounters repetitive.

The players may take a while to fully understand the grander political schemes threatening \gls{valley}, and even longer to come up with counter-schemes.
Bog-standard \gls{guard} missions can help kick-start \glspl{journey}, as they often involve entering the forest.
\iftoggle{judgement}{%
  You can find some random tables to generate missions in the \textit{Book of Judgement}, \autopageref{NGmissions}.
}{}

% Factions, death, letting go.
\subsubsection{Player Interference}
should be considered desirable.
Every \gls{segment} could change.
For example, if \pgls{pc} somehow kills the \gls{spiderqueen} during the second \gls{segment} of her \gls{sq}, the rest of the \glspl{segment} would disappear instantly.
However, most \glspl{sq} will not fade away so easily, because most of the protagonists come from one of five political factions.
If \gls{banditking} dies, \gls{traitor} (or even his brother \gls{sewerking}) can take his place.

Since the \glspl{pc} begin in the \gls{guard}, that makes them part of the \glsentrytext{establishment} at the \gls{campaign}'s start.
Of course, this may change soon, as \gls{sewerthief} will gently inquire how the \glspl{pc} feel about joining the \glspl{whiteBandits}, or they may find themselves making concrete plans with the \glspl{wolfhead}.

\end{multicols}

\setglossarypreamble[people]{
  These are the movers and shakers, the doers and thinkers.
  You can find each of them in the \glspl{sq}, sometimes as primary actors, other times wandering into the background of multiple threads and \glspl{segment}.
}

\printglossary[
  type=people,
  style=topicmcols,
]

