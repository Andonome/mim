\section{\Glsfmttext{digger} Catacombs}
\label{sewers}

\histEvent{100}{3}{\Glsfmttext{town}'s \glsfmttext{warden} commissions underground tunnels for transportation and storage.
The local \glsfmttext{templeOfSickness} use some sections as a catabomb}

\begin{multicols}{2}

\begin{exampletext}
  \noindent
  A century ago, the old \glsentrytext{warden} of \glsentrytext{town} commissioned tunnels to transport goods from the outer gates to the citadel.
  The \gls{healersGuild} also used the tunnels to place the bodies of the dead, then it expanded into a storage area for anyone who wanted to pay for them.
  The tunnels held myriad locked cupboards.

  \histEvent{40}{8}{\Glsfmttext{town}'s tunnels and catacombs collapse.
  The town leaves them abandoned}

  During the earthquakes of the season of Otsea%
  \exRef{judgement}{Judgement}{Otsea}
  it collapsed, and everyone left the area.

  \histEvent{30}{9}{\Glsfmttext{town}'s forsaken crawl down into the tunnels, using them as housing and shade}

  Soon after, someone noticed an opening in an old drain pipe, and entered.
  Others followed, and now a small population of people live, who have nowhere else to go.
  \Gls{town} began to gossip about burglars and cutthroats living in the sewers beneath them.
  Like most gossip, it was both true and misleading.
  Aside from the occasional nasty character, most are decent people who had nowhere else to go, and in time a little community formed in the tunnels beneath \gls{town}, based on long discussions, voting, and sharing what they could to survive.

  They call themselves the \glspl{digger}.
\end{exampletext}

\histEvent{4}{11}{%
  The \glsfmttext{sunGuard} attack the \glsfmttext{digger} catacombs, and set fire to bedding.
Smoke inhalation forced them to flee, and the \glsfmtplural{digger} learnt to use smoke to defend their small turf%
}
\begin{exampletext}
  \label{guardAttackHistory}
  Last \gls{cycle}, the \gls{sunGuard} decided to raid the sewers.
  A dozen descended into the tunnels, and began to beat anyone they found.
  They meant to send a brutal message that people should not live down beyond the \gls{warden}'s laws.

  When they found some old men, sitting and playing cards, they kicked them until their bones broke, then set fire to the bedding and straw in the room.
  Setting fires underground produces a lot of smoke, with nowhere to go, so the guards began to choke, and had to flee to the surface.
  This taught the \glspl{digger} a lesson in underground warfare, which they put to good use.

  The people in the sewer gathered supplies for the battles to come, and fix the various doors to the sewers which had rotted or had rusted hinges.
  Then they made a secret, small, storage room by covering a small door in clay, and stuffing it with bundles of twigs.

  The next time the \glspl{sunGuard} descended for a raid, the \glspl{digger} were ready.
  They had people listening near the entrance, who pulled out the twigs, set fire to them, and ran back to light more.
  Every door held the \glspl{sunGuard} in a smoke-filled room for another moment, while the \glspl{digger} scattered to rooms prepared with cloth around the door-frames, waiting to emerge once some of the smoke had cleared.

  During the fight, the \glspl{digger} captured \pgls{sunGuard}, and killed another.
  The \gls{sunGuard} no longer chase criminals into the tunnels beneath \gls{town}.
\end{exampletext}

The \nameref{sewers} has entrances in \glspl{area} \ref{farmExit}, \ref{sewerWaterHall}, \ref{sewerDrop}, \ref{sewerPigWalk}, and \ref{slum_exit}.

\paragraph{The narrow hallways}
make longer weapons difficult to use.%
\exRef{core}{Core Rules}{enclosedcombat}

\mapPic{t}{Dyson_Logos/sewer}{
  \ref{farmExit}/23/95,
  \rotatebox{60}{\normalsize\nameref{farmExit}}/14/85,
  %{\large\D}\ref{sewerCells}/21/75,
  %{\large\D}\ref{sewerCells}/52/06,
  \D\ref{sewerLockup}/07/41,
  \D\ref{sewerLockup}/37/03,
  \ref{oldlibrary}/28/49,
  \ref{sewerSticks}/86/63,
  \large S/83/65,
  \Large\ref{sewerKingRoom}/71/51,
  \Large\ref{wireTrap}/70/31,
  \ref{sewerWaterHall}/48/38,
  \rotatebox{-23}{\small\nameref{sewerWaterHall}}/43/32,
  \D\ref{sewerLockup}/61/42,
  %{\large\D}\ref{sewerCells}/21/75,
  \Large\ref{sewerFood}/70/69,
  \ref{underHall}/77/27,
  {\normalsize\rotatebox{-90}{\nameref{underHall}}}/87/27,
  \ref{citadelTunnel}/11/18,
  {\normalsize\nameref{citadelTunnel}}/05/09,
  %{\large\D}\ref{sewerCells}/21/16,
  \ref{sewerPigWalk}/84/82,
  {\small\rotatebox{45}{\nameref{sewerPigWalk}}}/90/81,
  \ref{sewerDrop}/92/70,
  {\normalsize\rotatebox{-90}{\nameref{sewerDrop}}}/97/69,
  \ref{slum_exit}/71/91,
  {\normalsize\nameref{slum_exit}}/71/98,
  \Large\ref{wireTrap}/49/76,
  \Large\ref{underGuard}/63/865,
  %{\large\D}\ref{sewerCells}/43/79,
  \ref{sewerShrine}/56/77,
  \Large\ref{sewerDog}/33/67,
}

\mapentry[farmExit]{Exit to the \Glsentrytext{jotter}}

This tunnel connects the \nameref{sewers} to \gls{traitor}'s house, just outside \gls{town}'s walls.
The \glspl{digger} smuggle goods through here for \glsentryfirst{pigowner}, and use it to travel, without moving through any official entrances to \gls{town}.

\Gls{traitor}'s house above has several rooms, as he's done rather well for himself, and hopes that once the revolution comes he'll be in an even better position.

\mapentry[oldlibrary]{the Old Library}
\index{Library!Ancient}

\begin{exampletext}
  The old library once held a treasure of autobiographies, dictated by the dying.
  This includes outdated maps of areas in \gls{valley}.
  The \glspl{digger} considered the ancient library too sacred to disturb at one point, but that changed only with the arrival of the \gls{bskulls}, at which point none of them wanted to go near the thing.

  It detects the little bits of death we all carry with us -- injury and tiredness -- and use the feeling of their position to curse them with a state of semi-undeath (even through walls).
  It likes to chatter, unendingly, about things it has never experienced, only learned from an undead priest who can't remember what the Sun looks like.

  When \gls{sewerking} stole the \gls{bskulls} \gls{artefact} from \gls{necromancer}, he took it out in this room to examine in, and immediately it cast \textit{Soul Specks} on him, which allowed it to speak with him.
  He avoided the \gls{artefact} for some time, until he learned that if the skull curses someone, and they die while the spell remains in effect, they become ghasts -- sentient undead.%
  \exRef{judgement}{Judgement}{ghast}

\end{exampletext}

\sidebox{
  \begin{boxtable}
    \textbf{\gls{tn}} & \textbf{Map Found} \\
    \hline
    8 & Alchemical Basement \\
    11 & \Gls{necromancer}'s Temple \\
  \end{boxtable}
}
Searching the books for meaningful information requires a Team Roll of \roll{Intelligence}{Academics}.
Success indicates that they have found an ancient map.
Find the maps in the handouts.

\begin{boxtext}
  Empty stone shelves show where an expansive library once provided the entire city knowledge, but not a scrap of paper remains.
  Three stone pillars divide the room, each with a brazier hanging in front of them by a chain.
  The central pillar's brazier is made from three human skulls.
\end{boxtext}

\artefact{bskulls}% Name
  {Three human skulls, tied together with their guts, each stuffed with their own inflated bladder.
  The Preservation spell cast upon it stops the bladders deflating.}% Body
  {2}% Intelligence
  {0}% Wits
  {3}% Charisma
  {to make everyone see as the dead see}% Mission
  {Preservation}% Base Spell
  {
    \setcounter{Fate}{2}
    \setcounter{Air}{2}
    \setcounter{Water}{1}
    \setcounter{Vigilance}{1}
    \setcounter{Academics}{1}
  }% Spheres

\showStdSpells[
    \revelationSpell
    \stepcounter{enc}
]

\mapentry[sewerLockup]{Dead Rooms}

\begin{exampletext}
  The \glspl{digger} have been kidnapping rich people from the \gls{templeOfSickness} above, and locking them in the various little rooms around the catacombs.
  The \gls{bskulls} casts \textit{Soul Specks} on them, though it remains too far away for any real conversation.
  They sit in the cells, hearing the unnatural gibbering of the chatty \gls{artefact}, until they die, and start to feel the unique hunger of undeath.

  The \glspl{digger} plan to have the ghasts storm \gls{town}'s citadel, once they have enough.%
  \footnote{See \vpageref{ghastPlan} for the plan.}
\end{exampletext}

Each storage cell has a bar across the front, and a few ghasts, stuffed in tightly.
Each was created by the \gls{bskulls}.

\paragraph{Anyone knocking off the bar}
will probably find a nasty surprise.
Consult the chart (no matter which door the \glspl{pc} open first, that's the first door).

\begin{nametable}{Cell Contents}
  \textbf{Door} & \textbf{Contents} \\\hline
  1st & 5 longswords. \\
  2nd & 7 ghasts jump out, and one rushes off to eat Gritbite (\vref{citadelTunnel}) \\
  3rd & String on the door indicates \gls{sewerthief} is taking a nap while he should be working. \\
  4th & An older woman, kidnapped from the \gls{templeOfSickness}.
    The \gls{bskulls} has cursed her, and she suffers from \textit{Torpid Flesh}, which makes her look rather ghastly.  When she hears the \glspl{pc} pass, she screams to be let out.  \\
  4th & 2 ghasts jump out, and one flees towards \nameref{farmExit}, above. \\
  5th+ & 5 ghasts jumps out. \\
\end{nametable}

\Gls{sewerthief} sleeps secretly by tying a piece of string round the door's bar, pushing the string inside the room, then balances the bar on top of the door.
When he closes the door, the bar falls into place, leaving only a small strip of string as evidence of the trickery.
The troupe can spot this, with a \roll{Wits}{Vigilance} roll (\tn[13]).

\paragraph{If the \glspl{pc} listen at the door,}
they find the dead are silent when not active, and don't respond to noise, as they are also deaf.%
\exRef{judgement}{Judgement}{undead_senses}

\newGhast[\npc{\T[7]\D}{\arabic{noAppearing} Ghasts}]

\mapentry[citadelTunnel]{Tunnel to Citadel}

\begin{exampletext}
  \Pgls{digger} is doing what they do.
  Gritbite digs whenever she can, so she can inch ever closer to \gls{townmaster}'s citadel.
  Every time her pick comes down, the thinks about the ghasts tearing \gls{townmaster} apart, and seeing a new order rise in \gls{town}.
\end{exampletext}

\paragraph{If the \glspl{pc} don't act extremely friendly,}
she flees.

No matter how they present, she won't tell them about the \glspl{digger}'s plan.

\begin{boxtext}
  A scraping sound comes from the tunnel ahead.
\end{boxtext}

\humanthief[\NPC{\F\Hu}{Gritbite}{red-faced, short hair}{huffs, loudly}{to see \glsentrytext{townmaster} die}]

\mapentry[sewerDog]{The Guard Dog}

\begin{exampletext}
  The \glspl{digger} found the \gls{bskulls} has no interests in animals (or at least dogs), so \gls{traitor} keep one of his dogs chained here at all times (in rotation, so they don't get bored).
\end{exampletext}

\paragraph{If Rover hears them,}
he begins barking and four of the armed \glspl{digger} in room \ref{underHall} (\nameref{underHall}) come to investigate.
They will walk a complete circuit round the area, while pre-emptively complaining about `those damned skulls' giving them the curse.
After doing that since walk, they leave.

The troupe can spot Rover before he spots them with a \roll{Wits}{Stealth} roll (\tn[10]).

\paragraph{If \pgls{digger} walks by,}
Rover does nothing -- he knows them all by scent alone.

\huntingdog[\npc{\A}{Rover}]

\mapentry[sewerWaterHall]{Drowned Hallway}

All the waste and filth from \nameref{stationDungeon} flows down a grate, and lands here.
The frigid water inflicts \pgls{fatigue}, and the filth demands a \roll{Strength}{Medicine} roll (\tn[10]) to avoid contracting a disease which slowly paralyses them.%
\exRef{judgement}{Judgement}{Corpse Hands}
The disease will not take effect for $1D6$ \glspl{interval}.

The roof lowers to meet the water, forcing heads to duck under.

\begin{boxtext}
  Going down the stairs you feel your feet hitting cold water.
  It's not clear how far the water goes down, but it's cold.
\end{boxtext}

\paragraph{Halfway through the icy slime,}
a little space to rise, and breathe, emerges.
And above, \emph{a roar echoes down}.
A tunnel leads straight upwards, and up to the grate sitting in the \nameref{stationDungeon} (room \vref{dunGrate}).

Neither \gls{sewerking}, nor anyone else, have any idea this tunnel exists.

However, if \gls{sewerking} has rescued his brother (see \autopageref{banditsCaught}), he finds out about this tunnel, then later traps it by placing two ghasts down here, chained to heavy weights.

\newGhast[\npc{\T[2]\D}{\arabic{noAppearing} Ghasts}]

\mapentry[wireTrap]{Wire Trap}

Both areas are barred from the outside (i.e. the lighter side on the map, \vpageref{Dyson_Logos/sewer}).
Opening the doors from the anterior side requires an \roll{Intelligence}{Larceny} roll (\tn[10]).

A thin wire, nailed across the rock walls, will pull the shins of anyone crossing it.
Seeing the wire in the dark requires a \roll{Wits}{Athletics} roll (\tn[12] by light, or 16 in darkness).

Failing while going uphill means the character falls Prone,%
\exRef{core}{Core Rules}{prone}
and those going downhill also suffer $1D6$ Damage.

\mapentry[sewerShrine]{Ashen Corpses}

\begin{exampletext}
  The \glspl{digger} once used this as a communal room, until the \gls{sunGuard} arrived and set fire to it.
  They've left the ashen corpses and burnt mattresses as a reminder, and a cenotaph.

  They had nowhere to bury the bodies, so they had to throw them down the drain (\gls{area} \vref{sewerDrop}).
\end{exampletext}

\begin{boxtext}
  Burnt bedding, and a trail of old, crusted blood have left a permanent smell in this room.
  At the far side is another exit.
\end{boxtext}

\mapentry[underGuard]{Guardroom}

Here four of the \glspl{digger} sit and play simple dice games to pass the time, or occasionally sleep in the foetid straw.

\paragraph{If the \glspl{pc} fight,}
the cutthroats try to ram them out the door, and run to alert the others in room \ref{underHall}.

\humanthief[\npc{\T[4]\F\M\Hu}{Four Cutthroats}]

\mapentry{\Glsentrytext{sewerking}'s Room}
\label{sewerKingRoom}

During the night, \gls{sewerking} sleeps here.
The other \glspl{digger} don't appreciate someone `being all posh' in their living space, and taking the best room.
But they put up with it, as \gls{sewerking} contributes a lot, and gets shit done.

The room contains:

\begin{itemize}
  \item
  A map of the area (check the handouts).
  \item
  A Wolf-Run potion.
  \item
  Many books.%
  \footnote{The books are variously written on history (real and imagined), \textit{The Art of Lies} (by an elvish author -- `Erend\"e'), and instructions on hosting a dinner party (it takes 280 pages to say `use seasoning`, and `have an anecdote').}
\end{itemize}

\talisman{Wolf-Run}% Name
  {detailed}% Enhancements
  {Wax}% Action
  {Air, Water}% Spheres
  {current \glspl{hp} plus \gls{dr}}% Resistance
  {Drinking the muddy potion increases the target's Speed by +\arabic{spellPlusOne}, while giving them a wolf-like outer layer of fur.
  The spell also inflicts a -\arabic{spellCost} Penalty to Intelligence, and forces the target to eat an extra \arabic{spellCost} meals per day, or suffer \glspl{fatigue} from starvation}% Summary
  {The effects endure until the target takes \gls{fatigue} Penalties due to starvation}% Details

\showTalisman

\paragraph{Picking the lock}
requires an \roll{Intelligence}{Larceny} roll (\tn[9]).

\begin{boxtext}
  The door opens to a noble's room, bearing a striking contrast to the dungeon around.
  The bed's well made, the sheets are silk, and various books sit on shelves.
  On the table sit various maps.
\end{boxtext}

\mapentry[underHall]{Grand Hall}

Here, the down-and-outs of \gls{town} gather, sleep, drink, and discuss ways to make money.
Permanent residents sleep in the alcoves to the side.
Theft is nearly impossible, as the \nameref{underHall} never sits empty.

At night, this place contains 30 people, and the air is foetid.
During the day, it contains 20.

Most of these people have no idea about the grader plan to fill the catacombs with ghasts.
They only know that they should never remove the bars from the doors, because those tunnels hold a curse.

\begin{boxtext}
  Around the next corner, the noxious smell of fat-candles stings your eyes.
  Myriad voices murmur, as if planning, quickly.
\end{boxtext}

\paragraph{If the \glspl{pc} approach from the lower doors,}
this shocks the crowd, and they run for the surface, all at once (\gls{area} \vref{slum_exit}).
Once half the crowd reach the slippery stairs, a couple at the top slip, and fall, creating a heavy cascade of bruised flesh and broken bones.

If the troupe follow the crowd, the first receives $1D6$ Damage, and gains $1D6$ \glspl{fatigue}, the second $1D3$ of each, and so on.

\paragraph{If the \glspl{pc} approach from the top door,}
(which leads along to \nameref{sewerKingRoom}) everyone will assume they're \glspl{guard}, trying to run from their duties.
The crowd will poke fun at them, then welcome them.

\begin{speechtext}

  ``Had enough monsters eh?
  Careful, this one's a proper monster!''

  ``\Glspl{bothy} toilets not good enough fer yer arse?''

  ``\Glspl{sunGuard} scaring yous?
  Go have a sit, I'll draw you a nice warm bath\ldots''

\end{speechtext}

\begin{description}
  \item[Doorcane] -- a woman with an abusive husband, who's fled to the safety of the \nameref{sewers}.
  \item[Chowdirge] -- an apprentice thief, learning from \gls{sewerthief}.
  \item[`Shitrat'] -- (real name `Starvale') the youngest son of \gls{townmaster}, here to find out what `real life is like'.%
  \footnote{Of course, \gls{townmaster} has no idea he comes here, and the undertown residents have no idea who his father is.  If a player makes note of the strange name, tell them that the boy sounds a bit strange (he's putting on a fake lower-class accent).}
\end{description}

\paragraph{If the \glspl{pc} corner the crowd and attack,}
they find a dozen ready and willing to fight.

\sewerking

\humanthief[\npc{\T[6]\E\Hu}{6 Sewer Bandits}]

\humanthief[\npc{\T[6]\E\Hu}{6 Sewer Bandits}]

\paragraph{Three rounds later,}
the \glspl{digger} from room \ref{underGuard} arrive (assuming they weren't killed already) with kindling and torches, and begin lighting fires.

\mapentry[sewerFood]{Food Storage}

\begin{exampletext}
  The basket of food in the basket contains a cursed apple, which contains an intense laxative.
  \Gls{sewerking} suspects one of the \glspl{digger} who live down here in the \nameref{sewers} steals food, so he left that tantalizing basket with a poisoned apple, injected with a laxative.
\end{exampletext}

\paragraph{Eating the apple}
demands a \roll{Wits}{Medicine} roll (\tn[11]) to notice the poison on the apple, if they try to eat it.%
\footnote{Feel free to roll for the characters so they're not aware there's a problem.}
Failure means that \pgls{interval} later, they receive 3~\glspl{fatigue}, then 2~\glspl{fatigue} on the next, then 1.

\begin{boxtext}
  Barrel after barrel fills the room, along with the smell of wine, apples, and vinegar.
  A little basket of choice snacks sits on top.
\end{boxtext}

\mapentry[sewerSticks]{Sticks}

The \glspl{digger} hid a small, wooden door by simply covering it in clay.
Inside, they keep large bundles of sticks and plenty of tinder boxes.

\begin{boxtext}
  As you push against the strangely-coloured section of the wall it creaks open, and bits of the covering fall onto the ground.
  The rings of shelves show kindling -- enough to start two-dozen fires, but no actual fire-wood in sight.
  On the immediate right, a dozen candles and tinder boxes.
\end{boxtext}

\mapentry[sewerDrop]{Sewer Entrance \& Abyss}

This artificial stream loops round from \gls{pig}, above.
The stream continues downwards to an underground abyss.

\begin{boxtext}
  The stairs go upwards for some time, and eventually arrive at a room filled with barrels of food, and a trapdoor above.
  You can hear a deep snore coming from just beyond the trapdoor.
\end{boxtext}

\paragraph{Anyone venturing down-river gets swept along,}
and dies in the unending blackness, unless they make a \roll{Dexterity}{Caving} roll, \tn[10], to cling onto the sides.
\Glspl{pc} can spend 5~\glspl{fp} instead to make the roll automatically.

\mapentry[sewerPigWalk]{Entrance to \glsentrytext{pig}}

Up these stairs, the troupe can reach the bowels of \gls{pig}.
\Gls{pigowner} understands what trouble she's in, and immediately accuses the characters of theft and calls on everyone in the room to kill them.

Calming down the patrons requires a \roll{Charisma}{Empathy} roll (\tn[10]).

\mapentry[slum_exit]{Stairway to the Slums}

An old drain leads a long way, and exits at multiple points along \gls{town}'s streets, then into the \nameref{greyDoulaShop} (area \vref{greyDoulaShop}).

\end{multicols}
