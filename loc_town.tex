\mapPic[\Large]{t}{Dyson_Logos/town}{
  \Huge 1/87/50,
  2/86/41,
  3/34/75,
  4/60/65,
  5/80/29,
  6/84/29,
  7/78/30,
  \rotatebox{9}{\footnotesize Big Docks}/60/48,
  \rotatebox{-9}{\footnotesize Shale Docks}/39/48,
  \rotatebox{-90}{\footnotesize Windy Bridge}/47/51,
  \rotatebox{-54}{\footnotesize Citadel Bridge}/72/55,
  \rotatebox{-9}{\scriptsize Cowbridge}/51/33,
  \rotatebox{-1}{\scriptsize Sheepbridge}/50/22,
  \rotatebox{-90}{\footnotesize Spy Grate}/23/54,
  {\footnotesize Market}/53/71,
  {\footnotesize Market}/60/35,
}

\begin{table*}[t]

\begin{rollchart}

  Number & Location \\\hline
  1 & The Citadel of \glsentrytext{townmaster}. \\
  2 & The Guard Station. \\
  3 & The White Horse Tavern. \\
  4 & The Paper Guild. \\
  5 & \Glsentrytext{pig}. \\
  6 & Butcher's entrance to the Lost Library in the sewers. \\
  7 & Temple of Ohta. \\

\end{rollchart}

\end{table*}

\subsection{The Market}
\label{greyMarket}
\index{Market!\glsentrytext{town}}

Every morning, wagons pour from the villages into \gls{town} with food, and enter the market.

Servants of the nobles arrive to buy large baskets.
\Gls{townmaster} and the guild masters always eat well.

The food sellers then take their coin, and move to stalls where the guilds sell their wares -- beer, weapons, contract-witnessing, blessings, clothes, carts, oil, and anything else a village might not produce.
Despite its remote location, one can buy almost anything in \gls{town}.

Next, the \gls{guard} take their share.
They may be few, but they buy a large portion with their heavy taxes on the guilds.

``No price is too high for safety'', they tell everyone inside \gls{town}'s walls.

While there isn't a legal difference between the guards in the city and the \gls{guard} outside, the difference is palatable.
Those inside are used to being heavy handed or violent with people, but they're terrified of the monsters outside, or anything unnatural.
Many have some distant family connection to \gls{townmaster}, ensuring their intense loyalty.

Servants of nobles or the guild then pay for the last of the good food at the market, and return home with what they can get.
Their employment alone keeps them out of the \gls{guard}.
When servants and lower-ranking guild members misbehave, they find themselves with a quick job outside the walls.
And the worse the deed, the closer they may come to the \gls{edge}.

The last group live in a state of constant questioning.
Will they find a little work, or steal today?
If they steal, will it be the gallows or the guard?
If someone makes them work in the \gls{guard}, what kind of creature will eat them?

Criminals taken into the \gls{guard} never return to \gls{town}.
People say they were `eaten by the forest'.

Take a look at the handouts for the \gls{town} market.

\subsection{The Paper Guild}
\label{paperGuild}
\index{Library!Paper Guild}

This massive library contains a wealth of unsorted books in piles, stacks, shelves, and occasionally used as doorstops (when Foxglove, the sorting-librarian, disapproves of the book).
Most books are arranged by size first, then either author (if the author wrote enough) or language (authors who write in multiple languages have some books placed by language first), then quality (with marks out of nineteen), and finally arranged by excellence of handwriting.

\humandiplomat[\NPC{\F\Hu}{Foxglove}{Running nose}{Sings `Nobody Knows the Trouble I've seen'}{C\'al\"e}]

While a number of librarians and priests of C\'al\"e within the Paper Guild outrank Foxglove, after sorting all of the books, only she knows how to find them.
As a result, all requests must go through her, and the amount of time she takes generally depends on her disposition towards that person.

Most research rolls here can bring up any number of items -- everything might be found, if only you can find what it lies under.
Anyone in Foxglove's good graces receives a +4 Bonus to research rolls.

\begin{itemize}

  \item
  Researching the local area might bring up a map of the whole of Maitavale (Intelligence + Wyldcrafting, TN 14).
  \item
  Researching old temples of Qualm\"e might bring up the map of \gls{necromancer}'s temple (from before it crumbled).
  \item
  Asking the young people who frequent the library about what they are doing reveals a young man who has mapped all of Greytown (Intelligence + Empathy, TN 10).
  \item
  Researching \gls{lostcity} yields a map of a small wing from the \gls{college} (Wits + Academics, TN 18).
  \item
  Researching old elvish settlements yields a map.
  The format makes little sense, the cartographer seems to name random places, and age has changed everything which might have been remotely useful.
  However, if you look closely, with an elvish eye, this map matches Maitavale.
  It also shows the location of two nura portals, marked 'gate place'.

\end{itemize}

These various maps can be found in the handouts.

\subsection{The Citadel}
\label{citadel}

The citadel is massive, and contains various floors.

\begin{enumerate}

  \item{Ground Floor: Outsiders}
    \begin{itemize}
      \item{Left Wing: Ballroom.}
      \item
      Left Wing: Guardroom.
      \item
      Right Wing: Dining Room.
      \item
      Right Wing: Servants' Quarters.
      \item
      Right Wing: Kitchen.
    \end{itemize}
  \item{First Floor: Insiders}
    \begin{itemize}
      \item
      Left Wing: Guest Beds.
      \item
      Left Wing: Office.
      \item
      Left Wing: Library.\index{Library!Modern}
      \item
      Right Wing: \Glsentrytext{townmaster}'s Sons' 9 Quarters (a nearby tree stands tall enough to access one room).
      \item
      Right Wing: Secret Stairway up to the floor above.
      \item
      Right Wing: Winery.
    \end{itemize}

  \item
  Second Floor: Others
    \begin{itemize}
      \item
      Left Wing: \gls{alchemist}, the Alchmist's Study.
      \item
      Left Wing: \Glsentrytext{townmaster}'s close servants' quarters.
      \item
      Right Wing: \Glsentrytext{townmaster}'s room.
      \item
      Right Wing: Treasury.
    \end{itemize}

\end{enumerate}

The lower floor holds fifteen guards in each wing.

\humansoldier[\npc{\T[5]\Hu}{The Citadel Guards}]

\paragraph{If trouble emerges in the Citadel,}
all the guards rush to the source of the noise, ready to prove themselves.

\humandiplomat[\npc{\T[9]\M\Hu}{\Glsentrytext{townmaster}'s Nine Sons}]

\paragraph{If \gls{townmaster}'s sons find intruders,}
they talk big, then surrender before the fight has begun, reminding the intruders that their father will pay handsomely.

\citadelAlchemist

\label{citadel_alchemist}

\paragraph{When \gls{alchemist} sees lawbreakers,}
he threatens the most awesome magic imaginable, even if his real powers leave much to be desired.

\townmaster


\subsection{The Temple of Qualm\"e}

The temple houses priests who white biographies (mostly for nobles), handle pensions (repaid on a per-family basis), practice starvation (so others can have more), and sing prayers for their ancestors to stay somewhere nice in the afterlife.

If the temple's groundskeeper -- Boris -- ever finds out what happened to the bodies they lovingly prepare before placing in caskets, he enters a rage and demands to join the PCs to confront the perpetrators immediately.

\humanpriest[\npc{\M}{Boris}]

