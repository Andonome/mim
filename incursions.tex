\resumecontents[Town]
\resumecontents[Forest]

\sidequest{Here and No Farther}\label{herenofarther}

\stopcontents[Town]
\stopcontents[Forest]

\startcontents[sq]

\sqminitoc

\noindent
Centuries ago, men created a logging city, and the elves destroyed it.%
\footnote{Other Side Quests offer explanations of why this happens.
The truth is that the humans were opening nura portals, and mostly destroyed the city themselves.
The elves just cleaned up.} 
Recently, \gls{townmaster} has decided to reclaim it, though nobody knows exactly where it is.
He sent a dozen trackers to find the place in the hopes of rebuilding it.
However, \gls{forestpriest}, won't tolerate new attempts to reclaim what has been buried in the city.
He has been turning anyone who finds or hunts for the place into animals.

\sqpart{Villages}% AREA
{\squash Introducing the Forest Priest}% NAME
{\Glsentrytext{forestpriest} walks with the troupe}% SUMMARY

\begin{boxtext}

  Looking behind you, there's a short, slender man with long, brown, woollen robes coming towards you on horseback.
  He smiles serenely through the gusts of wind.

\end{boxtext}

\Glsentrytext{forestpriest} has heard men are journeying to the forest to search for the lost city, so he has decided to track them down.
When he meets the characters on the road, whether they're journeying to or from \gls{town}, he's going the same way, and asks if he can walk with them.
If he gets a moment to make conversation, he tells the troupe this:

\begin{speechtext}

  I've found nura slugs in the area.
  They're slow, so I could run away easily, but it doesn't bode well.
  If the nura are on the rise in the area, they must be coming from somewhere.
  Hopefully \gls{king} will make sure to raise an army for everyone's protection.

\end{speechtext}

Play the next Side Quest's encounter immediately.
\paragraph{If the character encounter danger,}
he helps, possibly by giving a blessing to the troupe mid-battle, or possibly by changing opponents into animals with the Polymorph sphere.

\forestpriest

\sqpart{Villages}% AREA
{You See a Deer}% NAME
{A human has been transformed into a deer, and simply stops to stare at the characters}% SUMMARY
\label{seeADeer}

\begin{boxtext}

  A startled deer stops in front of you, then just stops and stares.

\end{boxtext}

\textbf{Background:}
Kinblot, the trader, did the monthly food drop-off for the woodspy bandits, and \gls{forestpriest} saw him, then intervened by turning him into a deer.

\paragraph{If the characters begin to question the odd behaviour of the deer,}
they can make a Wits + Wyldcrafting Group Roll to notice that something is wrong, TN 10.

\NPC{\M}{Kinblot the Trader}{Personable}{Ruffles hair}{Tribe}
\person{1}% STRENGTH
  {1}% DEXTERITY
  {0}% SPEED
  {{0}% INTELLIGENCE
  {0}% WITS
  {1}}% CHARISMA
  {0}% DR
  {1}% COMBAT
  {Empathy~1, Wyldcrafting~2}% SKILLS
  {Nothing, not even clothing.}% EQUIPMENT
  {}

\paragraph{If returned to human form,}
Kinblot can instantly identify \gls{forestpriest} as the man who cast the spell on him, though in deer form he will have a lot of trouble expressing himself, and can do little more than show distress around the priest.

Kinblot has no idea he works for the woodspy bandits.
He simply takes orders from \gls{traitor}, and assumes the supplies go to \gls{guard} scouts.

\paragraph{The characters may decide to visit \gls{forestpriest} in order to have the spell lifted,}
but \gls{forestpriest} simply tells them that this is an ordinary deer, enchanted to think that it is a man.
Spotting the lie requires a Wits + Empathy Group Roll, TN 11.

\label{deerDropOff}
\paragraph{If the troupe ask \gls{alchemist} in the citadel for help,}
he takes the deer for a while to `cure' it, fails, and eventually leaves it in the \gls{whitehorse} tavern, where it remains as an oddity to amuse local nobles.

\paragraph{If the troupe want to dispel the Polymorph spell,}
the TN is 13.
If they succeed, the game is up, and everyone will know about \gls{forestpriest}'s actions, although they will still not know his motives.

\Gls{traitor} will claim all food drop-offs were for \gls{guard} scouts and that he simply forgot the paperwork.

\sqpart{Villages}% AREA
{The Dead Messenger}% NAME
{A dead messenger still has his scroll}% SUMMARY

\textbf{Background:}
The Immortal Bandits have attacked a caravan and killed everyone, but none of those present were literate, and did not think about a scroll being useful.
The messenger was from \gls{redfall}, and was charged with delivering a message to \gls{townmaster} from \gls{nurabaron}.

\begin{boxtext}

  Conversation stops abruptly when you see death on the road ahead.
  A caravan of four carts lie stagnant.
  Ten bodies and three horses, all filled with arrows, too many to count, but enough to show a senseless and vicious attack.
  A little at the side of the road, a man lies bleeding from his arm, with a scroll still clutched in his hand.

\end{boxtext}

The messenger's name is Tobias of \gls{redfall}.
As the troupe arrives, he says ``Take this to the citadel'', then falls silent from blood-loss.

\paragraph{Keeping him alive}
requires a Wits plus Medicine roll at TN 10.

The message has an unknown seal, belonging to no family in the area.
The characters know that breaking someone else's seal is illegal, but if they do, they find an encrypted message.
If they pass an Intelligence + Academics roll, TN 9, they find it states:

\begin{speechtext}

  \Gls{greentower} is nearly complete.
  The men say they can hear noises underground sometimes, so we will have to investigate soon.

  Last shipment lost.
  Please send replacement soon.

\end{speechtext}

\paragraph{If the characters break the seal before handing the message in,} \gls{townmaster} will be livid, and have them arrested.

\paragraph{If he does not receive the message,}
he will not approach \gls{greentower} until he receives another message, much later.
If he receives the message and believes it to be intact, then he will send \gls{alchemist} to \gls{greentower} to inspect its progress.

\paragraph{\Glsentrytext{alchemist}'s Journey}
begins by going out with ten of the \gls{guard} to the villages to see a festival, then he requests they leave him.
Soon after, ten of the woodspy bandits, dressed as pilgrims of Laiqu\"e will escort him the rest of the way.

If the troupe attempt to follow \gls{alchemist}, have them make a \roll{Strength}{Stealth} check (\tn[9]) to endure the journey without being seen.
Failure entails being spotted.

\paragraph{If \gls{townmaster} finds out someone has read his letter,}
he will have \gls{captain} order the troupe to go on some mission beyond the \gls{edge}, or (if they are not in the \gls{guard}) will have them arrested for suspected banditry.

\humansoldier[\npc{\T[10]\Hu}{10 Woodspy Bandits}]

\sqpart{Forest}% AREA
{\Glsfmttext{greentower}}% NAME
{The Woodspy Bandits are setting up a secret operations base}% SUMMARY

Deep in the forest, \gls{townmaster} has commissioned masons to secretly start building a single tower from the fallen stone of \gls{lostcity}.
\footnote{See page \pageref{expanding_wilderness} for more on \glsentrytext{townmaster}'s motivations.}
This is the first foray into the deep wilderness.
The top is to be disguised by painting it green (the colouring is made by malachite, which can be expensive).

\begin{boxtext}

  In the deep forest, where there should really be nothing, you see a little tower, and a man hanging out the window, painting it green.
  With another day of painting, the green tower could have been made almost invisible from a distance.

\end{boxtext}

Half of the masons are active members of the Woodspy Bandits, lead by \gls{traitor}.
The other masons are aware they must keep the project a secret, but think that they work for \gls{king}, at least indirectly.
As a result, the group cannot attack the PCs without some way to explain their behaviour to the regular masons.

At this stage it's unlikely that they find the secret portal to catacombs of the old Temple of Qualm\"{e}, as the builders have no idea what lies under their feet.
However, it is possible for them to discover this information by obtaining old maps of \gls{lostcity}.

\paragraph{While the troupe introduce themselves,}
The tower is assaulted by a basilisk.
If the characters seek sanctuary in the tower, the workers will let them in, unless there is a good reason not to.
See page \pageref{green_tower} for the layout of \gls{greentower}.

The basilisk will remain outside for a day, and its stench quickly drives all the bandits to the top of the tower, where the swords are stored.
At this moment, \gls{traitor} realises the illegal swords sit in plain view, whether or not the troupe realize he is a traitor.

\label{traitor}
\traitor

\npc{\T[6]\M\Hu}{6 Woodspy Bandits}
\person{2}% STRENGTH
{0}% DEXTERITY 
{0}% SPEED
{{0}% INTELLIGENCE
{-1}% WITS
{0}}% CHARISMA
{0}% DR
{1}% COMBAT
{Crafts 2, Wyldcrafting~1}% SKILLS
{\shortsword}% EQUIPMENT
{}

\basilisk

\paragraph{If the troupe have met the Woodspy Bandits before,}
they can spot these are the same people with a Wits + Empathy Team Roll.%
\exRef{core}{Core Rules}{teamwork}

\paragraph{The Woodspy Bandits may also recall the characters}
(with the same roll).
If they do indeed recall who the characters are, they will try to attack by surprise in order to keep the characters from spilling their whereabouts to anyone.
A slow conspiracy develops within the keep, as each one gets called away to whisper with the others, one by one, until they all agree to let the characters spend the night there, and stab them in their sleep.

\paragraph{If the PCs bring news of \gls{greentower}'s location to the \gls{guard},}
footnote{See page \pageref{guardstation} for more on the guard station.}
then the Woodspy Bandits will be chased out of the area, in due time, once all the necessary paperwork has been completed.

\paragraph{If anyone in the troupe can use mana,}
they feel magical energy recharging them from below constantly (3 MP per turn).%
\footnote{See page \pageref{underGreenTower} for details on the tunnels below.}

\paragraph{Take a pen,}
and add the tower to your map of the area, wherever the troupe happen to be in the forest.

\sqpart{Town}% AREA
{\squash Rogue Sheep}% NAME
{The troupe find a person turned into a sheep}% SUMMARY

Run this encounter in Town, at the same time as the next Side Quest's part.

\begin{boxtext}

  A man in ragged clothing, missing a couple of fingers and a couple of teeth, chases a sheep down the road.  A woman shouts out \emph{``Oi, Trevor! That ain't your sheep!''}.

  \emph{``Well whose bloody is it then, bitch?!''}, the ragged man snarls back.

  The sheep runs into a tavern, and Trevor runs in after it.

\end{boxtext}

Yesterday, \gls{townmaster} and a number of his men met to discuss further building inside the deep forest, and hunting for \gls{lostcity}.
One of the maids who worships at the temple of Laiqu\"{e} on the Town's outskirts heard this and told \gls{forestpriest}.
\Gls{forestpriest} tracked down the main architect, Darren, cornered him in an empty ally, and turned him into a sheep with his polymorphing ability.

The sheep goes into \gls{pig} because it's the only place of safety he knows.  The sheep is obviously sentient, if anyone bothers to ask him clear questions.

Before the characters can decide what they're doing, start the next encounter in \gls{town}.

\sqpart{Town}% AREA
{\Gls{forestpriest} Exiled}% NAME
{The town crier announces that \gls{forestpriest} is a wanted criminal}% SUMMARY

\Gls{forestpriest} has been found to be a criminal after \gls{captain} stalked him, and found him turning a trader into a goat.
The goat ran away, and was murdered and eaten before \gls{captain} could find him, so `cannibalism', has been added to \gls{forestpriest}'s list of charges.
\Gls{forestpriest} has since fled to the deep forest to escape trial.

\begin{speechtext}

  Hear, ye! Hear, ye!

  Hell-slugs have been caught roaming (slowly) at the crossroads.
  Be on the lookout for dangerous creatures, even while travelling close to \gls{town}.

  \Gls{forestpriest}, former high priest of Laiqu\"{e}, has been caught using black arts, including cannibalism, and now carries a total reward of 200 silver pieces living, and 300 silver pieces dead.

  All elves in \gls{town} must carry with them a letter of registration, stating their business in town, and their current lodgings.
  Reporting an unregistered elf brings a 50cp reward.

\end{speechtext}

Elvish characters may have a difficult time with the registrations, since everyone in \gls{town} will ask them for documentation in the hopes of turning them in for a reward.
Even characters who were previously friendly towards the troupe's elves begin to view them as a potential source of income.

\paragraph{If the PCs were looking for \gls{forestpriest},}
their job just got harder.

\paragraph{If they find him,}
then he tells them about overhearing the Immortal Bandits' plan to create a \gls{blight} (see \autopageref{blightConspiracy}).

\stopcontents[sq]

