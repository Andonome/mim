\sidequest[Roads,Town]{Sewer Bandits}
\label{sewerking}

\histEvent{6071}{2}{The \glsfmtplural{digger} begin to fill \glsfmttext{town}'s catacombs with ghasts}
\begin{exampletext}
  \noindent
  The \glspl{digger} under \gls{town} have been working on a plan.
  Ever since \gls{sewerking} returned with the \gls{bskulls} \gls{artefact}, they noticed that people who get too close to it see strange visions and lose much of the sensation in their bodies.
  Once \pgls{sunGuard} fell victim to this effect while on a raid, they killed him, and found him turned into a ghast -- a sentient undead creature, with a broken mind and a killer instinct.%
  \exRef{judgement}{Judgement}{ghoul}

  \label{ghastPlan}
  \begin{description}
    \item[Phase I]
    Armed with this \gls{artefact}, their latest plan is to kidnapp the sick and decrepid people from the \gls{templeOfSickness} (\vpageref{townHealersGuild}), lock them in one of the many chambers underground, and wait for them to turn undead.
    \item[Phase II]
    involves digging a tunnel underneath \gls{town}'s citadel (\vpageref{citadel}).
    \item[Phase III]
    is to release the ghasts into the citadel, and let undead-nature take its course.
  \end{description}

  In the meantime, \gls{sewerking} has been working as a fence, using the tunnel to \gls{traitor}'s house, just outside \gls{town}'s wall.

\end{exampletext}

\sqpart{Roads}% AREA
{Art Collectors}% NAME
{\Glsentrytext{sewerking} introduces himself as a fence}% SUMMARY

\begin{exampletext}
  After raiding \gls{necromancer}'s crumbling temple with \gls{traitor} (\vpageref{rottenBreath}), \gls{sewerking} stashed the \gls{artefact}, and decided to sell his more mundane (and valuable) prizes in \gls{lakeside}.
  He travels by road to avoid the heavy taxes levied in \gls{town}.%
  \Gls{lakeside} always has someone willing to buy valuable items, without too many questions.%
  \footnote{See \vpageref{lakeside}.}
\end{exampletext}

\begin{boxtext}
  In the far distance, a trading caravan approaches.
  You can see the head rider's long hair trailing behind him in the \showTemperature\ wind.
  Both coaches have the usual archer on the top, though they don't look like \glspl{guard}.
\end{boxtext}

\setcounter{wounds}{2}

\sewerking

\humanarcher[\npc{\T[6]\Hu}{\arabic{noAppearing} \glspl{whiteBandits}}]

\Gls{sewerking} greets the troupe in a friendly way, and explains he's going to \gls{lakeside} to sell his wares for a `client'.
He refuses to divulge any information about this client, then begins hinting that if the troupe ever run into anything valuable, and need to sell it, he can be very discrete.

\begin{speechtext}
  If you brave warriors ever need to sell any `paintings' you find, you can always find me in \gls{whitehorse}, in \gls{town}.
  Just ask for me, and they'll show you right in.%
  \footnote{This is simply not true -- he keeps a very low profile.}

  Anyway, it's a dangerous road.
  Fancy turning the other way and joining me?
  You know, all the work you do in the \gls{guard} has real value.
  People don't tell you that enough, but you guys save everyone, every day, and nobody says `thank you'.
  Why don't you come and protect people on this road, instead of that one, and I'll say `thank you', at the end.
\end{speechtext}

\Gls{sewerking} then offers the troupe 1~\glspl{sp} per mile, in total, if they join him till \gls{lakeside}.
He will only travel by road, so he can avoid \gls{townmaster}'s taxes.

The length of the journey depends on how far this scene takes place from \gls{lakeside}.
\Gls{sewerking} takes a route not too far, or close to town, so encounters should consist of traders as much as beasts.

\paragraph{If anyone asks about the archers,}
\gls{sewerking} explains jovially.

\begin{speechtext}
  We come from the Temple of `Odd-Job', to save people from the god of messing-about and time-wasting!%
  \footnote{The characters will understand that this is a joke.}
\end{speechtext}

\paragraph{If the party press the issue,}
\gls{sewerking} explains that he comes from \gls{whiteplains}, and while his family were \glspl{warden}, they have become poor, due to the high taxes imposed by \gls{town}, so now he travels and takes work where he can, and hires people to help him.

\paragraph{If the \glspl{pc} abandon their mission,}
the next time they see \pgls{jotter}, the leader will have some explaining to do.
They should roll \roll{Intelligence}{Deceit} (\tn[12]), or suffer a demotion.
Fodder who behave in this way (the lowest rank) will find themselves escorted to the \gls{court} in \gls{town}, where \gls{townmaster} (or some other \gls{warden}) will sentence them.%
\exRef{judgement}{Judgement}{pitOfJustice}

\paragraph{The wagons hold,}

\begin{itemize}
  \item
  20 days' rations.
  \item
  2 ancient human skulls, with red stones inserted around the crown.
  \label{skullCrown}
  \item
  A stone statue of a half-dead man in repose, in a coffin.
\end{itemize}

\begin{boxtext}
  As the Sun casts red over the forest, darkness closes around you.
  Then in the distance, fire, then another, not far off.
\end{boxtext}

\sqpart{Town}% AREA
  {Missing People}% NAME
  {The \glsentrytext{healersGuild} have noticed aging \glsfmtplural{warden} going missing}% SUMMARY

\Gls{sewerking} and \gls{sewerthief} have wriggled through the hidden entrance from \gls{pig} (area \vref{pigShaft}) to the \gls{healersGuild}, and tricked an aging \gls{warden} into coming down to the sewers with them.

\sqpart{Town}% AREA
{Illegal Songs}% NAME
{A bard is caught singing about the wonders of \glsentrytext{lostcity}}% SUMMARY

\Gls{warningbard}, arrives to sing of the wonders of \gls{lostcity}, and how the `golden priests' in the \gls{healersGuild} would bring green-glowing mushrooms of great magical power up from the ground, to feed the people.%
\footnote{See \nameref{green_tower}, \vpageref{green_tower}, for more on these `priests'.}

\Gls{townmaster}, wandering by, overhears this and quietly asks nearby \gls{sunGuard} to arrest him for:

\begin{itemize}
  \item
  Dissent (talking about `priests' as if that were better than simple guilds).
  \item
  Singing out of key.
  \item
  Radical ideas.
  \item
  Performing without a licence.%
  \footnote{Performance licences aren't a thing.}
\end{itemize}

\paragraph{For the rest of the day,}
all of \gls{town} is full of talk about the golden priests of \gls{lostcity}.

\paragraph{If the troupe do not intervene,}
\gls{warningbard} will end up in the cells for six months.
\footnote{For the cells under the \glsentrytext{guard} station, see \nameref{guardstation}, \vpageref{guardstation}.}

\warningbard

\sqpart{Town}% AREA
{Unexpected Ghasts}% NAME
{The dead run through the streets}% SUMMARY

\Gls{sewerthief} forgot to lock a door properly, and a room of ghasts escaped via \nameref{slum_exit} (area \vref{slum_exit}).

\ghast[\npc{\D\Hu}{Old Ghast}]

\ghast[\npc{\D\Hu}{Young Ghast}]

\ghast[\npc{\T[3]\D\Hu}{Other Ghast}]

Any character can summon the nearby \gls{sunGuard} with a \roll{Strength}{Empathy} roll (\tn[7]), to shout out loud.
The guards take 5 rounds to arrive, but every Margin on the roll reduces that time by 1 round, to a minimum of 2 rounds.

\paragraph{At the end of the day,}
the ghasts will be put to the sword one way or another, but the question remains; what brought them here?

\paragraph{If the party track the ghouls,}
they won't find the city streets help much, and the bandits in the sewers will start to clear up the butchers as soon as the horde have gone.
The roll is \roll{Wits}{Vigilance} (\tn[14]).

